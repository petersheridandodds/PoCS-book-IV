% simple disease spreading
% general disease spreading

\section{Basic\ Contagion\ Models}

  \textbf{Contagion models}

  \textbf{Some large questions concerning network contagion:}
    
     
      For a given \alertb{spreading mechanism}
      on a given network,
      what's the \alert{probability}
      that there will be \alertb{global spreading}?
     If spreading does take off, how far will it go?
     
      How do the \alert{details} of the \alertb{network}
      affect the outcome?
     
      How do the \alert{details} of the \alertb{spreading mechanism}
      affect the outcome?
     
      What if the \alertb{seed} is one or many nodes?
    
  

  
   
    \alert{Next up}: We'll look at some fundamental
    kinds of spreading on generalized random networks.
  


  \textbf{Spreading mechanisms}

      
    \includegraphics[angle=-90,width=\textwidth]{rn_spread03}
    
    
    
      \alert{General spreading mechanism}:\\
      State of node $i$ depends on
      history of $i$ and $i$'s neighbors' states.
     
      \alert{Doses} of entity may be stochastic and history-dependent.
     
      May have \alertb{multiple, interacting entities} spreading at once.
    
  

  \textbf{Spreading on Random Networks}

  
   
    For random networks, we know local structure is pure branching.
   
    Successful spreading is $\therefore$ contingent on 
    \alert{single edges} infecting nodes.
              
        
                  
          
          \alertb{Success}\\
          \includegraphics[angle=-90,width=\textwidth]{rn_spread01}
                
        
                  
          
          \alertb{Failure:}\\
          \includegraphics[angle=-90,width=\textwidth]{rn_spread02}
                
         
    Focus on \alert{binary} case with edges and nodes
    either infected or not.
  
    \alert{First big question:} for a given network
    and contagion process, can global spreading from
    a single seed occur?
  


\section{Global\ spreading\ condition}

  \textbf{Global spreading condition}

  
  
    We need to find:\cite{dodds2011b}\\
    $\alertb{\gainratio}$ = the average \# of infected
    edges that one random infected edge brings about.
  
    Call $\alertb{\gainratio}$ the \alertb{gain ratio}.
   
    Define \alertb{$\infprob_{k1}$} as the probability that
    a node of degree $k$ is infected by
    a single infected edge.
  
    $$
    \gainratio
    =
    \sum_{k=0}^{\infty}
    {
      \underbrace{
        \frac{kP_k}{\tavg{k}}
      }_{
        \mbox{ \scriptsize
          \begin{tabular}{l}
            prob. of\\
            connecting to \\
            a degree $k$ node\\
          \end{tabular}
        }
      }
    }
    {
      \bullet
      \underbrace{
        (k-1)
      }_{
        \mbox{\scriptsize
          \begin{tabular}{l}
            \# outgoing \\
            infected \\
            edges
          \end{tabular}
        }
      }
    }
    {
      \bullet
      \underbrace{
        \infprob_{k1}
      }_{
        \mbox{\scriptsize
          \begin{tabular}{l}
            Prob. of \\
            infection
          \end{tabular}
        }
      }
    }
    $$
    $$
    {
      +
      \sum_{k=0}^{\infty}
      \overbrace{
        \frac{kP_k}{\tavg{k}}
      }
    }
    {
      \bullet
      \underbrace{0}_{
        \mbox{\scriptsize
          \begin{tabular}{l}
            \# outgoing \\
            infected \\
            edges
          \end{tabular}
        }
      }
    }
    {
      \bullet
      \underbrace{
        (1-\infprob_{k1})
      }_{
        \mbox{\scriptsize
          \begin{tabular}{l}
            Prob. of \\
            no infection
          \end{tabular}
        }
      }
    }
    $$      
      
    
  \textbf{Global spreading condition}

  
   
    Our global spreading condition is then:
    $$
    \boxed{
      \alertb{
        \gainratio
        =
        \sum_{k=0}^{\infty}
        \frac{kP_k}{\tavg{k}}
        \bullet
        (k-1)
        \bullet
        \infprob_{k1}
        > 1.
        }
      }
    $$
  
    \alert{Case 1:}
    {
      If $\infprob_{k1}=1$ 
    }
    {
      then
      $$
      \gainratio = 
        \sum_{k=0}^{\infty}
        \frac{kP_k}{\tavg{k}}
        \bullet
        (k-1)
        =
        \frac{\tavg{k(k-1)}}{\tavg{k}} > 1.
      $$
    }
  
    \alert{Good:} This is just our giant component condition again.
  

  \textbf{Global spreading condition}

  
   
    \alert{Case 2:}
    {
      If $\infprob_{k1}=\beta<1$ 
    }
    {
      then
      $$
      \gainratio = 
      \sum_{k=0}^{\infty}
      \frac{kP_k}{\tavg{k}}
      \bullet
      (k-1)
      \bullet
      \beta > 1.
      $$
    }
   
    A fraction (1-$\beta$) of edges do not transmit 
    infection.
   
    Analogous phase transition to giant component case
    but \alertb{critical value} of $\tavg{k}$ is \alertb{increased}.
   
    Aka \wordwikilink{http://en.wikipedia.org/wiki/Percolation\_theory}{bond percolation}.
   
    Resulting degree distribution $\tilde{P}_k$:
    $$
    \tilde{P}_k
    =
    \beta^k
    \sum_{i=k}^{\infty}
    \binom{i}{k}
    (1-\beta)^{i-k}
    P_i.
    $$
    \insertassignmentquestionsoft{07}{7}
   
    We can show $F_{\tilde{P}}(x) = F_{P}(\beta x + 1 - \beta)$.
  


\section{Social\ Contagion\ Models}

  \textbf{Global spreading condition}

  
  
    \alert{Cases 3, 4, 5, ...:}
    {
      Now allow $\infprob_{k1}$ to depend on $k$
    }
  
    \alertb{Asymmetry}: Transmission along an edge depends on
    node's degree at other end.
  
    Possibility: $\infprob_{k1}$ increases with $k$...
    {\alert{unlikely}}.
  
    Possibility: $\infprob_{k1}$ is not monotonic in $k$...
    {\alert{unlikely}}.
  
    Possibility: $\infprob_{k1}$ decreases with $k$...
    {\alert{hmmm}}.
  
    $\infprob_{k1} \searrow$ is a plausible representation
    of a simple kind of social contagion.
  
    \alert{The story:}\\
    More well connected people are harder to influence.
  


  \textbf{Global spreading condition}

  
   
    \alert{Example:} $\infprob_{k1}=1/k$.
   
    $$
    \gainratio
    =
    \sum_{k=\alert{1}}^{\infty}
    \frac{kP_k}{\tavg{k}}
    \bullet
    (k-1)
    \bullet
    \infprob_{k1}
    {
      =
      \sum_{k=1}^{\infty}
      (k-1)
      \bullet
      \frac{kP_k}{\tavg{k}}
      \bullet
      \frac{1}{k}
    }
    $$
    $$
    {
      =
      \sum_{k=1}^{\infty}
      \frac{P_k}{\tavg{k}}
      \bullet
      (k-1)
    }
    {
      =
      1 - \frac{1-P_0}{\tavg{k}}
    }
    $$
  
    Since $\gainratio$ is always less than $1$, no spreading
    can occur for this mechanism.
  
    Decay of $\infprob_{k1}$ is too fast.
  
    Result is independent of degree distribution.
  


  \textbf{Global spreading condition}

  
   
    \alert{Example:} $\infprob_{k1}=H(\frac{1}{k}-\phi)$\\
    where $0<\alert{\phi} \le 1$ is a \alert{threshold}
    and $H$ is the \wordwikilink{http://en.wikipedia.org/wiki/Heaviside_step_function}{Heaviside function}.
   
    Infection only occurs for nodes with \alertb{low} degree.
   
    Call these nodes \alert{vulnerables}:\\ 
    they flip
    when \alertb{only one} of their friends flips.
   
    $$
    \gainratio
    =
    \sum_{k=\alert{1}}^{\infty}
    \frac{kP_k}{\tavg{k}}
    \bullet
    (k-1)
    \bullet
    \infprob_{k1}
    {
      =
      \sum_{k=1}^{\infty}
      \frac{kP_k}{\tavg{k}}
      \bullet
      (k-1)
      \bullet
      H\left(
        \frac{1}{k}-\phi
      \right)
    }
    $$
    $$
    {
      =
      \sum_{k=1}^{\alert{\lfloor \frac{1}{\phi} \rfloor}}
      (k-1)
      \bullet
      \frac{kP_k}{\tavg{k}}
    \mbox{\quad where $\lfloor \cdot \rfloor$ means floor.}
    }
    $$
  

  \textbf{Global spreading condition}

  
  
    The uniform threshold model global spreading condition:
    $$
    \gainratio
    =
    \sum_{k=1}^{{\lfloor \frac{1}{\phi} \rfloor}}
    (k-1)
    \bullet
    \frac{kP_k}
    {\tavg{k}} 
    > 1.
    $$
   
    As \alert{$\phi \rightarrow 1$}, all nodes become resilient and $r \rightarrow 0$.
   
    As \alert{$\phi \rightarrow 0$}, all nodes become vulnerable and the contagion
    condition matches up with the giant component condition.
  
    \alert{Key}: If we fix $\phi$ and then vary $\tavg{k}$, we
    may see \alertb{two} phase transitions.
   Added to our standard giant component transition,
    we will see a cut off in spreading as nodes become more connected.
  


  Virtual contagion:
  \wordwikilink{http://en.wikipedia.org/wiki/Corrupted_Blood_incident}{Corrupted Blood},
  a 2005 virtual plague in World of Warcraft:
  
  \begin{center}
    \includegraphics[height=0.7\textheight]{WoW_Corrupted_Blood_Plague.jpg}
  \end{center}


\subsection{Network\ version}

  \textbf{Social Contagion}

  \textbf{Some important models (recap from CSYS 300)}
    
     Tipping models---Schelling (1971)\cite{schelling1971a,schelling1973a,schelling1978a}
      
      
        Simulation on checker boards.
      
        Idea of thresholds.
      
     Threshold models---Granovetter (1978)\cite{granovetter1978a}
     Herding models---Bikhchandani et al. (1992)\cite{bikhchandani1992a,bikhchandani1998a}
      
      
        Social learning theory, Informational cascades,...
      
    
  


  \textbf{Threshold model on a network}

  Original work:

  \bigskip

  \alert{``A simple model of global cascades on random networks''}\\
  D. J. Watts.  Proc. Natl. Acad. Sci., 2002\cite{watts2002a}

  \bigskip

  
   Mean field Granovetter model $\rightarrow$ network model
   Individuals now have a limited view of the world
  


  \textbf{Threshold model on a network}

  
   Interactions between individuals 
    now represented by a network
   Network is \alert{sparse}
   Individual $i$ has $k_i$ contacts
   Influence on each link is \alert{reciprocal} and of \alert{unit weight}
   Each individual $i$ has a fixed threshold $\phi_i$
   Individuals repeatedly poll contacts on network
   Synchronous, discrete time updating 
   Individual $i$ becomes active when\\
    number of active contacts $a_i \ge \phi_i k_i$
   Activation is permanent (SI)
  


%%  \textbf{Word-of-mouth contagion:}
%%  


  \textbf{Threshold model on a network}

      
    \setlength\fboxsep{0pt}
    \setlength\fboxrule{0.5pt}
    \fbox{\includegraphics[angle=0,width=1\textwidth]{contagioncondition3a}}%
    
    \setlength\fboxsep{0pt}
    \setlength\fboxrule{0.5pt}
    \fbox{\includegraphics[angle=0,width=1\textwidth]{contagioncondition3b}}%
    
    \setlength\fboxsep{0pt}
    \setlength\fboxrule{0.5pt}
    \fbox{\includegraphics[angle=0,width=1\textwidth]{contagioncondition3c}}%
  
  
   All nodes have threshold $\phi=0.2$.
  


%% previous work
%% definition of vulnerables
%% global condition



  \textbf{The most gullible}

  \textbf{Vulnerables:}
    
     Recall definition: individuals who can be activated by
    just one contact being active are \alert{vulnerables}.
     The vulnerability condition for node $i$:
      $1/k_i \ge \phi_i$.
    
      Means \# contacts  $k_{i} \le \lfloor 1/\phi_i \rfloor$.
    
      \alert{Key:} For global spreading events (cascades) on random networks, must have a
      \tc{blue}{\textit{global component of vulnerables}}\cite{watts2002a}
     For a uniform threshold $\phi$, our global spreading condition
      tells us when such a component exists:
      $$
      \gainratio
      =
      \sum_{k=1}^{{\lfloor \frac{1}{\phi} \rfloor}}
      \frac{kP_k}{\tavg{k}} 
      \bullet
      (k-1)
      > 1.
      $$
    
  


  \textbf{Example random network structure:}

      
    \includegraphics[width=\textwidth]{2011-04-04random-network-contagion-sketch_3a-tp-5.pdf}
    
    
     
      $\Omega_{\textnormal{crit}}$ = critical mass = global vulnerable component
     
      $\Omega_{\textnormal{trig}}$ = triggering component
     
      $\Omega_{\textnormal{final}}$ = potential extent of spread
     
      $\Omega$ = entire network
    
    \bigskip
  $$
  \Omega_{\textnormal{crit}} 
  \subset
  \Omega_{\textnormal{trig}};
  \
  \Omega_{\textnormal{crit}} 
  \subset
  \Omega_{\textnormal{final}};
  \
  \mbox{and}
  \
  \Omega_{\textnormal{trig}},
  \Omega_{\textnormal{final}} 
  \subset
  \Omega.
  $$


%%
%%   %%     \textbf{Cascade window}
%%     
%%     \textbf{When does a global cluster of vulnerables exist?}
%%       
%%        $z$ = average number of contacts per individual.
%%       
%%       
%%        
%%         \tc{blue}{Low $z$:} No cascades in poorly connected networks.\\
%%         No global clusters of any kind.
%%        
%%         \tc{blue}{High $z$:} Giant component exists but not enough vulnerables.
%%        
%%         \tc{blue}{Intermediate $z$:} Global cluster of vulnerables exists.\\
%%         Cascades are possible in \tc{red}{``Cascade window.''}
%%       
%%     
%%
%%  
%% previous work
%% explanation of cascade window


%% previous work
%% cascade window 1
%% figure of basic cascade window outline

  \textbf{Global spreading events on random networks}

      
    \setlength\fboxsep{0pt}
    \setlength\fboxrule{0.5pt}
    \fbox{\includegraphics[width=\textwidth]{watts2002a_fig2b}}\\
    \small{( n.b., $z = \tavg{k}$)}
    
    
    
      \alert{Top curve:} final fraction infected if successful.
    
      \alert{Middle curve:} chance of starting a global spreading event (cascade).
     
      \alert{Bottom curve:} fractional size of vulnerable subcomponent.\cite{watts2002a}
    
  
  
   
    Global spreading events occur only if size of vulnerable subcomponent $>0$.\\
   
    System is robust-yet-fragile just below upper boundary\cite{carlson1999a,carlson2000a,sornette2003a}
   
    `Ignorance' facilitates spreading.
  


  \textbf{Cascades on random networks}

      
    \includegraphics[width=\textwidth]{2011-04-04random-network-contagion-sketch_3c-tp-5.pdf}
    
     
      Above lower phase transition
    
    
    \includegraphics[width=\textwidth]{2011-04-04random-network-contagion-sketch_3b-tp-5.pdf}
    
     
      Just below upper phase transition
    
  


  \textbf{Cascades on random networks}

          
      \setlength\fboxsep{0pt}
      \setlength\fboxrule{0.5pt}
      \fbox{\includegraphics[width=\textwidth]{watts2002a_fig2a}}\\
      \small{( n.b., $z = \tavg{k}$)}
      
      
       Time taken for cascade to spread through network.\cite{watts2002a}
       Two phase transitions.
      
    
    
     Largest vulnerable component = \alert{critical mass}.
     Now have endogenous mechanism for spreading from
      an individual to the critical mass and then beyond.
    


  \textbf{Cascade window for random networks}

  \begin{center}
    \setlength\fboxsep{0pt}
    \setlength\fboxrule{0.5pt}
    \fbox{\includegraphics[width=.6\textwidth]{figtransitions_random01_noname}}\\
  \end{center}
  \small{( n.b., $z = \tavg{k}$)}
  
   Outline of cascade window for random networks. 
  


  \textbf{Cascade window for random networks}

  \includegraphics[angle=-90,width=\textwidth]{figcascadewindow_cut4.pdf}




\subsection{All-to-all\ networks}

  \textbf{Social Contagion}
  
  \textbf{Granovetter's Threshold model---recap}
          
      \includegraphics[width=\textwidth]{figthreshold_eg3_noname}
      
      
       Assumes deterministic response functions
       $\phi_\ast$ = threshold of an individual.
       $f(\phi_\ast)$ = distribution of thresholds in a population.
       $F(\phi_\ast)$ = cumulative distribution = $\int_{\phi_\ast'=0}^{\phi_\ast} f(\phi_\ast') \dee{\phi_\ast'}$
       $\phi_t$ = fraction of people `rioting' at time step $t$.
      
      
  

  \textbf{Social Sciences---Threshold models}
 
   
   
   At time $t+1$, fraction rioting
   = fraction with $\phi_\ast \le \phi_t$.
   
   \[ \phi_{t+1} = \int_{0}^{\phi_t} f(\phi_\ast) \dee{\phi_\ast}
   = \left. F(\phi_\ast) \right|_{0}^{\phi_t} = F(\phi_t) \]
   
   $\Rightarrow$ Iterative maps of the unit interval $[0, 1]$.
   


  \textbf{Social Sciences---Threshold models}

  Action based on perceived behavior of others.

  \includegraphics[width=1\textwidth]{figthreshold_noname}

  
   Two states: S and I
   Recover now possible (SIS)
   $\phi$ = fraction of contacts `on' (e.g., rioting)
   Discrete time, synchronous update (strong assumption!)
   This is a \alert{Critical mass model}
  



%% %%   \textbf{Social Sciences---Threshold models}
%% 
%%   \includegraphics[width=.45\textwidth]{figthreshold3_noname}
%%   \includegraphics[width=.45\textwidth]{figthresholdF3b_noname}
%% 
%%   
%%    Critical mass model
%%   
%% 
%% 
  \textbf{Social Sciences---Threshold models}

  \includegraphics[width=.45\textwidth]{figthreshold2_noname}
  \includegraphics[width=.45\textwidth]{figthresholdF2b_noname}\\

  
   Example of single stable state model
  



  \textbf{Social Sciences---Threshold models}

  \textbf{Implications for collective action theory:}
    
     Collective uniformity $\not\Rightarrow$ individual uniformity
     Small individual changes $\Rightarrow$ large global changes
    
  

  \textbf{Next:}
    
     Connect mean-field model to network model.
     Single seed for network model: $1/N \rightarrow 0$.
     Comparison between network and mean-field model
      sensible for vanishing seed size for the latter.
    
  



  \textbf{All-to-all versus random networks}
  \centering
  \includegraphics[height=.8\textheight]{figcascwind5_noname}\\

  


%% %%%%%%%%%%%%%%
%%   %% initiators %
%% %%%%%%%%%%%%%%
%% 
%%   %% cascade window
%%   %% comparison between different types of initiators
%%   %% 
%% %%
%% %%   \textbf{cascade initiators, $\phi=0.18$}
%%   Activate random individuals:\\
%%   \centering
%%   \includegraphics[width=0.62\textwidth]{figtest_nw_threshold_cwi04c_noname}\\
%%   Cascade initiators for $k_{\textnormal{init}}=1$, 2, 3, 4, 6, and 9.
%%   %% 
%%   %% 
%% 
%%   %% cascade window
%%   %% comparison between different types of initiators
%% 
%%   %%  
%% %% %%
%% %%   \textbf{cascade initiators, $\phi=0.18$}
%%   \centering
%%   \includegraphics[width=0.65\textwidth]{figtest_nw_threshold_cwi04i_noname}\\
%%   Averagely connected nodes versus nodes in top 10 percent.
%%   
%%   
%%   
%%   
%%   %% average degree of everyone versus initiators
%%   
%%   \centering
%% %% 
%% %%   \textbf{cascade initiators---average degree}
%%   \includegraphics[width=0.6\textwidth]{figtest_network_threshold2_19x1e_noname}\\
%%   \tc{blue}{dashed line:} mean degree of all individuals with $k>0$.\\
%%   \tc{blue}{solid line:} mean degree of cascade initiators.
%%   
%%   
%% %% 
%% 
%% %% distribution of initiator degrees
%% 
%% %%
%% %% %% %%    \textbf{cascade initiators---degree distributions}
%% %%   \[
%% %%   \begin{array}{l}
%% %%     P_{k,\textnormal{init}}  = [1 - (1-S)^k] \cdot P_k\\
%% %%     \\
%% %%     \qquad = P({\textnormal{cascade}} | k) \cdot P_k \\
%% %%     \\
%% %%     \\
%% %%     S = \mbox{size of} \\
%% %%     \qquad \mbox{vulnerable cluster} \\
%% %%   \end{array}
%% %%   \ \ \raisebox{-5cm}{
%% %%     \includegraphics[width=0.5\textwidth]{figtest_nw_thr2_06fh7_noname}
%% %%   }
%% %%   \]
%% %% 
%% %% 
%% %% 
%% %% 
%% %%   %% %%
%% %% %% %%    \textbf{cost/benefit analysis}
%% %%   Sampling individuals $\propto$ number sampled $n$.\\
%% %%   Triggering relatively very costly $\rightarrow$ trigger one individual only.\\
%% %%   Choose most connected individual from $n$ samples.
%% %%   \includegraphics[width=0.45\textwidth]{figtest_nw_thr2_06fcost3_noname}
%% %%   \includegraphics[width=0.45\textwidth]{figtest_nw_thr2_06fcost4_noname}\\
%% %%   $\bullet$ $n \nearrow$ as cost $\searrow$.\\
%% %%   $\bullet$ Chosen individual's degree increases slowly with $n$.
%% %%   
%% %% 
%% %% 
%% 
%% 
%% %% %% mean degree of early adopters
%% %% %% + way disease spreads
%% %% 
%% %%   %% %%
%% 
%% %%   \textbf{Early Adopters---Mean degree + Rate of adoption}
%% 
%%   \raisebox{-4cm}{    
%%     \includegraphics[width=0.62\textwidth]{figtest_nw_thr2_06eh2_6_noname}
%%   }
%%   \begin{tabular}{l}
%%     \tc{red}{---} vulnerables  \\
%%     \tc{blue}{---} non-vulnerables \\
%%     --- total
%%   \end{tabular}
%% 
%% \end{frame} 
%% 
%% %% % SIR comparison
%% %% 
%% %%
%% 
%% %%   \textbf{Comparison to disease spreading models}
%%   
%%   \raisebox{-4cm}{    
%%     \includegraphics[width=0.45\textwidth]{figtest_nw_SIR_01e2_noname}
%%   }
%%   \begin{tabular}{l}
%%     SIR model on random graph \\
%%     \\
%%     early adopters \textit{always}\\
%%     above average \\
%%   \end{tabular}
%% 
%%   Probability of connecting
%%   to a $k$ degree node $\propto k P_k$.
%%   
%% 
%% %%   \textbf{Early adopters---degree distributions}
%% 
%%   \begin{tabular}{ccccc}
%%     \includegraphics[width=0.2\textwidth]{figtest_nw_thr2_06e_1_mod_noname}&
%%     \includegraphics[width=0.2\textwidth]{figtest_nw_thr2_06e_2_mod_noname}&
%%     \includegraphics[width=0.2\textwidth]{figtest_nw_thr2_06e_3_mod_noname}&
%%     \includegraphics[width=0.2\textwidth]{figtest_nw_thr2_06e_4_mod_noname}\\
%%     \includegraphics[width=0.2\textwidth]{figtest_nw_thr2_06e_5_mod_noname}&
%%     \includegraphics[width=0.2\textwidth]{figtest_nw_thr2_06e_6_mod_noname}&
%%     \includegraphics[width=0.2\textwidth]{figtest_nw_thr2_06e_7_mod_noname}&
%%     \includegraphics[width=0.2\textwidth]{figtest_nw_thr2_06e_8_mod_noname}\\
%%     \includegraphics[width=0.2\textwidth]{figtest_nw_thr2_06e_9_mod_noname}&
%%     \includegraphics[width=0.2\textwidth]{figtest_nw_thr2_06e_10_mod_noname}&
%%     \includegraphics[width=0.2\textwidth]{figtest_nw_thr2_06e_11_mod_noname}&
%%     \includegraphics[width=0.2\textwidth]{figtest_nw_thr2_06e_12_mod_noname}\\
%% %%    \includegraphics{figtest_nw_thr2_06e_13_mod_noname.ps,width=0.2\textwidth}&
%% %%    \includegraphics{figtest_nw_thr2_06e_14_mod_noname.ps,width=0.2\textwidth}&
%% %%    \includegraphics{figtest_nw_thr2_06e_15_mod_noname.ps,width=0.2\textwidth}
%% %%    & 
%%   \end{tabular}
%% $$P_{k,t} \mbox{\ versus\ } k$$
%% 
%% %% 
%%    %%
%% 
%% %%     \textbf{The multiplier effect}
%%   
%%     \centering
%%     \includegraphics[width=0.9\textwidth]{fignw_threshold_gamma_multiplier13_avg_noname}
%%     
%%     Gamma distributed degrees (skewed)
%% %    \includegraphics[width=0.48\textwidth]{fignw_threshold_ramify_multiplier20_21comb3mod_noname}
%% %% 
%% %% 
%% 
%% %% 
%% %%%%%%%%%%%%%%%
%% %% 1c  unconnected, room for a general model
%% %%    rumours don't spread like SIR models
%% %%    diseases aren't like threshold models
%% 
%% 
%% %%%%%%%%%%%%%% extensions
%% %%%%%%%%%%%%%% threshold model
%% %%%%%%%%%%%%%% 1. group structure
%% 
%% %%   \textbf{Special subnetworks can act as triggers}
%% 
%%   \includegraphics[width=0.8\textwidth]{betheladder5}
%%   
%%    $\phi=1/3$ for all nodes
%%   
%%   
%%   
%% 
\neuralreboot{LMiYjkG4onM}{}{}{Pangolin happiness:}

\section{Theory}

  \textbf{Threshold contagion on random networks}

  \textbf{\alert{Three key pieces} to describe analytically:}
    
    
      The fractional size of the largest subcomponent of vulnerable nodes, 
      \alertb{$\Svuln$}.
    
      The chance of starting a global spreading event, \alertb{$\Ptrig = \Strig$}.
    
      The expected final size of any successful spread, \alertb{$S$}.
      
       n.b., the distribution of $S$ is almost always bimodal.
      

    
  


  \textbf{Example random network structure:}

      
    \includegraphics[width=\textwidth]{2011-04-04random-network-contagion-sketch_3a-tp-5.pdf}
    
    
     
      $\Omega_{\textnormal{crit}}$ = $\Omega_{\textnormal{vuln}}$ = critical mass = global vulnerable component
     
      $\Omega_{\textnormal{trig}}$ = triggering component
     
      $\Omega_{\textnormal{final}}$ = potential extent of spread
     
      $\Omega$ = entire network
    
    \bigskip
  $$
  \Omega_{\textnormal{crit}} 
  \subset
  \Omega_{\textnormal{trig}};
  \
  \Omega_{\textnormal{crit}} 
  \subset
  \Omega_{\textnormal{final}};
  \
  \mbox{and}
  \
  \Omega_{\textnormal{trig}},
  \Omega_{\textnormal{final}} 
  \subset
  \Omega.
  $$

\subsection{Spreading\ possibility}

  \textbf{Threshold contagion on random networks}

  
   
    \alert{First goal:} 
    Find the largest component of vulnerable nodes.
   
    Recall that for finding the giant component's size, 
    we had to solve:
    $$
    {
      F_{\pi}(x)
      =
      x F_{P}
      \left(
        F_{\rho} (x)
      \right)
    }
    \mbox{\ \  and \ }
    {
      F_{\rho}(x)
      =
      x F_{R}
      \left(
        F_{\rho} (x)
      \right)
    }
    $$
   
    We'll find a similar result for 
    the subset of nodes that are vulnerable.
   
    This is a node-based percolation problem.
   
    For a general monotonic threshold distribution \alert{$f(\phi)$},
    a degree $k$ node is vulnerable with probability
    $$
    \infprob_{k1} = \int_{0}^{1/k} f(\phi) \dee{\phi}.
    $$
  


  \textbf{Threshold contagion on random networks}

  
  
    Everything now revolves around the \alert{modified} generating function:
    $$
    F_P^{(\vuln)}(x) 
    = 
    \sum_{k=0}^{\infty}
    \infprob_{k1}
    P_k
    x^k.
    $$
  
    Generating function for friends-of-friends distribution is
    related in same way as before:
    $$
    F_R^{(\vuln)}(x) 
    = 
    \frac{
      \diff{}{x}{F}_P^{(\vuln)}(x)}
    {\diff{}{x} {F}_P^{(\vuln)}(x)|_{x=1}}.
    $$
  


  \textbf{Threshold contagion on random networks}

  
   Functional relations for component size g.f.'s
    are almost the same...
    $$
    {
      {F}_{\pi}^{(\vuln)}(x)
      =
    }
    {
      \underbrace{
        1-{F}_P^{(\vuln)}(1)
      }_{
        \mbox{
          \scriptsize
          \begin{tabular}{l}
            central node \\
            is not \\
            vulnerable
          \end{tabular}
        }
      }
      +
    }
    {
      x {F}_{P}^{(\vuln)}
      \left(
        {F}_{\rho}^{(\vuln)} (x)
      \right)
    }
    $$
    $$
    {
      {F}_{\rho}^{(\vuln)}(x)
      =
    }
    {
      \underbrace{
        1-{F}_R^{(\vuln)}(1)
      }_{
        \mbox{
          \scriptsize
          \begin{tabular}{l}
            first node \\
            is not \\
            vulnerable
          \end{tabular}
        }
      }
      +
    }
    {
      x {F}_{R}^{(\vuln)}
      \left(
        {F}_{\rho}^{(\vuln)} (x)
      \right)
    }
    $$
  
    Can now solve as before to find $S_{\textnormal{vuln}} = 1 - F_\pi^{(\vuln)}(1)$.
  


\neuralreboot{jofNR_WkoCE}{60}{120}{Vulpine vocalization:}

\subsection{Spreading\ probability}

  \textbf{Threshold contagion on random networks}

  
   \alert{Second goal:}
    Find probability of triggering largest vulnerable 
    component.
   
    Assumption is \alert{first node} is \alertb{randomly chosen}.
  
    \alertb{Same set up} as for vulnerable component except
    now we don't care if the initial node is vulnerable or not:
    $$
    {F}_{\pi}^{(\textnormal{trig)}}(x)
    =
    x \alert{{F}_{P}}
    \left(
      {F}_{\rho}^{(\vuln)} (x)
    \right)
    $$
    $$
    {F}_{\rho}^{(\vuln)}(x)
    =
    1-{F}_R^{(\vuln)}(1)
    +
    x {{F}_{R}^{(\vuln)}}
    \left(
      {F}_{\rho}^{(\vuln)} (x)
    \right)
    $$
  
    Solve as before to find $\Ptrig =  \Strig = 1 - F_\pi^{(\textnormal{trig)}}(1)$.
  

\begin{frame}[plain]
  \includegraphics[width=\textwidth]{2014-03-17advice-animals-cheezburger.jpg}

\subsection{Final\ size}

  \textbf{Threshold contagion on random networks}

  
  
    \alert{Third goal:} Find expected fractional size of spread.
  
    Not obvious even for uniform threshold problem.
  
    Difficulty is in figuring out if and when
    nodes that need $\ge 2$ hits switch on.
  
    Problem \alertb{solved} for infinite seed case by Gleeson and Cahalane:\\
    ``Seed size strongly affects cascades on random networks,'' 
    Phys.\ Rev.\ E, 2007.\cite{gleeson2007a}
  
    Developed further by Gleeson
    in ``Cascades on correlated and modular random networks,'' 
    Phys.\ Rev.\ E, 2008.\cite{gleeson2008a}
  


  \textbf{Expected size of spread}

  \textbf{Idea:}
    
     
      Randomly turn on a fraction $\phi_0$ of nodes at time $t=0$
     
      Capitalize on local branching network structure of random
      networks (again)
     
      Now think about what must happen for
      a specific node $i$ to become active at time $t$:
    [$\bullet$]
      \alertb{$t=0$:} $i$ is one of the seeds (prob = $\phi_0$)
    [$\bullet$]
      \alertb{$t=1$:} $i$ was not a seed but enough of $i$'s friends switched
      on at time $t=0$ so that $i$'s threshold is now exceeded.
    [$\bullet$] 
      \alertb{$t=2$:} enough of $i$'s friends and friends-of-friends switched
      on at time $t=0$ so that $i$'s threshold is now exceeded.
    [$\bullet$] 
      \alertb{$t=n$:} enough nodes within $n$ hops of $i$ 
      switched on at $t=0$ and their effects have propagated to reach $i$.
    
  



  \textbf{Expected size of spread}

  \includegraphics[angle=-90,width=\textwidth]{figfullspread.pdf}
  \includegraphics[angle=-90,width=\textwidth]{figfullspread2.pdf}
  \includegraphics[angle=-90,width=\textwidth]{figfullspread3.pdf}
  \includegraphics[angle=-90,width=\textwidth]{figfullspread4.pdf}
  \includegraphics[angle=-90,width=\textwidth]{figfullspread5.pdf}


  \textbf{Expected size of spread}

  \includegraphics[angle=-90,width=\textwidth]{figfullspread_circ1.pdf}
  \includegraphics[angle=-90,width=\textwidth]{figfullspread_circ2.pdf}
  \includegraphics[angle=-90,width=\textwidth]{figfullspread_circ3.pdf}
  \includegraphics[angle=-90,width=\textwidth]{figfullspread_circ4.pdf}
  \includegraphics[angle=-90,width=\textwidth]{figfullspread_circ5.pdf}



  \textbf{Expected size of spread}

  \textbf{Notes:}
    
     
      Calculations presume
      nodes do not become inactive (strong restriction, liftable)
    
      Not just for threshold model---works
      for a wide range of contagion processes.
    
      We can analytically determine the entire time evolution,
      not just the final size.
    
      We can in fact determine \\
      $\Prob$(node of degree $k$ switches on at time $t$).
    
      Even more, we can compute:
      $\Prob$(specific node $i$ switches on at time $t$).
    
      Asynchronous updating can be handled too.
    
  



  \textbf{Expected size of spread}

  \textbf{Pleasantness:}
    
     
      \alertb{Taking off from a single seed story} is about \alert{expansion} away from
      a node.
     
      \alertb{Extent of spreading story} is about \alert{contraction} at
      a node.
    
    \includegraphics[width=\textwidth]{contraction-expansion-tp-10}
  


  \textbf{Expected size of spread}

  
  
    \alert{Notation:}
    $ \phi_{k,t} = 
    \Prob(\mbox{a degree $k$ node is active at time $t$}) $.
  
    \alert{Notation:}
    $\infprob_{k j} = \Prob$ (a degree $k$ node becomes active
    if $j$ neighbors are active).
   
    Our starting point: $ \phi_{k,0} = \phi_0$.
  
    $ 
    \alertb{\binom{k}{j}
      \phi_0^{\, j}
      (1-\phi_0)^{k-j} }
    $ 
    =
    $\Prob$ ($j$ of a degree $k$ node's neighbors were seeded at time $t=0$).
   
    Probability a degree $k$ node was a seed at $t=0$ is $\alert{\phi_0}$ (as above).
   
    Probability a degree $k$ node was not a seed at $t=0$ is $\alert{(1-\phi_0)}$.
   
    Combining everything, we have:
    $$ 
    \phi_{k,1}
    = 
    \alert{\phi_{0}}
    + 
    \alert{(1-\phi_{0})}
    \sum_{j=0}^{k}
    \alertb{\binom{k}{j}
    \phi_0^{\, j}
    (1-\phi_0)^{k-j} }
  \infprob_{k j}.
    $$
  


  \textbf{Expected size of spread}

  
  
    For general $t$, we need to know
    the probability an edge coming into a degree $k$ node
    at time $t$ is active.
  
    \alert{Notation:} call this probability $\theta_t$.
  
    We already know $\theta_0 = \phi_0$.
  
    Story analogous to $t=1$ case.  For specific node $i$:
    $$
    \phi_{i,t+1}
    = 
    \alert{\phi_{0}}
    + 
    \alert{(1-\phi_{0})}
    \sum_{j=0}^{k_i}
    \alertb{\binom{k_i}{j}
    \theta_t^{\, j}
    (1-\theta_t)^{k_i-j}}
    \infprob_{k_i j}.
    $$
  
    Average over all nodes with degree $k$ to obtain expression for $\phi_{t+1}$:
    $$
    \phi_{t+1}
    = 
    \alert{\phi_{0}}
    + 
    \alert{(1-\phi_{0})}
    \sum_{k=0}^{\infty} P_k 
    \sum_{j=0}^{k}
    \alertb{\binom{k}{j}
    \theta_t^{\, j}
    (1-\theta_t)^{k-j}}
    \infprob_{kj}.
    $$
  
    So we need to compute $\theta_t$...  {\alertb{massive excitement...}}
  
  

  \textbf{Expected size of spread}
  
  \textbf{First connect $\theta_0$ to $\theta_1$:}
    
    
      $
      \theta_{1}
      =
      \phi_0 +
      $
      $$
      (1-\phi_0)
      \sum_{k=1}^{\infty}
      \alert{\frac{k P_k}{\tavg{k}}}
      \alertb{\sum_{j=0}^{k-1}}
      \binom{k-1}{j}
      \theta_{0}^{\ j}
      (1-\theta_{0})^{k-1-j}
      \infprob_{kj}
      $$
    
      $ \alert{\frac{kP_k}{\tavg{k}}} = Q_k$ = $\Prob$ (edge connects to a degree $k$ node).
    
      \alertb{$\sum_{j=0}^{k-1}$} piece gives $\Prob$ (degree node $k$ activates
      if $j$ of its $k-1$ incoming neighbors are active).
    
      $\phi_0$ and $(1-\phi_0)$ terms account for state of node at time $t=0$.
    
      See this all generalizes to give $\theta_{t+1}$ in terms of $\theta_{t}$...
    
  

  \textbf{Expected size of spread}
  
  \textbf{Two pieces: edges first, and then nodes}
    
    
      $
      \theta_{t+1}
      =
      \underbrace{\phi_0}_{\alertb{\mbox{exogenous}}} 
      $
      $$
      +
      (1-\phi_0)
      \underbrace{
      \sum_{k=1}^{\infty}
      \frac{k P_k}{\tavg{k}}
      \sum_{j=0}^{k-1}
      \binom{k-1}{j}
      \theta_{t}^{\ j}
      (1-\theta_{t})^{k-1-j}
      \infprob_{kj}
      }_{\alertb{\mbox{social effects}}}
      $$
      with $\theta_0 = \phi_0$.
    
      $ 
      \phi_{t+1}
      = 
      $
      $$
      \underbrace{\phi_0}_{\alertb{\mbox{exogenous}}} 
      + 
      (1-\phi_{0})
      \underbrace{
      \sum_{k=0}^{\infty}
      P_k
      \sum_{j=0}^{k}
      \binom{k}{j}
      \theta_t^{\, j}
      (1-\theta_t)^{k-j} 
      \infprob_{kj}
      }_{\alertb{\mbox{social effects}}}.
      $$
    
  

%% not true:
%%   
%%    
%%     \alert{Observe:} $\theta_{t+1}$ is an increasing function of $\theta_{t}$
%%    
%%     $\Rightarrow$ Stable attracting fixed point(s) must exist: $\theta_t \rightarrow \theta_\infty$
%%   



  \textbf{Comparison between theory and simulations}

      
    \includegraphics[width=\textwidth]{gleeson2007a_fig1.pdf}
    
    {\small From Gleeson and Cahalane\cite{gleeson2007a}}
    
    
     
      Pure random networks with simple threshold responses
     
      $R$ = uniform threshold (our $\phi_\ast$);
      $z$ = average degree; $\rho = \phi$; $q = \theta$; $N=10^5$.
     
      $\phi_0 = 10^{-3}$, $0.5 \times 10^{-2}$,
      and 
      $10^{-2}$.
    
      Cascade window is for $\phi_0 = 10^{-2}$ case.
    
      Sensible expansion of cascade window as $\phi_0$ increases.
    
  
  \textbf{Notes:}

    
    
      Retrieve cascade condition for 
      spreading from a single seed in limit $\phi_0 \rightarrow 0$.
     
      Depends on map $\theta_{t+1} = G(\theta_{t};\phi_0)$.
     
      First: if self-starters are present, some activation is assured:
      $$
      G(0;\phi_0) = 
      \sum_{k=1}^{\infty} 
      \frac{kP_k}{\tavg{k}}
      \bullet
      \infprob_{k0} > 0.
      $$
      meaning $\infprob_{k0}>0$ for at least one value of $k \ge 1$.
     
      If $\theta=0$ is a fixed point of $G$ (i.e., $G(0;\phi_0) = 0$)
      then spreading occurs if
      $$
      G'(0;\phi_0) = 
      \sum_{k=0}^{\infty} 
      \frac{k P_k}{\tavg{k}}
      \bullet
      (k-1) 
      \bullet
      \infprob_{k1} > 1.
      $$
      \insertassignmentquestionsoft{08}{8}
    

  \textbf{Notes:}

  \textbf{In words:}    
    
     
      If $G(0;\phi_0) > 0$, spreading must occur because
      some nodes turn on for free.
     
      If $G$ has an \alert{unstable fixed point} at $\theta = 0$,
      then cascades are also always possible.
    
  

  \textbf{Non-vanishing seed case:}
    
     
      Cascade condition is more complicated for
      $\phi_0 > 0$.
     
      If $G$ has a \alert{stable fixed point} at $\theta = 0$,
      and an \alert{unstable fixed point} for some $0 < \theta_\ast < 1$,
      then for $\theta_0  > \theta_\ast$, spreading takes off.
    
      Tricky point: $G$ depends on $\phi_0$, so as we change
      $\phi_0$, we also change $G$.
    
  


  \textbf{General fixed point story:}

      
    \includegraphics[angle=0,width=\textwidth]{figGfunction01.pdf}
    
    \includegraphics[angle=0,width=\textwidth]{figGfunction02.pdf}
    
    \includegraphics[angle=0,width=\textwidth]{figGfunction03.pdf}
  
  
   
    Given $\theta_0 (= \phi_0)$, $\theta_\infty$ will be 
    the nearest stable fixed point, either above or below.
   
    n.b., adjacent fixed points must have opposite stability types.
   
    \alert{Important:}
    Actual form of $G$ depends on $\phi_0$.  
  
    So choice of $\phi_0$ dictates both $G$ and starting
    point---can't start anywhere for a given $G$.
  


  \textbf{Comparison between theory and simulations}

      
    \includegraphics[width=\textwidth]{gleeson2007a_fig2.pdf}

    {\small From Gleeson and Cahalane\cite{gleeson2007a}}
    
    
     
      Now allow thresholds to be distributed
      according to a Gaussian with mean $R$.
     
      $R$ = \alert{0.2}, \alertb{0.362}, and 0.38; $\sigma = 0.2$.
    
      $\phi_0 = 0$ but some nodes have thresholds $\le 0$
      so effectively $\phi_0 > 0$.
    
      Now see a (nasty) discontinuous phase transition
      for low $\tavg{k}$.
    
  
  \textbf{Comparison between theory and simulations}

      
    \includegraphics[width=\textwidth]{gleeson2007a_fig3.pdf}

    {\small From Gleeson and Cahalane\cite{gleeson2007a}}
    
    
     
      Plots of stability points for $\theta_{t+1} = G(\theta_t; \phi_0)$.
     
      n.b.: 0 is not a fixed point here: $\theta_0 = 0$ always takes off.
     
      Top to bottom: $R$ = 0.35, 0.371, and 0.375.
     
      n.b.: higher values of $\theta_0$ for (b) and (c)
      lead to higher fixed points of $G$.
     
      Saddle node bifurcations 
      appear and merge (b and c).
    
  

  \textbf{Spreadarama}

  \textbf{Bridging to single seed case:}
    
     
      Consider largest vulnerable component
      as initial set of seeds.
    
      Not quite right as spreading must move
      through vulnerables.
    
      But we can usefully think of the vulnerable
      component as activating at time $t=0$
      because order doesn't matter.
    
      Rebuild $\phi_t$ and $\theta_t$ expressions...
    
  


  \textbf{Spreadarama}

  \textbf{Two pieces modified for single seed:}
    
    
      $
      \theta_{t+1}
      =
      \thetavuln 
      \ +
      $
      $$
      (1-\thetavuln)
      \sum_{k=1}^{\infty}
      \frac{k P_k}{\tavg{k}}
      \sum_{j=0}^{k-1}
      \binom{k-1}{j}
      \theta_{t}^{\ j}
      (1-\theta_{t})^{k-1-j}
      \infprob_{kj}
      $$
      with $\theta_0 = \thetavuln$ = $\Prob$ an edge leads
      to the giant vulnerable component (if it exists).
    
      $ 
      \phi_{t+1}
      = 
      \Svuln
      \ + 
      $
      $$
      (1-\Svuln)
      \sum_{k=0}^{\infty}
      P_k
      \sum_{j=0}^{k}
      \binom{k}{j}
      \theta_t^{\, j}
      (1-\theta_t)^{k-j} 
      \infprob_{kj}.
      $$
    
  



  \textbf{Time-dependent solutions}

  \textbf{Synchronous update}
    
     Done: Evolution of $\phi_t$ and $\theta_t$
      given exactly by the maps we have
      derived.
    
  

  \textbf{Asynchronous updates}
    
     
      Update nodes with probability $\alpha$.
     
      As $\alpha \rightarrow 0$, updates become
      effectively independent.
     
      Now can talk about $\phi(t)$ and $\theta(t)$.
    
  


\begin{comment}
  
\section{Appendix}

  \textbf{Threshold contagion on random networks}

  
   
    \alert{First goal:} 
    Find the largest component of vulnerable nodes.
   
    Recall that for finding the giant component's size, 
    we had to solve:
    $$
    {
      F_{\pi}(x)
      =
      x F_{P}
      \left(
        F_{\rho} (x)
      \right)
    }
    \mbox{\ \  and \ }
    {
      F_{\rho}(x)
      =
      x F_{R}
      \left(
        F_{\rho} (x)
      \right)
    }
    $$
   
    We'll find a similar result for 
    the subset of nodes that are vulnerable.
   
    This is a node-based percolation problem.
   
    For a general monotonic threshold distribution \alert{$f(\phi)$},
    a degree $k$ node is vulnerable with probability
    $$
    \infprob_{k1} = \int_{0}^{1/k} f(\phi) \dee{\phi}.
    $$
  


  \textbf{Threshold contagion on random networks}

  
  
    Everything now revolves around the \alert{modified} generating function:
    $$
    F_P^{(\vuln)}(x) 
    = 
    \sum_{k=0}^{\infty}
    \infprob_{k1}
    P_k
    x^k.
    $$
  
    Generating function for friends-of-friends distribution is
    related in same way as before:
    $$
    F_R^{(\vuln)}(x) 
    = 
    \frac{
      \diff{}{x}{F}_P^{(\vuln)}(x)}
    {\diff{}{x} {F}_P^{(\vuln)}(x)|_{x=1}}.
    $$
  


  \textbf{Threshold contagion on random networks}

  
   Functional relations for component size g.f.'s
    are almost the same...
    $$
    {
      {F}_{\pi}^{(\vuln)}(x)
      =
    }
    {
      \underbrace{
        1-{F}_P^{(\vuln)}(1)
      }_{
        \mbox{
          \scriptsize
          \begin{tabular}{l}
            central node \\
            is not \\
            vulnerable
          \end{tabular}
        }
      }
      +
    }
    {
      x {F}_{P}^{(\vuln)}
      \left(
        {F}_{\rho}^{(\vuln)} (x)
      \right)
    }
    $$
    $$
    {
      {F}_{\rho}^{(\vuln)}(x)
      =
    }
    {
      \underbrace{
        1-{F}_R^{(\vuln)}(1)
      }_{
        \mbox{
          \scriptsize
          \begin{tabular}{l}
            first node \\
            is not \\
            vulnerable
          \end{tabular}
        }
      }
      +
    }
    {
      x {F}_{R}^{(\vuln)}
      \left(
        {F}_{\rho}^{(\vuln)} (x)
      \right)
    }
    $$
  
    Can now solve as before to find $S_{\textnormal{vuln}} = 1 - F_\pi^{(\vuln)}(1)$.
  


  \textbf{Threshold contagion on random networks}

  
   \alert{Second goal:}
    Find probability of triggering largest vulnerable 
    component.
   
    Assumption is \alert{first node} is \alertb{randomly chosen}.
  
    \alertb{Same set up} as for vulnerable component except
    now we don't care if the initial node is vulnerable or not:
    $$
    {F}_{\pi}^{(\textnormal{trig)}}(x)
    =
    x \alert{{F}_{P}}
    \left(
      {F}_{\rho}^{(\vuln)} (x)
    \right)
    $$
    $$
    {F}_{\rho}^{(\vuln)}(x)
    =
    1-{F}_R^{(\vuln)}(1)
    +
    x {{F}_{R}^{(\vuln)}}
    \left(
      {F}_{\rho}^{(\vuln)} (x)
    \right)
    $$
  
    Solve as before to find $\Ptrig =  \Strig = 1 - F_\pi^{(\textnormal{trig)}}(1)$.
  

\end{comment}
