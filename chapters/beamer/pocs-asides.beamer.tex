\begin{frame}
  \frametitle{Applied knot theory:}

  ``Designing tie knots by random walks''\cite{fink1999a}
  Fink and Mao, Nature, 1999.

  \includegraphics[width=\textwidth]{fink1999a_fig1.pdf}

\end{frame}

\begin{frame}
  \frametitle{Applied knot theory:}
  
  
  \includegraphics[width=\textwidth]{fink1999a_tab1.pdf}

\end{frame}

\begin{frame}

  Ants!
\end{frame}


\section{Words}

\begin{frame}
  \frametitle{Culturomics:}

  \small{``Quantitative analysis of culture using millions of
    digitized books'' by Michel et al., Science, 2011\cite{michel2011a}}

  \includegraphics[width=0.45\textwidth]{michel2011a_fig3a.pdf} 
  \includegraphics[width=0.45\textwidth]{michel2011a_fig3e.pdf} \\
  \includegraphics[width=0.45\textwidth]{michel2011a_fig3f.pdf}
  \includegraphics[width=0.35\textwidth]{michel2011a_fig4f.pdf}

  {\small
    \wordwikilink{http://www.culturomics.org/}{http://www.culturomics.org/}\\
    \wordwikilink{http://ngrams.googlelabs.com/}{Google Books ngram viewer}
  }

\end{frame}

\section{People}

\begin{frame}
  \frametitle{Selflessness}

  Radiolab Podcast: 
  \wordwikilink{http://www.radiolab.org/2010/dec/14/equation-good/}{``An Equation for Good''}
  \bigskip
  \begin{columns}
    \column{0.6\textwidth}
    \includegraphics[width=\textwidth]{radiolab-2.pdf}
    \column{0.4\textwidth}
    \begin{itemize}
    \item 
      Natural selection, 
    \item 
      the `mystery of altrusim', 
    \item 
      George Price, 
    \item 
      madness.
    \end{itemize}
  \end{columns}

\end{frame}

\begin{frame}
  \frametitle{The Invention of Money}

  \begin{columns}
    \column{0.4\textwidth}
    \includegraphics[width=\textwidth]{2011-01-15thisamericanlife-inventionofmoney.jpg}
    \column{0.6\textwidth}
    \begin{itemize}
    \item 
      ``This American Life''
      Podcast on 
      \wordwikilink{http://www.thisamericanlife.org/radio-archives/episode/423/the-invention-of-money}{money and belief}
    \item 
      (1) Brazil and (2) the Fed...
    \end{itemize}
  \end{columns}

\end{frame}


\begin{frame}
  \frametitle{Dynamic networks: Server security}

  \begin{block}{Serving one html page with an image:}
    \bigskip
%%    \includegraphics[width=0.48\textwidth]{392_big02-windows-tp-3.pdf}
%%    \includegraphics[width=0.48\textwidth]{392_big01-linux-tp-3.pdf}
    \setlength\fboxsep{0pt}
    \setlength\fboxrule{1pt}
    \fbox{\includegraphics[width=0.48\textwidth]{392_big02-windows.jpg}
    \includegraphics[width=0.48\textwidth]{392_big01-linux.jpg}}
    \begin{itemize}
    \item 
      Map of system calls made by a Linux server running Apache and
      Windows server running IIS.  Which is which?
    \end{itemize}
  \end{block}
  {\tiny
  Taken from \wordwikilink{http://www.visualcomplexity.com/vc/project_details.cfm?id=392&index=392&domain=}{http://www.visualcomplexity.com}}
  
\end{frame}

\begin{frame}
  \frametitle{}
  
  \wordwikilink{http://en.wikipedia.org/wiki/Dunbar\'s\_number}{Dunbar's number} 
  and scaling.

  \cite{hill2008a}


\end{frame}

\begin{frame}
  \frametitle{The Teletherm:}
  


\end{frame}


\begin{frame}
  \frametitle{Social networks}

  What's the average number of friends?

\end{frame}

\section{Probability}

\begin{frame}
\frametitle{Homo probabilisticus?}

\begin{block}{The set up:}
  \begin{itemize}
  \item<2->
    A parent has two children.
  \end{itemize}
\end{block}

\begin{block}<3->{Simple probability question:}
  
  \begin{itemize}
  \item<3->
    What is the probability that both
    children are boys?
  \item<4-| handout=0| trans=0>
    \visible<4->{\alert{1/4...}}
  \end{itemize}
\end{block}

\end{frame}


\begin{frame}
\frametitle{Homo probabilisticus?}

\begin{block}{The next set up:}
  \begin{itemize}
  \item<2->
    A parent has two children.
  \item<3->
    We know one of them is a boy.
  \end{itemize}
\end{block}

\begin{block}<4->{The next probabilistic poser:}
  \begin{itemize}
  \item<4->
    What is the probability that both
    children are boys?
  \item<5-| handout=0| trans=0>
    \visible<5->{\alert{1/3...}}
  \end{itemize}
\end{block}
\end{frame}


\begin{frame}
  \frametitle{Homo probabilisticus?}

  \begin{block}{One more set up:}
    \begin{itemize}
    \item<2->
      A parent has two children.
    \item<3->
      We know one of them is a boy \alertb{born on a Tuesday}.
    \end{itemize}
  \end{block}
  
  \begin{block}<4->{One more probabilistic poser:}
    \begin{itemize}
    \item<4->
      What is the probability that both
      children are boys?
    \item<5-| handout=0| trans=0>
      \visible<5->{\alert{?}}
%%      \visible<5->{\alert{13/27...}}
    \end{itemize}
  \end{block}

\end{frame}


\begin{frame}
  \frametitle{Homo probabilisticus?}

  \begin{block}{One more set up:}
    \begin{itemize}
    \item<2->
      A parent has two children.
    \item<3->
      We know one of them is a boy \alertb{born on December 31}.
    \end{itemize}
  \end{block}
  
  \begin{block}<4->{One more probabilistic poser:}
    \begin{itemize}
    \item<4->
      What is the probability that both
      children are boys?
    \item<5-| handout=0| trans=0>
      \visible<5->{\alert{?}}
%%      \visible<5->{\alert{13/27...}}
    \end{itemize}
  \end{block}

\end{frame}

\begin{frame}
  \frametitle{Prediction:}

  \includegraphics[width=\textwidth]{extrapolating-tp-10.pdf}\\
  \wordwikilink{http://xkcd.com/605/}{http://xkcd.com/605/}

  \visible<2->{By the third trimester, there will be hundreds of babies inside you...}
\end{frame}

\section{Random}

\changelecturelogo{.18}{2011-02-07no-mouse-click-tp-10}

\begin{frame}
  \frametitle{What's this?}
  
  \includegraphics[width=\textwidth]{figtallestbuildings002_noname.pdf}

\end{frame}

\changelecturelogo{.18}{2011-02-07do-not-open-this-box-tp-10}


\begin{frame}

  \begin{block}{The Teletherm is nigh...}
    \begin{center}
    \includegraphics[width=0.5\textwidth]{figburlington_teletherm001_noname.pdf}
    \includegraphics[width=0.5\textwidth]{figcentralpark_teletherm001_noname.pdf}
    \end{center}
    \begin{itemize}
    \item 
      Hibernal Teletherm $\approx$ February 4.
    \item 
      Halfway between Winter Solstice and Spring Equinox
    \item 
      Bonus: \wordwikilink{http://en.wikipedia.org/wiki/Groundhog_Day}{Groundhog Day}, 
      \wordwikilink{http://en.wikipedia.org/wiki/Imbolc}{Imbolc}, \ldots
    \item 
      Aesteval Teletherm $\approx$ July 19 (164 days later).
    \end{itemize}
  \end{block}

\end{frame}

\changelecturelogo{.18}{2011-02-07do-not-click-here-tp-10}

\section{Videos}

\begin{frame}<1 | handout=0 | trans=1>
  \frametitle{Mimicry---the lying lyrebird}

  \begin{center}
    \includemovie[
    controls=true,
    toolbar=true,
    poster=lyrebird.jpg,
    ]{100mm}{75mm}{videos/2010/lyrebird.mp4}
  \end{center}

%    text=(tap, tap, tap, ...)

% \movie[borderwidth=5pt,%
% width=3cm,%
% height=2cm,%
% poster,%
% showcontrols=true%
% ]%
% {}%
% {videos/lyrebird.mp4} 

\end{frame}


\section{Good\ science}

\begin{frame}
  \frametitle{Whimsical but great example of real science:}

  \wordwikilink{http://www.sciencemag.org/content/early/2010/11/10/science.1195421}{``How Cats Lap: {W}ater Uptake by \textit{{F}elis catus}''}\\
  Reis et al., \textit{Science}, 2010.

  \medskip

  \includegraphics[width=\textwidth]{12cats_graphic-popup-v2.jpg}

  Amusing interview \wordwikilink{http://video.nytimes.com/video/2010/11/11/science/1248069317702/how-cats-lap.html}{here}

\end{frame}


\section{Wealth distribution}

\begin{frame}
  \frametitle{Wealth distribution in the United States:}

  According to a recent study by Norton and Ariely:\cite{norton2011a}
  \begin{itemize}
  \item<2->
    What percentage of all wealth is owned by 
    individuals grouped into quintiles?
  \item<3->
    How do people believe wealth is distributed?
  \item<4->
    How do people believe wealth should be distributed?
  \end{itemize}
  
\end{frame}

\begin{frame}

  \frametitle{Wealth distribution in the United States:}

  \includegraphics[width=\textwidth]{norton2011a_fig2-tp-10}
  
\end{frame}

\begin{frame}

  \frametitle{Wealth distribution in the United States:}

  \includegraphics[width=\textwidth]{norton2011a_fig3-tp-10}
  
\end{frame}

\section{Universality}

\begin{frame}
  \frametitle{Universal numbers}

  \begin{columns}
    \column{0.3\textwidth}
    %% picture of a hand
    %% picture of a Simpson's hand
    \column{0.7\textwidth}
    \begin{itemize}
    \item<1-> 
      Accidents of evolution give us 5 + 5 = 10 fingers
      and hence base 10.
    \item<2->
      We could be happy about 6, 8, or 12.
    \end{itemize}
  \end{columns}
  

\end{frame}

\begin{frame}
  \frametitle{}


  \begin{columns}
    \column{0.4\textwidth}
    \includegraphics[width=\textwidth]{470px-Babylonian_numerals.png}
    \column{0.6\textwidth}
    \begin{itemize}
    \item<1->
      Beep.
    \end{itemize}
  \end{columns}
\end{frame}

\begin{frame}
  \frametitle{Great moments in Universality}

  Phyllotaxis

  Add pictures
\end{frame}


\section{Quirkology}

\begin{frame}
  \frametitle{Richard Wiseman's research:}
  
  \begin{columns}
    \column{0.4\textwidth}
    \includegraphics[width=\textwidth]{quirkology-bookcover.pdf}\\
    \column{0.6\textwidth}
    \begin{itemize}
    \item<1-> 
      \wordwikilink{http://www.quirkology.com}{http://www.quirkology.com}
    \item<2-> 
      Letter writing exercise...
    \end{itemize}
  \end{columns}
  
\end{frame}

\begin{frame}
  \frametitle{Quirkology}

  \begin{block}<1->{People who draw letters so others can read them tend to:}
    \begin{itemize}
    \item<2-> 
      be high `self-monitors'
    \item<3-> 
      be concerned with how others see them
    \item<4-> 
      adapt better to social situations
    \item<5-> 
      skilled at altering how others see them
    \item<6-> 
      be more adept at lying...
    \end{itemize}    
  \end{block}

  \begin{block}<7->{People how draw letters so they can read them tend to:}
    \begin{itemize}
    \item<8->
      be low `self-monitors'
    \item<9->
      follow their inner values
    \item<10->
      remain the same across social settings
    \item<11->
      be less adept at lying...
    \end{itemize}
  \end{block}

\end{frame}


\begin{frame}
  \frametitle{Quirkology}

  \begin{itemize}
  \item<1->
    And those who convince themselves
    they drew their letters the opposite way
    to what they really did\ldots

    \bigskip

    \visible<2->{\alertb{are good at deceiving themselves.}}
  \end{itemize}

\end{frame}


\begin{frame}
  \frametitle{Slide mold and optimal networks}

\end{frame}

\begin{frame}
  \frametitle{Leaves}

  Magnasco's work on leaves

\end{frame}

\begin{frame}
  \frametitle{}

\end{frame}

\begin{frame}<handout: 0 | trans: 0>

  \wordwikilink{http://www.bigpumpkins.com/}{Big pumpkins}

\end{frame}




\begin{frame}
  London Taxis
\end{frame}

\begin{frame}

  \begin{block}{Highly relevant Applied Math seminar:}
    \begin{itemize}
    \item 
      \alertr{Title:} Data Mining Remotely Sensed Satellite Images,
      Transport Analysis and Modeling of Spatiotemporal Dynamical Systems
    \item 
      \alertr{Speaker:} 
      \wordwikilink{http://people.clarkson.edu/~ebollt/}{Erik Bollt},
      Clarkson University
    \item
      \alertr{Time:} 11:30am Thursday Aug 29, Kalkin 002.
    \end{itemize}
  \end{block}

  \begin{block}{}
    {\tiny %% \fontsize{6.8}{8.16}\selectfont
      \alertr{Abstract:} A broad range of scientific fields, such as
      climatology, oceanography, and fluid dynamics produce large data sets
      in the form of digital images or continuous-time, spatiotemporal video
      data from remotely sensed hyperspectral satellite data.  There have
      been terrific advancements in variational methods for image
      processing, and likewise in dynamical systems, there have been
      tremendous advancements in analyzing transport in complex
      spatiotemporal dynamical systems.  We will highlight our methods of
      image processing techniques specifically suited to the complex
      dynamical systems typical of fluid systems, and the computational
      tools of dynamical systems to infer transport issues. The
      Frobenius-Perron operator for a dynamical system allows transport
      modeling and phase decomposition into almost invariant sets, for
      discussion of coherent sets and transport.  Application of particular
      interest to us are remotely sensed ecological systems such as
      biological products including algae blooms, as well as the atmosphere
      of Jupiter.}
  \end{block}

\end{frame}


%% Petrichor
%% Nature of argillaceous odor
%% Bear and Thomas\cite{bear1964a}
%% Nature (London, United Kingdom) (1964), 201(4923), 993-5 CODEN: NATUAS; ISSN: 0028-0836.
%% 
%% Dunning Kruger

%% where's george maps
%% http://www.fastcoexist.com/1681677/a-new-map-of-the-us-created-by-how-our-dollar-bills-move#1

  

\section{Random\ things}

\begin{frame}
  \begin{center}
    \includegraphics[height=0.9\textheight]{lego-puzzles.jpg}
  \end{center}
\end{frame}

\subsection{Personality}

\begin{frame}
  \frametitle{Personality distributions}

  \begin{block}{Rentfrow, Gosling, and Potter\cite{rentfrow2008a}}
    ``A Theory of the Emergence, Persistence, and Expression of Geographic Variation in  Psychological Characteristics''

    \smallskip

    \textit{Perspectives on Psychological Science}

    \smallskip

    \textbf{Vol. 3}, pp. 339--369, 2008.
  \end{block}

\end{frame}

\begin{frame}
  \frametitle{Personality distributions}

  \begin{block}{Five Factor Model (FFM)}
    \begin{itemize}
    \item 
      Extraversion [E]
    \item 
      Agreeableness [A]
    \item 
      Conscientiousness [C]
    \item 
      Neuroticism [N]
    \item 
      Openness [O]
    \end{itemize}

    \bigskip
    
    \visible<2->{
      ``...a robust and widely accepted framework for 
      conceptualizing the structure of personality...
      Although the FFM is not universally accepted in the field...''\cite{rentfrow2008a}
    }
    
    \bigskip

    \visible<3->{
      \alert{Usual concern:} self-reported data.
    }
  \end{block}

\end{frame}

\begin{frame}
  \frametitle{Agreeableness:}

  \includegraphics[width=\textwidth]{rentfrow2008a_agreeableness.pdf}

\end{frame}

\begin{frame}
  \frametitle{Conscientiousness:}

  \includegraphics[width=\textwidth]{rentfrow2008a_conscientiousness.pdf}

\end{frame}

\begin{frame}
  \frametitle{Extraversion:}

  \includegraphics[width=\textwidth]{rentfrow2008a_extraversion.pdf}

\end{frame}

\begin{frame}
  \frametitle{Openness}

  \includegraphics[width=\textwidth]{rentfrow2008a_openness.pdf}

\end{frame}

\begin{frame}
  \frametitle{Neuroticism:}

  \includegraphics[width=\textwidth]{rentfrow2008a_neuroticism.pdf}

\end{frame}


\begin{frame}
  
Kenneth the Page's TV No-No Words:

Conflict
Urban
Woman
Divorce
Shows about shows
Writer
Justin Bartha
Dramedy
New York
Politics
High concept
Complex
Niche
Quality
Edgy
Blog
Immortal Characters
Foreign

\end{frame}


\changelecturelogo{.18}{2011-02-07no-mouse-click-tp-10}

\begin{frame}
  \frametitle{What's this?}
  
  \includegraphics[width=\textwidth]{figtallestbuildings002_noname.pdf}

\end{frame}


\changelecturelogo{.18}{icons-lightbulb-tp.pdf}

\begin{frame}

  \begin{block}<+->{Physicists say the craziest things:}
    \begin{itemize}
    \item<+->
      Ernest Rutherford:
      ``All science is either physics or stamp collecting.''
    \item<+->
      Richard Feynmann:
      ``Social science is an example of a science which is not a science.''
    \item<+->
      Sheldon Cooper's 
      \wordwikilink{http://www.youtube.com/watch?v=agzGlbRKzqw}{view}
      of the Social Sciences.
    \end{itemize}

  \end{block}


\end{frame}

\begin{frame}

  \begin{block}{\wordwikilink{http://tweetping.net}{Tweet Ping}}
    \includegraphics[height=0.85\textheight]{2013-01-29tweetping.jpg}
  \end{block}

\end{frame}

\changelecturelogo{.18}{2011-02-07do-not-open-this-box-tp-10}

\begin{frame}

  \begin{block}{The Teletherm is nigh...}
    \begin{center}
      \includegraphics[height=0.85\textwidth]{figteletherm_tmin_station006_1083_noname.pdf}
    \end{center}
  \end{block}

\end{frame}

\begin{frame}
  
\includegraphics[height=\textheight]{larson-polar-bear-penguins.jpg}

\end{frame}

\begin{frame}
  
\includegraphics[height=\textheight]{larson-ducks.jpg}

\end{frame}


\subsection{Probability}

\begin{frame}
\frametitle{Homo probabilisticus?}

\begin{block}{The set up:}
  \begin{itemize}
  \item<2->
    A parent has two children.
  \end{itemize}
\end{block}

\begin{block}<3->{Simple probability question:}
  \begin{itemize}
  \item<3->
    What is the probability that both
    children are girls?
  \item<8-| handout:0| trans:0>
    \visible<8->{\alert{1/4...}}
  \end{itemize}
\end{block}

\begin{block}<4->{The next set up:}
  \begin{itemize}
  \item<5->
    A parent has two children.
  \item<6->
    We know one of them is a girl.
  \end{itemize}
\end{block}

\begin{block}<7->{The next probabilistic poser:}
  \begin{itemize}
  \item<7->
    What is the probability that both
    children are girls?
  \item<9-| handout:0| trans:0>
    \visible<9->{\alert{1/3...}}
  \end{itemize}
\end{block}
\end{frame}


\begin{frame}
  \frametitle{Homo probabilisticus?}

  \begin{block}{Try this one:}
    \begin{itemize}
    \item<2->
      A parent has two children.
    \item<3->
      We know one of them is a girl \alertb{born on a Tuesday}.
    \end{itemize}
  \end{block}
  
  \begin{block}<4->{Simple question \#3:}
    \begin{itemize}
    \item<4->
      What is the probability that both
      children are girls?
    \item<8-| handout:0| trans:0>
      \visible<8->{\alert{?}}
%%      \visible<5->{\alert{13/27...}}
    \end{itemize}
  \end{block}

  \begin{block}<5->{Last:}
    \begin{itemize}
    \item<5->
      A parent has two children.
    \item<6->
      We know one of them is a girl \alertb{born on December 31}.
    \end{itemize}
  \end{block}
  
  \begin{block}<7->{And \ldots}
    \begin{itemize}
    \item<7->
      What is the probability that both
      children are girls?
    \item<9-| handout:0| trans:0>
      \visible<9->{\alert{?}}
%%      \visible<5->{\alert{13/27...}}
    \end{itemize}
  \end{block}

  

\end{frame}

\begin{frame}
  \frametitle{Prediction:}

  \includegraphics[width=\textwidth]{extrapolating-tp-10.pdf}\\
  \wordwikilink{http://xkcd.com/605/}{http://xkcd.com/605/}

  \visible<2->{By the third trimester, there will be hundreds of babies inside you...}
\end{frame}



\begin{frame}
  \frametitle{Applied knot theory:}

  ``Designing tie knots by random walks''\cite{fink1999a}
  Fink and Mao, Nature, 1999.

  \includegraphics[width=\textwidth]{fink1999a_fig1.pdf}

\end{frame}

\begin{frame}
  \frametitle{Sartorial topology:}
  
  \includegraphics[width=\textwidth]{fink1999a_tab1.pdf}

\end{frame}

\begin{frame}

  Ants!
\end{frame}


\subsection{Words}

\begin{frame}
  \frametitle{Culturomics:}

  \small{``Quantitative analysis of culture using millions of
    digitized books'' by Michel et al., Science, 2011\cite{michel2011a}}

  \includegraphics[width=0.45\textwidth]{michel2011a_fig3a.pdf} 
  \includegraphics[width=0.45\textwidth]{michel2011a_fig3e.pdf} \\
  \includegraphics[width=0.45\textwidth]{michel2011a_fig3f.pdf}
  \includegraphics[width=0.35\textwidth]{michel2011a_fig4f.pdf}

  {\small
    \wordwikilink{http://www.culturomics.org/}{http://www.culturomics.org/}\\
    \wordwikilink{http://ngrams.googlelabs.com/}{Google Books ngram viewer}
  }

\end{frame}

\subsection{People}

\begin{frame}
  \frametitle{Selflessness}

  Radiolab Podcast: 
  \wordwikilink{http://www.radiolab.org/2010/dec/14/equation-good/}{``An Equation for Good''}
  \bigskip
  \begin{columns}
    \column{0.6\textwidth}
    \includegraphics[width=\textwidth]{radiolab-2.pdf}
    \column{0.4\textwidth}
    \begin{itemize}
    \item 
      Natural selection, 
    \item 
      the `mystery of altrusim', 
    \item 
      George Price, 
    \item 
      madness.
    \end{itemize}
  \end{columns}

\end{frame}

\begin{frame}
  \frametitle{The Invention of Money}

  \begin{columns}
    \column{0.4\textwidth}
    \includegraphics[width=\textwidth]{2011-01-15thisamericanlife-inventionofmoney.jpg}
    \column{0.6\textwidth}
    \begin{itemize}
    \item 
      ``This American Life''
      Podcast on 
      \wordwikilink{http://www.thisamericanlife.org/radio-archives/episode/423/the-invention-of-money}{money and belief}
    \item 
      (1) Brazil and (2) the Fed...
    \end{itemize}
  \end{columns}

\end{frame}


\begin{frame}
  \frametitle{Dynamic networks: Server security}

  \begin{block}{Serving one html page with an image:}
    \bigskip
%%    \includegraphics[width=0.48\textwidth]{392_big02-windows-tp-3.pdf}
%%    \includegraphics[width=0.48\textwidth]{392_big01-linux-tp-3.pdf}
    \setlength\fboxsep{0pt}
    \setlength\fboxrule{1pt}
    \fbox{\includegraphics[width=0.48\textwidth]{392_big02-windows.jpg}
    \includegraphics[width=0.48\textwidth]{392_big01-linux.jpg}}
    \begin{itemize}
    \item 
      Map of system calls made by a Linux server running Apache and
      Windows server running IIS.  Which is which?
    \end{itemize}
  \end{block}
  {\tiny
  Taken from \wordwikilink{http://www.visualcomplexity.com/vc/project_details.cfm?id=392&index=392&domain=}{http://www.visualcomplexity.com}}
  
\end{frame}

\begin{frame}
  \frametitle{}
  
  \wordwikilink{http://en.wikipedia.org/wiki/Dunbar\'s\_number}{Dunbar's number} 
  and scaling.

  \cite{hill2008a}


\end{frame}

\begin{frame}
  \frametitle{The Teletherm:}
  


\end{frame}


\begin{frame}
  \frametitle{Social networks}

  What's the average number of friends?

\end{frame}


\subsection{Random}

\changelecturelogo{.18}{2011-02-07no-mouse-click-tp-10}

\begin{frame}
  \frametitle{What's this?}
  
  \includegraphics[width=\textwidth]{figtallestbuildings002_noname.pdf}

\end{frame}

\changelecturelogo{.18}{2011-02-07do-not-open-this-box-tp-10}

\begin{frame}

  \begin{block}{The Teletherm is nigh...}
    \begin{center}
    \includegraphics[width=0.5\textwidth]{figburlington_teletherm001_noname.pdf}
    \includegraphics[width=0.5\textwidth]{figcentralpark_teletherm001_noname.pdf}
    \end{center}
    \begin{itemize}
    \item 
      Hibernal Teletherm $\approx$ February 4.
    \item 
      Halfway between Winter Solstice and Spring Equinox
    \item 
      Bonus: \wordwikilink{http://en.wikipedia.org/wiki/Groundhog_Day}{Groundhog Day}, 
      \wordwikilink{http://en.wikipedia.org/wiki/Imbolc}{Imbolc}, \ldots
    \item 
      Aesteval Teletherm $\approx$ July 19 (164 days later).
    \end{itemize}
  \end{block}

\end{frame}

\changelecturelogo{.18}{2011-02-07do-not-click-here-tp-10}

\subsection{Videos}

\begin{frame}<1 | handout:0 | trans:1>
  \frametitle{Mimicry---the lying lyrebird}


%%   \begin{center}
%%     \includemovie[
%%     controls=true,
%%     toolbar=true,
%%     poster=lyrebird.jpg,
%%     ]{100mm}{75mm}{videos/2010/lyrebird.mp4}
%%   \end{center}

%    text=(tap, tap, tap, ...)

% \movie[borderwidth=5pt,%
% width=3cm,%
% height=2cm,%
% poster,%
% showcontrols=true%
% ]%
% {}%
% {videos/lyrebird.mp4} 

\end{frame}


\subsection{Good\ science}

\begin{frame}
  \frametitle{Whimsical but great example of real science:}

  \wordwikilink{http://www.sciencemag.org/content/early/2010/11/10/science.1195421}{``How Cats Lap: {W}ater Uptake by \textit{{F}elis catus}''}\\
  Reis et al., \textit{Science}, 2010.

  \medskip

  \includegraphics[width=\textwidth]{12cats_graphic-popup-v2.jpg}

  Amusing interview \wordwikilink{http://video.nytimes.com/video/2010/11/11/science/1248069317702/how-cats-lap.html}{here}

\end{frame}


\subsection{Wealth distribution}

\begin{frame}
  \frametitle{Wealth distribution in the United States:}

  According to a recent study by Norton and Ariely:~\cite{norton2011a}
  \begin{itemize}
  \item<2->
    What percentage of all wealth is owned by 
    individuals grouped into quintiles?
  \item<3->
    How do people believe wealth is distributed?
  \item<4->
    How do people believe wealth should be distributed?
  \end{itemize}
  
\end{frame}

\begin{frame}

  \frametitle{Wealth distribution in the United States:}

  \includegraphics[width=\textwidth]{norton2011a_fig2-tp-10}
  
\end{frame}

\begin{frame}

  \frametitle{Wealth distribution in the United States:}

  \includegraphics[width=\textwidth]{norton2011a_fig3-tp-10}
  
\end{frame}

\subsection{Universality}

\begin{frame}
  \frametitle{Universal numbers}

  \begin{columns}
    \column{0.3\textwidth}
    %% picture of a hand
    %% picture of a Simpson's hand
    \column{0.7\textwidth}
    \begin{itemize}
    \item<1-> 
      Accidents of evolution give us 5 + 5 = 10 fingers
      and hence base 10.
    \item<2->
      We could be happy about 6, 8, or 12.
    \end{itemize}
  \end{columns}
  

\end{frame}

\begin{frame}
  \frametitle{}


  \begin{columns}
    \column{0.4\textwidth}
    \includegraphics[width=\textwidth]{470px-Babylonian_numerals.png}
    \column{0.6\textwidth}
    \begin{itemize}
    \item<1->
      Beep.
    \end{itemize}
  \end{columns}
\end{frame}

\begin{frame}
  \frametitle{Great moments in Universality}

  Phyllotaxis

  Add pictures
\end{frame}


\subsection{Quirkology}

\begin{frame}
  \frametitle{Richard Wiseman's research:}
  
  \begin{columns}
    \column{0.4\textwidth}
    \includegraphics[width=\textwidth]{quirkology-bookcover.pdf}\\
    \column{0.6\textwidth}
    \begin{itemize}
    \item<1-> 
      \wordwikilink{http://www.quirkology.com}{http://www.quirkology.com}
    \item<2-> 
      Letter writing exercise...
    \end{itemize}
  \end{columns}
  
\end{frame}

\begin{frame}
  \frametitle{Quirkology}

  \begin{block}<1->{People who draw letters so others can read them tend to:}
    \begin{itemize}
    \item<2-> 
      be high `self-monitors'
    \item<3-> 
      be concerned with how others see them
    \item<4-> 
      adapt better to social situations
    \item<5-> 
      skilled at altering how others see them
    \item<6-> 
      be more adept at lying...
    \end{itemize}    
  \end{block}

  \begin{block}<7->{People how draw letters so they can read them tend to:}
    \begin{itemize}
    \item<8->
      be low `self-monitors'
    \item<9->
      follow their inner values
    \item<10->
      remain the same across social settings
    \item<11->
      be less adept at lying...
    \end{itemize}
  \end{block}

\end{frame}


\begin{frame}
  \frametitle{Quirkology}

  \begin{itemize}
  \item<1->
    And those who convince themselves
    they drew their letters the opposite way
    to what they really did\ldots

    \bigskip

    \visible<2->{\alertb{are good at deceiving themselves.}}
  \end{itemize}

\end{frame}


\begin{frame}
  \frametitle{Slide mold and optimal networks}

\end{frame}

\begin{frame}
  \frametitle{Leaves}

  Magnasco's work on leaves

\end{frame}

\begin{frame}
  \frametitle{}

\end{frame}

\begin{frame}<handout: 0 | trans: 0>

  \wordwikilink{http://www.bigpumpkins.com/}{Big pumpkins}

\end{frame}




