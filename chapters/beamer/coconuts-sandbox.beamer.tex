\begin{frame}

  \frametitle{Damuth's Law}
  
  \displaypaper{damuth1981a}{1}


% Damuth's law
Damuth's law relates the population density, $\rho$, of mammalian primary consumers
to their body mass, $m$.  In support of the scaling relationship,
Damuth cites statistics on 307 different groups of mammals\cite{damuth81}.  A more extensive
set of data is used in a more recent paper\cite{damuth93}, where a least squared
regression test of $\log\rho$ vs. $m$ leads to an exponent of $-0.77 \pm 0.05$.
The spread of population density, $\delta\rho$, is typically two to three orders of magnitude
which is not surprising given the varied ecological conditions over which the data was collected.

Because of the large spread of the data, it is clear that many fits
may lead to similar values of least square error.  Given a possible allometric exponent,
$\alpha$, the power law of best fit may be calculated directly from the data,
$\rho(m)=C(\alpha)m^{\alpha}$.  The flucutations, $\rho_i'$ about this possible scaling law are
\begin{equation}
\rho_i'={\rho_i\over C(\alpha)m^{\alpha}}.
\end{equation}
Since evolution is a growth process, it is natural to study the correlation coefficient, $r$,
of $\log\rho_i'$ vs. $m_i$ as a function of the scaling exponent, $\alpha$.  We find $r=0$ 
when $\alpha=0.65$, much closer to 2/3 than to 3/4.  For $\alpha=0.66$, $r=0.009$, while
for $\alpha=0.75$, $r=0.078$, nearly an order of magnitude difference.  

Damuth's law was also intended to give ecologists
a sense of population densities in (relatively) isolated
ecological communities.  In support of this proposition, Damuth extracts the
109 species whose population density was sampled in ecological communities
of three or more species.  Damuth concludes that the mean allometric exponent
of these pooled communities is -0.74, e.g.\ ecological communities obey the 
same allometric scaling structure as the entire ecological community.
This result is tenuous at best.  A closer examination of Damuths constructed communities
reveals that seven of the fifteen examples have less than five individual species,
which leads Damuth to offer the warning that the "individual values for
these regressions are not very reliable".

If we only consider communities with seven or more species, e.g.\ all
of the communities whom Damuth considers to have reliable regressions,
we find something completely different.  The mean of the scaling exponents for the 
communities with reliable regressions is $-0.68\pm 0.10$.
Using 10000 bootstrap samples, we calculated a 95\% bootstrap interval on the 
mean of the exponent.  
The mean exponent from bootstrap analysis is $-0.6803\pm 0.03167$, and the 95\% bootstrap confidence 
interval stretches from -0.6200 to -0.7438.  Note that this interval does not include -3/4!

Any scaling law based on the behavior of a limited number of ecological
communities is clearly subject to criticism.  Our objective here is
to show that the results for population density in ecological communities
subject to similar ecological conditions is by no means inconsistent
with a 2/3 law for basal metabolic rate.  There is still a great need
for further data collection, studies on the role of sample size\cite{woman},
as well as further studies on equipartitioning of ecological resources.

\end{frame}

\begin{frame}
  Effects of size and temperature on metabolic rate
JF Gillooly, JH Brown, GB West, VM Savage, EL Charnov
science 293 (5538), 2248-2251	1577	2001

Effects of size and temperature on developmental time
JF Gillooly, EL Charnov, GB West, VM Savage, JH Brown
Nature 417 (6884), 70-73
\end{frame}

\begin{frame}
  Add Damuth!!!

\wordwikilink{https://en.wikipedia.org/wiki/Cope\%27s_rule}{Cope's rule}
\end{frame}

\begin{frame}
  \url{http://www.theatlantic.com/health/archive/2012/06/science-confirms-bieber-fever-is-more-contagious-than-the-measles/258460/}
\end{frame}

\begin{frame}
  Death rate as a function of day of the year.

  What is going on?
\end{frame}

\begin{frame}
  Campaign 

  Donation

  Conatagion.

\url{https://www.technologyreview.com/s/600708/the-social-network-illusion-that-could-turn-this-election-on-its-head/}

\displaypaper{traag2016a}{}

\end{frame}


\begin{comment}


\begin{frame}
  Curvature in metabolic scaling
\end{frame}

\begin{frame}
  \url{http://arxiv.org/abs/1602.01103}

"Winning Arguments: Interaction Dynamics and Persuasion Strategies in Good-faith Online Discussions”

\end{frame}

\begin{frame}

A general model for ontogenetic growth
GB West, JH Brown, BJ Enquist
Nature 413 (6856), 628-631
  
  Add all this:

  Effects of size and temperature on metabolic rate
  JF Gillooly, JH Brown, GB West, VM Savage, EL Charnov
  science 293 (5538), 2248-2251	
\end{frame}


\begin{frame}
  
  Add Damuth.

\end{frame}




\begin{frame}
  South Korean marriage.
\end{frame}

\begin{frame}
  Title: Winning Arguments: Interaction Dynamics and Persuasion Strategies in
 Good-faith Online Discussions
Authors: Chenhao Tan, Vlad Niculae, Cristian Danescu-Niculescu-Mizil, Lillian
 Lee
Categories: cs.SI cs.CL physics.soc-ph
Comments: 12 pages, 10 figures, to appear in Proceedings of WWW 2016, data and
 more at https://chenhaot.com/pages/changemyview.html
DOI: 10.1145/2872427.2883081
\\
 Changing someone's opinion is arguably one of the most important challenges
of social interaction. The underlying process proves difficult to study: it is
hard to know how someone's opinions are formed and whether and how someone's
views shift. Fortunately, ChangeMyView, an active community on Reddit, provides
a platform where users present their own opinions and reasoning, invite others
to contest them, and acknowledge when the ensuing discussions change their
original views. In this work, we study these interactions to understand the
mechanisms behind persuasion.
 We find that persuasive arguments are characterized by interesting patterns
of interaction dynamics, such as participant entry-order and degree of
back-and-forth exchange. Furthermore, by comparing similar counterarguments to
the same opinion, we show that language factors play an essential role. In
particular, the interplay between the language of the opinion holder and that
of the counterargument provides highly predictive cues of persuasiveness.
Finally, since even in this favorable setting people may not be persuaded, we
investigate the problem of determining whether someone's opinion is susceptible
to being changed at all. For this more difficult task, we show that stylistic
choices in how the opinion is expressed carry predictive power.
\\ ( http://arxiv.org/abs/1602.01103 ,  2773kb)
\end{frame}

\begin{frame}
  
cosmological networks
\url{http://arxiv.org/abs/1310.6272}

\url{https://scholar.google.com/citations?view_op=view_citation&hl=en&user=ybhDQgsAAAAJ&citation_for_view=ybhDQgsAAAAJ:Se3iqnhoufwC}

\end{frame}


\framedisplaypaper{youn2016a}{1}{fig2}



\begin{frame}
  Northwestern guy's work on optimal networks.

  Supply networks---population, reservoir size.

\end{frame}

\begin{frame}

"Quantifying origin and character of long-range correlations in
narrative texts"; S. Drozdz, P. Oswiecimka, A. Kulig, J. Kwapien,
K. Bazarnik, I. Grabska-Gradzinska, J. Rybicki, M. Stanuszek;
Information Sciences, vol. 331, 32-44, 20 February 2016; DOI:
10.1016/j.ins.2015.10.023



  \wordwikilink{http://www.eurekalert.org/pub\_releases/2016-01/thni-twg012116.php}{}

 drodz2016a_fig1.jpg
 drodz2016a_fig2.jpg
 drodz2016a_fig3.jpg

\end{frame}

http://journals.aps.org/pre/abstract/10.1103/PhysRevE.80.036115

Community detection in networks with positive and negative links

V. A. Traag and Jeroen Bruggeman
Phys. Rev. E 80, 036115 – Published 21 September 2009

\begin{frame}
  
Conway's Doomsday Rule:

\url{https://en.wikipedia.org/wiki/Doomsday_rule}
\end{frame}

\begin{frame}
  branching networks in ecological food networks
\end{frame}

\begin{frame}
  branching networks in Wikipedia
\end{frame}


\begin{frame}
  arcsine for sports 
\end{frame}



  \begin{frame}
    Explosive percolation
  \end{frame}

\begin{frame}
  Add meandering 
\end{frame}

\begin{frame}
  Add references to papers trying to solve river networks \ldots
\end{frame}

\begin{frame}
  Add reference to MIT paper on river network evolution.
\end{frame}


\begin{frame}
  Find more review papers
\end{frame}

\begin{frame}
  
  Add city and transportation networks

  Scaling ...

  Bettencourt ...

  Scaling of city road networks ...

  What about the edge of a lake.

\end{frame}

\begin{frame}
  Edit Bruce's video down to something like 4 or 5 minutes.
\end{frame}


  


\begin{frame}
  Come back to Krapivsky
\end{frame}

\begin{frame}
  Add githubbery
\end{frame}

\begin{frame}
  Add HOT for networks!!!!
\end{frame}

\begin{frame}
  Find a more organic approach to creating HOT models.
\end{frame}


\begin{frame}
  Create an enormous showcase of networks.
\end{frame}

\begin{frame}
  Contagion:

  Echo chambers
  \url{https://www.washingtonpost.com/news/energy-environment/wp/2016/01/04/heres-how-scientific-misinformation-such-as-climate-doubt-spreads-through-social-media/}
\end{frame}

\begin{frame}

  Star wars:

  \url{http://evelinag.com/blog/2015/12-15-star-wars-social-network/index.html#.Voagdja3jWp}
\end{frame}

\begin{frame}
http://www.sciencemag.org/content/350/6267/1488.4.full
• EDITORS' CHOICE
POLITICAL SCIENCE
Predicting protests via tweets
• Barbara R. Jasny

Although many investigations have attempted to link the use of social media with protests, they have usually been based on events
occurring in one country. Steinert-Threlkeld et al. collected nearly 14 million tweets and protest records from 16 countries in
the Middle East and North Africa from 1 November 2010, through 31 December 2011, which includes the period of the Arab Spring
protests. They studied the coordination of tweets (i.e., the progressive use of smaller numbers of protest-related hashtags by
multiple users) and found that increased coordination was strongly associated with increased protests the next day. This was not
the result of a few highly tweeted events or a few digital activists or international attempts to draw attention to the events.


\end{frame}

\begin{frame}
  \url{https://en.wikipedia.org/wiki/The_Garden_of_Forking_Paths}
\end{frame}



\begin{frame}
  Prediction that Star Wars would suck  
\end{frame}

\begin{frame}

http://arxiv.org/abs/1510.08542
  Network structure at multiple scales via a new network statistic: the onion decomposition

Laurent Hébert-Dufresne, Joshua A. Grochow, Antoine Allard
(Submitted on 29 Oct 2015)
We introduce a new network statistic that measures diverse structural properties at the micro-, meso-, and macroscopic scales, while still being easy to compute and easy to interpret at a glance. Our statistic, the onion spectrum, is based on the onion decomposition, which refines the k-core decomposition, a standard network fingerprinting method. The onion spectrum is exactly as easy to compute as the k-cores: it is based on the stages at which each vertex gets removed from a graph in the standard algorithm for computing the k-cores. But the onion spectrum reveals much more information about a network, and at multiple scales; for example, it can be used to quantify node heterogeneity, degree correlations, centrality, and tree- or lattice-likeness of the whole network as well as of each k-core. Furthermore, unlike the k-core decomposition, the combined onion-degree spectrum immediately gives a clear local picture of the network around each node which allows the detection of interesting subgraphs whose topological structure differs from the global network organization. This local description can also be leveraged to easily generate samples from the ensemble of networks with a given joint onion-degree distribution. We demonstrate the utility of the onion spectrum both on several standard graph models and on many real-world networks.
\end{frame}
\begin{frame}
  Add explosive percolation.
\end{frame}



%% \adverstisement{Broccoli}{}{link}{figure}

%% for assignments
%% for each one, add some gephi action
%% for exploring what they're computing
%% give them a few networks to play around with
%% for each assignment.

\begin{frame}

  \begin{block}{Key element for division approach:}
    \begin{itemize}
    \item<1-> 
      Recomputing betweenness.
    \item<2-> 
      \alert{Reason:} Possible to have a low betweenness
      in links that connect large communities
      if other links carry majority of shortest paths.
    \end{itemize}
  \end{block}

  \begin{block}<3->{When to stop?:}
    \begin{itemize}
    \item<4-> 
      How do we know which
      divisions are meaningful?
    \item<5-> 
      \alert{Modularity measure:}
      difference in fraction of within component
      nodes to that expected for randomized version:\\
      \smallskip
      \uncover<6->{
      $
      Q = 
      \sum_{i}
      [e_{ii} - a_{i}^2]
      $\\
      \smallskip
      where $e_{ij}$ is the fraction of (undirected) edges
      travelling between identified communities $i$ and $j$,
      and $a_i = \sum_{j}e_{ij}$ is the fraction of edges with
      at least one end in community $i$.
    }
    \end{itemize}
  \end{block}

\end{frame}

\begin{frame}
  \frametitle{Hierarchy by division}

  \begin{block}{Key element:}
    \begin{itemize}
    \item<1-> 
      Recomputing betweenness.
    \item<2-> 
      \alert{Reason:} Possible to have a low betweenness
      in links that connect large communities
      if other links carry majority of shortest paths.
    \end{itemize}
  \end{block}

  \begin{block}<3->{When to stop?:}
    \begin{itemize}
    \item<4-> 
      How do we know which
      divisions are meaningful?
    \item<5-> 
      \alert{Modularity measure:}
      difference in fraction of within component
      nodes to that expected for randomized version:\\
      \smallskip
      \uncover<6->{
      $
      Q = 
      \sum_{i}
      [e_{ii} - a_{i}^2]
      $\\
      \smallskip
      where $e_{ij}$ is the fraction of (undirected) edges
      travelling between identified communities $i$ and $j$,
      and $a_i = \sum_{j}e_{ij}$ is the fraction of edges with
      at least one end in community $i$.
    }
    \end{itemize}
  \end{block}

\end{frame}

\begin{frame}

  \begin{block}{Modularity measure:}
    \begin{itemize}
    \item 
      Define 
    \end{itemize}

    $$  
    Q = 
    \sum_{i}
    [e_{ii} - (\sum_j e_{ij})^2]
    =
    {\textnormal{Tr}} \textmatrix{E} - || \textmatrix{E}^2 ||_1,
    $$
  \end{block}

\end{frame}



\begin{frame}


  Structure detection:

  From Lipari summer school:

%%

jure leskovec:

bipartite affiliation graph model

community detection

affiliation graph model

%%%

Add Louvain method

Limits of modularity:

Can fail for random graphs.

Spinglass models

Look at fluctuations of modularity.

Resolution limit
Fortunato and Barthelemy, 2007

Infomap (Rosvall and Bergstrom, 2008)
Random walks and geometric maps


Methods for finding overlapping communities.

Clique percolation method.

How to test clustering algorithms.

LFR benchmark
lancichinetti et al 2008

Mutual information

Danon et al. (2005) using GN benchmark

\end{frame}

%% Get this to work:
%%  \pdfmouseovercomment{Does this work?}{Beep}


\begin{frame}
  Include Romu's famous paper
\end{frame}

\begin{frame}
  Add Callaway to random networks
  Point out growing mechanism.
\end{frame}

\begin{frame}
  Move Newman's generating function madness to 
  the end of assortativity.

  Add our work?
\end{frame}




\begin{frame}<handout: 0 | trans: 0>

\wordwikilink{http://amaznode.fladdict.net/}{Fun with amazon's recommender system}.

[amaznode.fladdict.net]
  
\end{frame}

\begin{frame}<handout: 0 | trans: 0>

  \wordwikilink{http://www.ted.com/index.php/talks/blaise_aguera_y_arcas_demos_photosynth.html}{Photosynth}.

\end{frame}

\begin{frame}
  Spreaders?

\url{http://dish.andrewsullivan.com/2014/03/20/the-conspiratorial-sort/}
\end{frame}


\begin{frame}
  
  %% show examples of how the following works

  \begin{block}{Some extra thinking:}
    \begin{itemize}
    \item 
      (
    \end{itemize}
    
  \end{block}
  

  $ \mbox{$x^k$ piece of $\left( \sum_{i'=0}^\infty V_{i'}x^{i'}\right)^j$}$

\end{frame}


\begin{frame}

Harold Ramis

``What did you do Ray?''

It's the Stay Puft Marshmallow Man

Connect Ghostbusters to kid's showing self control
over marshmallows.

\end{frame}



\begin{frame}
  Random network applications

  Add code to site, github.
\end{frame}

\begin{frame}
Random network applications
  Piece on Ugander 2013a

  Types of motifs possible.

\end{frame}





\section{Nutshell}

\begin{frame}[label=]
  \frametitle{Nutshell:}

  \begin{block}{Optimal supply networks:}
    \begin{itemize}
    \item<1-> Be
    \item<2->
    \item<3->
    \item<4->
    \item<5->
    \end{itemize}
  \end{block}

\end{frame}



\begin{frame}
  Ugander's

  Structural Diversity work.

  Beautiful.
\end{frame}


\begin{frame}
  Add page on 
  
  ``Virality Prediction and Community Structure in Social Networks''

  \cite{weng2013a}

  Weng, Men?er, Ahn
\end{frame}

\begin{frame}

  ``Subgraph Frequencies: Mapping the Empirical and Extremal Geography of
  Large Graph Collections''

  To add to Motifs

  Ugander: Neighborhood graphs.

  Triad space.

  Triadic closure (Rapaport)

  Suppression of certain graph types.

  Extremal graph theory.

  Razborov, flag albebras, 2007.

  Not a lot of open triads.

  \cite{ugander2013a}
\end{frame}


\begin{frame}
  
For optimal supply networks
Metastable states in random media.  Optimal paths.
P. Jogi and Sornette, 1998, Phys Rev. E 57, 6931--6943.
 
 \cite{xie2007a}
 Geographical networks evolving with an optimal policy
\end{frame}


%% For random networks:

\subsection{Simple analysis}

\begin{frame}

\end{frame}







%% build course map figures

\begin{frame}
  \frametitle{Weighted networks}

  %% Barrat et al. 2004
  
\end{frame}

http://prl.aps.org/abstract/PRL/v112/i4/e048701
James P. Gleeson1,*, Jonathan A. Ward2, Kevin P. O’Sullivan1, and William T. Lee1
1MACSI, Department of Mathematics and Statistics, University of Limerick, Limerick, Ireland
2Centre for the Mathematics of Human Behaviour, Department of Mathematics and Statistics, University of Reading, Whiteknights RG6 6AH, United Kingdom
Selected for a Viewpoint in Physics Published 30 January 2014; received 31 May 2013
See accompanying Physics Synopsis
Heavy-tailed distributions of meme popularity occur naturally in a model of meme diffusion on social networks. Competition between multiple memes for the limited resource of user attention is identified as the mechanism that poises the system at criticality. The popularity growth of each meme is described by a critical branching process, and asymptotic analysis predicts power-law distributions of popularity with very heavy tails (exponent α<2, unlike preferential-attachment models), similar to those seen in empirical data.

%% Add a drawn picture for "and in detail" section

%% Get into hierarchies.

\begin{frame}
For branching networks:
  
Cite woldenberg1966a
\cite{woldenberg1966a}
Allometry, Herbert Simon!

Describe process of network growth...
Extract pictures from
\cite{glock1931a}

Precipitation of DEMS
\cite{gregory1994a}

Add page on Caldarelli 1997
``Randomly pinned landscape evolution''

\end{frame}


\begin{frame}

add slide about galactic network
to intro
  
\end{frame}

%% weighted networks
%% Barrat et al. 2004

\begin{frame}
  
\end{frame}


%% far lands or bust
%% \insertvideo{teU1nRc0bso}{980}{1100}{Crafting landscapes}
%% far lands or bust
%% \insertvideo{7uBRfjlv480}{1410}{1530}{Crafting landscapes}


\insertvideo{lFg21x2sj-M}{}{}{Ant colony structure}

\insertvideo{1IugvemOyZY}{}{}{Ant colony structure}




\begin{frame}
      The \syllabuslink{\coursewebsite/docs/\courseprefix}.  
\end{frame}



%% To do:
%% 
%% stories as theme
%% 
%% outline
%%
%% make a poster (movie like)
%% 
%% logo
%% 
%% New theme for website
%%
%% script for putting videos up
%% 
%% add more videos throughout
%% 
%% sketch outline of video shell
%% tape images on
%% 
%% Lada Adamic

%% london biking



 


\begin{frame}

  Supernatural clip:

  Clowns, S7 or S8?

  Terrifying Octopus.

\end{frame}

\begin{frame}
  Networks are a mess.
  
  Join-the-dots gag.
  
  
\end{frame}

\begin{frame}
  Katz Centrality


\end{frame}

\begin{frame}
  \frametitle{Maximum flow problem:}

  To add to supply networks...
  
  \wordwikilink{http://www.technologyreview.com/article/26869/}{http://www.technologyreview.com/article/26869/}

\end{frame}

\section{Beep}

\begin{frame}
  \frametitle{Where is Barry?}

    \includegraphics[width=\textwidth]{datamining-core-2006-06-27-tp-10.pdf}

\end{frame}

\begin{frame}
  \frametitle{Yes...}

  \begin{itemize}
  \item 
    Weird
  \end{itemize}

\end{frame}

\subsection{Sub Beep 1}

\begin{frame}
  \frametitle{Yes...}

  \begin{itemize}
  \item 
    Weird
  \end{itemize}

\end{frame}

\subsection{Sub Beep 2}

\begin{frame}
  \frametitle{Yes...}

  \begin{itemize}
  \item 
    Weird
  \end{itemize}

\end{frame}

\section{Random\ walks\ on\ networks}

\begin{frame}
  \frametitle{Yes...}

  \begin{itemize}
  \item 
    Weird
  \end{itemize}

\end{frame}


\subsection{Sub Beep 1}

\begin{frame}
  \frametitle{Yes...}

  \begin{itemize}
  \item 
    Weird
  \end{itemize}

\end{frame}

\renewcommand{\insertlecturelogo}{
      \includegraphics[height=0.1\textheight]{seal4.pdf}
      X
}


\subsection{Sub Beep 2}

\begin{frame}
  \frametitle{Yes...}

  \begin{itemize}
  \item<1-> 
    Weird
  \end{itemize}

\end{frame}

\renewcommand{\insertlecturelogo}{
      \includegraphics[height=0.1\textheight]{icons-lightbulb2.pdf}
}

%% \begin{frame}
%% 
%% Next time: add in time to equilibrium, max lambda, noh2006a.pdf.
%% 
%% Also:
%% 
%% Laplacian matrix
%% 
%% Admittance matrix
%% 
%% Similar to a shifted version of the Laplacian
%% 
%% Kirchoff's laws
%% 
%% Connection to transportation problems
%% 
%% Electricity
%%   
%% \end{frame}

\begin{frame}
  \frametitle{Random walks on networks---basics:}

  \begin{itemize}
  \item<1->
    Imagine a single random walker moving
    around on a network.
  \item<2-> 
    At $t=0$, start walker at node $j$ and 
    take time to be discrete.
  \item<3-> 
    \alert{Q:} What's the long term probability distribution
    for where the walker will be?
  \item<4-> 

    Define \alertb{$p_i(t)$} as the probability
    that at time step $t$, our walker is at node $i$.
  \item<5-> 
    We want to characterize the evolution
    of $\vec{p}(t)$.
  \item<6-> 
    First task: connect $\vec{p}(t+1)$ to $\vec{p}(t)$.
  \item<7-> 
    \visible<7->{Let's call our walker \alert{Barry}.}
  \item<8-> 
    \visible<8->{Unfortunately for Barry,
      he lives on a high dimensional graph and is far from home.}
  \item<9->
    \visible<9->{Worse still: Barry is \alertb{hopelessly drunk}.}
  \end{itemize}

\end{frame}

\renewcommand{\insertlecturelogo}{
  \includegraphics[height=0.1\textheight]{scrabbleletters2-tp.pdf}
}

\begin{frame}
  \frametitle{Where is Barry?}

  \begin{itemize}
  \item<1->
    Consider simple directed, ergodic (strongly connected) networks.
  \item<2->
    As usual, represent network by \alert{adjacency matrix $A$}
    where
    $$
    \begin{array}{l}
      a_{ij}=1 \ \mbox{if $i$ has an edge leading to $j$}, \\
      a_{ij}=0 \ \mbox{otherwise.}
    \end{array}
    $$
  \item<3->
    Barry is at node $i$ at time $t$ with probability $p_i(t)$.
  \item<4->
    In the next time step he 
    \alertb{randomly lurches} toward one of $i$'s neighbors.
  \item<5->
    Equation-wise:
    $$
      p_i(t+1) = \sum_{j=1}^{n} \frac{1}{k_i}  a_{ji} p_j(t).
    $$
    where $k_i$ is $i$'s degree.
    \visible<6->{Note: $k_i = \sum_{j=1}^{n} a_{ij}$.}
  \end{itemize}

\end{frame}

\begin{frame}
  \frametitle{Where is Barry?}

  \begin{itemize}
  \item<1->
    Linear algebra-based excitement:
    $
    p_i(t+1) = \sum_{j=1}^{n} a_{ji} \frac{1}{k_j} p_j(t)
    $
    is more usefully viewed as
    $$
    \vec{p}(t+1) 
    = 
    A^{\textnormal{T}} K^{-1}
    \vec{p}(t) 
    $$
    where $[K_{ij}] = [\delta_{ij} k_i]$ 
    has node degrees on the main diagonal
    and zeros everywhere else.
  \item<2->
    So... we need to find the \alert{dominant eigenvalue} 
    of $A^{\textnormal{T}} K^{-1}$.
  \item<3->
    Expect this eigenvalue will be 1 (doesn't make sense
    for total probability to change).
  \item<4->
    The corresponding eigenvector will be the limiting
    probability distribution (or invariant measure).
  \item<5->
    Extra concerns: multiplicity of eigenvalue = 1,
    and network connectedness.
  \end{itemize}

\end{frame}

\begin{frame}
  \frametitle{Where is Barry?}

  \begin{itemize}
  \item<1->
    By inspection, we see that
    $$
    \vec{p}(\infty) = \frac{1}{\sum_{i=1}^{n} k_i} \vec{k}
    $$
    satisfies
    $
    \vec{p}(\infty)
    = 
    A^{\textnormal{T}} K^{-1}
    \vec{p}(\infty)
    $
    with eigenvalue 1.
  \item<2->
    We will find Barry at node $i$ with probability
    proportional to its degree $k_i$.
  \item<3->
    Nice implication: probability of finding Barry travelling along
    any edge is \alert{uniform}.
  \item<4->
    Diffusion in real space smooths things out.
  \item<5->
    On networks, uniformity occurs on edges.
  \item<6->
    So in fact, diffusion in real space is \alertb{about the edges too}
    but we just don't see that.
  \end{itemize}

\end{frame}

\begin{frame}
  \frametitle{Other pieces:}

  \begin{itemize}
  \item<1->
    Goodness: $A^{\textnormal{T}} K^{-1}$ is similar to a real symmetric matrix
    if $A = A^{\textnormal{T}}$.
  \item<2->
    Consider the transformation $M = \alertb{K^{-1/2}}$:
    $$
    \alertb{K^{-1/2}}
    \alert{A^{\textnormal{T}} K^{-1}}
    \alertb{K^{1/2}}
    =
    \alertb{K^{-1/2}}
    \alert{A^{\textnormal{T}}}
    \alertb{K^{-1/2}}.
    $$
  \item<3->
    Since $A^{\textnormal{T}} = A$, we 
    have 
    $$
    ({K^{-1/2}}
    {A}
    {K^{-1/2}})^{\textnormal{T}}
    =
    {K^{-1/2}}
    {A}
    {K^{-1/2}}.
    $$
  \item<4->
    Upshot: $A^{\textnormal{T}} K^{-1} = A K^{-1}$ has real eigenvalues and a complete
    set of orthogonal eigenvectors.
  \item<5->
    Can also show that maximum eigenvalue magnitude is indeed 1.
  \item<6->
    Other goodies: next time round.
  \end{itemize}

\end{frame}





\end{comment}
