\section{History}

  \textbf{Some problems for people thinking about people?:}
  
  \textbf{How are social networks structured?}
    
     How do we define connections?
     How do we measure connections?
     (remote sensing, self-reporting)
    
  

  \textbf{What about the dynamics of social networks?}
    
     How do social networks evolve? 
     How do social movements begin? 
     How does collective problem solving work? 
     How is information transmitted through social networks?
    
  


  \textbf{Social Search}

  \textbf{A small slice of the pie:}
    
     
      \alert{Q.} Can people pass messages between distant individuals 
      using only their existing social connections?
    
      \alert{A.} Apparently yes...
    
  

  \textbf{Handles:}
    
     
      The Small World Phenomenon
    
      or ``Six Degrees of Separation.''
    
  


  \textbf{The problem}

  \textbf{Stanley Milgram et al., late 1960's:}
    
     Target person worked in Boston as a stockbroker.
     296 senders from Boston and Omaha.
     20\% of senders reached target.
     average chain length $\simeq$ 6.5.
    
  


   \textbf{The problem}

        
     \textbf{Lengths of successful chains:}
       \includegraphics[width=\textwidth]{figbasicmilg_noname}
     
     
     From Travers and Milgram (1969) in Sociometry:\cite{travers1969a}\\
     \alertb{``An Experimental Study of the Small World Problem.''}
   

  \textbf{The problem}

  \alertb{Two features}
  characterize a social `Small World':

  
   Short paths exist
  [] and 
   People are good at finding them.
  


\section{An online experiment}


  \textbf{Social Search}


  \textbf{Milgram's small world experiment with e-mail\cite{dodds2003b}}
    \begin{center}
      \includegraphics[width=0.8\textwidth]{window2}
    \end{center}
  


  \textbf{Social search---the Columbia experiment}

  
   
    60,000+ participants in 166 countries
   
    18 targets in 13 countries including
    
     
      a professor at an Ivy League university,\\
     
      an archival inspector in Estonia,\\
     
      a technology consultant in India,\\
     
      a policeman in Australia,\\
    [] 
      and 
     
      a veterinarian in the Norwegian army.
    
  
    24,000+ chains
  


  \textbf{Social search---the Columbia experiment}

  
  
    Milgram's participation rate was roughly 75\%
  
    Email version: Approximately 37\% participation rate.
  
    Probability of a chain of length 10 getting through:
    $$.37^{10} \simeq 5 \times 10^{-5}$$
  
  $\Rightarrow$ 384 completed chains (1.6\% of all chains).
  
  

  \textbf{Social search---the Columbia experiment}

  % results

  
    
    Motivation/Incentives/Perception matter.
    
    If target \textit{seems} reachable\\
    $\Rightarrow$ participation more likely.
    
    Small changes in attrition rates\\
    $\Rightarrow$ large changes in completion rates
    
    e.g., $\searrow$ 15\% in attrition rate \\
    $\Rightarrow$ $\nearrow$ 800\% in completion rate
  




  \textbf{Social search---the Columbia experiment}

  \textbf{Successful chains disproportionately used}
    
     
      weak ties (Granovetter)
     
      professional ties (34\% vs.\ 13\%)
     
      ties originating at work/college
     
      target's work (65\% vs.\ 40\%)
    
  
  
  \textbf{\ldots and disproportionately avoided}
    
     
      hubs (8\% vs. 1\%) (+ no evidence of funnels)
    
      family/friendship ties (60\% vs. 83\%)
    
  

  \textbf{Geography $\rightarrow$ Work}
    



  \textbf{Social search---the Columbia experiment}


  Senders of successful messages showed\\
  \tc{blue}{little absolute dependency} on
  
  
    age, gender
  
    country of residence
   
    income
   
    religion
   
    relationship to recipient
  

  \bigskip

  {
    Range of completion rates for subpopulations: \\
    \mbox{} \hfill 30\% to 40\%
  }



  \textbf{Social search---the Columbia experiment}

% Age 30-39   39.3%
% Australia     40.0%
% Graduate level education 41.9%
% Gender Male 39.6%
% occupation > 20 counts: mass media 47.0%
% position 
% high school student 31.0%
% college student 32.2%
% retired 32.9%
% nreligion christian 36.2%, buddhism 33.5%, islam 32.3%


% Age 17 or under 32.8%
% Canada 34.7%
% Elementary school 28.3%
% Gender female 37.1%
% occupation > 20 counts: consumer services 29.2%
% position 
% `other' 40%
% specialist/engineer 39.8%
% university student 39.8%
% religion none 40.5%

% 69 countries
% Canada, Italy, France, U.S.
% Australia, Germany, Norway, Finland

% uber sender
% mass media
% > 100k
% graduate
% male

Nevertheless, some weak discrepencies do exist...

\textbf{An above average connector:}
  Norwegian, secular male, aged 30-39, earning over \$100K, 
  with graduate level education working in mass media or science,
  who uses relatively weak ties to people
  they met in college or at work.


\textbf{A below average connector:}
  Italian, Islamic or Christian female earning less than \$2K,
  with elementary school education and retired,
  who uses strong ties to family members.



  \textbf{Social search---the Columbia experiment}

  \textbf{Mildly bad for continuing chain:}
    choosing recipients because 
    \alert{``they have lots of friends''}
    or because they will 
    \alert{``likely continue the chain.''}
  

  \textbf{Why:}
    
     
      Specificity important
     
      Successful links used relevant information.\\
      (e.g. connecting to someone who shares same profession as target.)
    
  



% %  \textbf{Social search---the Columbia experiment}
%
%  \includegraphics[height=0.86\textheight]{figsw_2_r_invert_all3_mod_noname}
%
%
  \textbf{Social search---the Columbia experiment}
  \textbf{Basic results:}
    
    
    
      $\avg{L} = 4.05$ for all completed chains
    
      $L_\ast$ = Estimated `true' median chain length (zero attrition)
    
      Intra-country chains: $L_\ast = 5$ 
    
      Inter-country chains:
      $L_\ast = 7$ 
    
      All chains:
      $L_\ast = 7$ 
    
      Milgram:
      $L_\ast \simeq$ 9
    
  


\section{Previous theoretical work}

  \textbf{Previous work---short paths}
  
  
  
    Connected \alertb{random networks}
    have short average path lengths:
    $$\tavg{d_{AB}} \sim \log(N)$$
  []
    $N$ = population size,
  []
    $d_{AB}$ = distance between nodes $A$ and $B$.
  
  \alert{But: social networks aren't random...}
  




  \textbf{Previous work---short paths}

      
    \includegraphics[width=\textwidth]{clustering}
    
    Need \alert{``clustering''} (your friends are likely to know each other):
  


  \textbf{Non-randomness gives clustering}

  \begin{center}
    \includegraphics[height=0.65\textheight]{lattice3}
  \end{center}

  $d_{AB}=10$ $\rightarrow$ too many long paths.


  \textbf{Randomness + regularity}

  \begin{center}
    \includegraphics[height=0.65\textheight]{latticeshortcut3}
  \end{center}

  \alert{Now have $d_{AB}=3$}
  \hfill $\tavg{d}$ decreases overall

  \textbf{Small-world networks}

  Introduced by\\
  Watts and Strogatz (Nature, 1998)\cite{watts1998a}\\
  ``Collective dynamics of `small-world' networks.''

  \textbf{Small-world networks were found everywhere:}
    
     neural network of C. elegans,
     semantic networks of languages,
     actor collaboration graph,
     food webs,
     social networks of comic book characters,...
    
  

  \textbf{Very weak requirements:}
    
     \alert{local regularity}
      {+ random \alertb{short cuts}}
    
  

  

  \textbf{Toy model}

    \includegraphics[width=\textwidth]{watts1998a_fig1.pdf}


  \textbf{The structural small-world property}

    \includegraphics[width=\textwidth]{watts1998a_fig2.pdf}




  \textbf{Previous work---finding short paths}


  But are these short cuts findable?

  \bigskip

  {\alert{No.}}

  \bigskip

  {
  Nodes \alertb{cannot} find each other quickly\\ 
  with \alertb{any local search method}.
  }



  \textbf{Previous work---finding short paths}

  
   What can a local search method reasonably use?
    How to find things without a map?
   \alertb{Need some measure of distance between friends
      and the target.}
  
  
  \bigskip

  \textbf{Some possible knowledge:}
    
     Target's identity
     Friends' popularity 
     Friends' identities 
     Where message has been 
    
  


  \textbf{Previous work---finding short paths}

  Jon Kleinberg (Nature, 2000)\cite{kleinberg2000a}\\
   ``Navigation in a small world.''

   \bigskip
   
   \textbf{Allowed to vary:}
     
      local search algorithm
     [] and
      network structure.
     
   


  \textbf{Previous work---finding short paths}

  \textbf{Kleinberg's Network:}
    
    
      Start with
      regular d-dimensional cubic lattice.
     
      Add local links so 
      nodes know all nodes within a distance $q$.
    
      Add $m$ short cuts per node.
      
      Connect $i$ to $j$ with probability 
      $$ p_{ij} \propto {d_{ij}}^{-\alpha}. $$
    
  

  
   
    \alert{$\alpha=0$}: random connections.
    
    \alert{$\alpha$ large}: reinforce local connections.
   
    \alert{$\alpha=d$}: same number of connections at all scales.
  



  \textbf{Previous work---finding short paths}

  \textbf{Theoretical optimal search:}
    
     
      ``Greedy'' algorithm.
     
      Same number of connections at all scales: $\alpha=d$.
    

    \bigskip
    {
      Search time grows slowly with system size (like $\log^2N$).
      }

 %  For $\alpha \ne d$, polynomial factor $N^\beta$ appears.

    \bigskip
    {
      \alert{But: social networks aren't lattices plus links.}
    }
    
  
  



  \textbf{Previous work---finding short paths}

  
   
    If networks have \alertb{hubs} can 
    also search well: Adamic et al. (2001)\cite{adamic2001a}
    $$ P(k_i) \propto k_i^{-\gamma}$$
    where $k$ = degree of node $i$ (number of friends).
  
    Basic idea: get to hubs first\\
    (airline networks).
     
    \alert{But: hubs in social networks are limited.}

  

\section{An improved model}

  \textbf{The problem}

  If there are no hubs and no underlying lattice,
  how can search be efficient?

  \includegraphics[width=0.45\textwidth]{barenetwork}%
  \raisebox{8ex}{\begin{tabular}{l}
      \\
      Which friend of \alertb{a} is closest \\
      to the target \alertb{b}?\\
      \\
      What does `closest' mean?\\
      \\
      What is
      `social distance'?  \\
      \end{tabular}}



  \textbf{The model}

  One approach: incorporate \alertb{identity}.\\
  \small{(See ``Identity and Search in Social Networks.'' Science, 2002,  Watts, Dodds, and Newman\cite{watts2002b})}

  \bigskip

  \textbf{\alertb{Identity is formed from attributes such as:}}
    
     
      Geographic location
     
      Type of employment
     
      Religious beliefs
     
      Recreational activities.
    
  

  \bigskip

  {
    \alertb{Groups} are formed by people with at least one similar attribute.
  }

  \bigskip

  {
    Attributes $\Leftrightarrow$ 
    Contexts $\Leftrightarrow$ 
    Interactions $\Leftrightarrow$ 
    Networks.
  }


  \textbf{Social distance---Bipartite affiliation networks}

  \centering
  \includegraphics[height=0.75\textheight]{bipartite}

% boards of directors
% movies
% transportation




  \textbf{Social distance---Context distance}

  \centering
  \includegraphics[width=\textwidth]{bipartite2}


  \textbf{The model}

  Distance between two individuals $x_{ij}$ 
  is the height of lowest common ancestor.

  \begin{center}
    \includegraphics[width=0.8\textwidth]{fig01_hierarchy_againA}
  \end{center}

  \alertb{$x_{ij}=3$, $x_{ik}=1$, $x_{iv}=4$.}


  \textbf{The model}

  
   
    Individuals are more
    likely to know each other the closer they are
    within a hierarchy.
   
    Construct $z$ connections for each node
    using
    \alertb{$$p_{ij} =c\exp\{-\alpha x_{ij}\}.$$}
   
    \alert{$\alpha=0$}: random connections.
   
    \alert{$\alpha$ large}: local connections.
  



  \textbf{Social distance---Generalized context space}

  \centering
  \includegraphics[width=1\textwidth]{generalcontext2}

  (Blau \& Schwartz, Simmel, Breiger)

% %   \textbf{The model}
% 
%   Six propositions about social networks:\\
%   (Blau \& Schwartz, Simmel, Breiger)
% 
%   \alert{P1:} Individuals have identities and belong to
%   various groups that reflect these identities.
% 
%   \alert{P2:} Individuals break down
%   the world into a hierarchy of categories.
% 
% % 
% % % %   \textbf{The model}
% % 
% %   A Geographic example: The United States.
% % 
% %   \alertb{Level 1:} The country.
% % 
% %   \alertb{Level 3:} Regions: South, North East, Midwest, West coast, South West, Alaska.
% % 
% %   \alertb{Level 4:} States within regions\\ (New York, Connecticut, Massachusetts,\ldots).
% % 
% %   \alertb{Level 5:} Cities/areas within States\\ (New York city, Boston, the Berkshires).
% % 
% %   \alertb{Level 6:} Suburbs/towns/smaller cities\\ (Brooklyn, Cambridge).
% %   
% %   \alertb{Level 7:} Neighborhoods\\ (the Village, Harvard Square).
% % % 

 
% %   \textbf{The model}
% 
%   \alert{P4:}  Each attribute
%   of identity $\equiv$ hierarchy.
% 

  \textbf{The model}

  \begin{center}
    \includegraphics[width=\textwidth]{fig01_hierarchy_againD}
  \end{center}

  \begin{center}

    $\vec{v}_i = [ 1 \  1 \ 1 ]^T$, $\vec{v}_j = [ 8 \ 4 \ 1]^T$ \hfill
    Social distance:\\
    \alertb{$x^1_{ij} = 4$, \ $x^2_{ij} = 3$, \ $x^3_{ij} = 1$.}
    \hfill
    $ \boxed{y_{ij} = \min_h x^h_{ij}.} $

  \end{center}


% %   \textbf{The model}
% 
%   \alert{P5:}   ``Social distance'' is the minimum distance
%   between two nodes in all hierarchies.
% 
%   $$ \boxed{y_{ij} = \min_h x^h_{ij}.} $$
% 
% \vfill
% 
%   Previous slide:
%   \begin{center}
%     
%     \alertb{$x^1_{ij} = 4$, \ $x^2_{ij} = 3$, \ $x^3_{ij} = 1$.}
% 
%     $\Rightarrow  y_{ij} = 1$.
% 
%   \end{center}
% 
% 

  \textbf{The model}

  Triangle inequality doesn't hold:

  \begin{center}
    \includegraphics[width=1\textwidth]{fig01_hierarchy_againE}
  \end{center}

  \begin{center}
    \alert{$y_{ik} = 4 > y_{ij} + y_{jk} = 1 + 1 = 2.$}
  \end{center}
 

  \textbf{The model}

  
   
    Individuals know the identity
    vectors of
    
     
      themselves,
      
      their friends,
    []  
      and
      
      the target.
    
  
    Individuals can estimate the social distance
    between their friends and the target.
  
    Use a greedy algorithm + allow searches to fail randomly.
  
  


   \textbf{The model-results---searchable networks}
 
   $\alpha=0$ versus $\alpha=2$ for $N \simeq 10^5$:
   \centering
   \includegraphics[height=0.4\textheight]{figHalphavar02ultp_talk2_noname}%
 \raisebox{12ex}{
   \begin{tabular}{l}
   \alertb{$q \ge r$} \\
   \alert{$q<r$} \\
   $r= 0.05$
 \end{tabular}}

$q$ = probability an arbitrary message
chain reaches a target.


 A few dimensions help.\\
 Searchability decreases as population increases.\\
 Precise form of hierarchy largely doesn't matter.



 
  \textbf{The model-results}

  Milgram's Nebraska-Boston data:

      
    \includegraphics[width=\textwidth]{figmilgram_talk_noname}%
    
    \textbf{Model parameters:}
      
      
        $N=10^8$, 
      
        $z=300$, $g=100$,
      
        $b=10$,  
      
        $\alpha=1$, $H=2$; 
      []
      
        $\tavg{L_{\textnormal{model}}} \simeq 6.7$
      
        ${L_{\textnormal{data}}} \simeq 6.5$
      
    
  

  \textbf{Social search---Data}

  \textbf{Adamic and Adar (2003)}
    
    
      For HP Labs, found probability of connection
      as function of organization distance
      well fit by exponential distribution.
    
      Probability of connection as function of
      real distance $\propto 1/r$.
    
  


  \textbf{Social Search---Real world uses}

  
   
  Tags create identities for objects
   
  Website tagging:
  \url{http://www.del.icio.us}
   
  (e.g., Wikipedia)
   
  Photo tagging:
  \url{http://www.flickr.com}
   
  Dynamic creation of metadata
  plus links between information objects.
   
  Folksonomy: collaborative creation of metadata
  
  

  \textbf{Social Search---Real world uses}

  \textbf{Recommender systems:}
    
    
      Amazon uses people's actions to build
      effective connections between books.  
    
      Conflict between `expert judgments' and\\
      tagging of the hoi polloi.
    
  

%  Q: Does tagging lead to a flat structure or 
%  can we identify categories?  (Community detection.)

  % some information scientists decry tagging
  % as poorly directed



  \textbf{Conclusions}

  
  
    Bare networks are typically unsearchable.
   
    Paths are findable if nodes understand how network is formed.
   
    Importance of identity (interaction contexts).
   
    Improved social network models.
   
    Construction of peer-to-peer networks.
   
    Construction of searchable information databases.
  




