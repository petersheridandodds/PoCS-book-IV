%% definition 

%% how to measure r

%% how to build in practice

%% how to deal with analytically...

%% giant component calculation

%% general assortativity

\section{Definition}

  \textbf{Basic idea:}

  
  
    Random networks with arbitrary degree distributions
    cover much territory but do not represent
    all networks.
  
    Moving away from pure random networks was a key first step.
  
    We can extend in many other directions and a natural
    one is to introduce correlations between 
    different kinds of nodes.
  
    Node attributes may be anything, e.g.:
    
     degree
     demographics (age, gender, etc.)
     group affiliation
    
  
    We speak of mixing patterns, correlations, biases...
  
    Networks are still random at base but now have more 
    global structure.
  
    Build on work by Newman\cite{newman2002a,newman2003e},
    and Bogu\~{n}\'{a} and Serano.\cite{boguna2005a}.
  
  

\section{General\ mixing}

  \textbf{General mixing between node categories}

  
  
    Assume types of nodes are countable, and are
    assigned numbers 1, 2, 3, \ldots.
  
    Consider networks with directed edges.
    {
      $$
      \emn
      = \Prob 
      \left(
        \begin{array}{ll}
          \mbox{an edge connects a node of type $\mu$} \\ 
          \mbox{to a node of type $\nu$}
        \end{array}
      \right)
      $$
    }
    {
      $$
      a_{\mu} 
      = \Prob (\mbox{an edge comes from a node of type $\mu$})
      $$
    }
    {
      $$
      b_{\nu} 
      = \Prob (\mbox{an edge leads to a node of type $\nu$})
      $$
    }  
  
    Write $\m{E} = [\emn]$, $\vec{a} = [a_\mu]$, and $\vec{b} = [b_\nu]$.
  
    Requirements:
    $$
    \sum_{\mu\ \nu} \emn = 1,
    \
    \sum_{\nu} \emn = a_{\mu},
    \
    \mbox{and}
    \sum_{\mu} \emn = b_{\nu}.
    $$
  


%% %%   \textbf{Connection to degree distribution:}
%% 
%%   
%% 
  \textbf{Notes:}

  
   Varying $\emn$ allows us to move
    between the following:
    
     
      \alert{Perfectly assortative networks} where
      nodes only connect to like nodes, and the
      network breaks into subnetworks.\\
      {
        Requires 
        \alertb{$
        \emn = 0 
        $
        if $\mu \ne \nu$ and 
        $\sum_{\mu} {\emm} = 1$.}
      }
    
      \alert{Uncorrelated networks} (as we have studied so far)\\
      {
        For these we must have independence:
        \alertb{$
        \emn = a_{\mu} b_{\nu}.
        $}
      }
    
      \alert{Disassortative networks} where nodes connect
      to nodes distinct from themselves.
    
  
    Disassortative networks can be hard to build
    and may require constraints on the $\emn$.
  
    Basic story: level of assortativity reflects
    the degree to which nodes are connected to
    nodes within their group.
  


  \textbf{Correlation coefficient:}

  
   Quantify the level of assortativity
    with the following \alert{assortativity coefficient}\cite{newman2003e}:
    $$
    r = 
    \frac{
      \sum_{\mu} \emm - \sum_{\mu} a_{\mu} b_{\mu}
      }
    {
      1 - \sum_{\mu} a_{\mu} b_{\mu}
    }
    =
    \frac{
      {\rm Tr} \, \m{E} - || E^2 ||_1
    }
    {
      1 - || E^2 ||_1
    }
    $$
    where $|| \cdot ||_1$ is the $1$-norm = sum of a matrix's entries.
  
    ${\rm Tr} \, \m{E}$ is the fraction of edges that are within groups.
  
    $|| E^2 ||_1$ is the fraction of edges that would be within
    groups if connections were random.
  
    $1 - || E^2 ||_1$ is a normalization factor so $r_{\rm max} = 1$.
  
    When $\rm{Tr}\, \emm = 1$, we have $r=1$. \alert{\checkmark}
  
    When $\emm = a_{\mu} b_{\mu}$, we have $r=0$. \alert{\checkmark}
  


  \textbf{Correlation coefficient:}

  \textbf{Notes:}
  
  
    $r=-1$ is inaccessible if three or more types are present.
  
    Disassortative networks
    simply have nodes connected to unlike nodes---no measure
    of how unlike nodes are.
  
    Minimum value of $r$ occurs when all links between
    non-like nodes: $\rm{Tr}\, \emm = 0$.
  
    $$
    r_{\rm min}
    =
    \frac{
      - || E^2 ||_1
    }
    {
      1 - || E^2 ||_1
    }
    $$
    where $-1 \le r_{\rm min} < 0$.
  
  


\section{Assortativity\ by\ degree}

  \textbf{Scalar quantities}

  
  
    Now consider nodes defined by a scalar integer quantity.
  
    Examples: age in years, height in inches, number of friends, ...
  
    $\ejk$ = $\Prob$ (a randomly chosen edge connects a node with
    value $j$ to a node with value $k$).
  
    $a_{j}$ and $b_{k}$ are defined as before.
  
    Can now measure correlations between nodes based 
    on this scalar quantity using standard 
    \wordwikilink{http://en.wikipedia.org/wiki/Correlation\#Non-parametric_correlation_coefficients}{Pearson correlation coefficient}:
    {
      $$
      r = 
      \frac{\sum_{j\, k} j\, k (\ejk - a_j b_k)}
      {\sigma_a\, \sigma_b}
      =
      \frac{
        \tavg{jk} - \tavg{j}_a\tavg{k}_b
      }
      {
        \sqrt{\tavg{j^2}_a - \tavg{j}_a^2}
        \sqrt{\tavg{k^2}_b - \tavg{k}_b^2}
      }
      $$
%%      where $\sigma_a^2 = \sum_{j} j^2 a_j - [\sum_{j} j a_j]^2$.
      }
    
      This is the observed normalized deviation from randomness
      in the product $jk$.
    


  \textbf{Degree-degree correlations}
  
  
  
    Natural correlation is between the degrees
    of connected nodes.
  
    Now define $\ejk$ with a slight twist:
    $$
    \ejk
      = \Prob 
      \left(
        \begin{array}{ll}
          \mbox{an edge connects a \alertb{degree $j+1$ node}} \\ 
          \mbox{to a \alertb{degree $k+1$ node}}
        \end{array}
      \right)
    $$
    {
      $$
      = \Prob 
      \left(
        \begin{array}{ll}
          \mbox{an edge runs between a \alert{node of in-degree $j$}}\\
          \mbox{and a \alert{node of out-degree $k$}}
        \end{array}
      \right)
      $$
    }
  
    Useful for calculations (as per $R_k$)
  
    \alert{Important:} Must separately define $P_0$
    as the $\{\ejk\}$ contain no information about
    isolated nodes.
  
    Directed networks still fine but we will assume
    from here on that $\ejk = \ekj$.
  


  \textbf{Degree-degree correlations}

  
  
    Notation reconciliation for undirected networks:
    $$
    r = 
    \frac{\sum_{j\, k} j\, k (\ejk - R_j R_k)}
    {\sigma_R^2}
    $$
    where, as before, $R_k$ is the probability that a 
    randomly chosen edge leads to a node of degree $k+1$,
    and
    $$
    \sigma_R^2 = 
    \sum_j j^2 R_j - 
    \left[
      \sum_j j R_j
    \right]^2.
    $$
  


  \textbf{Degree-degree correlations}
  
  \textbf{Error estimate for $r$:}
    
    
      Remove edge $i$ and recompute $r$ 
      to obtain $r_i$.
    
      Repeat for all edges and compute
      using the 
      \wordwikilink{http://en.wikipedia.org/wiki/Resampling_(statistics)\#Jackknife}{jackknife method}\cite{efron1981a}
      $$
      \sigma_r^2 
      =
      \sum_{i} (r_i - r)^2.
      $$
     Mildly sneaky as variables need to be
      independent for us to be truly happy and edges
      are correlated...
    
  



  \textbf{Measurements of degree-degree correlations}
  
  \includegraphics[width=\textwidth]{newman2003e_tab2.pdf}

  
   Social networks tend to be assortative (homophily)
   Technological and biological networks tend to be disassortative
  


\section{Contagion}

\subsection{Spreading\ condition}

  \textbf{Spreading on degree-correlated networks}

  
   
    Next: Generalize our work for random networks to
    degree-correlated networks.
   
    As before, by allowing that a node of degree $k$
    is activated by one neighbor with probability
    $\infprob_{k1}$, we can handle various problems:
    
     
      find the giant component size.
     
      find the probability and extent of spread
      for simple disease models.
     
      find the probability of spreading
      for simple threshold models.
    
    
  


  \textbf{Spreading on degree-correlated networks}

  
   
    \alert{Goal:} Find $f_{n,j}$ = $\Prob$ an edge 
    emanating from a degree $j+1$ node leads to
    a finite active subcomponent of size $n$.
  
    Repeat: a node of degree $k$ is in the game with
    probability $\infprob_{k1}$.
  
    Define $\vec{\infprob}_1 = [\infprob_{k1}]$.
  
    \alert{Plan:} Find the generating function
    $\alertb{F_j(x;\vec{\infprob}_1)} = \sum_{n=0}^\infty f_{n,j} x^n$.
  


  \textbf{Spreading on degree-correlated networks}

  
   Recursive relationship:
    \begin{align}
      \nonumber
      F_j(x;\vec{\infprob}_1)
      & =
      x^{0}
      \sum_{k=0}^{\infty} 
      \frac{e_{jk}}{R_j}
      ( 1 - \infprob_{k+1,1} )
      \\ 
      \nonumber
      & +
      x
      \sum_{k=0}^{\infty} 
      \frac{e_{jk}}{R_j}
      \infprob_{k+1,1}
      \left[
        F_k(x;\vec{\infprob}_1)
      \right]^k.
    \end{align}
  
    \alert{First term} =
    $\Prob$ that the first node we reach is not in the game.
  
    \alert{Second term} involves
    $\Prob$ we hit an active node which has $k$ outgoing edges.
  
    Next: find average size of active components reached
    by following a link from a degree $j+1$ node = $F_{j}'(1;\vec{\infprob}_1)$.
  



  \textbf{Spreading on degree-correlated networks}

  
  
    Differentiate $F_{j}(x;\vec{\infprob}_1)$,
    set $x=1$, and rearrange.
  
    We use $F_k(1;\vec{\infprob}_1)=1$ which
    is true when no giant component exists.
    {We find:}
    {
      $$
      R_j F_{j}'(1;\vec{\infprob}_1)
      =
      \sum_{k=0}^{\infty}
      e_{jk}
      \infprob_{k+1,1}
      + 
      \sum_{k=0}^{\infty}
      k e_{jk}
      \infprob_{k+1,1}
      F_k'(1;\vec{\infprob}_1).
      $$
    }
    
      Rearranging and introducing a sneaky $\delta_{jk}$:
      $$
      \sum_{k=0}^\infty
      \left(
        \delta_{jk} R_k 
        -
        k \infprob_{k+1,1} e_{jk}
      \right)
      F_k'(1;\vec{\infprob}_1)
      =
      \sum_{k=0}^{\infty}
      e_{jk}
      \infprob_{k+1,1}.
      $$
  
  


  \textbf{Spreading on degree-correlated networks}

  
  
    In matrix form, we have
    $$
    \mathbf{A}_{\mathbf{E},\vec{\infprob}_1 }
    \vec{F}'(1;\vec{\infprob}_1)
    =
    \mathbf{E} \vec{\infprob}_1
    $$
    where 
    \begin{gather}
      \nonumber
      \left[ \mathbf{A}_{\mathbf{E},\vec{\infprob}_1 } \right]_{j+1,k+1}
      = 
      \delta_{jk} R_{k} 
      -
      k \infprob_{k+1,1} e_{jk},
      \\
      \nonumber
      \left[ \vec{F}'(1;\vec{\infprob}_1)  \right]_{k+1}
      = 
      F_{k}'(1;\vec{\infprob}_1),
      \\
      \nonumber
      \left[ \mathbf{E} \right]_{j+1,k+1}
      = 
      e_{jk},
      \
      \mbox{and} \  \left[ \vec{\infprob}_1 \right]_{k+1}
      =
      \infprob_{k+1,1}.
    \end{gather}
  

  \textbf{Spreading on degree-correlated networks}

  
  
    So, in principle at least:
    $$
    \vec{F}'(1;\vec{\infprob}_1)
    =
    \mathbf{A}_{\mathbf{E},\vec{\infprob}_1 }^{-1}
    \, \mathbf{E} \vec{\infprob}_1.
    $$
  
    Now: as $\vec{F}'(1;\vec{\infprob}_1)$, 
    the average size of an active component
    reached along an edge, increases, we move towards
    a transition to a giant component.
  
    Right at the transition, the average component
    size explodes.
  
    Exploding inverses of matrices occur
    when their determinants are 0. 
  
    The condition is therefore: 
    $$\det{\m{A}_{\mathbf{E},\vec{\infprob}_1}} = 0$$.
 


  \textbf{Spreading on degree-correlated networks}

  
   General condition details:
    $$
    \det{\m{A}_{\mathbf{E},\vec{\infprob}_1}}
    =
    \det
    \left[
      \delta_{jk} R_{k-1} 
      -
      (k-1) \infprob_{k,1} e_{j-1,k-1}
    \right] = 0.
    $$
  
    The above collapses to our standard
    contagion condition when $e_{jk} = R_j R_k$.
  
    When $\vec{\infprob}_1=\infprob \vec{1}$, we have the condition
    for a simple disease model's successful spread
    $$
    \det
    \left[
      \delta_{jk} R_{k-1} 
      -
      \infprob (k-1)   e_{j-1,k-1}
    \right] = 0.
    $$
   
    When $\vec{\infprob}_1=\vec{1}$, we have the condition
    for the existence of a giant component:
    $$
    \det
    \left[
      \delta_{jk} R_{k-1} 
      -
      (k-1) e_{j-1,k-1}
    \right] = 0.
    $$
  
    Bonusville: We'll find a much better
    version of this set of conditions later...
  


\subsection{Triggering\ probability}

  \textbf{Spreading on degree-correlated networks}

  \textbf{We'll next find two more pieces:}
    
     
      $\Ptrig$, the probability of starting a cascade
    
      $S$, the expected extent of activation given
      a small seed.
    

    \textbf{Triggering probability:}
      
       
        Generating function:
        $$
        H(x;\vec{\infprob}_1)
        = 
        x
        \sum_{k=0}^\infty
        P_k
        \left[
          F_{k-1}(x;\vec{\infprob}_1)
        \right]^k.
        $$
      
        Generating function for vulnerable
        component size is more complicated.
      
    
    
  


  \textbf{Spreading on degree-correlated networks}


  
   
    Want probability of \alert{not reaching} a finite component.
    \begin{align}
      \nonumber
      \Ptrig  = \Strig 
      = & 
      1 -
      H(1;\vec{\infprob}_1) 
      \\
      \nonumber
      = &
      1
      -
      \sum_{k=0}^\infty
      P_k
      \left[
        F_{k-1}(1;\vec{\infprob}_1)
      \right]^k.
    \end{align}
  
    Last piece: we have to compute $F_{k-1}(1;\vec{\infprob}_1)$.
  
    Nastier (nonlinear)---we have to solve the 
    recursive expression we started with when $x=1$:
    $
      F_j(1;\vec{\infprob}_1)
      =
      \sum_{k=0}^{\infty} 
      \frac{e_{jk}}{R_j}
      ( 1 - \infprob_{k+1,1} )
      + $\\
      \mbox{} \hfill
      $
      \sum_{k=0}^{\infty} 
      \frac{e_{jk}}{R_j}
      \infprob_{k+1,1}
      \left[
        F_k(1;\vec{\infprob}_1)
      \right]^k.
    $
  
    Iterative methods should work here.
  


\subsection{Expected size}

  \textbf{Spreading on degree-correlated networks}

  
  
    \alert{Truly final piece:} 
    Find final size using approach of Gleeson\cite{gleeson2008a},
    a generalization of that used for uncorrelated random networks.
  
    Need to compute $\theta_{j,t}$, the probability that 
    an edge leading to a degree $j$ node is infected at time $t$.
  
    Evolution of edge activity probability:
    $$
    \theta_{j,t+1}
    =
    G_j(\vec{\theta}_t)
    =
    \phi_0 + 
    (1-\phi_0) \times
    $$
    $$
    \sum_{k=1}^{\infty}
    \frac{e_{j-1,k-1}}{R_{j-1}}
    \sum_{i=0}^{k-1}
    \binom{k-1}{i}
    \theta_{k,t}^{\, i}
    (1-\theta_{k,t})^{k-1-i}
    \infprob_{ki}.
    $$
   
    Overall active fraction's evolution:
    $$
    \phi_{t+1}
    =
    \phi_0
    +
    (1-\phi_0)
    \sum_{k=0}^{\infty}
    P_k
    \sum_{i=0}^{k}
    \binom{k}{i}
    \theta_{k,t}^{\, i}
    (1-\theta_{k,t})^{k-i}
    \infprob_{ki}.
    $$
  


  \textbf{Spreading on degree-correlated networks}

  
  
    As before, these equations give the actual evolution
    of $\phi_t$ for synchronous updates.
  
    Contagion condition follows from $\vec{\theta}_{t+1} = \vec{G}(\vec{\theta}_{t})$.
  
    %% Need small $\vec{\theta}_{0}$ to take off so we 
    Expand $\vec{G}$ around $\vec{\theta}_{0}=\vec{0}$.
    {
      $$
      \theta_{j,t+1} =
      G_j(\vec{0})
      + 
      \sum_{k=1}^\infty
      \alert{\partialdiff{G_j(\vec{0})}{\theta_{k,t}}}
      \theta_{k,t}
      +
      \frac{1}{2!}
      \sum_{k=1}^\infty
      \partialdiffsq{G_j(\vec{0})}{\theta_{k,t}}
      \theta_{k,t}^2
      +
      \ldots
      $$
    }
  
    If $G_j(\vec{0}) \ne 0$ for at least one $j$, always have some infection.
  
    If $G_j(\vec{0}) = 0 \, \forall \, j$, want largest eigenvalue
    \alertb{$\left[\partialdiff{G_j(\vec{0})}{\theta_{k,t}}\right] > 1$}.
   
    Condition for spreading is therefore
    dependent on eigenvalues of this matrix:
    $$
    \partialdiff{G_j(\vec{0})}{\theta_{k,t}} 
    = 
    \frac{e_{j-1,k-1}}{R_{j-1}}
    (k-1)
    \infprob_{k1}
    $$
    \insertassignmentquestionsoft{09}{9}



  \textbf{How the giant component changes with assortativity:}

      
    \includegraphics[width=\textwidth]{newman2002a_fig1.pdf}\\
    {\tiny from Newman, 2002\cite{newman2002a}}
    
    
     
      More assortative networks percolate for
      lower average degrees
     
      But disassortative networks end up with
      higher extents of spreading.
    
  