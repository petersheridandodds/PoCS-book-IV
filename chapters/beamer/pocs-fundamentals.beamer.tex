\section{Data}

\changelecturelogo{0.18}{MacGillivray_William_John_Dory_cut.png}

\begin{frame}
  
  \showtarotcards{0.30}{
    overview,
    manifesto,
    scaling,
    power-law-size-distributions,
    random-walks,
    variable-transformation,
    rich-get-richer,
    efficient-language,
    data-poor,
  }
\end{frame}

\begin{frame}
  
  \showtarotcards{0.30}{
    overview,
    manifesto,
    scaling,
    power-law-size-distributions,
    random-walks,
    variable-transformation,
    rich-get-richer,
    efficient-language,
    data-poor,
    pouring-data,
  }
\end{frame}

\begin{frame}

  \wordwikilink{http://www.economist.com/node/15557443}{Data, Data, Everywhere---the Economist, Feb 25, 2010}

  \begin{columns}
    \column{0.45\textwidth}
    \includegraphics[width=\textwidth]{201009SRC696-economist.png}
    \begin{itemize}
    \item 
      Exponential growth: $\sim$ 60\% per year.
    \end{itemize}
    \column{0.55\textwidth}
    \begin{block}{Big Data Science:}
    \begin{itemize}
    \item 
      2013: year traffic on Internet estimate to reach 2/3 Zettabytes \\
      (1ZB = $10^3$EB = $10^6$PB = $10^9$TB)
    \item 
      Large Hadron Collider: 40 TB/second.\\
    \item 
      2016---Large Synoptic Survey Telescope:\\
      140 TB every 5 days.
    \item 
      Facebook: $\sim$ 250 billion photos (mid 2013)
    \item 
      Twitter: $\sim$ 500 billion tweets (mid 2013)
    \end{itemize}
    \end{block}      
  \end{columns}

\end{frame}

\begin{frame}[plain]
  \frametitle{No really, that's a lot of data}

  \includegraphics[width=1.2\textwidth]{201009SRC722-economist.png}
\end{frame}


\begin{frame}
  \frametitle{Big Data---Culturomics:}

  \small{``Quantitative analysis of culture using millions of
    digitized books'' by Michel et al., Science, 2011\cite{michel2011a}}

  \includegraphics[width=0.45\textwidth]{michel2011a_fig3a.pdf} 
  \includegraphics[width=0.45\textwidth]{michel2011a_fig3e.pdf} \\
  \includegraphics[width=0.45\textwidth]{michel2011a_fig3f.pdf}
  \includegraphics[width=0.35\textwidth]{michel2011a_fig4f.pdf}

  {\small
    \wordwikilink{http://www.culturomics.org/}{http://www.culturomics.org/}\\
    \wordwikilink{http://ngrams.googlelabs.com/}{Google Books ngram viewer}
  }

\end{frame}


\begin{frame}
  \frametitle{Basic Science $\simeq$ Describe + Explain:}

  \begin{columns}
    \column{0.4\textwidth}
    \includegraphics[width=\textwidth]{lordkelvin-aip.jpg}
    \column{0.6\textwidth}
    \begin{block}{Lord Kelvin (possibly):}
      \begin{itemize}
      \item<+->
        \alertg{``To measure is to know.''}
      \item<+-> 
        \alertg{``If you cannot measure it, you cannot improve it.''}
      \end{itemize}
    \end{block}
    \begin{block}<+->{Bonus:}
      \begin{itemize}
      \item<+->
        \visible<+->{
          \alertg{``X-rays will prove to be a hoax.''}
        }
      \item<+->
        \visible<+->{
          \alertg{``There is nothing new to be discovered in physics now, 
            All that remains is more and more precise measurement.''}
        }
      \end{itemize}
    \end{block}

  \end{columns}

\end{frame}


\begin{frame}
  \frametitle{The Newness of being a Scientist (1833 on):}

  \includegraphics[width=\textwidth]{2013-01-14ngrams-scientist.jpg}

  \begin{block}{}
    \begin{itemize}
    \item<+-> 
      Etymology 
      \wordwikilink{http://en.wikipedia.org/wiki/Scientist\#Historical\_development\_and\_etymology\_of\_the\_term}{here}.
    \item<+-> 
      ``Scientists are the people who ask a question about a phenomenon and proceed to \alertg{systematically} go about answering the question themselves. They are by nature curious, creative and well organized.''
    \end{itemize}
  \end{block}

  
\end{frame}


%% \begin{frame}
%%   \frametitle{Outreach}
%% 
%%   \begin{center}
%%     \includegraphics[height=0.8\textheight]{complexity-society-frontpage.pdf}
%%   \end{center}
%% 
%% \end{frame}
%% 
%% \begin{frame}
%%   \frametitle{Outreach}
%% 
%%   ``The society objectives are to promote the \alertb{theory of complexity} in
%%   education, government, the health service and business as well as the
%%   beneficial application of complexity in a wide variety of social,
%%   economic, scientific and technological contexts such as sources of
%%   competitive advantage, business clusters and knowledge management.''
%% 
%%   \medskip
%% 
%%   \visible<2->{
%%     ``Complexity includes ideas such as complex adaptive systems,
%%     self-organisation, co-evolution, agent based computer models, chaos,
%%     networks, emergence, and fractals.''
%%   }
%% 
%% \end{frame}

%% \changelecturelogo{0.18}{MacGillivray_William_John_Dory_cut.png}
%% \changelecturelogo{0.18}{2014-08-16pocsmap-emergence-reductionism-manifesto_postcard1_polaroid-logo.png}

\changelecturelogo{0.18}{2014-08-16pocsmap-emergence-reductionism-manifesto_postcard0_polaroid-logo.png}

\section{Emergence}

\begin{frame}
  \includegraphics[width=\textwidth]{2014-08-16pocsmap-emergence-reductionism-manifesto_postcard0_polaroid.png}
\end{frame}

\begin{frame}
  \includegraphics[width=\textwidth]{2014-08-16pocsmap-emergence-reductionism-manifesto_postcard1_polaroid.png}
\end{frame}

\begin{frame}
  
  \showtarotcards{0.30}{
    overview,
    manifesto,
    scaling,
    power-law-size-distributions,
    random-walks,
    variable-transformation,
    rich-get-richer,
    efficient-language,
    data-poor,
    pouring-data,
    emergence-of-structure,
  }

\end{frame}

\begin{frame}
  
  \showtarotcards{0.30}{
    overview,
    manifesto,
    scaling,
    power-law-size-distributions,
    random-walks,
    variable-transformation,
    rich-get-richer,
    efficient-language,
    data-poor,
    pouring-data,
    emergence-of-structure,
    emergence-of-destruction,
  }
\end{frame}

\begin{frame}
  
  \showtarotcards{0.30}{
    overview,
    manifesto,
    scaling,
    power-law-size-distributions,
    random-walks,
    variable-transformation,
    rich-get-richer,
    efficient-language,
    data-poor,
    pouring-data,
    emergence-of-structure,
    emergence-of-destruction,
    emergence-of-thinking,
  }
\end{frame}

\begin{frame}
  
  \showtarotcards{0.30}{
    overview,
    manifesto,
    scaling,
    power-law-size-distributions,
    random-walks,
    variable-transformation,
    rich-get-richer,
    efficient-language,
    data-poor,
    pouring-data,
    emergence-of-structure,
    emergence-of-destruction,
    emergence-of-thinking,
    emergence-of-stories,
  }
\end{frame}


%% \changelecturelogo{0.18}{MacGillivray_William_John_Dory_cut.png}

\changelecturelogo{0.18}{2014-08-16pocsmap-emergence-reductionism-manifesto_postcard1_polaroid-logo.png}

\begin{frame}
  \frametitle{Emergence:}

  \begin{block}{The Wikipedia on \alertb{Emergence}:
      \hfill
      \includegraphics[width=.07\textwidth]{wikipedia-tp.pdf}
    }
    ``In philosophy, systems theory and the sciences, emergence refers to
    the way complex systems and patterns arise out of a multiplicity of
    relatively simple interactions. 
    \visible<2->{... \alertb{emergence is central
        to the physics of complex systems and yet very controversial}.''}
  \end{block}

\begin{block}<3->{}
    The philosopher G. H. Lewes first
    used the word explicity in 1875.
\end{block}

\end{frame}

\begin{frame}<1 | handout:0 | trans:1>
  \frametitle{Fireflies $\Rightarrow$ Synchronized Flashes:}

  \localvideo{fireflies-strogatz.mov}

  \small

  Film: Sir David Attenborough, BBC.

  Voiceover: Steve Strogatz on 
  \wordwikilink{http://www.radiolab.org/2007/aug/14/}{Radiolab's Emergence, S1E3}.

\end{frame}

\begin{frame}
%%  \frametitle{Emergence:}

  \begin{block}{Emergence:}
    Tornadoes, financial collapses, human emotion aren't
    found in water molecules, dollar bills, or carbon atoms.
    %%      There's no tornado in a water molecule, no financial
    %%      collapse in a single dollar bill,
    %%      no human emotion in a carbon atom;
    %%      these are `emergent' properties.
  \end{block}

  \begin{block}<2->{Examples:}
    \begin{itemize}
    \item<2-> 
      Fundamental particles $\Rightarrow$ Life, the Universe, and Everything
    \item<2-> 
      Genes $\Rightarrow$ Organisms
    \item<2->
      Neurons etc. $\Rightarrow$ Brain $\Rightarrow$ Thoughts
    \item<2-> 
      People $\Rightarrow$ Religion, Collective behaviour
    \item<2->
      People $\Rightarrow$ The Web
    \item<2->
      People $\Rightarrow$ Language, and rules of language
    \item<2->
      ? $\Rightarrow$ time; 
      ? $\Rightarrow$ gravity;
      ? $\Rightarrow$ reality.
    \end{itemize}
  \end{block}

  \begin{block}<3->{}
    \alertb{``The whole is more than the sum of its parts''}
    --Aristotle
  \end{block}

\end{frame}


\begin{frame}
  \frametitle{Emergence:}

  \begin{block}{\wordwikilink{http://en.wikipedia.org/wiki/Thomas\_Schelling}{Thomas Schelling}
      \smallskip
      (Economist/Nobelist):}
    \medskip
    \begin{columns}
      \column{0.3\textwidth}
      \includegraphics[width=\textwidth]{eggs1.jpg}\\
      \includegraphics[width=\textwidth]{eggs2.jpg}\\
      \includegraphics[width=\textwidth]{eggs3.jpg}\\
      \wordwikilink{http://www.youtube.com/watch?v=JjfihtGefxk}{[youtube]}
      \column{0.7\textwidth}
      \begin{center}
      \includegraphics[width=0.7\textwidth]{thomas_schelling_nobelprice_eco_2005.png}
      \begin{itemize}
      \item 
        ``Micromotives and Macrobehavior''\cite{schelling1978a}
        \begin{itemize}
        \item
          Segregation\cite{schelling1971a,schelling2006a}
        \item
          Wearing hockey helmets\cite{schelling1973a}
        \item
          Seating choices
        \end{itemize}
      \end{itemize}
      \end{center}
    \end{columns}
  \end{block}

\end{frame}



\begin{frame}
  \frametitle{Emergence:}

  \begin{block}{
      \wordwikilink{http://en.wikipedia.org/wiki/Friedrich_Hayek}{Friedrich Hayek} 
      \smallskip
      (Economist/Philospher/Nobelist):
    }
    \begin{itemize}
    \item<1-> Markets, legal systems, political systems are emergent and not designed.
    \item<2->   
      `Taxis' = made order (by God, Sovereign, Government, ...)
    \item<3->
      `Cosmos' = grown order
    \item<4->
      Archetypal limits of \alertb{hierarchical} and \alertr{decentralized} structures.
    \item<5->
      \alertb{Hierarchies} arise once problems are solved.\cite{dodds2003c}
    \item<6->
      \alertr{Decentralized structures} help solve problems.
    \item<7->
      Dewey Decimal System versus tagging.
    \end{itemize}
  \end{block}

\end{frame}

\begin{frame}
  \frametitle{Emergence:}

  \begin{block}{
      \wordwikilink{http://en.wikipedia.org/wiki/James\_Samuel\_Coleman}{James Coleman} 
      \smallskip
      in \textit{Foundations of Social Theory}:}
    \includegraphics[width=0.9\textwidth]{coleman.pdf}
  \end{block}

  \bigskip

  \begin{block}{}
  \begin{itemize}
  \item 
    Understand macrophenomena arises from microbehavior
    which in turn depends on macrophenomena.\cite{coleman1994a}
  \item<2->
    More on Coleman 
    \wordwikilink{http://en.wikipedia.org/wiki/James_Samuel_Coleman}{here}.
  \end{itemize}
  \end{block}

\end{frame}


\begin{frame}
  \frametitle{The emergence of taste:}

  \begin{block}{}
    \begin{itemize}
    \item 
      Molecules $\Rightarrow$ Ingredients $\Rightarrow$ Taste 
    \item 
      See Michael Pollan's
      \wordwikilink{http://www.nytimes.com/2007/01/28/magazine/28nutritionism.t.html}{article on nutritionism} in the New York Times, January 28, 2007.

      \medskip

      \includegraphics[width=0.7\textwidth]{2007-01-28unhappymeals.jpg}\\
      {\tiny \wordwikilink{nytimes.com}{nytimes.com}}
    \end{itemize}
  \end{block}

\end{frame}


\begin{frame}
  \frametitle{Reductionism}

  \begin{block}{Reductionism and food:}
    \begin{itemize}
%    \item 
%      {\small \alertb{\textit{Unhappy Meals}}, Michael Pollan, NY Times, January 2007}
    \item<2-> 
      \alertb{Pollan:}
      ``even the simplest food is a
      hopelessly complex thing to study, a virtual wilderness of chemical
      compounds, many of which exist in complex and dynamic relation to one
      another...''
    \item<3->
      ``So ... break the
      thing down into its component parts and study those one by one, even
      if that means ignoring complex interactions and contexts, as well as
      the fact that the whole may be more than, or just different from, the
      sum of its parts. This is what we mean by reductionist science.''
    \end{itemize}
  \end{block}

\end{frame}

\begin{frame}
  \frametitle{Reductionism}

  \begin{block}{}
  \begin{itemize}
  \item<1-> ``people don't eat nutrients, they eat foods, and foods can behave
    very differently than the nutrients they contain.''
  \item<2-> Studies suggest diets high in fruits and vegetables help prevent cancer.
  \item<3-> So...  find the nutrients responsible and eat more of them
  \item<4-> But ``in the case of \alertb{beta
    carotene ingested as a supplement}, scientists have discovered that it
    actually \alertr{increases the risk of certain cancers}. Oops.''
  \end{itemize}
  \end{block}

\end{frame}


\begin{frame}
  \frametitle{Reductionism}

  \begin{columns}
    \column{0.7\textwidth}
    \begin{block}{\alertg{Thyme's known antioxidants:}}
      4-Terpineol, alanine, anethole, apigenin, ascorbic acid, beta
      carotene, caffeic acid, camphene, carvacrol, chlorogenic acid,
      chrysoeriol, eriodictyol, eugenol, ferulic acid, gallic acid,
      gamma-terpinene isochlorogenic acid, isoeugenol, isothymonin,
      kaempferol, labiatic acid, lauric acid, linalyl acetate, luteolin,
      methionine, myrcene, myristic acid, naringenin, oleanolic acid,
      p-coumoric acid, p-hydroxy-benzoic acid, palmitic acid, rosmarinic
      acid, selenium, tannin, thymol, tryptophan, ursolic acid, vanillic
      acid.
    \end{block}
    \column{0.3\textwidth}
    \includegraphics[width=\textwidth]{thyme-lrg.jpg}\\
    {\mbox{} \hfill \tiny [cnn.com]}
  \end{columns}

\end{frame}

\begin{frame}
  \frametitle{Reductionism}

  \begin{block}{}
  ``It would be great to know how this all works, but \alertb{in the meantime} we
  can enjoy thyme in the knowledge that it probably doesn't do any harm
  (since people have been eating it forever) and that it may actually do
  some good (since people have been eating it forever) and that even if
  it does nothing, we like the way it tastes.''

  \bigskip

  \visible<2->{
    \alertb{Gulf between theory and practice (see baseball and bumblebees).}
  }
  \end{block}

\end{frame}

\begin{frame}
%%  \frametitle{Emergent Catastrophic Failure:}
\frametitle{This is a Collateralized Debt Obligation:}

  \includegraphics[width=\textwidth]{2011-10-25flickr-hotdog-2527649390_600d768d96_b-cropped-transparent.png}

  %% http://www.flickr.com/photos/68662010@N00/2527649390/sizes/l/in/photostream/
  
\end{frame}

\begin{frame}

  \begin{center}
    
    \begin{itemize}
    \item 
    ``The Universe is made of stories, not of atoms.''
    \end{itemize}

    \includegraphics[height=0.5\textheight]{universe-is-made-of-stories.jpg}

    \begin{itemize}
    \item<+->
      From ``The Speed of Darkness'' (1968) by
      \wordwikilink{http://en.wikipedia.org/wiki/Muriel\_Rukeyser}{Muriel Rukeyser}
    \item<+->
      Quoted by Metatron in Supernatural, Meta Fiction, S9E18.
    \end{itemize}
  \end{center}

\end{frame}

\begin{frame}
%%  \frametitle{\wordwikilink{http://en.wikipedia.org/wiki/Terry\_Pratchett}{Pratchett} on stories:}
  \frametitle{\wordwikilink{http://en.wikipedia.org/wiki/Terry\_Pratchett}{(Sir Terry) Pratchett's}
    \wordwikilink{http://wiki.lspace.org/wiki/Narrativium}{Narrativium}:
%%    and 
%%    \wordwikilink{http://wiki.lspace.org/wiki/Narrative\_Causality}{Narrative Causality}:
  }

  \begin{columns}
    \column{0.3\textwidth}
    \includegraphics[width=\textwidth]{4463841109_f3dbc05754_pratchett.jpg}
    \column{0.7\textwidth}
    \begin{block}<+->{}
        \begin{itemize}
        \item<+-> 
          ``The most common element on the disc, although not
          included in the list of the standard five: earth, fire, air,
          water and surprise. It ensures that everything runs properly
          as a story.''
        \item<+->
          ``A little narrativium goes a long way: the simpler the story,
          the better you understand it. Storytelling is the opposite of
          reductionism: 26 letters and some rules of grammar are no story
          at all.''
        \end{itemize}
      \end{block}
    \end{columns}

%%     \begin{block}<+->{}
%%       \begin{itemize}
%%       \item
%%         ``Heroes only win when outnumbered, and things which have a
%%         one-in-a-million chance of succeeding often do so.''
%%       \end{itemize}
%%     \end{block}

\end{frame}

\begin{frame}
  \frametitle{Emergence:}

  \begin{block}<+->{Higher complexity:}
    \begin{itemize}
    \item 
      Many system scales (or levels) \\ 
      that
      interact with each other.
    \item<+->
      Potentially much harder to explain/understand.
    \end{itemize}
  \end{block}

  \begin{block}<+->{Even mathematics:\cite{foote2007a}}
    \begin{columns}
      \column{0.02\textwidth}
      \column{0.2\textwidth}
      \includegraphics[width=0.97\textwidth]{kurt_godel.jpg}
      \column{0.6\textwidth}
      \wordwikilink{http://bit.ly/VdbsWU}{G\"{o}del's Theorem}:\\
      we can't prove every theorem that's true \ldots
%%      \column{0.2\textwidth}
%%      \includegraphics[width=0.97\textwidth]{hofst.jpg}
    \end{columns}
  \end{block}

  \begin{block}<+->{}
    \begin{itemize}
    \item 
      Suggests a \alertb{strong form of emergence}:
      \uncover<+->{
        Some phenomena cannot be analytically deduced
        from elementary aspects of a system.
      }
    \end{itemize}
  \end{block}

\end{frame}


\begin{frame}
  \frametitle{Emergence:}

  \begin{block}{}
    Roughly speaking, there are \alertg{two types} of \alertb{emergence}:  
  \end{block}

  \begin{block}<2->{\alertb{I. Weak emergence:}}
    System-level phenomena is
    different from that of its constituent parts
    yet can be connected theoretically.
  \end{block}

  \begin{block}<3->{\alertb{II. Strong emergence:}}
    System-level 
    phenomena fundamentally cannot
    be deduced from how parts interact.
  \end{block}

%%   \smallskip
%%   \visible<4->{
%%     Strong emergence could be called \alertr{magic}...\\
%%     See Bedau (1997)\cite{bedau1997a}
%%     }

\end{frame}

\begin{frame}
  \frametitle{Emergence:}

  \begin{block}{}
    \begin{itemize}
      %% \item<+->
      %%   Complex Systems enthusiasts often decry \alertb{reductionist} approaches \ldots
      %% \item<+->
      %%   But reductionism seems to be misunderstood.
    \item<+->
      \alertb{Reductionist} techniques can explain weak emergence.
    \item<+->
      \alertb{Magic} explains strong emergence.\cite{bedau1997a}
    \item<+->
      But: maybe \alertb{magic} should be interpreted
      as an \alertb{inscrutable yet real mechanism} that cannot
      ever be \alertr{simply described}.    
    \item<+->
      Gulp.
    \end{itemize}
  \end{block}

\end{frame}


\begin{frame}

  \includemedia[
  label=audio.radiolab-limits,
  activate=pageopen,
  addresource=sound/2010/radiolab041610c.mp3,
  flashvars={
    source=sound/2010/radiolab041610c.mp3
    &autoPlay=0
    &showinfo=1
  },
  transparent
  ]{\color{grey}\framebox[\linewidth][c]{Limits of Science | Radiolab}}{APlayer.swf}
  
  \begin{block}{}
    \begin{columns}
      \column{0.02\textwidth}
      \column{0.18\textwidth}
      \includegraphics[width=\textwidth]{radiolab-limits-150430.jpg}
      \column{0.78\textwidth}
      Listen to Steve Strogatz, Hod Lipson, and Michael Schmidt (Cornell) in the 
      \wordwikilink{http://www.radiolab.org/2010/apr/05/limits-of-science/}{last piece}
      (11:16) on Radiolab's show 
      \wordwikilink{http://www.radiolab.org/2010/apr/05/}{`Limits'} (April 5, 2010).
      %% (51:40)
      \column{0.02\textwidth}
    %% \url{http://blogs.wnyc.org/radiolab/2010/04/05/limits/}
    \end{columns}
    \bigskip
    \includegraphics[width=\textwidth]{radiolab-limits-science-tp-1.pdf}

    Pair with some 
    \wordwikilink{https://www.youtube.com/watch?v=z7VYVjR_nwE}{slow tv}

    Bonus: 
    \wordwikilink{http://www.youtube.com/watch?v=6XSncCrQzwk}{Mike Schmidt's talk on Eureqa}
    at \newline
    \wordwikilink{http://www.uvm.edu/~tedxuvm/?Page=archive/2011/default.php}{UVM's 2011 TEDx event ``Big Data, Big Stories.''}

\end{block}

\end{frame}


\changelecturelogo{0.18}{MacGillivray_William_John_Dory_cut.png}

\section{Self-Organization}

\begin{frame}
  \frametitle{Definitions}

  \begin{block}<+->{}
    ``\wordwikilink{http://en.wikipedia.org/wiki/Self-organization}{Self-organization} 
    is a process in which the internal organization 
    of a system, normally an open system, increases in complexity without 
    being guided or managed by an outside source.''
    \uncover<+->{(also: Self-assembly)}
  \end{block}

  \medskip

%%   \begin{block}<2->{}
%%     \begin{itemize}
%%     \item<2-> 
%%       Self-organization refers to a broad array of decentralized processes 
%%       that lead to emergent phenomena.
%%     \end{itemize}
%%   \end{block}
%% 
%% \end{frame}
%% 

%% \begin{frame}
%%   \frametitle{

   \begin{block}<+->{Examples:}
     \begin{itemize}
     \item
       Molecules/Atoms liking each other $\rightarrow$ \\
       \mbox{} \hfill Gases, liquids, and solids.
     \item
       Spin alignment $\rightarrow$ Magnetization.
     \item 
       Protein folding.
     \item 
       Imitation $\rightarrow$ Herding, flocking, mobs, \ldots
     \end{itemize}

     \medskip

     \visible<+->{
       Fundamental question: how likely is `complexification'?
     }
   \end{block}

 \end{frame}

%% \begin{frame}
%%   \frametitle{Buzzword Definitions}
%% 
%%   \alertb{Emergence but no Self-Organization?}
%% 
%%   \bigskip
%% 
%%   \visible<2->{
%%     H$_2$0 molecules $\Rightarrow$ Water
%%   }
%% 
%%   \bigskip
%%   
%%   \visible<3->{
%%     Random walks $\Rightarrow$ Normal distributions
%%   }
%%   
%% \end{frame}
%% 
%% 
%% \begin{frame}
%%   \frametitle{Buzzword Definitions}
%% 
%%   \alertb{Self-organization but no Emergence?}
%% 
%%   \medskip
%% 
%%   \visible<2->{
%%     Water above and near the freezing point.
%%   }
%% 
%%   \medskip
%% 
%%   \visible<3->{
%%     Emergence may be limited to a low scale of a system.
%%   }
%% 
%% \end{frame}

%% \begin{frame}
%%  \frametitle{Economics}
%%Eric Beinhocker (\textit{The Origin of Wealth}):
%%
%%{\small
%%
%%\begin{tabular}{lll}
%% & Complexity Economics  &  Traditional Economics \\
%%Dynamic & 
%%Open, dynamic, non-linear systems, far from equilibrium & 
%%Closed, static, linear systems in equilibrium \\
%%Agents  &
%%Modelled individually; use inductive rules of thumb to make decisions; have incomplete information; are subject to errors and biases; learn to adapt over time  &
%%Modelled collectively; use complex deductive calculations to make decisions; have complete information; make no errors and have no biases; have no need for learning or adaptation (are already perfect) \\
%%Networks  &  
%%Explicitly model bi-lateral interactions between individual agents; networks of relationships change over time &
%%Assume agents only interact indirectly through market mechanisms (e.g. auctions) \\
%%Emergence   &    
%%No distinction between micro/macro economics; 
%%macro patterns are emergent result of micro level behaviours and interactions.  &  
%%Micro-and macroeconomics remain separate disciplines \\
%%Evolution   &    
%%The evolutionary process of differentiation, selection and amplification provides the system with novelty and is responsible for its growth in order and complexity  &
%%No mechanism for endogenously creating novelty, or growth in order and complexity \\
%%\end{tabular}
%%
%%}
%% 
%% \end{frame}

%% \begin{frame}
%%   \frametitle{Economics}
%% 
%%   Eric Beinhocker (\textit{The Origin of Wealth}):\cite{beinhocker2006a}
%% 
%%   \begin{block}{\alertb{Dynamic:}}
%%     \begin{itemize}
%%     \item 
%%       \tc{red}{Complexity Economics:}
%%       Open, dynamic, non-linear systems, far from equilibrium
%%     \item 
%%       \tc{red}{Traditional Economics:}
%%       Closed, static, linear systems in equilibrium
%%     \end{itemize}
%%   \end{block}
%% 
%% \end{frame}
%% 
%% \begin{frame}
%%   \frametitle{Economics}
%% 
%%   \begin{block}{\alertb{Agents:}}
%%     \begin{itemize}
%%     \item 
%%       \tc{red}{Complexity Economics:}
%%       
%%       Modelled individually; use inductive rules of thumb to make
%%       decisions; have incomplete information; are subject to errors
%%       and biases; learn to adapt over time
%% 
%%     \item \tc{red}{Traditional Economics:} 
%%       Modelled collectively; use
%%       complex deductive calculations to make decisions; have complete
%%       information; make no errors and have no biases; have no need for
%%       learning or adaptation (are already perfect)
%%     \end{itemize}
%%     
%%   \end{block}
%% 
%% \end{frame}
%% 
%% \begin{frame}
%%   \frametitle{Economics}
%% 
%%   \begin{block}{\alertb{Networks:}}
%%     \begin{itemize}
%%     \item 
%%       \tc{red}{Complexity Economics:}
%%       Explicitly model bi-lateral interactions between individual agents; networks of relationships change over time
%%     \item 
%%       \tc{red}{Traditional Economics:}
%%       Assume agents only interact indirectly through market mechanisms (e.g. auctions)
%%     \end{itemize}
%%   \end{block}
%%   
%% 
%% \end{frame}
%% 
%% \begin{frame}
%%   \frametitle{Economics}
%% 
%%   \begin{block}{\alertb{Emergence:}}
%%     \begin{itemize}
%%     \item
%%       \tc{red}{Complexity Economics:}
%%       No distinction between micro/macro economics; 
%%       macro patterns are emergent result of micro level behaviours and interactions
%%     \item
%%       \tc{red}{Traditional Economics:}
%%       Micro-and macroeconomics remain separate disciplines
%%     \end{itemize}
%%   \end{block}
%% 
%% \end{frame}
%% 
%% \begin{frame}
%%   \frametitle{Economics}
%% 
%%   \begin{block}{\alertb{Evolution:}}
%%     \begin{itemize}
%%     \item 
%%       \tc{red}{Complexity Economics:}
%%       
%%       The evolutionary process of differentiation, selection and
%%       amplification provides the system with novelty and is
%%       responsible for its growth in order and complexity
%% 
%%     \item 
%% 
%%       \tc{red}{Traditional Economics:}
%%   
%%       No mechanism for endogenously creating novelty, or growth in
%%       order and complexity
%% 
%%     \end{itemize}
%%     
%%   \end{block}
%%   
%% \end{frame}



\section{Modeling}

%% \begin{frame}
%%   \frametitle{Models}
%% 
%%   \begin{block}{\alertb{Nino Boccara} in \textit{Modeling Complex Systems}:}
%%     ``Finding the emergent global behavior of a large
%%     system of interacting agents using methods is usually
%%     hopeless, and researchers therefore must rely on
%%     computer-based models.''
%%   \end{block}
%% 
%%   \begin{block}{Focus is on (toy) dynamical systems models:}
%%     \begin{itemize}
%%     \item<+-> 
%%       Differential and difference equation models.
%%     \item<+-> 
%%       Cellular automata.
%%     \item<+-> 
%%       Past: Simulations done on simple 2-d lattics.
%%     \item<+-> 
%%       2000--: Simulations on complex networks.
%%     \end{itemize}
%%   \end{block}
%% 
%% \end{frame}

\begin{frame}
  \frametitle{}

  \begin{block}{Tools and techniques:}
    \begin{itemize}
    \item<2-> 
      Differential equations, difference equations, linear algebra,
      stochastic models.
    \item<3-> 
      Statistical techniques for comparisons and descriptions.
    \item<4-> 
      Methods from statistical mechanics and computer science.
    \item<5-> 
      Machine learning (but beware the black box).
    \item<6-> 
      Computer modeling, everything from
      \begin{itemize}
      \item 
        Artisanal toy models (e.g., using \wordwikilink{http://ccl.northwestern.edu/netlogo/}{Netlogo})
      \item 
        to kitchen sink models.
      \end{itemize}
    \end{itemize}
  \end{block}

  \begin{block}<7->{Key advance (to repeat):}
    \begin{itemize}
    \item<7-> Representation of \alertb{complex interaction patterns}
      as \alertb{complex networks}.
    \item<8-> The driver: \alertr{Massive amounts of Data}
    \end{itemize}
  \end{block}

\end{frame}

\begin{frame}
  \frametitle{Rather silly but great example of real science:}

  \wordwikilink{http://www.sciencemag.org/content/early/2010/11/10/science.1195421}{``How Cats Lap: {W}ater Uptake by \textit{{F}elis catus}''}\\
  Reis et al., \textit{Science}, 2010.

  \medskip

  \includegraphics[width=\textwidth]{12cats_graphic-popup-v2.jpg}

  Amusing interview \wordwikilink{http://video.nytimes.com/video/2010/11/11/science/1248069317702/how-cats-lap.html}{here}

\end{frame}

\section{Statistical\ Mechanics}

\begin{frame}
  
  \showtarotcards{0.30}{
    overview,
    manifesto,
    scaling,
    power-law-size-distributions,
    random-walks,
    variable-transformation,
    rich-get-richer,
    efficient-language,
    data-poor,
    pouring-data,
    emergence-of-structure,
    emergence-of-destruction,
    emergence-of-thinking,
    emergence-of-stories,
    mechanics-statistical,
  }
\end{frame}

\begin{frame}
  
  \showtarotcards{0.30}{
    overview,
    manifesto,
    scaling,
    power-law-size-distributions,
    random-walks,
    variable-transformation,
    rich-get-richer,
    efficient-language,
    data-poor,
    pouring-data,
    emergence-of-structure,
    emergence-of-destruction,
    emergence-of-thinking,
    emergence-of-stories,
    mechanics-statistical,
    thresholds-of-percolation,
  }
\end{frame}


\begin{frame}
%%  \frametitle{Models}

  \begin{block}{}
    \begin{itemize}
    \item<1-> 
      Statistical Mechanics is \alertr{``a science of collective behavior.''}
    \item<1->
      \alertb{Simple rules} give rise to \alertb{collective phenomena.}
    \end{itemize}
  \end{block}

  \begin{block}{Percolation:}
    \includegraphics[width=\textwidth]{loandhi.png}
    
    Snared from 
    \wordwikilink{http://wwwf.imperial.ac.uk/~mgastner/percolation/percolation.html}{Michael Gastner's page on percolation}
  \end{block}

\end{frame}


\begin{frame}
  %% \frametitle{The stuff of statistical mechanics:}
  \small

    \begin{block}{\normalsize \wordwikilink{http://en.wikipedia.org/wiki/Ising_model}{The Ising Model} of a ferromagnet:}
      \small
      \begin{columns}
        \column{0.02\textwidth}
        \column{0.27\textwidth}
        \includegraphics[width=\textwidth]{2013-01-21ising-model-sketch-crop-tp-1.pdf}\\
        \includegraphics[width=\textwidth]{2013-01-22ising-model-sketch-crop-tp-1.pdf}
        \column{0.69\textwidth}
        \begin{itemize}
        \item<+-> 
          Each atom is assumed to have a local spin 
          that can be \alertr{up} or \alertr{down}: $ S_i = \pm 1$.
        \item<+-> 
          Spins are assumed to be arranged on a lattice.
        \item<+-> 
          In isolation, spins like to align with each other.
        \item<+-> 
          Increasing temperature breaks these alignments.
        \item<+-> 
          The \wordwikilink{http://en.wikipedia.org/wiki/Drosophila}{drosophila} of statistical mechanics.
        \item<+-> 
          Criticality: Power-law distributions at critical points.
        \end{itemize}
        \column{0.02\textwidth}
      \end{columns}
    \end{block}
    \begin{block}<+->{Example 2-d Ising model simulation:}
      \small
      \wordwikilink{http://dtjohnson.net/projects/ising}{http://dtjohnson.net/projects/ising}
    \end{block}

\end{frame}


\begin{frame}
  \frametitle{Phase diagrams}

  \includegraphics[height=0.7\textheight]{Phase-diag.png}

  \medskip

  Qualitatively distinct macro states.

\end{frame}

\begin{frame}
  \frametitle{Phase diagrams}

  Oscillons, bacteria, traffic, snowflakes, ...

  \medskip
  
  \includegraphics[width=0.45\textwidth]{osc1.pdf}
  \includegraphics[width=0.45\textwidth]{osc2.pdf}

  \medskip

  Umbanhowar et al., \textit{Nature}, 1996\cite{umbanhowar1996a}

\end{frame}

\begin{frame}
  \frametitle{Phase diagrams}

  \begin{center}
    \includegraphics[height=0.8\textheight]{oscillon_phasediagram.pdf}
  \end{center}

\end{frame}

\begin{frame}
  \frametitle{Phase diagrams}

  \begin{center}
    \includegraphics[height=0.7\textheight]{bacteria_phase_diagram.jpg}
  \end{center}

  {\tiny $W_0$ = initial wetness, $S_0$ = initial nutrient supply\\
    \url{http://math.arizona.edu/~lega/HydroBact.html}}

\end{frame}

\begin{frame}
  \frametitle{Ising model}

  \begin{block}{Analytic issues:}
    \begin{itemize}
    \item<1-> 1-d: simple (Ising \& Lenz, 1925)
    \item<2-> 2-d: hard (Onsager, 1944)
    \item<3-> 3-d: extremely hard...
    \item<4-> 4-d and up: simple.
    \end{itemize}
  \end{block}

\end{frame}

\begin{frame}
  \frametitle{Statistics}

  \begin{block}{Historical surprise:}
  \begin{itemize}
  \item<+-> Origins of Statistical Mechanics are in the studies of people...
    (Maxwell and co.)
  \item<+-> Now physicists are using their techniques to study everything else
    including people...
  \item<+-> See Philip Ball's ``Critical Mass''\cite{ball2004a}
  \end{itemize}
  \end{block}

  \begin{block}<+->{Beyond Statistical Mechanics:}
    \begin{itemize}
    \item<+->
      Analytic approaches have their limits,
      especially in evolutionary, algorithm-rich systems.
    \item<+->
      Algorithmic methods and simulation techniques will continue
      to rise in importance.
    \end{itemize}
  \end{block}

\end{frame}

\section{Nutshell}

\begin{frame}
  \frametitle{Nutshell}

  \begin{block}{}
  \begin{itemize}
  \item<+->
    The central concepts \alertg{Complexity} and
    \alertg{Emergence} are \alertb{not precisely defined}.
  \item<+->
    There is \alertb{no general theory of Complex Systems}.
  \item<+->
    But the problems exist...\\
    \qquad \qquad Complex (Adaptive) Systems abound...
  \item<+->
    Framing from the Manifesto: Science's focus is moving to Complex Systems 
    \alertg{because it finally can}.
  \item<+->
    \alertb{We use whatever tools we need.}
%%  \item<+->
%%    Reality is theoretically weak.
  \item<+->
    Science $\simeq$ Describe + Explain.
  \end{itemize}
  \end{block}

\end{frame}

