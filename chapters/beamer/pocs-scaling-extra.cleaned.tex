  \textbf{First returns}



  \textbf{Normality}

derive using characteristic functions

The normal (or Gaussian) distribution is
$$
P(x) 
=
\frac{1}{\sqrt{2\pi}\sigma}
e^{-\frac{()}{2\sigma^2}}
$$


  \textbf{Normality}

Random walk/binomial derivation



  \textbf{Normality}

Scaling of random walks


  \textbf{Normality}

Deriving the normal distribution using 
the renormalization group approach.

Hierarchical approach


  \textbf{Normality}

Fourier transforms and
characteristic equations


  \textbf{Normality}

Cumulants


  \textbf{RG}

The RG approach works when


  \textbf{Normality}

The RG approach fails when


  \textbf{Normality}

Anderson's 1972 paper `More is Different'
does not suggest RG should work 
very generally!

%% opposite to what sornette says


  \textbf{Percolation}

Percolation example

Real space renormalization


  \textbf{Normality}

Normalization:

%%%%%%%%%%%%%%%%%%%
% levy distributions
% stable distributions
%%%%%%%%%%%%%%%%%%%



%%%%%%%%%%%%%%%%%%%
% mechanisms and reasons for power law distributions
%%%%%%%%%%%%%%%%%%%


  \textbf{Mechanisms}

Generally, heavy-tailed distributions of
part size appears when there is a low cost in combining or growing
of parts.



  \textbf{Mechanisms}

Beware of PLIPLO...

Power law in, power law out.


  \textbf{Mechanisms}

Change of variable:


  \textbf{Mechanisms}

Distribution of the gravitational force:

$$ P(F) \propto F^{-5/2} $$



  \textbf{Mechanisms}

\textbf{Random copying with innovation}

Fundamental type of growth or agglomeration.

First described by G. Udny Yule in 
``a mathematical theory of evolution, based on the 
conclusions of Dr. J. C. Willis, F.R.S.
\textit{Phil. Trans. B.}, \textbf{Vol.} 213, pp. 21--, 1924.

``On a class of skew distribution functions''\\
Herbert Simon, \textit{Biometrika}, \textbf{Vol.} 42, pp. 425--440, 1955.


  \textbf{Measuring exponents}

River network data from my thesis

Show problems with measurement




  \textbf{Measuring exponents}

$\chi^2$ test


  \textbf{Measuring exponents}

Kolmogorov-Smirnov test



  \textbf{Measuring exponents}

Number of sexual partners



  \textbf{Statistical models}

$$y = c (x + x_0)^\alpha$$

$$y = c x^\alpha e^{-x/x_c}$$


  \textbf{Measuring exponents}

Finite size scaling


  \textbf{Measuring exponents}

Breaks in scaling



  \textbf{Scaling in nature}

Stefan-Boltzmann relation for radiated energy:

$$\diff{E}{t} = \sigma S T^4$$



  \textbf{Physics}

Critical Phenomena in physics



  \textbf{Examples}

Random walks in one dimension:\\
typical displacement $x \propto$ (time)$^{1/2}$


(Essence of the central limit theorem)


%%%%%%%%%%%%%%%%%%%%%%%%%%%%%%%%%%%%%
% scaling in math
%%%%%%%%%%%%%%%%%%%%%%%%%%%%%%%%%%%%%


  \textbf{Random walks}

Scaling from randomness:



$$\tavg{x} \propto t^{1/2}$$

%%%%%%%%%%%%%%%%%%%%%%%%%%%%%%%%%%%%%
% mechanisms and explanations
%%%%%%%%%%%%%%%%%%%%%%%%%%%%%%%%%%%%%


  \textbf{Dimensional Analysis}

Dimensional analysis:


  \textbf{Cleverness}

Turbulence

Atomic bomb


  \textbf{$\pi$ theorem}

Buckingham's $\pi$ theorem (1914).


%%%%%%%%%%%%%%%%%%%%%%%%%%%%%%%%%%%%%
% fractals
%%%%%%%%%%%%%%%%%%%%%%%%%%%%%%%%%%%%%



  \textbf{Geometry}

Okay, okay, okay: `fractals.'



  \textbf{Benford's law}


  \textbf{Turbulence}

% http://www.efluids.com/efluids/gallery/gallery_pages/jet_cfd_page.jsp

\includegraphics[width=0.48\textwidth]{jet_cfd.jpg}
\raisebox{17ex}{
    \parbox{.48\textwidth}{
      \small
      Big whirls have little whirls\\
      That heed on their velocity, \\
      And little whirls have littler whirls \\
      And so on to viscosity.

      \hfill---Lewis Richardson
      
      ??? laws
    }
  }

%% [A play on Jonathan Swift's "Great fleas have little fleas upon their backs to bite 'em, And little fleas have lesser fleas, and so ad infinitum." (1733)].


  \textbf{Examples}

Scaling in elementary laws of physics:

Inverse-square law of gravity
and Coulomb's law: 
$$F \propto \frac{m_1 m_2}{r^{2}}
\quad \mbox{and} \quad
F \propto \frac{q_1 q_2}{r^{2}}$$

$\Rightarrow$ Force is diminished by expansion of
space away from source.  

(The square is $d-1=3-1=2$, the dimension of
a sphere's surface.)


% check out http://complexsystems.lri.fr/Main/tiki-index.php


%% \newslide{Measuring exponents}
%% 
%% Ordinary least squares (OLS) linear regression
%% 
%% \newslide{Measuring exponents}
%% 
%% Check residuals obey a normal distribution
%% 
%% \newslide{Measuring exponents}
%% 
%% Define lognormal
%% 
%% 
%% \newslide{Measuring exponents}
%% 
%% Major Axis (RMA) linear regression
%% 
%% 
%% aka Standardized Major Axis and 
%% originally Reduced Major Axis.
%% 
%% citation
%% 
%% Raynor?


%% give explanation for white/grey matter scalin


  \textbf{Allometry}

White matter = connections\\
Grey matter = computation units

\begin{center}
\includegraphics[height=0.7\textheight]{zhang2000fig2.jpg}  
\end{center}

\hfill{\tiny(from Zhang \& Sejnowski, PNAS, 2000)}


  \textbf{Examples}

Explanation

\hfill{\tiny(from Zhang \& Sejnowski, PNAS, 2000)}



  \textbf{Examples}

1/f noise



\begin{comment}

percolation

\end{comment}



  \textbf{Normality}

  derive using characteristic functions

  The normal (or Gaussian) distribution is
  $$
  P(x) 
  =
  \frac{1}{\sqrt{2\pi}\sigma}
  e^{-\frac{()}{2\sigma^2}}
  $$


  \textbf{Normality}

  Random walk/binomial derivation



  \textbf{Normality}

  Scaling of random walks


  \textbf{Normality}

  Deriving the normal distribution using 
  the renormalization group approach.

  Hierarchical approach


  \textbf{Normality}

  Fourier transforms and
  characteristic equations


  \textbf{Normality}

  Cumulants


  \textbf{RG}

  The RG approach works when


  \textbf{Normality}

  The RG approach fails when


  \textbf{Normality}

  Anderson's 1972 paper `More is Different'
  does not suggest RG should work 
  very generally!

  %% opposite to what sornette says


  \textbf{Percolation}

  Percolation example

  Real space renormalization


  \textbf{Normality}

  Normalization:

%%%%%%%%%%%%%%%%%%%
  %% levy distributions
  %% stable distributions
%%%%%%%%%%%%%%%%%%%



%%%%%%%%%%%%%%%%%%%
  %% mechanisms and reasons for power law distributions
%%%%%%%%%%%%%%%%%%%


  \textbf{Mechanisms}

  Generally, heavy-tailed distributions of
  part size appears when there is a low cost in combining or growing
  of parts.



  \textbf{Mechanisms}

  Beware of PLIPLO...

  Power law in, power law out.


  \textbf{Mechanisms}

  Change of variable:


  \textbf{Mechanisms}

  Distribution of the gravitational force:

  $$ P(F) \propto F^{-5/2} $$



  \textbf{Mechanisms}

  \textbf{Random copying with innovation}

  Fundamental type of growth or agglomeration.

  First described by G. Udny Yule in 
  ``a mathematical theory of evolution, based on the 
  conclusions of Dr. J. C. Willis, F.R.S.
  \textit{Phil. Trans. B.}, \textbf{Vol.} 213, pp. 21--, 1924.

  ``On a class of skew distribution functions''\\
  Herbert Simon, \textit{Biometrika}, \textbf{Vol.} 42, pp. 425--440, 1955.


  \textbf{Measuring exponents}

  River network data from my thesis

  Show problems with measurement




  \textbf{Measuring exponents}

  $\chi^2$ test


  \textbf{Measuring exponents}

  Kolmogorov-Smirnov test



  \textbf{Measuring exponents}

  Number of sexual partners



  \textbf{Statistical models}

  $$y = c (x + x_0)^\alpha$$

  $$y = c x^\alpha e^{-x/x_c}$$


  \textbf{Measuring exponents}

  Finite size scaling


  \textbf{Measuring exponents}

  Breaks in scaling



  \textbf{Scaling in nature}

  Stefan-Boltzmann relation for radiated energy:

  $$\diff{E}{t} = \sigma S T^4$$



  \textbf{Physics}

  Critical Phenomena in physics



  \textbf{Examples}

  Random walks in one dimension:\\
  typical displacement $x \propto$ (time)$^{1/2}$


  (Essence of the central limit theorem)


%%%%%%%%%%%%%%%%%%%%%%%%%%%%%%%%%%%%%
  %% scaling in math
%%%%%%%%%%%%%%%%%%%%%%%%%%%%%%%%%%%%%


  \textbf{Random walks}

  Scaling from randomness:



  $$\tavg{x} \propto t^{1/2}$$

%%%%%%%%%%%%%%%%%%%%%%%%%%%%%%%%%%%%%
  %% mechanisms and explanations
%%%%%%%%%%%%%%%%%%%%%%%%%%%%%%%%%%%%%


  \textbf{Dimensional Analysis}

  Dimensional analysis:


  \textbf{Cleverness}

  Turbulence

  Atomic bomb


  \textbf{$\pi$ theorem}

  Buckingham's $\pi$ theorem (1914).


%%%%%%%%%%%%%%%%%%%%%%%%%%%%%%%%%%%%%
  %% fractals
%%%%%%%%%%%%%%%%%%%%%%%%%%%%%%%%%%%%%



  \textbf{Geometry}

  Okay, okay, okay: `fractals.'



  \textbf{Benford's law}


  \textbf{Turbulence}

  %% http://www.efluids.com/efluids/gallery/gallery_pages/jet_cfd_page.jsp

  \includegraphics[width=0.48\textwidth]{jet_cfd.jpg}
  \raisebox{17ex}{
    \parbox{.48\textwidth}{
      \small
      Big whirls have little whirls\\
      That heed on their velocity, \\
      And little whirls have littler whirls \\
      And so on to viscosity.

      \hfill---Lewis Richardson
      
      ??? laws
    }
  }

  %% [A play on Jonathan Swift's "Great fleas have little fleas upon their backs to bite 'em, And little fleas have lesser fleas, and so ad infinitum." (1733)].


  

Published online 7 July 2006 | Nature | doi:10.1038/news060703-17

News
Van Gogh painted perfect turbulence

The disturbed artist intuited the deep forms of fluid flow.
  %% http://www.nature.com/news/2006/060703/full/060703-17.html



  \textbf{Examples}

  Scaling in elementary laws of physics:

  Inverse-square law of gravity
  and Coulomb's law: 
  $$F \propto \frac{m_1 m_2}{r^{2}}
  \quad \mbox{and} \quad
  F \propto \frac{q_1 q_2}{r^{2}}$$

  $\Rightarrow$ Force is diminished by expansion of
  space away from source.  

  (The square is $d-1=3-1=2$, the dimension of
  a sphere's surface.)


  %% check out http://complexsystems.lri.fr/Main/tiki-index.php



  \textbf{Measuring exponents}
  %% 
  %% Ordinary least squares (OLS) linear regression
  %% 

  \textbf{Measuring exponents}
  %% 
  %% Check residuals obey a normal distribution
  %% 

  \textbf{Measuring exponents}
  %% 
  %% Define lognormal
  %% 
  %% 

  \textbf{Measuring exponents}
  %% 
  %% Major Axis (RMA) linear regression
  %% 
  %% 
  %% aka Standardized Major Axis and 
  %% originally Reduced Major Axis.
  %% 
  %% citation
  %% 
  %% Raynor?


  %% give explanation for white/grey matter scalin


  \textbf{Allometry}

  White matter = connections\\
  Grey matter = computation units

  \begin{center}
    \includegraphics[height=0.7\textheight]{zhang2000fig2.jpg}  
  \end{center}

  \hfill{\tiny(from Zhang \& Sejnowski, PNAS, 2000)}


  \textbf{Examples}

  Explanation

  \hfill{\tiny(from Zhang \& Sejnowski, PNAS, 2000)}



  \textbf{Examples}

  1/f noise


%% \subsection{Geometry}


%% \subsection{Solutions to Equations}

%% Percolation


