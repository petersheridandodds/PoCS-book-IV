\changelecturelogo{.18}{sliderule_small}

\begin{frame}
  \frametitle{Stories---The Fraction Assassin:}

  \includegraphics[width=0.9\textwidth]{2014-08-26pocs-sketch-metabolism-assassin_shadow.png}
\end{frame}


\section{Metabolism\ and\ Truthicide}

%% "In the criminal justice system, the people are represented by two
%% separate yet equally important groups: the police who investigate
%% crime and the district attorneys who prosecute the offenders. These
%% are their stories."

\begin{frame}
  
  \begin{block}<+->{Law and Order, Special Science Edition: Truthicide Department}
    \uncover<+->{``In the scientific integrity system known as peer review,}
    \uncover<+->{the people are represented by two highly overlapping yet equally important groups:}
    \uncover<+->{the independent scientists who review papers} 
    \uncover<+->{and the scientists who punish those who publish garbage.}
    \uncover<+->{This is one of their stories.''}
  \end{block}

\end{frame}

\begin{frame}
  \frametitle{Animal power}

  \begin{block}{Fundamental biological and ecological constraint:}
    $$
    \alertb{\boxed{P = c\, M^{\, \alpha}}}
    $$
    $$P = \mbox{basal metabolic rate}$$
    $$M = \mbox{organismal body mass}$$

    %%  Mammals, poikilotherms, birds, trees,\\
    %%  bacteria, rocks,\ldots\\
    \begin{columns}
      \column{0.025\textwidth}
      \column{0.3\textwidth}
      \includegraphics[width=\textwidth]{shrew.jpg}
      \column{0.025\textwidth}
      \column{0.3\textwidth}
      \begin{overprint}
        \onslide<1| handout:0| trans:0>
        \onslide<2-| handout:1| trans:1>
        \includegraphics[width=\textwidth]{shrew-elephant-tp-5.pdf}
      \end{overprint}
      \column{0.025\textwidth}
      \column{0.3\textwidth}
      \includegraphics[width=\textwidth]{250px-Re-exposure_of_elephant_-_lahugala_park1.jpg}
      \column{0.025\textwidth}
    \end{columns}
  \end{block}

\end{frame}

\begin{frame}
  \frametitle{$P=c\, M^{\, \alpha}$}

  \begin{block}{}
  Prefactor 
  \alertb{$c$} 
  depends on 
  \alertg{body plan} and \alertg{body temperature}:
  
  \bigskip

  \begin{overprint}
    \onslide<1 | handout:0| trans:0>
    \onslide<2-| handout:1| trans:1>
    \settablerowcolours
    \begin{tabular}{rl}
      Birds & 39--\tempc{41}\\
      Eutherian Mammals & 36--\tempc{38}\\
      Marsupials & 34--\tempc{36}\\
      Monotremes & 30--\tempc{31}
    \end{tabular}

    \bigskip

    \begin{center}
      \includegraphics[height=0.3\textheight]{Platypus.jpg}
      \quad
      \includegraphics[height=0.3\textheight]{Echidna.jpg}
    \end{center}
  \end{overprint}
  \end{block}

%% \newslide{Metabolism}
%% 
%% But it's a controversial business:
%% 
%% Some claim $\alpha=3/4$, others claim $2/3$,
%% and yet others say it's neither.
%% 
%% \mbox{}\hfill long story...

\end{frame}

\begin{frame}
  \frametitle{What one might expect:}

  \begin{block}<1->{$\alpha=2/3$ \visible<2->{because \ldots}}
    \begin{itemize}
    \item<2-> 
      Dimensional analysis suggests\\ 
      an energy balance surface law:\\
      $$ P \propto S \propto V^{2/3} \propto M^{\, 2/3}$$
    \item<3->
      Assumes isometric scaling (not quite the spherical cow).
    \item<4->
      \alertg{Lognormal fluctuations:}\\
      \quad Gaussian fluctuations in $\log{P}$ around $\log{cM^\alpha}$.
    \item<5->
      \wordwikilink{http://en.wikipedia.org/wiki/Stefan-Boltzmann_law}{Stefan-Boltzmann law} for radiated energy:\\
      $$\diff{E}{t} = \sigma \varepsilon S T^4 \propto S $$ 
    \end{itemize}

  \end{block}

\end{frame}

\begin{frame}
  \frametitle{The prevailing belief of the Church of Quarterology:}

  \bigskip

  \begin{block}{}
  $$ \boxed{\alpha = 3/4} $$

  \bigskip

  $$ P \propto M^{\, 3/4} $$

  \bigskip

  \begin{center}
    \visible<2->{\alertg{Huh?}}
  \end{center}
  \end{block}

\end{frame}


\begin{frame}
  \frametitle{The prevailing belief of the Church of Quarterology:}


  \begin{block}{Most obvious concern:}
    $$
    3/4 - 2/3 = 1/12
    $$
    \begin{itemize}
    \item<+->
      An exponent higher than 2/3 points suggests
      a fundamental inefficiency in biology.
    \item<+->
      Organisms must somehow be running `hotter' than
      they need to balance heat loss.
    \end{itemize}
  \end{block}

\end{frame}

\begin{frame}
  \frametitle{Related putative scalings:}

  \begin{block}{Wait! There's more!:}
  \begin{itemize}
  \item 
  number of capillaries $\propto M^{\, 3/4}$
  \item 
  time to reproductive maturity $\propto M^{\, 1/4}$
  \item 
  heart rate $\propto M^{\, -1/4}$
  \item 
  cross-sectional area of aorta $\propto M^{\, 3/4}$
  \item 
  population density $\propto M^{\, -3/4}$
  \end{itemize}
  \end{block}

\end{frame}


\begin{frame}
  \frametitle{The great `law' of heartbeats:}

  \begin{block}<1->{Assuming:}
    \begin{itemize}
    \item 
      Average lifespan $\propto M^{\beta}$
    \item 
      Average heart rate $\propto M^{-\beta}$
    \item 
      Irrelevant but perhaps $\beta = 1/4$.
    \end{itemize}
  \end{block}

  \begin{block}<2->{Then:}
    \begin{itemize}
    \item<3->
      \begin{center}
        Average number of heart beats in a lifespan
        \visible<4->{
          $\simeq \ \mbox{(Average lifespan)} \times  \mbox{(Average heart rate)} $\\
        }
        \visible<5->{
          $ \propto M^{\beta - \beta} $\\
        }
        \visible<6->{
          $ \alertg{\propto M^{0}} $
        }
      \end{center}
    \item<7-> 
      \visible<7->{
        Number of heartbeats per life time is independent of organism size!
    }
    \item<8-> 
      \visible<8->{
        $\approx$ 1.5 billion....
    }
    \end{itemize}
  \end{block}

\end{frame}

\begin{frame}

  \showtarotcards{0.35}{
%%  \dealnewtarotcard{0.35}{
    john-dory,
    overview,
    complex-networks,
    random-networks,
    scale-free-networks,
    small-world-networks,
    theory-six-degrees,
    landscapes-of-forking-paths,
    networks-of-blood,
    trees-of-reality,
    orders-of-streams,
    laws-of-branching,
    unknown-mechanism,
    law-of-optimal-forks,
    church-of-quarterology,
%%    truthicide,
}

\end{frame}

\section{Death\ by\ fractions}

\begin{frame}
  \frametitle{A theory is born:}

  1840's: Sarrus and Rameaux\cite{sarrus1838a} first suggested $\alpha=2/3$.

  \begin{center}
    \includegraphics[height=0.3\textheight]{Smokestack_small.jpg}
  \end{center}

\end{frame}

\begin{frame}
  \frametitle{A theory grows:}

  1883: Rubner\cite{rubner1883a} found $\alpha \simeq 2/3$.

  \begin{center}
    \includegraphics[height=0.3\textheight]{bio_dogs.jpg}
  \end{center}

\end{frame}

\begin{frame}
  \frametitle{Theory meets a different `truth':}

  1930's: Brody, Benedict study mammals.\cite{brody1945a}

  Found $\alpha \simeq 0.73 $ (standard).

  \begin{center}
    \includegraphics[height=0.3\textheight]{haybale_small}
  \end{center}

\end{frame}

\begin{frame}
  \frametitle{Our hero faces a shadowy cabal:}

  \begin{center}
    \includegraphics[height=0.3\textheight]{sliderule_small}
  \end{center}

  \begin{itemize}
  \item
  1932: Kleiber analyzed 13 mammals.\cite{kleiber1932a}
  \item 
  Found $\alpha=0.76$ and suggested $\alpha=3/4$.
  \item<+-> 
    Scaling law of Metabolism became
    known as 
    \wordwikilink{http://en.wikipedia.org/wiki/Kleiber's\_law}{Kleiber's Law} (2011 Wikipedia entry is embarrassing).
  \item<+-> 
    1961 book: ``The Fire of Life.  An Introduction to Animal Energetics''.\cite{kleiber1961a}
  \end{itemize}

\end{frame}

\begin{frame}
  \frametitle{When a cult becomes a religion:}

  1950/1960: Hemmingsen\cite{hemmingsen1950a,hemmingsen1960a}\\
  Extension to unicellular organisms.\\
  $\alpha=3/4$ assumed true.

  \begin{center}
    \includegraphics[height=0.3\textheight]{slimemold}
  \end{center}

\end{frame}

\begin{frame}
  \frametitle{Quarterology spreads throughout the land:}

  \small
  \begin{block}{The Cabal assassinates 2/3-scaling:}
    \begin{itemize}
    \item 
      1964: Troon, Scotland.
    \item 
      3rd Symposium on Energy Metabolism.
    \item 
      $\alpha=3/4$ made official \ldots \hfill \visible<2->{\ldots 29 to zip.}
    \end{itemize}

    \begin{center}
      %% \includegraphics[height=0.3\textheight]{poll_card06_front}
      \includegraphics[height=0.24\textheight]{ballotbox}
    \end{center}

    \begin{itemize}
    \item<3->
      But the Cabal slipped up by \alertr{publishing the conference proceedings} \ldots
    \item<4->
      ``Energy Metabolism; Proceedings of the 3rd symposium
      held at Troon, Scotland, May 1964,''
      Ed. Sir Kenneth Blaxter\cite{blaxter1965a}
    \end{itemize}
  \end{block}

\end{frame}

\begin{frame}

  \showtarotcards{0.35}{
%%  \dealnewtarotcard{0.35}{
    john-dory,
    overview,
    complex-networks,
    random-networks,
    scale-free-networks,
    small-world-networks,
    theory-six-degrees,
    landscapes-of-forking-paths,
    networks-of-blood,
    trees-of-reality,
    orders-of-streams,
    laws-of-branching,
    unknown-mechanism,
    law-of-optimal-forks,
    church-of-quarterology,
    truthicide,
}

\end{frame}


\begin{frame}
  \frametitle{An unsolved truthicide:}

  \begin{block}<+->{So many questions ...}
    \begin{itemize}
    \item<+-> 
      Did the truth kill a theory?  Or did a theory kill the truth?
    \item<+-> 
      Or was the truth killed by just a lone, lowly hypothesis?
    \item<+-> 
      Does this go all the way to the top?\\ 
      \uncover<+->{To the National Academies of Science?}
    \item<+-> 
      Is 2/3-scaling really dead?
    \item<+-> 
      Could 2/3-scaling have faked its own death?
    \item<+-> 
      What kind of people would vote on scientific facts?
    \end{itemize}
  \end{block}

\end{frame}

\begin{frame}
  \frametitle{Modern Quarterology, Post Truthicide}

  \begin{block}{}
  \begin{itemize}
  \item 3/4 is held by many to be the one true exponent.
  \end{itemize}

  \bigskip 

  \begin{columns}
    \column{0.05\textwidth}
    \column{0.2\textwidth}
    \includegraphics[width=\textwidth]{inthebeatofaheart.jpg}
    \column{0.05\textwidth}
    \column{0.7\textwidth}
    \textit{In the Beat of a Heart: Life, Energy, and the Unity of Nature}---by John Whitfield
  \end{columns}

  \bigskip 

  \begin{itemize}
  \item<2-> 
    \visible<2->{But: much controversy ...}
  \item<3-> 
    \visible<3->{See `Re-examination of the ``3/4-law'' of metabolism'\\
      by the Heretical Unbelievers Dodds, Rothman, and Weitz\cite{dodds2001d}}, and ensuing madness...
  \end{itemize}
  \end{block}

%% todo 
  %% add riisgard

  %% ???
  %% \newslide{Even in physics \ldots}
  %%
  %%   Physics:
  %%
  %%  Ratio of electron mass and charge.
  %% 


\end{frame}


\begin{frame}
  \frametitle{Some data on metabolic rates}

  \begin{block}{}
  \begin{columns}
    \column{0.67\textwidth}
    \includegraphics[width=\textwidth]{figheusner391_v2}    
    \column{0.33\textwidth}
    \begin{itemize}
    \item<1->
      Heusner's data (1991)\cite{heusner1991a}
    \item<1-> 391 Mammals
    \item<1-> \alertb{blue line}: 2/3
    \item<1-> \alertg{red line}: 3/4.
    \item<1-> ($B = P$)
    \end{itemize}
  \end{columns}
  \end{block}
  
\end{frame}


\begin{frame}
  \frametitle{Some data on metabolic rates}

  \begin{block}{}
  \begin{columns}
    \column{0.67\textwidth}
    \includegraphics[width=\textwidth]{figbennettbirds398_v2_noname}
    \column{0.33\textwidth}
    \begin{itemize}
    \item<1->
      Bennett and Harvey's data (1987)\cite{bennett1987a}
    \item<1-> 398 birds
    \item<1-> \alertb{blue line}: 2/3
    \item<1-> \alertg{red line}: 3/4.
    \item<1-> ($B = P$)
    \end{itemize}
  \end{columns}

  \begin{itemize}
  \item 
    Passerine vs. non-passerine issue...
  \end{itemize}

  \end{block}

\end{frame}

\section{Measuring\ allometric\ exponents}

\begin{frame}
  \frametitle{Linear regression}

  \begin{block}{Important:}
    \begin{itemize}
    \item<1-> 
      Ordinary Least Squares (OLS) Linear regression 
      is only appropriate for analyzing
      a dataset $\{(x_i,y_i)\}$
      when we know the $x_i$ are measured without error.
    \item<2-> 
      Here we assume that measurements of mass $M$
      have less error than measurements of metabolic rate $B$.
    \item<3-> 
      Linear regression assumes Gaussian errors.
    \end{itemize}
  \end{block}

\end{frame}



\begin{frame}
  \frametitle{Measuring exponents}

  \begin{block}{More on regression:}
    If \alertb{(a)} we don't know what the errors of either variable are,\\
    \medskip
    \visible<2->{or \alertb{(b)} no variable can be considered independent,\\}
    \medskip
    \visible<3->{then we need to use\\ Standardized Major Axis Linear Regression.\cite{samuelson1942a,rayner1985a}\\}
    \medskip
    \visible<4->{(aka Reduced Major Axis = RMA.)}
  \end{block}
\end{frame}

\begin{frame}
  \frametitle{Measuring exponents}

  \begin{block}{For Standardized Major Axis Linear Regression:}
    $$
    \mbox{slope}_{\mbox{\tiny SMA}}
    =
    \frac{
      \mbox{standard deviation of $y$ data}
    }
    {
      \mbox{standard deviation of $x$ data}
    }
    $$
    \begin{itemize}
    \item<+-> 
      Very simple!
    \item<+-> 
      Minimization of sum of areas of triangles induced by
      vertical and horizontal residuals with best fit line.
    \item<+-> 
      The only linear regression that is 
      \wordwikilink{http://en.wikipedia.org/wiki/Total\_least\_squares\#Scale\_invariant\_methods}{Scale invariant}.
    \item<+-> 
      Attributed to Nobel Laureate economist
      \wordwikilink{http://en.wikipedia.org/wiki/Paul\_Samuelson}{Paul Samuelson},\cite{samuelson1942a}
      but discovered independently
      by others.
    \item<+-> 
      \#somuchwin
    \end{itemize}
  \end{block}

\end{frame}

\begin{frame}
  \frametitle{Measuring exponents}

  \begin{block}{Relationship to ordinary least squares regression is simple:}
  \begin{eqnarray*}
    \mbox{slope}_{\mbox{\tiny SMA}} & = & r^{-1} \times 
    \mbox{slope}_{\mbox{\tiny OLS $y$ on $x$}} \\ 
    & = & r \times \mbox{slope}_{\mbox{\tiny OLS $x$ on $y$}}
  \end{eqnarray*} \inv
  where $r$ = standard correlation coefficient:
  $$
  r = \frac{
    \sum_{i=1}^{n} (x_i - \bar{x})(y_i - \bar{y})
  }
  {
    \sqrt{\sum_{i=1}^{n} (x_i - \bar{x})^2}
    \sqrt{\sum_{i=1}^{n} (y_i - \bar{y})^2}
  }
  $$
  \end{block}

\end{frame}

\begin{frame}
  \frametitle{Heusner's data, 1991 (391 Mammals)}

  \begin{block}{}
  \begin{center}
    \settablerowcolours
     \begin{tabular}{c|c|c}
       range of $M$ & $N$ & $\nalpha$ \\ \hline
       & & \\
       $\leq 0.1$ kg    &  167          & $0.678 \pm 0.038$ \\
       & & \\
       $\leq 1$  kg     &  276          & $0.662 \pm 0.032$ \\
       & & \\
       $\leq 10$ kg     &  357          & $0.668 \pm 0.019$ \\
       & & \\
       $\leq 25$ kg     &  366          & $0.669 \pm 0.018$ \\
       & & \\
       $\leq 35$ kg     &  371          & $0.675 \pm 0.018$ \\
       & & \\
       $\leq 350$ kg    &  389          & $0.706 \pm 0.016$ \\
       & & \\
       $\leq 3670$ kg   &  391          & $0.710 \pm 0.021$ \\
     \end{tabular}
  \end{center}
  \end{block}

\end{frame}

\begin{frame}
  \frametitle{Bennett and Harvey, 1987 (398 birds)}

  \begin{block}{}
  \begin{center}
    \settablerowcolours
    \begin{tabular}{c|c|c}
      $\Mmax$   & $N$ & $\nalpha$  \\ \hline
      & & \\
      $\leq 0.032$   & 162 & $0.636 \pm 0.103$ \\
      & & \\
      $\leq  0.1$    & 236 & $0.602 \pm 0.060$ \\
      & & \\
      $\leq 0.32$    & 290 & $0.607 \pm 0.039$ \\
      & & \\
      $\leq    1$    & 334 & $0.652 \pm 0.030$ \\
      & & \\
      $\leq  3.2$    & 371 & $0.655 \pm 0.023$ \\
      & & \\
      $\leq   10$    & 391 & $0.664 \pm 0.020$ \\
      & & \\
      $\leq   32$    & 396 & $0.665 \pm 0.019$ \\
      & & \\
      $\leq  100$    & 398 & $0.664 \pm 0.019$ \\
    \end{tabular}
  \end{center}
  \end{block}

\end{frame}

\begin{frame}
  \frametitle{Fluctuations---Things look normal...}

  \begin{block}{}
  \begin{center}
    \includegraphics[height=0.65\textheight]{figmetascalingfn2}

    \begin{itemize}
    \item 
     $  P(B\, |M) = 1/M^{2/3} f(B/M^{2/3})$\\
    \item 
      Use a Kolmogorov-Smirnov test.
    \end{itemize}
  \end{center}
  \end{block}

\end{frame}

\begin{frame}
  \frametitle{Hypothesis testing}

  \begin{block}{}
  \alertb{Test to see if $\alpha'$ is consistent
    with our data $\{(M_i,B_i)\}$:}

  $$ H_0: \alpha = \alpha' \mbox{\ and\ } H_1: \alpha \ne \alpha'.$$

  \begin{itemize}
  \item<2->
    Assume each $\mathbf{B}_i$ (now a random variable)
    is normally distributed
    about $\alpha' \log_{10} M_i + \log_{10} c$.
  \item<3-> 
    Follows that the measured $\alpha$ for
    one realization obeys
    a $t$ distribution with $N-2$ degrees of freedom.
  \item<4-> 
    Calculate a $p$-value: probability that the measured
    $\alpha$ is as least as different to our hypothesized
    $\alpha'$ as we observe.
  \item<5-> 
    See, for example, DeGroot and Scherish, ``Probability and Statistics.''\cite{degroot1975a}
  \end{itemize}
  \end{block}

\end{frame}

\begin{frame}
  \frametitle{Revisiting the past---mammals}

  \begin{block}{Full mass range:}
    \settablerowcolours
    \begin{tabular}{ccccccc}
             & $N$ & $\nalpha$ & $p_{2/3}$ & $p_{3/4}$ \\ \hline
              & & & & & & \\
      Kleiber             &  13 & 0.738 & $<10^{-6}$ & 0.11 \\
              & & & & & & \\
      Brody               &  35 & 0.718 & $<10^{-4}$ & $<10^{-2}$ \\
              & & & & & & \\
      Heusner             & 391 & 0.710 & $<10^{-6}$ & $<10^{-5}$ \\
              & & & & & & \\
      Bennett   & 398 & 0.664 & 0.69 &     $<10^{-15}$ \\
      and Harvey & & & & & & \\
    \end{tabular}
  \end{block}

\end{frame}

\begin{frame}
  \frametitle{Revisiting the past---mammals}

  \small
  \begin{block}{$M \leq 10$ kg:}
    \settablerowcolours
  \begin{tabular}{ccccccc}
    & $N$ & $\nalpha$ & $p_{2/3}$ & $p_{3/4}$ \\ \hline
    & & & & & & \\
    Kleiber        &   5 & 0.667 &  0.99 & 0.088 \\
    & & & & & & \\
    Brody          &  26 & 0.709   & $<10^{-3}$ & $<10^{-3}$ \\
    & & & & & & \\
    Heusner        & 357 & 0.668 &   0.91 &   $<10^{-15}$ \\
  \end{tabular}
  \end{block}

  \begin{block}{$M \geq 10$ kg:}
      \settablerowcolours
  \begin{tabular}{ccccccc}
    & $N$ & $\nalpha$ &  $p_{2/3}$ & $p_{3/4}$ \\ \hline
    & & & & & & \\
    Kleiber        &   8 & 0.754 & $<10^{-4}$ & 0.66 \\
    & & & & & & \\
    Brody          &   9 & 0.760 & $<10^{-3}$ & 0.56 \\
    & & & & & & \\
    Heusner        &  34 & 0.877 & $< 10^{-12}$ & $<10^{-7}$ \\
  \end{tabular}
  \end{block}


%% \newslide{Correlations in residuals}
%%
%%  \vspace{5mm}
%%  \begin{center}
%%    \includegraphics[angle=90,height=0.9\textheight]{figheusner_stats}
%%  \end{center}


% The individual plots correspond
% to the following ranges: 
% (a) $M<3.2$ kg, (b) $M<10$ kg, (c) $M<32$ kg, (d) all mammals.
% For all mass ranges considered, $p_{2/3}>0.05$ and $p_{3/4} \ll 10^{-4}$.  

\end{frame}

\begin{frame}
  \frametitle{Analysis of residuals}

  \begin{block}{}
  \begin{enumerate}
  \item<1->
    Presume an exponent of your choice: 2/3 or 3/4.
  \item<2->
    Fit the prefactor ($\log_{10} c$) and then
    examine the residuals:
    $$ 
    r_i = \log_{10} B_i - (\alpha' \log_{10} M_i - \log_{10} c).
    $$
  \item<3->
    $H_0$: residuals are uncorrelated\\
    $H_1$: residuals are correlated.
  \item<4->
    Measure the correlations in the residuals
    and compute a $p$-value.
  \end{enumerate}
  \end{block}

\end{frame}

\begin{frame}
  \frametitle{Analysis of residuals}

  \begin{block}{}
  We use the spiffing 
  \wordwikilink{http://en.wikipedia.org/wiki/Spearman's_rank_correlation_coefficient}{Spearman Rank-Order Correlation Coefficient}
  \end{block}

  \smallskip

  \begin{block}<2->{Basic idea:}
    \begin{itemize}
    \item<2->
      Given $\{(x_i,y_i)\}$, rank 
      the $\{x_i\}$ and $\{y_i\}$ separately from
      smallest to largest.  Call these ranks $R_i$ and $S_i$.
    \item<3->
      Now calculate correlation coefficient for ranks, $r_s$:
    \item<4->
      {\small
        $$ r_s 
        = 
        \frac{
          \sum_{i=1}^{n} (R_i - \bar{R})(S_i - \bar{S})
        }
        {
          \sqrt{\sum_{i=1}^{n} (R_i - \bar{R})^2}
          \sqrt{\sum_{i=1}^{n} (S_i - \bar{S})^2}
        }
        $$
      }
    \item<5->
      Perfect correlation: $x_i$'s and $y_i$'s both
      increase monotonically.
    \end{itemize}
  \end{block}

\end{frame}


\begin{frame}
  \frametitle{Analysis of residuals}

  \begin{block}{We assume all rank orderings are equally likely:}
    \begin{itemize}
    \item<2->
      $r_s$ is distributed according to a
      \wordwikilink{http://en.wikipedia.org/wiki/Student's_t-distribution}{Student's $t$-distribution}
      with $N-2$ degrees of freedom.
    \item<3->
      Excellent feature: Non-parametric---real distribution
      of $x$'s and $y$'s doesn't matter.
    \item<4->
      Bonus: works for non-linear monotonic relationships as well.
    \item<5->
      See \wordwikilink{http://www.nr.com/}{Numerical Recipes in C/Fortran} 
      which contains many good things.\cite{press1992a}
    \end{itemize}
  \end{block}

\end{frame}

\begin{frame}
  \frametitle{Analysis of residuals---mammals}

  \begin{block}{}
    \begin{columns}
      \column{0.7\textwidth}
      \includegraphics[width=\textwidth]{figmammals_pv_log_noname}
      \column{0.3\textwidth}
      \begin{enumerate}
      \item[(a)] 
        $M<3.2$ kg,
      \item[(b)]
        $M<10$ kg,
      \item[(c)] 
        $M<32$ kg, 
      \item[(d)] 
        all mammals.
      \end{enumerate}
    \end{columns}
  \end{block}

% The individual plots correspond
% to the following ranges: 
% (a) $M<3.2$ kg, (b) $M<10$ kg, (c) $M<32$ kg, (d) all mammals.
% For all mass ranges considered, $p_{2/3}>0.05$ and $p_{3/4} \ll 10^{-4}$.  


\end{frame}

\begin{frame}
  \frametitle{Analysis of residuals---birds}

  \begin{block}{}
  \begin{columns}
    \column{0.7\textwidth}
    \includegraphics[width=\textwidth]{figbirds_pv_log_noname}
    \column{0.3\textwidth}
    \begin{enumerate}
    \item[(a)] 
      $M<0.1$ kg,
    \item[(b)]
      $M<1$ kg,
    \item[(c)] 
      $M<10$ kg, 
    \item[(d)] 
      all birds.
    \end{enumerate}
  \end{columns}
  \end{block}

% The individual plots correspond
% to the following ranges: 
% (a) $M<3.2$ kg, (b) $M<10$ kg, (c) $M<32$ kg, (d) all mammals.
% For all mass ranges considered, $p_{2/3}>0.05$ and $p_{3/4} \ll 10^{-4}$.  

\end{frame}


%% \newslide{\small Fluctuations}
%% 
%%   \vspace{5mm}
%%   \begin{center}
%%     \begin{tabular}{l|ccccc}
%%       & range  & $\sigma$ &    $p$  & $\sigma^{\ast}$ & $p^{\ast}$ \\ \hline
%%       & & & & & \\
%%       mammals & $M < 1$  & 0.153    & 0.232   & 0.120 & 0.307 \\
%%       & & & & & \\
%%       mammals & $M < 10$ & 0.153    & 0.093   & 0.120 & 0.135 \\
%%       & & & & & \\
%%       birds   & all    & 0.132    & 0.032   & 0.115 & 0.573 \\ 
%%     \end{tabular}
%% 
%%     $$  P(B\, |M) = 1/M^{2/3} f(B/M^{2/3})$$
%%   \end{center}

% \end{comment}

\begin{frame}
%%  \frametitle

  \begin{block}{Other approaches to measuring exponents:}
  \begin{itemize}
  \item 
    Clauset, Shalizi, Newman: ``Power-law distributions in empirical data''\cite{clauset2009b}\\
    SIAM Review, 2009.
  \item 
    See Clauset's page on \wordwikilink{http://tuvalu.santafe.edu/~aaronc/powerlaws/}{measuring power law exponents}
    (code, other goodies).
  \end{itemize}
  \end{block}

\end{frame}


\begin{frame}
  \frametitle{Impure scaling?:}

  \begin{block}{}
  \begin{itemize}
  \item<1-> 
    So: The exponent $\alpha = 2/3$ works for all birds and
    mammals up to 10--30 kg
  \item<2-> 
    For mammals $>$ 10--30 kg, maybe we have a new scaling regime
  \item<3-> 
    Possible connection?: 
    Economos (1983)---limb length break in scaling around 20 kg\cite{economos1983a}
  \item<4-> 
    But see later: non-isometric growth leads to \alertb{lower} metabolic scaling.  Oops.
  \end{itemize}
  \end{block}

\end{frame}

\begin{frame}
  \frametitle{The widening gyre:}

  %% summarize empirical work since ours

  \begin{block}{Now we're really confused (empirically):}
    \begin{itemize}
    \item<+-> 
      White and Seymour, 2005: unhappy with large herbivore measurements\cite{white2005a}.
      Pro $2/3$: Find $\alpha \simeq 0.686 \pm 0.014$.
%%    \item<2-> 
%%      White ... \cite{white2007a}
    \item<+-> 
      Glazier, BioScience (2006)\cite{glazier2006a}:
      ``The 3/4-Power Law Is Not Universal: {E}volution of Isometric, 
      Ontogenetic Metabolic Scaling in Pelagic Animals.''
    \item<+-> 
      Glazier, Biol. Rev. (2005)\cite{glazier2005a}:
      ``Beyond the 3/4-power law': variation in the intra- 
      and interspecific scaling of metabolic rate in animals.''
    \item<+-> 
      Savage et al., PLoS Biology (2008)\cite{savage2008a}
      ``Sizing up allometric scaling theory''
      Pro $3/4$: problems claimed to be finite-size scaling.
%%    \item<6-> 
%%      Mori et al.\cite{mori2010a}
%%    \item<7-> 
%%      Add more.
    \end{itemize}
  \end{block}

\end{frame}


\section{River\ networks}

\begin{frame}[label=]
%%  \frametitle{Basic basin quantities: $a$, $l$, $L_\parallel$,
%%  $L_\perp$:}

  \frametitle{Somehow, optimal river networks are connected:}

  \begin{block}{}
  \begin{columns}
    \column<1->{0.6\textwidth}
    \includegraphics[width=\textwidth]{basin.pdf}
    \column<1->{0.4\textwidth}
    \begin{itemize}
    \item<1-> 
      \alertg{$a$} = drainage basin area
    \item<1-> 
      \alertg{$\msl$} = length of longest (main) stream 
%      (which may be fractal)
    \item<1-> 
      \alertg{$L = L_\parallel$} = longitudinal length of basin
%    \item<1-> 
%      \alertg{$L = L_\perp$} = width of basin
    \end{itemize}
  \end{columns}
  \end{block}

\end{frame}

\begin{frame}
  \frametitle{Mysterious allometric scaling in river networks}

  \begin{block}{}
  \begin{itemize}
  \item<1->
    1957: J. T. Hack\cite{hack1957a}\\
    ``Studies of Longitudinal Stream Profiles in Virginia and Maryland''
    $$
    \msl \sim a^{\, h}
    $$
    $$ h \sim 0.6 $$
  \item<2->
    Anomalous scaling: we would expect $h=1/2$...
  \item<3->
    Subsequent studies: $0.5 \lesssim h \lesssim 0.6 $
  \item<4->
    Another quest to find \alertb{universality/god}...
  \item<5->
    \alertg{A catch:} studies done on small scales.
  \end{itemize}
  \end{block}

\end{frame}

\begin{frame}
  \frametitle{Large-scale networks:}

  \begin{block}{(1992) Montgomery and Dietrich\cite{montgomery1992a}:}
    \bigskip
    \includegraphics[width=0.9\textwidth]{montgomery1992a_fig2.pdf}
    \begin{itemize}
    \item<1->
      \alertb{Composite data set:} includes everything from unchanneled valleys up
      to world's largest rivers.
    \item<1->
      Estimated fit: $$ L \simeq 1.78 a^{\, 0.49} $$
    \item<1->
      Mixture of basin and main stream lengths.
    \end{itemize}    
  \end{block}

\end{frame}

\begin{frame}
  \frametitle{World's largest rivers only:}

  \begin{block}{}
  \begin{center}
    \includegraphics[width=0.5\textwidth]{figworldhack002.pdf}
  \end{center}

  \begin{itemize}
  \item<1-> 
    Data from Leopold (1994)\cite{leopold1994a,dodds2000a}
  \item<1-> 
    Estimate of Hack exponent: \alertg{$ h = 0.50 \pm 0.06$}
  \end{itemize}
  \end{block}

\end{frame}

\section{Earlier\ theories}

%% empirical studies of both examples
%% trend away from simple story
%% and then 



\begin{frame}
  \frametitle{Earlier theories (1973--):}

  \begin{block}{Building on the surface area idea:}
    \begin{itemize}
    \item<+-> 
      McMahon (70's, 80's): Elastic Similarity\cite{mcmahon1973a,mcmahon1983a}
    \item<+-> 
      Idea is that organismal shapes scale allometrically
      with 1/4 powers {\small (like trees...)}
    \item<+-> 
      Disastrously, cites Hemmingsen~\cite{hemmingsen1960a} for
      surface area data.
    \item<+-> 
      Appears to be true for ungulate legs \ldots\cite{mcmahon1975b}
    \item<+-> 
      Metabolism and shape never properly connected.
    \end{itemize}
  \end{block}

\end{frame}

\framedisplaypaper{mcmahon1973a}{2}{fig3}

\begin{frame}

  \begin{center}
    \includegraphics[width=\textwidth,height=0.5\textheight,keepaspectratio]{hemmingsen196a_pp42-43.pdf}    
  \end{center}

  \begin{itemize}
  \item 
    Hemmingsen's ``fit'' is for a 2/3 power, notes possible 10 kg transition.~\cite{hemmsingsen1960a}
  \item 
    p 46: ``The energy metabolism thus definitely varies interspecifically
    over similar wide weight ranges with a higher power of the body
    weight than the body surface.'' 
  \end{itemize}

\end{frame}

\begin{frame}
  \frametitle{Earlier theories (1977):}

  \begin{block}<1->{Building on the surface area idea...}
    \begin{itemize}
    \item<1->
      Blum (1977)\cite{blum1977a} speculates on four-dimensional biology:
      $$P \propto M^{\, (d-1)/d}$$
    \item<2->
      $d=3$ gives $\alpha = 2/3$
    \item<3->
      $d=4$ gives $\alpha = 3/4$
    \item<4-> 
      So we need another dimension...
    \item<5->
      Obviously, a bit silly\ldots\cite{speakman1990a}
    \end{itemize}
  \end{block}

\end{frame}


\begin{frame}
  \frametitle{Nutrient delivering networks:}

  \begin{block}{}
    \begin{itemize}
    \item<1-> 
      1960's: Rashevsky considers
      blood networks and finds a $2/3$ scaling.
    \item<2-> 
      1997: West \etal\cite{west1997a} use a network story to find $3/4$ scaling.
    \end{itemize}
    \begin{overprint}
      \onslide<1 | handout:0| trans:0>
      \onslide<2-| handout:1| trans:1>
      \begin{center}
        \includegraphics[width=0.75\textwidth]{west97figure1}
      \end{center}
    \end{overprint}
    
  \end{block}

\end{frame}

\changelecturelogo{.18}{cheat-to-win-bracelet-the-onion-tp-1.pdf}


\changelecturelogo{.18}{sliderule_small}

\begin{frame}
  \frametitle{Nutrient delivering networks:}

      \begin{block}{West et al.'s assumptions:}
        \begin{enumerate}
        \item<1-> 
          hierarchical network
        \item<2-> 
          capillaries (delivery units) invariant
        \item<3-> 
          network impedance is minimized via evolution
        \end{enumerate}
      \end{block}

      \begin{block}<4->{Claims:}
        \begin{itemize}
        \item<4->
          $P \propto M^{\, 3/4}$
        \item<5-> 
          networks are fractal
        \item<6-> 
          quarter powers everywhere
        \end{itemize}
      \end{block}

\end{frame}

\begin{frame}
  \frametitle{Impedance measures:}

  \begin{block}{}
    \begin{itemize}
    \item 
      Poiseuille flow (outer branches):
      $$ Z = \frac{8\mu}{\pi} \sum_{k=0}^N \frac{\ell_k}{r_k^4 N_k} $$
    \item 
      Pulsatile flow (main branches):
      $$ Z \propto \sum_{k=0}^N \frac{h_k^{1/2}}{r_k^{5/2} N_k} $$
    \item 
      Wheel out Lagrange multipliers \ldots
    \item 
      Poiseuille gives $P \propto M^1$ with a logarithmic correction.
    \item 
      Pulsatile calculation explodes into flames.
    \end{itemize}
  \end{block}

\end{frame}

\begin{frame}
  \frametitle{Not so fast \ldots}

  \begin{block}{Actually, model shows:}
    \begin{itemize}
    \item<1->
      $P \propto M^{\, 3/4}$ does not follow for pulsatile flow
    \item<2->
      networks are not necessarily fractal.
    \end{itemize}
  \end{block}

  \begin{block}<3->{Do find:}
    \begin{itemize}
    \item<3->
      Murray's cube law (1927) for outer branches:\cite{murray1927a}
      $$r_0^{3} = r_1^{3} + r_2^{3}$$
    \item<4->
      Impedance is distributed evenly.
    \item<5->
      Can still assume networks are fractal.
    \end{itemize}
  \end{block}

\end{frame}


\begin{frame}
  \small
  \frametitle{Connecting network structure to $\alpha$}

  \begin{block}{}
  \begin{enumerate}
  \item<1->
    Ratios of network parameters:
    $$ 
    R_n = \frac{n_{k+1}}{n_k},
    \
    R_\ell = \frac{\ell_{k+1}}{\ell_k},
    \
    R_r = \frac{r_{k+1}}{r_k}
    $$
%  \item<2->
%    Can estimate $V_{\textrm{blood} \propto (\beta^2 \gamma)^{-N} \propto M$\\
%    \visible<3->{(also problematic due to  prefactor issues)}
  \item<2->
    Number of capillaries $\propto P \propto M^\alpha$.
    \visible<3->{
      $$ \Rightarrow \ \ \boxed{\alertb{\alpha = -\frac{\ln{R_n}}{\ln{R_r^2 R_\ell}}}}$$
      (also problematic due to  prefactor issues)
    }
  \end{enumerate}
  \end{block}

  \bigskip

  \begin{block}<4->{Obliviously soldiering on, we could assert:}

    \begin{columns}
      \column{0.6\textwidth}
      \begin{itemize}
      \item<4->
        area-preservingness: $R_r = R_n^{-1/2}$ 
      \item<4->
        space-fillingness: $R_\ell = R_n^{-1/3}$ 
      \end{itemize}
      \column{0.4\textwidth}
      $$\Rightarrow \alertg{\alpha=3/4}$$
    \end{columns}
  \end{block}

\end{frame}


%% \begin{frame}
%%   \frametitle{Data from real networks: much variation}
%% 
%% %% ignore vein data
%%   {\small
%%     \begin{center}
%%     \begin{tabular}{c|ccc|c}
%%       Network & $R_n$ & $R_r^{-1}$ & $R_l^{-1}$ & $\alpha$ \\
%%       \hline
%%       & & & & \\
%%       West \etal\      & --   & --   & --   &  0.75   \\
%%       & & & & \\
%% %      Dimensional analysis               & --   & --   & --   & 1/2  & 1/2  & 2/3   \\
%%       rat (PAT)           & 2.76 & 1.58 & 1.60 & 0.73  \\
%%       & & & & \\
%%       cat (PAT)           & 3.67 & 1.71 & 1.78 & 0.79  \\
%%       {\small (Turcotte \etal\cite{turcotte1998a})}
%%       & & & & \\
%%       & & & & \\
%%       dog (PAT)           & 3.69 & 1.67 & 1.52 & 0.90  \\
%% %%      dog (PVT)           & 3.76 & 1.70 & 1.56 & 0.40 & 0.34 & 0.88  \\
%%       & & & & \\
%%       pig (LCX)           & 3.57 & 1.89 & 2.20 & 0.62  \\
%%       pig (RCA)           & 3.50 & 1.81 & 2.12 & 0.65  \\
%%       pig (LAD)           & 3.51 & 1.84 & 2.02 & 0.65  \\
%% %%      pig (TV)            & 3.05 & 1.73 & 1.61 & 0.49 & 0.43 & 0.71  \\
%% %%      pig (SV)            & 3.37 & 1.72 & 1.77 & 0.45 & 0.47 & 0.73  \\ 
%%       & & & & \\
%%       human (PAT)         & 3.03 & 1.60 & 1.49 & 0.83  \\
%% %%      human (PVT)         & 3.30 & 1.68 & 1.68 & 0.43 & 0.43 & 0.77  \\
%%       human (PAT)         & 3.36 & 1.56 & 1.49 &  0.94  \\
%% %%      human (PVT)         & 3.33 & 1.58 & 1.50 & 0.38 & 0.34 & 0.91  \\ 
%%     \end{tabular}
%%     \end{center}
%%     }
%% 
%% \end{frame}

\begin{frame}

  \begin{block}{Data from real networks:}
%% ignore vein data
  {\small
    \begin{center}
      \settablerowcolours
    \begin{tabular}{c|ccc|cc|c}
      Network & $R_n$ & $R_r$ & $R_\ell$ & $-\frac{\ln{R_r}}{\ln{R_n}}$ & 
      $-\frac{\ln R_\ell}{\ln{R_n}}$  & $\alpha$ \\
      \hline
      & & & & & & \\
      West \etal\      & --   & --   & --   & 1/2  & 1/3  & 3/4   \\
      & & & & & & \\
%      Dimensional analysis               & --   & --   & --   & 1/2  & 1/2  & 2/3   \\
      rat (PAT)           & 2.76 & 1.58 & 1.60 & 0.45 & 0.46 & 0.73  \\
      & & & & & & \\
      cat (PAT)           & 3.67 & 1.71 & 1.78 & 0.41 & 0.44 & 0.79  \\
      {\tiny (Turcotte \etal\cite{turcotte1998a})}
      & & & & & & \\
      & & & & & & \\
      dog (PAT)           & 3.69 & 1.67 & 1.52 & 0.39 & 0.32 & 0.90  \\
%%      dog (PVT)           & 3.76 & 1.70 & 1.56 & 0.40 & 0.34 & 0.88  \\
      & & & & & & \\
      pig (LCX)           & 3.57 & 1.89 & 2.20 & 0.50 & 0.62 & 0.62  \\
      pig (RCA)           & 3.50 & 1.81 & 2.12 & 0.47 & 0.60 & 0.65  \\
      pig (LAD)           & 3.51 & 1.84 & 2.02 & 0.49 & 0.56 & 0.65  \\
%%      pig (TV)            & 3.05 & 1.73 & 1.61 & 0.49 & 0.43 & 0.71  \\
%%      pig (SV)            & 3.37 & 1.72 & 1.77 & 0.45 & 0.47 & 0.73  \\ 
      & & & & & & \\
      human (PAT)         & 3.03 & 1.60 & 1.49 & 0.42 & 0.36 & 0.83  \\
%%      human (PVT)         & 3.30 & 1.68 & 1.68 & 0.43 & 0.43 & 0.77  \\
      human (PAT)         & 3.36 & 1.56 & 1.49 & 0.37 & 0.33 & 0.94  \\
%%      human (PVT)         & 3.33 & 1.58 & 1.50 & 0.38 & 0.34 & 0.91  \\ 
    \end{tabular}
    \end{center}
    }
  \end{block}

\end{frame}



\begin{frame}
  
  \begin{block}{Some people understand it's truly a disaster:}

    \displayamazonbook{lane2005a}

    \bigskip

    \alertdg{``As so often happens in science, the apparently solid foundations
      of a field turned to rubble on closer inspection.''}
  \end{block}

\end{frame}

\begin{frame}
  \frametitle{Let's never talk about this again:}

  \displaypaper{west1999a}{2}

  \begin{itemize}
  \item<+-> 
    No networks: Scaling argument for energy exchange area $a$.
  \item<+-> 
    Distinguish between biological and physical length scales
    (distance between mitochondria versus cell radius).
  \item<+-> 
    Buckingham $\pi$ action.~\cite{buckingham1914a}
  \item<+-> 
    Arrive at $a \propto M^{D/D+1}$ and $\ell \propto M^{1/D}$.
  \item<+-> 
    New disaster: after going on about fractality of
    $a$, then state $v \propto a \ell$ in general.
  \end{itemize}
\end{frame}

\begin{frame}
  \frametitle{Really, quite confused:}

  \begin{block}{Whole 2004 issue of Functional Ecology addresses the problem:}
  \begin{itemize}
  \item<+->
     J. Kozlowski, M. Konrzewski. ``Is West, Brown and Enquist's model of allometric scaling mathematically correct and biologically relevant?'' Functional Ecology 18: 283--9, 2004.\cite{kozlowski2004a}
   \item<+->
     J. H. Brown, G. B. West, and B. J. Enquist.
     ``Yes, West, Brown and Enquist's model of allometric scaling is both mathematically correct and biologically relevant.''
     Functional Ecology 19: 735--738, 2005.\cite{brown2005a}
   \item<+->
     J. Kozlowski, M. Konarzewski. ``West, Brown and Enquist's model of allometric scaling again: the same questions remain.'' Functional Ecology 19: 739--743, 2005.
  \end{itemize}
  \end{block}

\end{frame}

\begin{frame}
  \frametitle{Simple supply networks:}

  \begin{block}{}
  \begin{columns}
    \column{0.65\textwidth}
    \includegraphics[width=\textwidth]{banavar1999a_fig1.pdf}
    \column{0.35\textwidth}
    \begin{itemize}
    \item 
      Banavar et al., Nature, (1999)\cite{banavar1999a}.
    \item 
      Flow rate argument.
    \item 
      Ignore impedance.
    \item 
      Very general attempt to find most efficient transportation networks.
    \end{itemize}
  \end{columns}
    
  \end{block}


\end{frame}


\begin{frame}
  \frametitle{Simple supply networks}

  \begin{block}{}
  \begin{itemize}
  \item<1->
    Banavar \etal\ find `most efficient' networks with
    $$P \propto M^{\, d/(d+1)}$$
  \item<2->
    ... but also find 
    $$V_{\textrm{network}} \propto M^{\, (d+1)/d}$$
  \item<3->
    $d=3$:
    $$V_{\textrm{blood}} \propto M^{\, 4/3}$$
  \item<4-> Consider a 3 g shrew with $V_{\textrm{blood}}$ = $0.1V_{\textrm{body}}$
  \item<5-> $\Rightarrow$
    3000 kg elephant with \alertb{$V_{\textrm{blood}}$ = $10V_{\textrm{body}}$}
  \end{itemize}
  \end{block}

\end{frame}

\begin{frame}
  \frametitle{Simple supply networks}

  \begin{block}{Such a pachyderm would be rather miserable:}
    \includegraphics[width=\textwidth]{elephant-blood-bank.pdf}
  \end{block}
  
\end{frame}

%% \begin{comment}

\section{Geometric\ argument}

\begin{frame}
  \frametitle{Geometric argument}

  \begin{block}{}
    \displaypaper{dodds2010a}{1}
  \begin{itemize}
  \item<2-> 
    Consider \alertg{one source} supplying \alertg{many sinks} in a \alertb{$d$-dim.} 
    volume
    in a \alertb{$D$-dim.} ambient space.
  \item<3->
    Assume \alertb{sinks are invariant}.
  \item<4->
    Assume sink density \alertg{$\rho = \rho(V)$}.
  \item<5-> 
    Assume some cap on flow speed of material.
  \item<6-> 
    See network as a bundle of virtual vessels:
    \begin{center}
      \begin{overprint}
        \onslide<1-6 | handout:0| trans:0>
        \onslide<7-| handout:1| trans:1>
        \begin{center}
          \includegraphics[angle=-90,width=0.8\textwidth]{virtualvessels4.pdf}
        \end{center}
      \end{overprint}
    \end{center}
  \end{itemize}
  \end{block}

\end{frame}

\begin{frame}
  \frametitle{Geometric argument}

  \begin{block}{}
  \begin{itemize}
  \item<1-> 
    \alertg{Q:} how does the number of sustainable
    sinks $N_{\textrm{sinks}}$
    scale with volume $V$ for the most efficient network design?
  \item<2-> 
    \alertg{Or:} what is the highest $\alpha$ for $N_{\textrm{sinks}} \propto V^{\alpha}$?
  \end{itemize}
  \end{block}

\end{frame}

\begin{frame}
  \frametitle{Geometric argument}

  \begin{block}{}
  \begin{itemize}
  \item<1-> Allometrically growing regions:
%  \item<2-> Family = a basic shape $\Omega$ indexed by volume $V$.
  \begin{center}
    \includegraphics[angle=-90,width=0.8\textwidth]{shapescaling}    
  \end{center}
  \bigskip
%  \item<3-> Orient shape to have dimensions $L_1 \times L_2 \times  ... \times L_d$
%  \item<4-> In 2-d,
%    $L_1 \propto A^{\gamma_1}$ and $L_2 \propto A^{\gamma_2}$
%    where $A$ = area.
  \item<1-> Have $d$ length scales which scale
    as 
    {
      $$
      \alertb{L_i} \propto \alertb{V}^{\alertb{\gamma_i}}
      \mbox{\ where $\gamma_1 + \gamma_2 + \ldots + \gamma_d = 1$.}
      $$
    }
  \item<1-> 
    For \alertg{isometric} growth, $\gamma_i = 1/d$.
  \item<1->
    For \alertg{allometric} growth, 
    we must have at least two of the $\{\gamma_i\}$ being different
%  \item<6-> For above example, width grows faster than
%    height: $\gamma_1 > \gamma_2$.
  \end{itemize}
  \end{block}

\end{frame}

\begin{frame}

  \begin{block}{Spherical cows and pancake cows:}

    \begin{itemize}
    \item<+-> 
      \alertg{Question:}
      How does the surface area $S_{\textrm{cow}}$ of our two types
      of cows scale with cow volume $V_{\textrm{cow}}$?
      \insertassignmentquestionsoft{3}{3}
    \item<+-> 
      \alertg{Question:}
      For general families of regions,
      how does surface area $S$ scale with 
      volume $V$?
      \insertassignmentquestionsoft{3}{3}
    \end{itemize}
  \end{block}

\end{frame}

\begin{frame}
  \frametitle{Geometric argument}

  \begin{block}{}
  \begin{itemize}
  \item<1-> Best and worst configurations (Banavar et al.)
    \begin{center}
      \includegraphics[angle=-90,width=0.8\textwidth]{efficientnetworks5.pdf}
    \end{center}
    \bigskip
  \item<2-> \alertg{Rather obviously:}\\
    $\min V_{\textrm{net}} \propto \sum$
    distances
    from source to sinks.

%  \item<3-> See what this means for sink density $\rho$ if sinks do not
%    change their feeding habits with overall size.
  \end{itemize}
  \end{block}

\end{frame}

\begin{frame}
  \frametitle{Minimal network volume:}

  \begin{block}{Real supply networks are close to optimal:}
  \includegraphics[width=\textwidth]{gastner2006a_fig1.pdf}
  \bigskip\\
  {\small 
    Gastner and Newman (2006):
    ``Shape and efficiency in spatial distribution networks''\cite{gastner2006a} 
  }
  \end{block}

\end{frame}

\begin{frame}
  \displaypaper{tero2010a}{2}

  \begin{center}
    \includegraphics[height=0.5\textheight]{slime_mold_1-660x501.jpg}
  \end{center}

  Urban deslime in action:  
  \wordwikilink{https://www.youtube.com/watch?v=GwKuFREOgmo}{https://www.youtube.com/watch?v=GwKuFREOgmo}

\end{frame}

\begin{frame}
  \frametitle{Minimal network volume:}

  \begin{block}{We add one more element:}
    \includegraphics[width=\textwidth]{shapes-virtualv-4c.pdf}
    \begin{itemize}
    \item Vessel cross-sectional area
      may vary with distance from the source.
    \item
      Flow rate increases as cross-sectional area decreases.
    \item e.g., a collection network may
      have vessels tapering as they approach
      the central sink.
    \item
      Find that vessel volume $v$ must scale
      with vessel length $\ell$ to affect overall
      system scalings.
    \end{itemize}
  \end{block}
\end{frame}

\begin{frame}
  \frametitle{Minimal network volume:}

  \begin{block}{Effecting scaling:}
    \includegraphics[width=\textwidth]{shapes-virtualv-4c.pdf}
    \begin{itemize}
    \item
      Consider vessel radius $r \propto (\ell+1)^{-\epsilon}$,
      tapering from $r=r_{\max}$ where $\epsilon \ge 0$.
    \item
      Gives
      $
      v \propto \ell^{1-2\epsilon}
      $ if $\epsilon < 1/2$
    \item
      Gives
      $
      v \propto 1 - \ell^{-(2\epsilon-1)} \rightarrow 1$ for large $\ell$
      if $\epsilon > 1/2$
    \item
      Previously, we looked at $\epsilon=0$ only.
    \end{itemize}
  \end{block}
\end{frame}

\begin{frame}
%%  \frametitle{Minimal network volume:}

  \begin{block}{Minimal network volume:}
    For $0 \le \epsilon < 1/2$, approximate network volume by integral over region:
    $$ 
    \alertb{\min V_{\textnormal{net}}}  \propto 
    \int_{\volume{V}} \alertb{\rho} \, ||\vec{x}||^{1-2\epsilon} \, \dee{\vec{x}} 
    $$
    %% \visible<2->{
    %%   $$
    %%   \rightarrow 
    %%   \rho V^{1+\gamma_{\max}}
    %%   \int_{\volume{c}} (c_1^{2} u_1^2 + \ldots + c_k^{2} u_k^2 )^{(1-2\epsilon)/2}
    %%   \dee{\vec{u}}
    %%   $$
    %% }
    \insertassignmentquestion{3}{3}
    \visible<2->{
      $$
      \propto
      \alert{ \rho V^{1+\gamma_{\max}(1-2\epsilon)} } 
      \
      \mbox{where}
      \
      \gamma_{\max} = \max_{i} \gamma_i.
      $$
    }
    \visible<3->{
      For $\epsilon > 1/2$, find simply that 
      $$
      \alertb{\min V_{\textnormal{net}}}  
      \propto 
      \rho V
      $$
    }
    \begin{itemize}
    \item<4->
      So if supply lines can taper fast enough and without
      limit, minimum network volume can be made negligible.
      % \item<5->
      %   \alert{The problem:} must eventually reach a limiting speed
      %   or size (e.g., blood velocity and cells).
    \end{itemize}
  \end{block}

\end{frame}

\begin{frame}
%%  \frametitle{Geometric argument}

  \begin{block}{For $0 \le \epsilon < 1/2$:}
    \begin{itemize}
    \item<1-> 
      $
      \boxed{\alert{
          \min V_{\textnormal{net}} 
          \propto
          \rho V^{1+\gamma_{\max}(1-2\epsilon)} 
        }}
      $
    \item<2-> 
      If scaling is \alertb{isometric}, we have $\gamma_{\max} = 1/d$:
      $$
      \min V_{\textnormal{net/iso}} 
      \propto
      \rho V^{1+(1-2\epsilon)/d}
      $$
    \item<3-> 
      If scaling is \alertb{allometric}, we have
      $\gamma_{\max} = \gamma_{\textnormal{allo}} > 1/d$:
      and 
      $$
      \min V_{\textnormal{net/allo}} 
      \propto
      \rho V^{1+(1-2\epsilon)\gamma_{\textnormal{allo}}}
      $$
    \item<4-> 
      Isometrically growing volumes 
      \alert{require less network volume} 
      than allometrically growing volumes:
      $$
      \frac{\min V_{\textnormal{net/iso}}}{\min V_{\textnormal{net/allo}}} \rightarrow 0 
      \mbox{\ as $V \rightarrow \infty$}
      $$
    \end{itemize}    
    
  \end{block}
\end{frame}

\begin{frame}
%%  \frametitle{Geometric argument}

  \begin{block}{For $\epsilon > 1/2$:}
    \begin{itemize}
    \item<1-> 
      $
      \boxed{\alert{
          \min V_{\textnormal{net}} 
          \propto
          \rho V
        }}
      $
    \item<2-> 
      Network volume scaling is now independent 
      of overall shape scaling.
    \end{itemize}
  \end{block}

  \medskip

  \begin{block}<3->{Limits to scaling}
    \begin{itemize}
    \item 
      Can argue that $\epsilon$ must effectively be 0
      for real networks over large enough scales.
    \item 
      Limit to how fast material can move,
      and how small material packages can be.
    \item 
      e.g., blood velocity and blood cell size.
    \end{itemize}
  \end{block}
\end{frame}


\subsection{Real\ networks}
%% \subsection{Blood\ networks}

\begin{frame}
  \frametitle{Blood networks}

  \begin{block}{}
    \begin{itemize}
    \item<1-> Velocity at capillaries and 
      aorta approximately constant across body size\cite{weinberg2006a}: 
      $\epsilon = 0$.
    \item<2-> \alert{Material costly} $\Rightarrow$ expect lower optimal bound of 
      $V_{\textnormal{net}} \propto \rho V^{(d+1)/d}$ to be followed closely.
    \item<3->
      For cardiovascular networks, \alert{$d=D=3$}.
    \item<4->
      Blood volume scales linearly with body
      volume\cite{stahl1967a}, $V_{\textnormal{net}} \propto V$.
    \item<5->
      Sink density must $\therefore$ decrease as volume increases:
      $$
      \alertb{\rho \propto V^{-1/d}}.
      $$
    \item<6->
      Density of suppliable sinks \alert{decreases} with organism size.
    \end{itemize}      
  \end{block}

\end{frame}


\begin{frame}
  \frametitle{Blood networks}

  \begin{block}{}
    \begin{itemize}
    \item<1-> Then $P$, the rate of overall energy 
      use in $\Omega$, can at most scale with volume as
      $$
      P \propto \rho V 
      \visible<2->{
        \propto \rho \, M
      }
      \visible<3->{
        \propto M^{\, (d-1)/d}
      }
      $$
    \item<4-> 
      For $d=3$ dimensional organisms, we have 
      $$\alertb{\boxed{ P \propto M^{\, 2/3}}}$$
    \item<5-> 
      Including other constraints may raise scaling exponent
      to a higher, less efficient value.
    \item<6->
      \alertb{Exciting bonus:} 
      Scaling obtained by the supply network story and the surface-area law
      \alert{only match} for isometrically growing shapes.\\
      \insertassignmentquestionsoft{3}{3}
    \end{itemize}    
  \end{block}

\end{frame}

\begin{frame}
  \frametitle{Recall:}

  \begin{block}{}
  \begin{itemize}
  \item<+-> 
    The exponent $\alpha = 2/3$ works for all birds and
    mammals up to 10--30 kg
  \item<+-> 
    For mammals $>$ 10--30 kg, maybe we have a new scaling regime
  \item<+-> 
    Economos: limb length break in scaling around 20 kg
  \item<+-> 
    White and Seymour, 2005: unhappy with large herbivore measurements.
    Find $\alpha \simeq 0.686 \pm 0.014$
  \end{itemize}
  \end{block}

\end{frame}

\begin{frame}
  \frametitle{Prefactor:}

  \begin{block}<1->{\wordwikilink{http://en.wikipedia.org/wiki/Stefan-Boltzmann_law}{Stefan-Boltzmann law:}}
    \begin{itemize}
    \item<+->
      $$\diff{E}{t} = \sigma S T^{4}$$
      where $S$ is surface and $T$ is temperature.
    \item<+-> 
      Very rough estimate of prefactor based on scaling
      of normal mammalian body temperature and surface
      area $S$:
      $$B \simeq 10^{5}M^{2/3} \mbox{erg/sec}.$$
    \item<+->
      Measured for $M \leq 10$ kg:
      $$B=2.57\times 10^5M^{2/3} \mbox{erg/sec}.$$
    \end{itemize}
  \end{block}

\end{frame}

%% \subsection{River\ networks}

\begin{frame}

  \begin{block}{River networks}
    \begin{itemize}
    \item<+-> 
      View river networks as collection networks.
    \item<+-> 
      Many sources and one sink.
    \item<+-> 
      $\epsilon$?
    \item<+-> 
      Assume $\rho$ is constant over time and $\epsilon=0$:
      $$V_{\textnormal{net}} \propto \rho V^{(d+1)/d} = \mbox{constant} \times V^{\, 3/2} $$
    \item<+-> 
      Network volume grows faster than
      basin `volume' (really area).
    \item<+-> 
      \alert{It's all okay:}\\ 
      Landscapes are $d$=2 surfaces living in $D$=3 dimensions.
    \item<+->
      Streams can grow not just in width but in depth...
    \item<+->
      If $\epsilon > 0$, $V_{\textnormal{net}}$ will grow more slowly
      but 3/2 appears to be confirmed from real data.
    \end{itemize}
  \end{block}

\end{frame}

%%%%%%%%%%%%%%%%%%%%%%%%%%%%%%%%%%%%%%%%
%% integrate below with above



\begin{frame}

  \begin{block}{Hack's law}
  \begin{itemize}
  \item<1-> Volume of water in river network can be calculated 
    by adding up basin areas
  \item<2-> Flows sum in such a way that 
    $$ V_{\textrm{net}} = \sum_{\mbox{\scriptsize all pixels}} a_{\mbox{\scriptsize pixel $i$}} $$
  \item<3-> Hack's law again:
    $$
    \ell \sim a^{\, h}
    $$
  \item<4-> 
    Can argue     
    $$ V_{\textrm{net}} \propto V_{\textrm{basin}}^{1+h} = a_{\textrm{basin}}^{1+h}$$
    where 
    $h$ is Hack's exponent.
  \item<5-> 
    $\therefore$ minimal volume calculations gives 
    $$
    \boxed{
      h=1/2
    }
    $$
  \end{itemize}
  \end{block}

\end{frame}

\begin{frame}
  \frametitle{Real data:}

  \begin{block}{}
  \begin{columns}
    \column{0.4\textwidth}
    \begin{itemize}
    \item<1-> Banavar et al.'s approach\cite{banavar1999a} is okay 
      because $\rho$ \alertb{really is constant}.
    \item<3-> \alertg{The irony:} shows optimal basins are isometric
    \item<4-> Optimal Hack's law: $\msl \sim a^{h}$ with
      $h=1/2$ 
    \item<5-> \visible<5->{(Zzzzz)}
    \end{itemize}
    \column{0.6\textwidth}
    \begin{overprint}
      \onslide<2-| handout:1| trans:1>
      \includegraphics[width=\textwidth]{banavar1999a_fig2.pdf}\\
      {\small From Banavar et al. (1999)\cite{banavar1999a}}
    \end{overprint}
  \end{columns}
  \end{block}

\end{frame}

\begin{frame}
  \frametitle{Even better---prefactors match up:}

  \begin{block}{}
  \begin{center}
    \includegraphics[width=0.8\textwidth]{figwatervolume02_noname.pdf}
  \end{center}
  \end{block}

\end{frame}

\begin{frame}
  \frametitle{The Cabal strikes back:}

  \begin{block}{}
    \begin{itemize}
    \item<1-> 
      Banavar et al., 2010, PNAS:\\
      ``A general basis for quarter-power scaling in animals.''\cite{banavar2010a}
    \item<2->
      ``It has been known for decades that the metabolic 
      rate of animals scales with body mass with an exponent 
      that is almost always $<1$, $>2/3$, and often very close to $3/4$.''
    \item<3->
      Cough, cough, cough, hack, wheeze, cough.
    \end{itemize}
  \end{block}

\end{frame}

\begin{frame}
  \frametitle{Stories---Darth Quarter:}

  \includegraphics[width=0.9\textwidth]{2014-08-26pocs-sketch-metabolism-darth-quarter_shadow.png}
\end{frame}


\begin{frame}
  
  \begin{block}{Some people 
      \wordwikilink{http://www.science20.com/mark_changizi/peter_sheridan_dodds_theoretical_biologys_buzzkill}{understand it's truly a disaster:}}
    \begin{center}
      \includegraphics[height=0.8\textheight]{2010-02-09changizi-buzzkill.jpg}
    \end{center}
  \end{block}

\end{frame}

\begin{frame}
  \frametitle{The unnecessary bafflement continues:}

%%  \displaypaper{price2012a}{1}
  
  \begin{block}{``Testing the metabolic theory of ecology''\cite{price2012a}}
    C. Price, J. S. Weitz, V. Savage, J. Stegen, A. Clarke, D. Coomes, P. S. Dodds, R. Etienne, A. Kerkhoff, K. McCulloh, K. Niklas, H. Olff, and N. Swenson\\
    Ecology Letters, \textbf{15}, 1465--1474, 2012.
  \end{block}

\end{frame}

\begin{frame}
\frametitle{Artisanal, handcrafted silliness:}

\begin{block}{
  ``Critical truths about power laws''\cite{stumpf2012a}\\
  Stumpf and Porter, Science, 2012
  }
  \includegraphics[width=0.5\textwidth]{stumpf2012a_fig1_top}
  \includegraphics[width=0.5\textwidth]{stumpf2012a_fig1_bottom}
  \begin{itemize}
  \item 
    Call generalization of Central Limit Theorem,
    stable distributions.  Also: PLIPLO action.
  \item<2->
    Summary: Wow.
  \end{itemize}
\end{block}

\end{frame}


\section{Conclusion}

\begin{frame}
  \small
  \frametitle{Conclusion}

  \begin{block}{}
  \begin{itemize}
  \item<+->
    Supply network story consistent with dimensional analysis.
  \item<+->
    Isometrically growing regions can be
    more efficiently supplied than allometrically growing ones.
  \item<+-> 
    Ambient and region dimensions matter\\ ($D=d$ versus $D>d$).
  \item<+-> 
    Deviations from optimal scaling suggest inefficiency
    (e.g., gravity for organisms, geological boundaries).
  \item<+-> 
    Actual details of branching networks not that important.
  \item<+->
    Exact nature of self-similarity varies.
  \item<+->
    2/3-scaling lives on, largely in hiding.
  \item<+->
    3/4-scaling?  Jury ruled a mistrial.
  \item<+->
    The truth will out.
    \uncover<+->{Maybe.}
  \end{itemize}
  \end{block}


\end{frame}

