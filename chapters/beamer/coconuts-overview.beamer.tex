%% fix the syllabus thing!

%% incorporate stanford large network collection

%% Erd\"{o}s-Bacon number
%% http://en.wikipedia.org/wiki/Erdős–Bacon_number


%%%%%%%%%%%%%%%%%%%%%%%%%%%%%%%%%%%%%
% very clear summary slide 
%%%%%%%%%%%%%%%%%%%%%%%%%%%%%%%%%%%%%

\begin{frame}

%%  \showtarotcards{0.25}{
  \dealnewtarotcard{0.35}{
    john-dory,
    overview,
    complex-networks,
%%    random-networks,
%%    scale-free-networks,
%%    small-world-networks,
%%    theory-six-degrees,
%%    landscapes-of-forking-paths,
%%    law-of-optimal-forks,
%%    laws-of-branching,
%%    networks-of-blood,
%%    orders-of-streams,
%%    trees-of-reality,
%%    unknown-mechanism,
}

\end{frame}


\begin{frame}
%%  \frametitle{Computational Story Lab:}

  \includegraphics[width=\textwidth]{2015-09-29collaborators-twitter.pdf}\\
  Funding: NSF, NASA, MITRE.
  
%%  \includegraphics[width=\textwidth]{collaborators-onehappybird-no-bliss1.jpg}
%%  \includegraphics[width=\textwidth]{collaborators-onehappybird-no-bliss1-tp-1.pdf}
%%  \includegraphics[width=\textwidth]{collaborators-onehappybird-no-bliss1-tp-3.pdf}
  %%  \includegraphics[width=\textwidth]{collaborators-onehappybird.jpg}
  %%   \includegraphics[width=\textwidth]{/Users/danforth/Local/collaborators-twitter.pdf}

\end{frame}

\changelecturelogo{0.18}{lightbulb-idea-calculus-circle-tp-1.pdf}

\begin{frame}
  %% display of papers by images

  \setlength{\fboxsep}{0pt}
  \setlength{\fboxrule}{0.5pt}

  \newcommand{\paperfigureheight}{0.115}
  \input{2016-01-18paper-tiles}

\end{frame}

%% heroes

\changelecturelogo{.18}{conkslogo200.png}

\section{Orientation}

\subsection{Course\ Information}

\begin{frame}
  \frametitle{Basics:}

  \begin{block}{}
  \begin{itemize}
  \item 
    \alertb{Instructor:} \myname
  \item 
    \alertb{Lecture room and meeting times:}\\
    \lectureroom, \meetingtimes
  \item 
    \alertb{Office:} \officelocation
  \item 
    \alertb{email:} \myemail
  \item 
    \alertb{Course Website:} \newline
    {\small
      \wordwikilinklong{\coursewebsite}{\coursewebsite}}
  \item
    \alertb{Course Twitter handle:} \coursehandle
  \item
    \alertb{Course hashtag:} \coursehashtag
  \end{itemize}
  \end{block}

\end{frame}

\begin{frame}
%%  \frametitle{Admin:}

  \begin{block}<1->{Potential paper products:}
    \begin{itemize}
    \item<1-> 
%% fix!
%%      The \syllabuslink{\coursewebsite/docs/\courseprefix}.
      The
      \wordwikilink{http://www.uvm.edu/~pdodds/teaching/courses/2014-01UVM-303/docs/2014-01UVM-303syllabus.pdf}{Syllabus}
      and a 
      \wordwikilink{http://www.uvm.edu/~pdodds/teaching/courses/2014-01UVM-303/docs/2014-01UVM-303poster.pdf}{Poster}.
      %% and a 
%%      \posterlink{\coursewebsite/docs/\courseprefix}.
    \end{itemize}
  \end{block}

  \begin{block}<2->{Office hours:}
    \begin{itemize}
    \item<2-> 
      \officehours, \\ \officelocation
    \end{itemize}
  \end{block}

  \begin{block}<3->{Graduate Certificate:}
    \begin{itemize}
    \item
      Principles of Complex Systems is one of two core requirements
      for UVM's five course
      \wordwikilink{http://www.uvm.edu/~cems/complexsystems/?Page=gradApplications/certificate.php}{Certificate of Graduate Study in Complex Systems}.
    \item
     Other required course:  Prof.\ Maggie Eppstein's ``Modelling Complex Systems'' (CSYS/CS 302).
    \item
      coCoNuTS: The Sequel to PoCS: ``Complex Networks'' (CSYS/MATH 303).
    \end{itemize}
  \end{block}

\end{frame}


\begin{frame}
  \small

  \frametitle{Details regarding these artisanal slides:}
%% made from
%% locally-sourced, grade B bits:}

  \begin{block}<1->{}
    \begin{itemize}
    \item<+->
      Three versions (all in pdf): 
      \begin{enumerate}
      \item 
        Presentation,
      \item
        Flat Presentation,
      \item
        Handout (3x2 slides per page).
      \end{enumerate}
    \item<+->
      Presentation versions are \alertb{hyperly navigable}:\newline
      \insertbackfindforwardnavigationsymbol
      $\equiv$
      back + search + forward.
    \item<+->
      Web links look 
      \wordwikilink{http://www.google.com}{like this}
      and are eminently clickable.
    \item<+->
      References in slides link to full citation at end.\cite{anderson1972a}
    \item<+->
      Citations contain links to pdfs for papers (if available).
    \item<+->
      Some books will be linked to on amazon.
    \item<+->
      Brought to you by a frightening melange of 
      \wordwikilink{http://en.wikipedia.org/wiki/XeTeX}{\XeLaTeX}, 
      \wordwikilink{http://en.wikipedia.org/wiki/Beamer\_(LaTeX)}{Beamer}, 
      \wordwikilink{http://www.perl.org/}{perl}, 
      \wordwikilink{http://www.ctan.org/tex-archive/macros/latex/contrib/perltex/}{PerlTeX},
      \wordwikilink{http://en.wikipedia.org/wiki/Command-line_interface}{fevered command-line madness},
      and 
      \wordwikilink{http://www.youtube.com/watch?v=uprjmoSMJ-o}{an almost fanatical devotion}
      to the indomitable \wordwikilink{http://en.wikipedia.org/wiki/Emacs}{emacs}.\newline
      \uncover<+->{\alertr{\#superpowers}}
    \end{itemize}
  \end{block}

\end{frame}

\begin{frame}
  \small

  \begin{block}{More super exciting details:}
    \begin{itemize}
    \item
      This is Season \courseseason\ of \coursename.
    \item
      Lectures will be called Episodes.
    \item
      All lectures are 
      \wordwikilink{http://en.wikipedia.org/wiki/Bottle_episode}{bottle} 
      \wordwikilink{http://tvtropes.org/pmwiki/pmwiki.php/Main/BottleEpisode}{episodes}.
    \item
      \wordwikilink{http://tvtropes.org}{Other tropes} will be involved.
    \end{itemize}
  \end{block}

  \begin{center}

    \youtubevideo{6lzrm3SQynQ}{30}{60}

    \begin{itemize}
    \item
      Last coCoNuTs Episodes are \wordwikilink{http://www.uvm.edu/~pdodds/teaching/courses/2013-08UVM-300/content/lectures.html}{here}.
    \end{itemize}
  \end{center}
  
\end{frame}

\begin{frame}

  \begin{block}{Wonderful foundational support for PoCS and CoNKS has come from the NSF:}
    \begin{itemize}
    \item 
    ``CAREER: Explorations of Complex Social and Psychological
    Phenomena through Multiscale Online Sociological Experiments,
    Empirical Studies, and Theoretical Models.'' 2009--2015.
    \item 
      SES Division of Social and Economic Sciences\\
      SBE Directorate for Social, Behavioral \& Economic Sciences
    \item 
      Abstract is \wordwikilink{http://www.nsf.gov/awardsearch/showAward?AWD_ID=0846668}{here}.
    \end{itemize}
  \end{block}

%%   \begin{block}<+->{}
%%     \begin{itemize}
%%       \item
%%         People have 
%%         \wordwikilink{http://www.uvm.edu/~pdodds/teaching/courses/2015-08UVM-300/content/comments.html}{said
%%           nice things about PoCS}
%%     \end{itemize}
%%   \end{block}

  \begin{block}<+->{}
    \begin{itemize}
    \item
      Last season's Episodes are \wordwikilink{http://www.uvm.edu/~pdodds/teaching/courses/\lastcourseyear-\lastcoursemonth\institution-\coursenumber/content/lectures.html}{here}.
    \end{itemize}
  \end{block}

\end{frame}

\begin{frame}
  \frametitle{Team coCoNuTs}

  \begin{block}{We'll be carrying on with the PoCS Slack:}
    \begin{itemize}
    \item<+-> 
      Place for discussions about all things PoCS/coCoNuTs
      including assignments and projects.
    \item<+-> 
      Once invited, please sign up here:
      \href{http://teampocs.slack.com}{http://teampocs.slack.com}
    \item<+->
      Very good: Install Slack app on laptops, tablets, phone.
    \item<+->
      Everyone will behave wonderfully.
    \end{itemize}    

  \includegraphics[width=0.9\textwidth]{Slack-Colour-RGB.png}

  \end{block}

\end{frame}

\begin{frame}
  \frametitle{Grading breakdown:}

  \begin{block}{}
    \begin{itemize}
    \item 
      \alertb{Projects/talks (36\%)}---Students will 
      work on semester-long projects.  Students will develop a
      proposal in the first few weeks of the course which will be discussed
      with the instructor for approval.  

      \smallskip

      Details: 12\% for the first talk, 12\% for the final talk,
      and 12\% for the written project.
    \end{itemize}
  \end{block}
  \begin{block}{}
    \begin{itemize}
    \item 
      \alertb{Assignments (60\%)}---All assignments will be 
      of equal weight and there will be 10 $\pm$ 1 of them.
    \end{itemize}
  \end{block}
  \begin{block}{}
    \begin{itemize}
    \item 
      \alertb{General attendance/Class participation (4\%)}
    \end{itemize}
  \end{block}

\end{frame}

\begin{frame}
  \frametitle{How grading works:}

  \begin{block}{Questions are worth 3 points according to the following scale:}
    \begin{itemize}
    \item 
      3 = correct or very nearly so.
    \item 
      2 = acceptable but needs some revisions.
    \item 
      1 = needs major revisions.
    \item 
      0 = way off.
    \end{itemize}

  \end{block}

\end{frame}


\begin{frame}
  \frametitle{Important things:}

  \begin{block}{}
    \importantdates
  \end{block}

  \begin{block}{}
    \textbf{Do} check the course Twitter account, \coursehandle, for
    updates regarding the course (part of the course site).
  \end{block}

  \begin{block}{}
    \alertb{Academic assistance:} Anyone who requires assistance in any way 
    (as per the ACCESS program
    or due to athletic endeavors), please see or contact me as soon as possible.
  \end{block}

\end{frame}

\begin{frame}[plain]
  \frametitle{Schedule in detail:}

  \tiny
  \begin{block}{}
    \settablerowcolours
    \lectureschedule
  \end{block}

\end{frame}

\subsection{Projects}

\begin{frame}
  \frametitle{Projects}

  \begin{block}{}
  \begin{itemize}
  \item<+->
    Semester-long projects.
  \item<+-> 
    Possible theme: Stories, Narratives, and Language.
  \item<+-> 
    Develop proposal in first few weeks.
  \item<+-> 
    May range from novel research to investigation of an established area of complex systems.
  \item<+-> 
    Two talks + written piece + Project on Github Pages.
  \item<+->
    Usage of 
    \wordwikilink{http://www.uvm.edu/~vacc/}{the VACC}
    is encouraged (ability to code well = super powers).
  \item<+->
    Massive data sets available, including Twitter.
  \item<+->
    Academic output (journal papers) resulting from Principles
    of Complex Systems and Complex Networks can be found
    \wordwikilink{\coursewebsite/output/}{here}.  Add more!
  \item<+->
    We'll go through a list of possible projects soon.
  \end{itemize}
  \end{block}

\end{frame}

\begin{frame}
%%  \frametitle{Projects}

  \begin{block}{\wordwikilink{http://www.uvm.edu/~pdodds/fama/2015/06/04/the-narrative-hierarchystories-and-storytelling-on-all-scales/}{The
      narrative hierarchy---Stories and Storytelling on all Scales:}}
    \begin{columns}
      \column{0.3\textwidth}
      \includegraphics[width=\textwidth]{2014-11-15narrative-hierarchy-sketches-stories_001_base_1200px-tp-5.png}\\
      \includegraphics[width=\textwidth]{2014-11-15narrative-hierarchy-sketches-stories_001_broken-story_1200px-tp-5.png}\\
      \includegraphics[width=\textwidth]{2014-11-15narrative-hierarchy-sketches-stories_001_broken_1200px-tp-5.png}
      \column{0.7\textwidth}
      \begin{itemize}
      \item 
        1 to 3 word encapsulation = a soundbite = a buzzframe,
      \item 
        1 sentence, title,
      \item 
        few sentences, a haiku,
      \item 
        a paragraph, abstract,
      \item 
        short paper, essay,
      \item 
        long paper,
      \item 
        chapter,
      \item 
        book,
      \item 
        \ldots
      \end{itemize}
    \end{columns}

  \end{block}

\end{frame}



%% \begin{frame}
%%   \frametitle{Projects}
%% 
%% The ``narrative hierarchy''---explaining things on many scales:
%% 
%% 1 to 3 word encapsulation, a soundbite,\\
%% a sentence/title,\\
%% a few sentences,\\
%% a paragraph,\\
%% a short paper,\\
%% a long paper,\\
%% a chapter,\\
%% a book,\\
%% $\vdots$
%% 
%% \end{frame}

\section{The\ rise\ of\  networks}

\begin{frame}
  \frametitle{Key Observation:}

  \begin{itemize}
  \item <1->
    Many \alert{complex systems}\\ 
    can be viewed as \alert{complex networks}\\
    of physical or abstract interactions.
  \item <2->
    Opens door to mathematical and numerical analysis.
  \item <3-> 
    Dominant approach of last decade of 
    a \alertb{theoretical-physics/stat-mechish} flavor.
  \item <4-> 
    Mindboggling amount of work published 
    on complex networks since 1998 \ldots
  \item <5-> 
    \ldots due to your typical theoretical physicist:
    \begin{overprint}
      \onslide<1-5 | handout:0 | trans: 0>
      \onslide<6- | handout:1 | trans: 1>
      \smallskip
      \begin{columns}
        \column{0.3\textwidth}
        \includegraphics[width=\textwidth]{piranha3-tp.pdf}
        \column{0.7\textwidth}
        \begin{itemize}
        \item \textit{Piranha physicus}
        \item<7-> Hunt in packs.
        \item<8-> Feast on new and interesting ideas \\
          {\small (see chaos, cellular automata, \ldots)}
        \end{itemize}
      \end{columns}
    \end{overprint}
  \end{itemize}

\end{frame}

%%%%%%%%%%%%%%%%%%%%%%%%%%%%%%%%%%%%%
% popularity
%%%%%%%%%%%%%%%%%%%%%%%%%%%%%%%%%%%%%

\begin{frame}
  \frametitle{Popularity (according to Google Scholar)}

  \begin{block}<1->{``Collective dynamics of `small-world' networks''\cite{watts1998a}}
    \begin{itemize}
    \item[] 
      Duncan Watts and Steve Strogatz\\
      Nature, 1998
    \item[] 
      \wordwikilink{http://scholar.google.com/citations?view\_op=view\_citation\&hl=en\&user=LhOAiXMAAAAJ\&citation\_for\_view=LhOAiXMAAAAJ:u5HHmVD\_uO8C}
      {Times cited: \uncover<2->{\alert{$\sim 28,017$}} }
      {\tiny(as of January 18, 2016)}
        %% http://scholar.google.com/citations?view_op=view_citation&hl=en&user=LhOAiXMAAAAJ&citation_for_view=LhOAiXMAAAAJ:u5HHmVD_uO8C
        %% 
      \end{itemize}
    \end{block}

  \begin{block}<1->{``Emergence of scaling in random networks''\cite{barabasi1999a}}
    \begin{itemize}
    \item[] 
      L\'{a}szl\'{o} Barab\'{a}si and R\'{e}ka Albert\\
      Science, 1999
    \item[] 
      \wordwikilink{http://scholar.google.com/citations?view\_op=view\_citation\&hl=en\&user=vsj2slIAAAAJ\&citation\_for\_view=vsj2slIAAAAJ:u5HHmVD\_uO8C}
      {Times cited: \uncover<3->{\alert{$\sim 24,236$}}}
      {\tiny(as of January 18, 2016)}
      %% https://scholar.google.com/citations?view_op=view_citation&hl=en&user=vsj2slIAAAAJ&citation_for_view=vsj2slIAAAAJ:u5HHmVD_uO8C
    \end{itemize}
  \end{block}
\end{frame}

\section{Models}


\begin{frame}
  \frametitle{Models}

  \begin{block}{Some important models:}
    \begin{enumerate}
    \item<2-> generalized random networks \alert{(touched on in 300)}
    \item<3->
      \wordwikilink{http://en.wikipedia.org/wiki/Scale-free_network}{scale-free
        networks} \alert{(partly covered in 300)}
    \item<4-> \wordwikilink{http://en.wikipedia.org/wiki/Small-world_network}{small-world networks} \alert{(covered in 300)}
    \item<5-> statistical generative models ($p^\ast$)
    \item<6-> generalized affiliation networks \alert{(covered in 300)}
    \end{enumerate}
  \end{block}
  
\end{frame}

\begin{frame}
  \showtarotcards{0.35}{
%%  \dealnewtarotcard{0.25}{
    john-dory,
    overview,
    complex-networks,
    random-networks,
%%    scale-free-networks,
%%    small-world-networks,
%%    theory-six-degrees,
}
\end{frame}


\begin{frame}
  \frametitle{Models}

  \begin{block}<1->{1. generalized random networks:}
    \begin{itemize}
    \item<2-> Arbitrary degree distribution $P_k$.
    \item<3-> Wire nodes together randomly.
    \item<4-> Create ensemble to test deviations from randomness.
    \item<5-> Interesting, applicable, rich mathematically.
    \item<6->
      We will have fun with these things \ldots
    \end{itemize}
  \end{block}
  
\end{frame}

\begin{frame}
  \showtarotcards{0.35}{
%%  \dealnewtarotcard{0.25}{
    john-dory,
    overview,
    complex-networks,
    random-networks,
    scale-free-networks,
%%    small-world-networks,
%%    theory-six-degrees,
}
\end{frame}


\begin{frame}
  \frametitle{Models}

  \begin{block}{2. `scale-free networks':}
    \begin{columns}
      \column{0.03\textwidth}
      \column{0.33\textwidth}
      \bigskip
      \includegraphics[height=\textwidth]{nw_powerlaw_graphviz03maincluster_1}\\
      \bigskip
      \small{$\gamma$ = 2.5},
      \small{$\tavg{k}$ = 1.8},
      \small{$N = 150$}
      \column{0.64\textwidth}
      \begin{itemize}
      \item<2-> Introduced by Barabasi and Albert\cite{barabasi1999a}
      \item<2-> Generative model
      \item<3-> Preferential attachment model with growth:
      \item<4->  $P[\textnormal{attachment to node}\ i] \propto k_{i}^\alpha$.
      \item<5-> Produces \alert{$P_k \sim k^{-\gamma}$} when $\alpha=1$.
      \item<6-> Trickiness: other models generate skewed degree distributions.
      \end{itemize}
    \end{columns}
  \end{block}

\end{frame}

\begin{frame}
  \showtarotcards{0.35}{
%%  \dealnewtarotcard{0.25}{
    john-dory,
    overview,
    complex-networks,
    random-networks,
    scale-free-networks,
    small-world-networks,
%%    theory-six-degrees,
}
\end{frame}

\begin{frame}
  \frametitle{Models}

  \begin{block}<1->{3. small-world networks}
    \begin{itemize}
    \item Introduced by Watts and Strogatz\cite{watts1998a}
    \end{itemize}
    \medskip
    \visible<2->{Two scales:}
    \begin{itemize}
    \item<3-> \alert{local regularity} (an individual's friends know each other)
    \item<4-> \alert{global randomness} (shortcuts).
    \end{itemize}

    \begin{columns}
      \column{0.65\textwidth}
      \begin{itemize}
      \item<5-> Shortcuts allow disease to jump
      \item<6-> Number of infectives increases exponentially in time
      \item<7-> Facilitates synchronization
      \end{itemize}
      \column{0.35\textwidth}
      \begin{overprint}
        \onslide<1-2| handout:0| trans:0>
        \onslide<3| handout:0| trans:0>
        \includegraphics[height=0.3\textheight]{lattice4}
        \onslide<4-| handout:1| trans:1>
        \includegraphics[height=0.3\textheight]{latticeshortcut4}
      \end{overprint}
    \end{columns}
  \end{block}
  
\end{frame}

\begin{frame}
  \showtarotcards{0.35}{
%%  \dealnewtarotcard{0.25}{
    john-dory,
    overview,
    complex-networks,
    random-networks,
    scale-free-networks,
    small-world-networks,
    theory-six-degrees,
}
\end{frame}

\begin{frame}
  \frametitle{Models}

  \begin{block}{5. generalized affiliation networks}
    \bigskip
    \begin{center}
      \includegraphics[height=0.6\textheight]{bipartite}
    \end{center}
  \end{block}

  Bipartite affiliation networks: boards and directors, movies and actors.

\end{frame}

\begin{frame}
  \frametitle{Models}

  \begin{block}{5. generalized affiliation networks}
    \centering
    \includegraphics[width=1\textwidth]{bipartite2}
  \end{block}

\end{frame}

\begin{frame}
  \frametitle{Models}

  \begin{block}{5. generalized affiliation networks}
    \includegraphics[width=1\textwidth]{generalcontext2}  
    \begin{itemize}
    \item Blau \& Schwartz\cite{blau1984a}, Simmel\cite{simmel1902a}, Breiger\cite{breiger1974a}, Watts \etal\cite{watts2002b}
    \end{itemize}
  \end{block}


\end{frame}

\section{Resources}

\begin{frame}
  \frametitle{Bonus materials:}

  \begin{block}<1->{Textbooks:}
    \begin{itemize}
      \item \small
        Mark Newman (Physics, Michigan)\\
        \alertb{``Networks: An Introduction''}
        \wikilink{http://www.amazon.com/Networks-Introduction-Mark-Newman/dp/0199206651}
      \item \small
        David Easley and Jon Kleinberg (Economics and Computer Science, Cornell)\\
        \alertb{``Networks, Crowds, and Markets: Reasoning About a Highly Connected World''}
        \wikilink{http://www.cs.cornell.edu/home/kleinber/networks-book/}
    \end{itemize}
  \end{block}

\end{frame}

\begin{frame}
  \frametitle{Bonus materials:}

  \begin{block}<1->{Review articles:}
    \begin{itemize}
    \item
      S. Boccaletti et al.,\\
      Physics Reports, 2006,\\
      \alertb{``Complex networks: structure and dynamics''}\cite{boccaletti2006a}\\
      \wordwikilink{http://scholar.google.com/citations?view\_op=view\_citation\&hl=en\&user=BEC76f4AAAAJ\&citation\_for\_view=BEC76f4AAAAJ:u5HHmVD\_uO8C}{Times
        cited: \alert{$\sim$ 6,034}}
      {\tiny(as of January 18, 2016)}
      %% http://scholar.google.com/citations?view\_op=view\_citation&hl=en&user=BEC76f4AAAAJ&citation\_for\_view=BEC76f4AAAAJ:u5HHmVD\_uO8C
    \item 
      M. Newman,\\
      SIAM Review, 2003,\\
      \alertb{``The structure and function of complex networks''}\cite{newman2003a}\\
      \wordwikilink{https://scholar.google.com/citations?view\_op=view\_citation\&hl=en\&user=rQ68pVwAAAAJ\&citation\_for\_view=rQ68pVwAAAAJ:\_FxGoFyzp5QC}
      {Times cited: \alert{$\sim$ 13,536}}
      {\tiny(as of January 18, 2016)}
      %% 
      %% http://scholar.google.com/scholar?cites=12945060519911641528\&as\_sdt=5,46\&sciodt=0,46\&hl=en
    \item 
      R.\ Albert and A.-L.\ Barab\'{a}si\\
      Reviews of Modern Physics, 2002,\\
      \alertb{``Statistical mechanics of complex networks''}\cite{albert2002a}\\
      \wordwikilink{http://scholar.google.com/citations?view\_op=view\_citation\&hl=en\&user=d27Ji6kAAAAJ\&citation\_for\_view=d27Ji6kAAAAJ:WF5omc3nYNoC}
      {Times cited: \alert{$\sim$ 16,041}} 
      {\tiny(as of January 18, 2016)}
      %% https://scholar.google.com/citations?view_op=view_citation&hl=en&user=d27Ji6kAAAAJ&citation_for_view=d27Ji6kAAAAJ:WF5omc3nYNoC
    \end{itemize}
  \end{block}

\end{frame}

\section{Nutshell}

\begin{frame}[label=]
  \frametitle{Nutshell:}

  \begin{block}{Overview Key Points:}
    \begin{itemize}
    \item<1->
      The field of complex networks came into
      existence in the late 1990s.
    \item<2->
      Explosion of papers and interest since 1998/99.
    \item<3->
      Hardened up much thinking about complex systems.
    \item<4->
      Specific focus on networks that are 
      \alert{large-scale}, 
      \alertb{sparse}, 
      \alert{natural} or \alert{man-made}, 
      \alertb{evolving} and \alertb{dynamic}, 
      and 
      (crucially) \alert{measurable}.
    \item<5->
      Three main (blurred) categories: 
      \begin{enumerate}
      \item 
      \alert{Physical} (e.g., river networks),
      \item 
      \alert{Interactional} (e.g., social networks),
      \item 
      \alert{Abstract} (e.g., thesauri).
      \end{enumerate}
    \end{itemize}
    
  \end{block}

\end{frame}

\begin{frame}[label=]
  \frametitle{Nutshell:}

  \begin{block}{Overview Key Points (cont.):}
    \begin{itemize}
    \item<1->
      Obvious connections with the vast
      extant field of graph theory.
    \item<2->
      But focus on dynamics is more of a physics/stat-mech/comp-sci
      flavor.
    \item<3->
      Two main areas of focus:
      \begin{enumerate}
      \item 
        \alertb{Description:} Characterizing very large networks
      \item
        \alertb{Explanation:} Micro story $\Rightarrow$ Macro features
      \end{enumerate}
    \item<4->
      Some essential structural aspects are understood: degree distribution, clustering,
      assortativity, group structure, overall structure, \ldots
    \item<5->
      Still much work to be done, especially with respect to dynamics \ldots
      \uncover<6->{\alert{exciting!}}
    \end{itemize}
    
  \end{block}

\end{frame}


\begin{frame}
  \frametitle{Neural solace---Temporal social networks:}

  \begin{block}{
      \wordwikilink{http://flowingdata.com/2015/12/15/a-day-in-the-life-of-americans}
      {Visualizing a day in the life of Americans}
    }
    
    \begin{center}
    \includegraphics[width=0.8\textwidth]{2016-01-18people-moving-flowing-data.png}
    \end{center}

    \begin{itemize}
    \item   
      Source: Flowing Data/Nathan Yau.
    \end{itemize}
  \end{block}

  
\end{frame}
