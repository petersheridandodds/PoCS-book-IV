%% for good reason, this is a famous photo
%% \url{http://en.wikipedia.org/wiki/Afghan_Girl}

%% cass
%% 1930s 
%% 
%% bach's cello suite
%% spent 20 years
%% steve mcqueen


%% NFL draft picks
%% cool visualization
%% http://www.nytimes.com/interactive/2013/04/25/sports/football/picking-the-best-in-the-nfl-draft.html?smid=tw-share

%% great:
%% ``leapfrogging''
%% http://dish.andrewsullivan.com/2013/03/18/bestsellers-by-the-numbers/
%% 
%% 
%% Moby Dick
%% 
%% Maurice Sendak
%% 
%% 
%% %%% 
%% 
%% http://www.smithsonianmag.com/history-archaeology/When-Republicans-Were-Blue-and-Democrats-Were-Red-176776491.html?c=y&page=1#?utm_source=twitter.com&utm_medium=socialmedia&utm_campaign=20121101-1&utm_content=whenrepoublicanswereblueanddemocratswerered
%% 
%% 
%% %% preface with estimation.body.tex
%% 
%% http://andrewsullivan.thedailybeast.com/2011/09/the-key-to-groupthink.html
%% 
%% @Article{guimera2011a,
%%   author =       {Guimer\`{a}, Roger and Sales-Pardo, Marta},
%%   journal =      {PLoS ONE},
%%   OPTpublisher = {Public Library of Science},
%%   title =        {Justice Blocks and Predictability of U.S. Supreme
%%                   Court Votes},
%%   year =         {2011},
%%   OPTmonth =     {11},
%%   volume =       {6},
%%   OPTurl =       {http://dx.doi.org/10.1371\%2Fjournal.pone.0027188},
%%   pages =        {e27188},
%%   OPTnumber =    {11},
%%   OPTdoi =       {10.1371/journal.pone.0027188}
%% }        

%% Find picture from national geographic...
%% The haunted eyes.
%% 
%% figures/local/1983afghan_girl_national_geographic.jpg
%% http://www.nationalgeographic.com/ngm/100best/multi1_interview.html
%% 
%% Talk about inherent quality...


%% build on from musiclab
%% 
%% do some of this beforehand?
%% 
%% 


%% 
%
%%  \textbf{IV. Online experiments}
%
%%  \tc{blue}{\ding{228} .}
%
%%  Is the Mona Lisa really the Mona Lisa of paintings?
%%  
%% 

%% to add
%% Rosen

%% super rich story
%% http://news.bbc.co.uk/2/hi/business/7118991.stm

  

\section{Winning:\ it's\ not\ for\ everyone}

\subsection{Superstars}

  \textbf{Where do superstars come from?}

  
    Rosen (1981): \alert{``The Economics of Superstars''}\cite{rosen1981a}
  

  \textbf{Examples:}
  
   Full-time Comedians ($\approx 200$)
   Soloists in Classical Music
   Economic Textbooks (the usual myopic example)
  
  

  
    
     {Highly skewed distributions again...}
    
  
  

  \textbf{Superstars}

  \textbf{Rosen's theory:}
    
     Individual quality $q$ maps to reward $R(q)$
     $R(q)$ is `convex' ($\tdiffdiff{R}{q} > 0$)
     Two reasons:
      
       \alert{Imperfect substitution:}\\
        {A very good surgeon is worth many mediocre ones}
       \alert{Technology:}\\
        {Media spreads \& technology reduces cost of reproduction
          of books, songs, etc.}\\
      
    {Joint consumption versus public good}
    {No social element---success follows `inherent quality'}
    
    
  
  

  \textbf{Superstars}
 
  \textbf{Adler (1985): \alert{``Stardom and Talent''}\cite{adler1985a}}
    
     
      Assumes extreme case of equal `inherent quality'
     
      Argues desire for coordination in knowledge and culture
      leads to differential success
     
      Success can be purely a social construction
     
      (How can we measure `inherent quality'?)
    
  
  

  \textbf{Voting}
 
  \textbf{Evidence from the web suggestions (Huberman et al.)}
    
     Easy decisions (yes/no) lead to bandwagoning
      
       e.g. jyte.com
      
     More costly evaluations lead to oppositional votes
      
       e.g. amazon.com
      
    
    
     \alert{Self-selection:} Costly voting may lower incentives for those
      who agree with the current assessment and increase
      incentives for those who disagree.
    
  
  

  \textbf{Voting}
 
  \textbf{Score-based voting versus rank-based voting:}
    
     
      Balinski and Laraki\cite{balinski2007a}\\
      \alertb{``A theory of measuring, electing, and ranking''}\\
      Proc. Natl. Acad. Sci., pp. 8720--8725 (2007)
    
    
  
  



  \textbf{Voting}

  \textbf{
      Laureti et al. (2004): 
      \alert{``Aggregating partial, local evaluations to achieve global ranking''}\cite{laureti2004b}
    }    
    
     
      Model: participants rank $n$ objects based on underlying quality $q$
     
      Assume evaluation of object $i$ is a random variable with mean $q_i$ 
     
      Choose objects based on votes:\\
      $$ p_i(t) \propto v_i(t)^{\alpha} \ \mbox{or} \ p_i(t) \propto q_i v_i(t)^{\alpha}. $$
     
      If $\alpha<1$, correct quality ordering is uncovered
     
      If $\alpha>1$, some objects are never evaluated and mistakes are made...
     
      Related to Adler's approach
    
  

%   One model: Evaluators compare pairs of objects
%    
%     Choose objects at random
%     Choose objects based on votes: $p_i(t) \propto v_i(t)^{\alpha}$
%    
  

  \textbf{Dominance hierarchies}

  \textbf{Chase et al. (2002): \alert{``Individual differences versus social dynamics in the formation of animal dominance hierarchies''}\cite{chase2002a}}
    
     
      The aggressive female Metriaclima zebra:
    
    \begin{center}
      \includegraphics[width=0.5\textwidth]{maylandia_lombardoi_wiki.jpg}
      \includegraphics[width=.07\textwidth]{wikipedia-tp-3.pdf}
    \end{center}
    
     
      Pecking orders for fish... 
    
  
  

  \textbf{Dominance hierarchies}

  \textbf{Fish forget---changing of dominance hierarchies:}
    \includegraphics[width=0.48\textwidth]{chase2002afig1_1}%
    \includegraphics[width=0.48\textwidth]{chase2002afig1_2}%
    
    22 observations: about 3/4 of the time, hierarchy changed
    
  
  

  \textbf{Dominance hierarchies}

  \includegraphics[width=\textwidth]{chase2002afig2}%

  
    
    Group versus isolated interactions produce different hierarchies
    
  
  


\subsection{Musiclab}

  \textbf{Music Lab Experiment}
  \includegraphics[width=0.45\textwidth]{MusicLab_mainlogo}%
  \hfill
  \includegraphics[width=0.45\textwidth]{instructions-tp.pdf}

  \begin{tabular}{ll}
    48 songs \\
    30,000 participants\\    % teenagers
  \end{tabular}
  \hfill
  \begin{tabular}{ll}
    multiple `worlds'\\
    Inter-world variability\\
  \end{tabular}

  
   How probable is the world?  
   Can we estimate variability?
   Superstars dominate but are unpredictable.  Why?
  



  \textbf{Music Lab Experiment}

    \includegraphics[width=0.9\textwidth]{ml_info-v1-original}

    Salganik et al. (2006)
    \alertg{``An experimental study of inequality and unpredictability in an artificial cultural market''}\cite{salganik2006a}


  \textbf{Music Lab Experiment}

  \begin{tabular}{cc}
    Experiment 1 & Experiments 2--4 \\
    \includegraphics[width=0.48\textwidth]{ml_info-v1-original} & 
    \includegraphics[width=0.48\textwidth]{ml_info-v2-original} \\
  \end{tabular}





  \textbf{Music Lab Experiment}

  \includegraphics[width=0.45\textwidth]{ml_ms_noinfo_info_v1_rank}
  \includegraphics[width=0.45\textwidth]{ml_ms_noinfo_info_v2_rank}
  
  
   
    Variability in final rank.
  


  \textbf{Music Lab Experiment}

  \includegraphics[width=0.45\textwidth]{ml_ms_noinfo_info_v1}
  \includegraphics[width=0.45\textwidth]{ml_ms_noinfo_info_v2}
  
  
   
    Variability in final number of downloads.
  
  


%% %%    \textbf{Music Lab Experiment}
%% 
%%   \includegraphics[width=0.45\textwidth]{ml_compare_gini_v1v2_unordered}
%%   \raisebox{-1cm}{\includegraphics[width=0.5\textwidth]{ml_compare_unpred_v1v2}}
%%   
%%   Stronger social signal $\Rightarrow$ \\
%%   Stronger inequality and less predictability.
%%  
%% 

%% 
%% \includegraphics[width=0.3\textwidth]{ml_compare_gini_v1v2_unordered.pdf}
%% \includegraphics[width=0.3\textwidth]{ml_compare_unpred_v1v2.pdf}
%% \includegraphics[width=0.3\textwidth]{ml_exp34_v3.pdf}
%% \includegraphics[width=0.3\textwidth]{ml_expdesign_ajs.pdf}
%% \includegraphics[width=0.3\textwidth]{ml_info-v1-original.pdf}
%% \includegraphics[width=0.3\textwidth]{ml_info-v2-original.pdf}
%% \includegraphics[width=0.3\textwidth]{ml_listenchoice_v1v2_user.pdf}
%% \includegraphics[width=0.3\textwidth]{ml_ms_noinfo_info_v1.pdf}
%% \includegraphics[width=0.3\textwidth]{ml_ms_noinfo_info_v1_rank.pdf}
%% \includegraphics[width=0.3\textwidth]{ml_ms_noinfo_info_v2.pdf}
%% \includegraphics[width=0.3\textwidth]{ml_ms_noinfo_info_v2_rank.pdf}
%% \includegraphics[width=0.3\textwidth]{ml_pair1_34.pdf}
%% \includegraphics[width=0.3\textwidth]{ml_pair2_34.pdf}



  \textbf{Music Lab Experiment}
  \centering
  \includegraphics[width=0.45\textwidth]{ml_compare_gini_v1v2_unordered}

  
   Inequality as measured by Gini coefficient:
    $$
    G= \frac{1}{(2N_{\textrm{s}}-1)} \sum_{i=1}^{N_{\textrm{s}}} \sum_{j=1}^{N_{\textrm{s}}} | m_i - m_j |    
    $$
  


  \textbf{Music Lab Experiment}
  \centering
  \includegraphics[width=0.45\textwidth]{ml_compare_unpred_v1v2}

  
    Unpredictability
    $$ 
    U = 
    \frac{1}
    %% {N_{\textrm{s}(N_{\textrm{w}-1)N_{\textrm{w}/2}
    {N_{\textrm{s}}\binom{N_{\textrm{w}}}{2}}
    \sum_{i=1}^{N_{\textrm{s}}}
    \sum_{j=1}^{N_{\textrm{w}}}
    \sum_{k=j+1}^{N_{\textrm{w}}}
    | m_{i,j} - m_{i,k} |  
    $$
  


  \textbf{Music Lab Experiment}

  \textbf{Sensible result:}
  
   Stronger social signal leads to \alert{greater following and greater inequality}.
  
  

  \textbf{Peculiar result:}
  
   Stronger social signal leads to greater \alert{unpredictability}.
  
  

  \textbf{Very peculiar observation:}
  
   The most unequal distributions would suggest the greatest
    variation in underlying `quality.'
   But success may be due to social construction through \alert{following}.
    {\alert{(so let's tell a story...\cite{sunstein2006a,taleb2007a}})}
  
  

%%   Stronger social signal $\Rightarrow$ \\
%%   Stronger inequality and less predictability.



  \textbf{Music Lab Experiment---Sneakiness}

  \includegraphics[width=0.45\textwidth]{ml_pair1_34}
  \includegraphics[width=0.45\textwidth]{ml_pair2_34}

  
    Inversion of download count
    The pretend rich get richer ...
    ... but at a slower rate
  


%% \section{Final\ words}
%% 
%% %% 
%%   \textbf{Final words:}
%% 
%%   \textbf{Modern science in three steps:}
%%     
%%     
%%       Find interesting/meaningful/important phenomena
%%       involving spectacular amounts of data.
%%     
%%       Describe what you see.
%%     
%%       Explain it.
%%     
%%   
%% 
%% 
