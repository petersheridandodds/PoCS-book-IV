%%%%%%%%%%%%%%%%%%%%%%%%%%%%%%%%%%%%%
% very clear summary slide 
%%%%%%%%%%%%%%%%%%%%%%%%%%%%%%%%%%%%%

\section{Plan}

  \textbf{Something of a plan:}

  
   \alert{Lecture 1:} Overview; Background
   \alert{Lecture 2:} Random, Scale-free, \\
    \qquad \qquad and Small-World networks
   \alert{Lecture 3:} Models of Contagion
   \alert{Lecture 4:} Transportation networks; \\
    \qquad \qquad Discovering structure
  


  \textbf{Exciting details regarding these slides:}

  
  
    Three versions (all in pdf): 
    
     
      Presentation,
    
      Flat Presentation,
    
      Handout (2x2).
    
  
    Presentation versions are \alertb{navigable} 
    and hyperlinks are \alertb{clickable}.
  
    Web links look 
    \wordwikilink{http://www.google.com}{like this}. 
  
    References in slides link to full citation at end.\cite{anderson1972a}
  
    Citations contain links to papers in pdf (if available).
  
    50 hours of lectures $\rightarrow$ 5 hours.
  
    Brought to you by a concoction of \LaTeX, Beamer, and perl.
  

%%%%%%%%%%%%%%%%%%%%%%%%%%%%%%%%%%%%%
% basic definitions
%%%%%%%%%%%%%%%%%%%%%%%%%%%%%%%%%%%%%

\section{Basic definitions}

  \textbf{Basic definitions}

  \textbf{Complex System---Some ingredients:}
    
    
      Distributed system of many interrelated parts
    
      No centralized control 
    
      Nonlinear relationships
    
      Existence of feedback loops
    
      Complex systems are open (out of equilibrium)
    
      Presence of Memory
    
      Modular (nested)/multiscale structure
    
      Opaque boundaries
    
      Emergence---`More is Different'\cite{anderson1972a}
    
  
  \mbox{}
  \hfill
  \includegraphics[width=.07\textwidth]{wikipedia.jpg}


  \textbf{Basic definitions}

% dictionary definition

  \textbf{{Complex:} (Latin = with + fold/weave (com + plex))}
    \alert{Adjective}
    
     Made up of multiple parts; intricate or detailed.
     Not simple or straightforward.
    
  
\mbox{} \hfill \includegraphics[width=.07\textwidth]{wikipedia.jpg}


  \includegraphics[width=\textwidth]{network_dictionary_cut.pdf}

  \textbf{Thesaurus deliciousness:}

  \begin{center}
    \includegraphics[width=0.9\textwidth]{network_thesaurus_cut.pdf}
  \end{center}





%% %%   \textbf{Basic definitions}
%% 
%%   \alert{Network:} (net + work, 1500's)
%%   \hfill
%%   \includegraphics[width=.07\textwidth]{wikipedia.jpg}
%% 
%%   \textbf{\alert{Noun:}}
%%     
%%      Any interconnected group or system
%%      Multiple computers and other devices connected together to share information
%%     
%%   
%% 
%%   \textbf{\alert{Verb:}}
%%     
%%      To interact socially for the purpose of getting connections or personal advancement
%%      To connect two or more computers or other computerized devices
%%     
%%   
%% 
%% 
  \textbf{Ancestry:}

  From Keith Briggs's excellent
  \wordwikilink{http://keithbriggs.info/network.html}{etymological investigation:}

  \medskip

      
    
    
     
      Opus reticulatum:
     
      A Latin origin?
    
    
    \includegraphics[width=\textwidth]{opus_reticulatum.jpg}\\
    {\tiny [http://serialconsign.com/2007/11/we-put-net-network]}
  

  \textbf{Ancestry:}

  \textbf{First known use: Geneva Bible, 1560}
    `And thou shalt make unto it a grate like networke of brass (Exodus xxvii 4).'
  

  \textbf{From the OED via Briggs:}
    
     
      1658--: reticulate structures in animals
     
      1839--: rivers and canals
     
      1869--: railways
     
      1883--: distribution network of electrical cables
     
      1914--: wireless broadcasting networks
    
  



  \textbf{Ancestry:}

  \textbf{Net and Work are venerable old words:}
    
    
      \alert{`Net'} first used to mean spider web 
      {\small (King {\AE}lfr\'{e}d, 888)}.
    
      \alert{`Work'} appear to have long meant purposeful action.
    
  

      
    \includegraphics[width=\textwidth]{briggs2005a_fig1}
    
    \includegraphics[width=\textwidth]{briggs2005a_fig2}
  
  
  
    `Network' = something built
    based on the idea of natural, flexible lattice or web.
   
    c.f., ironwork, stonework, fretwork.
  


  \textbf{Key Observation:}

  
   
    Many \alert{complex systems}\\ 
    can be viewed as \alert{complex networks}\\
    of physical or abstract interactions.
   
    Opens door to mathematical and numerical analysis.
    
    Dominant approach of last decade of 
    a \alertb{theoretical-physics/stat-mechish} flavor.
    
    Mindboggling amount of work published 
    on complex networks since 1998...
    
    ... largely due to your typical theoretical physicist:
          
      
      \smallskip
              
        \includegraphics[width=\textwidth]{piranha3.jpg}
        
        
         \textit{Piranha physicus}
         Hunt in packs.
         Feast on new and interesting ideas \\
          {\small (see chaos, cellular automata, ...)}
        
            


%%%%%%%%%%%%%%%%%%%%%%%%%%%%%%%%%%%%%
% popularity
%%%%%%%%%%%%%%%%%%%%%%%%%%%%%%%%%%%%%

  \textbf{Popularity (according to ISI)}

  \textbf{``Collective dynamics of `small-world' networks''\cite{watts1998a}}
    
     
      Watts and Strogatz\\
      Nature, 1998
     
      \alert{$\approx 3752$} citations {\tiny(as of June 5, 2009)}\\
     
      Over 1100 citations in 2008 alone.
    
  

  \textbf{``Emergence of scaling in random networks''\cite{barabasi1999a}}
    
     
      Barab\'{a}si and Albert\\
      Science, 1999
     
      \alert{$\approx 3860$} citations {\tiny(as of June 5, 2009)}
     
      Over 1100 citations in 2008 alone.
    
  

\section{Popularity according to books}

  \textbf{Popularity according to books:}

  \showbook{tippingpoint.jpg}
  {The Tipping Point: How Little Things can make a Big Difference}
  {Malcolm Gladwell\cite{gladwell2000a}}

  \bigskip

  \showbook{nexus.jpg}
  {Nexus: Small Worlds and the Groundbreaking Science of Networks}
  {Mark Buchanan}


  \textbf{Popularity according to books:}

  \showbook{linked.jpg}
  {Linked: How Everything Is Connected to Everything Else and What It Means}
  {Albert-Laszlo Barab\'{a}si}

  \bigskip

  \showbook{sixdegrees.jpg}
  {Six Degrees: The Science of a Connected Age}
  {Duncan Watts\cite{watts2003a}}


%% %%   \textbf{Books}
%% 
%% \showbook{handbookgraphs.jpg}
%% {Handbook of Graphs and Networks}
%% {editors: Stefan Bornholdt and H. G. Schuster\cite{bornholdt2003a}}
%% 
%% \bigskip
%% 
%% \showbook{evolutionofnetworks.jpg}
%% {Evolution of Networks}
%% {S. N. Dorogovtsev and J. F. F. Mendes\cite{dorogovtsev2003a}}
%% 
%% %% 
%% %%   \textbf{Books}
%% 
%% \showbook{socialnetworkanalysis.jpg}
%% {Social Network Analysis}
%% {Stanley Wasserman and Kathleen Faust\cite{wasserman1994a}}
%% 
%% \bigskip
%% 
%% \showbook{inthebeatofaheart.jpg}
%% {In the Beat of a Heart: Life, Energy, and the Unity of Nature}
%% {John Whitfield}
%% 
%% 
  \textbf{Numerous others:}
  
  \small
  
    
     
      \alertb{Complex Social Networks}---F. Vega-Redondo\cite{vega-redondo2007a}
     
      \alertb{Fractal River Basins: Chance and Self-Organization}---I. Rodr\'{\i}guez-Iturbe and A. Rinaldo\cite{rodriguez-iturbe1997a}
     
      \alertb{Random Graph Dynamics}---R. Durette
     
      \alertb{Scale-Free Networks}---Guido Caldarelli
     
      \alertb{Evolution and Structure of the Internet: A Statistical Physics Approach}---Romu Pastor-Satorras and Alessandro Vespignani
     
      \alertb{Complex Graphs and Networks}---Fan Chung
     
      \alertb{Social Network Analysis}---Stanley Wasserman and Kathleen Faust
     
      \alertb{Handbook of Graphs and Networks}---Eds: Stefan Bornholdt and H. G. Schuster\cite{bornholdt2003a}
     
      \alertb{Evolution of Networks}---S. N. Dorogovtsev and J. F. F. Mendes\cite{dorogovtsev2003a}
    


%%%%%%%%%%%%%%%%%%%%%%%%%
%% Observations
%%%%%%%%%%%%%%%%%%%%%%%%%

  \textbf{More observations}

  
  
    But surely \alert{networks aren't new}...
  
    Graph theory is well established...
  
    Study of social networks started in the 1930's...
  
    So why all this `new' research on networks?
  
    \alert{Answer:} \alertb{Oodles of Easily Accessible Data.}
  
    We can now inform (alas) our theories \\
    with a much more measurable reality.$^\ast$
   
    A worthy goal: establish \alertb{mechanistic explanations}.

    \medskip
    {
    {\small 
      $\mbox{}^\ast$\textit{If this is upsetting, maybe string theory is for you...}}
  }
  

  \textbf{More observations}

  
  
    \alertb{Web-scale} data sets can be overly \alert{exciting}.
  
  
  \textbf{Witness:}
    
    
      The End of Theory: The Data Deluge Makes the Scientific Theory Obsolete (Anderson, Wired)
      \wikilink{http://www.wired.com/science/discoveries/magazine/16-07/pb\_theory\#}
    
      ``The Unreasonable Effectiveness of Data,''\\ Halevy et al.\cite{halevy2009a}.
    
  

  \textbf{But:}
  
   
    For scientists, description is only part of the battle.
   
    We still need to \alertb{understand}.
  
  



%%%%%%%%%%%%%%%%%%%%%%%%
%% Basic definitions
%%%%%%%%%%%%%%%%%%%%%%%%

  \textbf{Super Basic definitions}

  \textbf{\alert{Nodes} = A collection of entities 
      which have properties that
      are somehow related to each other}
    
      
      e.g., people, forks in rivers, proteins, webpages, organisms,...
    
  

  \textbf{\alert{Links} = Connections between nodes}
    
    
      \alert{Links} may be directed or undirected.
    
      \alert{Links} may be binary or weighted.
    
  

  {
    Other spiffing words: vertices and edges.
  }


%% %%   \textbf{Basic definitions}
%% 
%%   \textbf{\alert{Links} = Connections between nodes}
%%     
%%     
%%       \alert{links}
%%       
%%        
%%       may be real and fixed (rivers),
%%        
%%       real and dynamic (airline routes), 
%%        
%%       abstract with physical impact (hyperlinks),
%%        
%%       or purely abstract (semantic connections between concepts).
%%       
%%     
%%       \alert{Links} may be directed or undirected.
%%     
%%       \alert{Links} may be binary or weighted.
%%     
%%   
%% 
%% 
  \textbf{Super Basic definitions}

  \textbf{\alert{Node degree} = Number of links per node}
    
     Notation: Node $i$'s degree = $k_i$.
     $k_i$ = 0,1,2,\ldots.
     Notation: the average degree of a network = $\avg{k}$ \\
      {(and sometimes $z$)}
    
      Connection between number of edges $m$ and average degree:
      $$
      \tavg{k} = \frac{2m}{N}.
      $$
    
      \alertb{Defn:} ${\cal N}_i$ = the set of $i$'s $k_i$ neighbors
    
  


  \textbf{Super Basic definitions}

  \textbf{Adjacency matrix:}
    
    
      We represent a directed network by a 
      matrix $A$ with link weight $a_{ij}$ for nodes $i$ and $j$
      in entry $(i,j)$.
    
      e.g.,
      $$
      A = \left[
        \begin{array}{ccccc}
          0 & 1 & 1 & 1 & 0\\
          0 & 0 & 1 & 0 & 1\\
          1 & 0 & 0 & 0 & 0 \\
          0 & 1 & 0 & 0 & 1 \\
          0 & 1 & 0 & 1 & 0 \\
        \end{array}
      \right]
      $$
    
      (n.b., for numerical work, we 
      always use sparse matrices.)
    
  



%%%%%%%%%%%%%%%%%%%%%%%%%%%%%%%%%%%%%
% examples
%%%%%%%%%%%%%%%%%%%%%%%%%%%%%%%%%%%%%

\section{Examples of Complex Networks}

  \textbf{Examples}

  \textbf{So what passes for a complex network?}
    
     Complex networks are \alert{large} (in node number)
     Complex networks are \alert{sparse} (low edge to node ratio)
     Complex networks are usually \alert{dynamic} and \alert{evolving}
     Complex networks can be social, economic, natural, informational, abstract, ...
    
  



  \textbf{Examples}

  \textbf{Physical networks}
          \column {0.5\textwidth}
        
         River networks
         Neural networks
         Trees and leaves
         Blood networks
        

      
        
         The Internet
         Road networks
         Power grids
        

      

  \medskip

  \begin{columns}[t]
    
      \includegraphics[height=.28\textheight]{internet_opte_com.png} 
%%      {\centering \tiny (\url{opte.com})}
    
    \includegraphics[height=.28\textheight]{Rivierescr.jpg}
    
    \includegraphics[height=.28\textheight]{BoucleSach_imacr.jpg} 
  
  
   \alert{Distribution} (branching) vs.\ \alert{redistribution} (cyclical)
  


  \textbf{Examples}

      
    \textbf{Interaction networks}
      
        The Blogosphere
        Biochemical networks
        Gene-protein networks
        Food webs: who eats whom
        The World Wide Web (?)
        Airline networks
        Call networks (AT\&T)
        The Media
        Paper citations
      
    
    
    \includegraphics[width=\textwidth]{datamining-core-2006-06-27.png}\\
    {\tiny \wordwikilink{http://datamining.typepad.com}{datamining.typepad.com}}
  

  \textbf{Examples}

      
    \textbf{Interaction networks: social networks}
      
       Snogging
       Friendships
       Acquaintances
       Boards and directors
       Organizations %% formal and informal ties
       
        \wordwikilink{http://www.myspace.com}{myspace.com}, 
        \wordwikilink{http://www.facebook.com}{facebook.com}
      
    
    
    \includegraphics[width=\textwidth]{bearman_sex_network.jpg}\\
    {\tiny (Bearman \etal, 2004)} 
  
  
  
  `Remotely sensed' by:
  email activity, 
  instant messaging, 
  phone logs {\alert{(*cough*)}}.
  


  \textbf{Examples}

    \includegraphics[width=\textwidth]{bearman_sex_network.jpg}\\


  \textbf{Examples}

  \textbf{Relational networks}
    
     Consumer purchases \\
      {(Wal-Mart: $\approx 1 \ \mbox{petabyte} \ = 10^{15} \ \mbox{bytes}$)}
     Thesauri: Networks of words generated by meanings
     Knowledge/Databases/Ideas
     Metadata---Tagging:
      \wordwikilink{http://del.icio.us}{del.icio.us},
      \wordwikilink{http://www.flickr.com}{flickr}
    
  
      
    \includegraphics{delicious.pdf}
  

  \textbf{Clickworthy Science:}

    \includegraphics[height=0.75\textheight]{bollen2009a_fig5.pdf}\\
    {\small Bollen et al.\cite{bollen2009a}}


%%%%%%%%%%%%%%%%%%%%%%%%%%%%%%%%%%%%%
% properties
%%%%%%%%%%%%%%%%%%%%%%%%%%%%%%%%%%%%%

\section{Properties of Complex Networks}

  

  \textbf{A notable feature of large-scale networks:}
    
    
      Graphical renderings are often just a big mess.
              
        
                  
          \includegraphics[height=\textwidth]{nw_purerandom_graphviz01_10}
          
          
          [] 
            $\Leftarrow$ Typical hairball
           
            number of nodes $N$ = 500
           
            number of edges $m$ = 1000
           
            average degree $\tavg{k}$ = 4
          
                  
      And even when renderings somehow look good:\\
      {
      \alertb{``That is a very graphic analogy which aids 
      understanding wonderfully while being,
      strictly speaking, wrong in every possible way''} \\
      {\small
      said Ponder [Stibbons]
      ---\textit{Making Money}, T. Pratchett.
      }
      }
     
      We need to extract \alert{digestible, meaningful aspects}.
    
  


  \textbf{Properties}

  \textbf{Some key features of real complex networks:}
      
    
    
     Degree distribution
     Assortativity
     Homophily
     Clustering
     Motifs
     Modularity
    
    
    
     Concurrency
     Hierarchical scaling
     Network distances
     Centrality
     Efficiency
     Robustness
    
    
    

  
   Coevolution of network \alertb{structure}
    \\ and \alertb{processes} on networks.
  


  \textbf{Properties}

  \textbf{1. Degree distribution $P_k$}
    
    
      $P_k$ is the probability that a randomly selected
      node has degree $k$
    
      \alertb{Big deal:} Form of $P_k$ key to network's behavior
    
      \alert{ex 1:}
      \erdosrenyi\ random networks have a Poisson
      distribution:
      $$ P_k = e^{-\tavg{k}} \tavg{k}^k / k! $$
    
      \alert{ex 2:}
      \alertb{``Scale-free'' networks:}
      $P_k \propto k^{-\gamma}$ $\Rightarrow$ `hubs'
    
      We'll come back to this business soon...
    
  


%% %%   \textbf{Properties}
%% 
%%   \textbf{1. Degree distribution $P_k$}
%%     
%%     
%%       \alert{ex 2:}
%%       \alert{``Scale-free'' networks:}
%%       $P_k \propto k^{-\gamma}$ $\Rightarrow$ `hubs'
%%     
%%       Link cost controls skew
%%     
%%       Hubs may facilitate or impede contagion
%%     
%%   
%%  
%% 

  \textbf{Properties}

  \textbf{2. Assortativity/3. Homophily:}
    
     Social networks: \wordwikilink{http://en.wikipedia.org/wiki/Homophily}{Homophily} = birds of a feather
     e.g., degree is standard property for sorting:\\
      measure degree-degree correlations.
    
      \alert{Assortative} network:\cite{newman2002a} 
      similar degree nodes connecting to each other.\\
      
      
      {\textit{Often \alertb{social}: company directors, coauthors, actors.}}
      
    
      \alert{Disassortative} network: high degree nodes connecting to low degree nodes.\\
      
       
      {\textit{Often \alertb{techological} or \alertb{biological}: 
        Internet, protein interactions, neural networks, food webs.}}
      
    
  


  \textbf{Properties}

  \textbf{4. Clustering:}
    
     Your friends tend to know each other.
     Two measures:
      $$ C_1 = \avg{\frac{\sum_{j_1 j_2 \in {\cal N}_i} a_{j_1 j_2}}{k_i(k_i-1)/2}}_{i} 
      \mbox{\ due to Watts \& Strogatz\cite{watts1998a}}
      $$  
      $$ C_2 = \frac{3 \times \rm\#triangles}{\rm \#triples} 
      \mbox{\ due to Newman\cite{newman2003a}}
      $$ 
     $C_1$ is the \alert{average fraction} 
      of 
      \alertb{pairs of neighbors} who are \alertb{connected}.
    
      Interpret $C_2$ as probability two of a node's friends
      know each other.
    
  


%% %%   \textbf{Properties---First clustering measure:}
%% 
%%   
%%    $C_1$ is the \alert{average fraction} 
%%     of 
%%     \alertb{pairs of neighbors} who are \alertb{connected}.
%%    Fraction of pairs of neighbors who are connected is
%%     $$ \frac{\sum_{j_1 j_2 \in {\cal N}_i} a_{j_1 j_2}}{k_i(k_i-1)/2} $$
%%     where
%%     $k_i$ is node $i$'s degree, and 
%%     ${\cal N}_i$ is the set of $i$'s neighbors.
%%   
%%     Averaging over all nodes, we have
%%     $$ C_1 = \frac{1}{n}{\sum_{i=1}^{n}\frac{\sum_{j_1 j_2 \in {\cal N}_i} a_{j_1 j_2}}{k_i(k_i-1)/2}} 
%%     { = \avg{\frac{\sum_{j_1 j_2 \in {\cal N}_i} a_{j_1 j_2}}{k_i(k_i-1)/2}}_{i} }$$  
%%    For sparse networks, $C_1$ tends to discount
%%     highly connected nodes.
%%   
%% 
%% %% 
%% 
%% %%   \textbf{Properties---Triples and triangles:}
%% 
%%   
%%   
%%     \alertb{Defn:} Nodes $i_1$, $i_2$, and $i_3$ form a \alert{triple}
%%     around $i_1$ if $i_1$ is connected to $i_2$ and $i_3$.
%%   
%%     \alertb{Defn:} Nodes $i_1$, $i_2$, and $i_3$ form a \alert{triangle}
%%     if each pair of nodes is connected
%%    
%%    The definition
%%     $$ C_2 = \frac{3 \times \rm\#triangles}{\rm \#triples} $$ 
%%     measures the fraction of \alertb{closed triples}.
%%    
%%     The \alert{`3'} appears because for each triangle, \\
%%     we have 3 closed triples.
%%   
%%     Interpret $C_2$ as probability two of a node's friends
%%     know each other.
%%    Social Network Analysis (SNA): fraction of
%%     \alertb{transitive triples}.
%%    In general, \alert{$C_1 \ne C_2$}.
%%   
%% 
%% 
  \textbf{Properties}

  \textbf{5. Motifs:}
  
   
    Small, recurring functional subnetworks 
  
    e.g., Feed Forward Loop:
          
      \begin{center}
        \includegraphics[width=0.45\textwidth]{feedforwardloop}%
      \end{center}
        Shen-Orr, Uri Alon, \etal\cite{shen-orr2002a}
  %% , Wiggins \etal
  
  


  \textbf{Properties}

  \textbf{6. modularity:}
    \begin{center}
      \begin{tabular}{c}
        \includegraphics[height=0.6\textheight]{ncaa_annotated}\\
        Clauset \etal, 2006\cite{clauset2006a}: NCAA football
      \end{tabular}
    \end{center}
  


  \textbf{Properties}

  \textbf{7. Concurrency:}
    
     
      Transmission of a contagious element
      only occurs during contact\cite{kretzschmar1996a}
     
      Rather obvious but easily missed in a simple model
     
      Dynamic property---static networks are not enough
     
      Knowledge of previous contacts crucial
     
      \alert{Beware} cumulated network data!
    
  


  \textbf{Properties}

  \textbf{8. Horton-Strahler stream ordering:}
    
     Metrics for branching networks:
      
       
        Method for ordering streams hierarchically
      
        Reveals fractal nature of natural branching networks
      
        Hierarchy is not pure but mixed (Tokunaga).\cite{tokunaga1966a,dodds1999a}
      
        Major examples: rivers and blood networks.
        %%      Number: $R_n = N_{\omega}/N_{\omega+1}$ 
        %%      
        %%      Segment length: $R_l = \tavg{l_{\omega+1}}/\tavg{l_{\omega}}$ 
        %%      
        %%      Area/Volume: $R_a = \tavg{a_{\omega+1}}/\tavg{a_{\omega}}$ 
      
    
  
      
    \begin{center}
      \includegraphics[height=0.4\textheight]{network1a}%
      \includegraphics[height=0.4\textheight]{network2b} 
      \includegraphics[height=0.3\textheight]{network3c}%
    \end{center}
    
   \alertb{Beautifully described} but \alert{poorly explained}.
  


  \textbf{Properties}

  \textbf{9. Network distances:}
    \textbf{\alert{(a) shortest path length $d_{ij}$:}}
      
       Fewest number of steps between nodes $i$ and $j$.      
       (Also called the chemical distance between $i$ and $j$.)
      
    
    \textbf{\alert{(b) average path length $\tavg{d_{ij}}$:}}
      
        Average shortest path length in whole network.
        
        Good algorithms exist for calculation.
       
        Weighted links can be accommodated.
      
    

  


  \textbf{Properties}

  \textbf{9. Network distances:}
    \textbf{\alert{(c) Network diameter $d_{\rm max}$:}}
      
      
        Maximum shortest path length in network.
      
    
    \textbf{\alert{(d) Closeness $d_{\rm cl} = [\sum_{ij} d_{ij}^{\ -1} / \binom{n}{2}]^{-1}$:}}
      
       
        Average `distance' between any two nodes.
      
        Closeness handles disconnected networks ($d_{ij}=\infty$)
      
        $d_{\rm cl} = \infty$ only when all nodes are isolated.
%%      
%%        Closeness perhaps compresses too much into one number.
      
    
  


  \textbf{Properties}

  \textbf{10. Centrality:}
    
     Many such measures of a node's `importance.'  
     \alert{ex 1:} Degree centrality: $k_i$.
     \alert{ex 2:} Node $i$'s betweenness \\
      = fraction of shortest paths that pass through $i$.
     \alert{ex 3:} Edge $\ell$'s betweenness \\
      = fraction of shortest paths that travel along $\ell$.
     \alert{ex 4:} Recursive centrality: Hubs and Authorities
      (Jon Kleinberg\cite{kleinberg1998a})
    
    
  





\section{Nutshell}

\begin{frame}[label=]
  \textbf{Nutshell:}

  \textbf{Overview Key Points:}
    
    
      The field of complex networks came into
      existence in the late 1990s.
    
      Explosion of papers and interest since 1998/99.
    
      Hardened up much thinking about complex systems.
    
      Specific focus on networks that are 
      \alert{large-scale}, 
      \alertb{sparse}, 
      \alert{natural} or \alert{man-made}, 
      \alertb{evolving} and \alertb{dynamic}, 
      and 
      (crucially) \alert{measurable}.
    
      Three main (blurred) categories: 
      
       
      \alert{Physical} (e.g., river networks),
       
      \alert{Interactional} (e.g., social networks),
       
      \alert{Abstract} (e.g., thesauri).
      
    
    
  


\begin{frame}[label=]
  \textbf{Nutshell:}

  \textbf{Overview Key Points (cont.):}
    
    
      Obvious connections with the vast
      extant field of graph theory.
    
      But focus on dynamics is more of a physics/stat-mech/comp-sci
      flavor.
    
      Two main areas of focus:
      
       
        \alertb{Description:} Characterizing very large networks
      
        \alertb{Explanation:} Micro story $\Rightarrow$ Macro features
      
    
      Some essential structural aspects are understood: degree distribution, clustering,
      assortativity, group structure, overall structure,...
    
      Still much work to be done, especially with respect to dynamics...
    
    
  

