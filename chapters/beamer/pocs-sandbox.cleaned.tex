

%% \adverstisement{Broccoli}{}{link}{figure}

%% for assignments
%% for each one, add some gephi action
%% for exploring what they're computing
%% give them a few networks to play around with
%% for each assignment.


  \textbf{Key element for division approach:}
    
     
      Recomputing betweenness.
     
      \alert{Reason:} Possible to have a low betweenness
      in links that connect large communities
      if other links carry majority of shortest paths.
    
  

  \textbf{When to stop?:}
    
     
      How do we know which
      divisions are meaningful?
     
      \alert{Modularity measure:}
      difference in fraction of within component
      nodes to that expected for randomized version:\\
      \smallskip
      {
      $
      Q = 
      \sum_{i}
      [e_{ii} - a_{i}^2]
      $\\
      \smallskip
      where $e_{ij}$ is the fraction of (undirected) edges
      travelling between identified communities $i$ and $j$,
      and $a_i = \sum_{j}e_{ij}$ is the fraction of edges with
      at least one end in community $i$.
    }
    
  


  \textbf{Hierarchy by division}

  \textbf{Key element:}
    
     
      Recomputing betweenness.
     
      \alert{Reason:} Possible to have a low betweenness
      in links that connect large communities
      if other links carry majority of shortest paths.
    
  

  \textbf{When to stop?:}
    
     
      How do we know which
      divisions are meaningful?
     
      \alert{Modularity measure:}
      difference in fraction of within component
      nodes to that expected for randomized version:\\
      \smallskip
      {
      $
      Q = 
      \sum_{i}
      [e_{ii} - a_{i}^2]
      $\\
      \smallskip
      where $e_{ij}$ is the fraction of (undirected) edges
      travelling between identified communities $i$ and $j$,
      and $a_i = \sum_{j}e_{ij}$ is the fraction of edges with
      at least one end in community $i$.
    }
    
  



  \textbf{Modularity measure:}
    
     
      Define 
    

    $$  
    Q = 
    \sum_{i}
    [e_{ii} - (\sum_j e_{ij})^2]
    =
    {\textnormal{Tr}} \textmatrix{E} - || \textmatrix{E}^2 ||_1,
    $$
  



\begin{comment}



  Structure detection:

  From Lipari summer school:

%%

jure leskovec:

bipartite affiliation graph model

community detection

affiliation graph model

%%%

Add Louvain method

Limits of modularity:

Can fail for random graphs.

Spinglass models

Look at fluctuations of modularity.

Resolution limit
Fortunato and Barthelemy, 2007

Infomap (Rosvall and Bergstrom, 2008)
Random walks and geometric maps


Methods for finding overlapping communities.

Clique percolation method.

How to test clustering algorithms.

LFR benchmark
lancichinetti et al 2008

Mutual information

Danon et al. (2005) using GN benchmark


%% Get this to work:
%%  \pdfmouseovercomment{Does this work?}{Beep}


  Include Romu's famous paper

  Add Callaway to random networks
  Point out growing mechanism.

  Move Newman's generating function madness to 
  the end of assortativity.

  Add our work?

%%   \displaypaper{}

  Glock's paper


%%   \displaypaper{}

  River network piracy

  Add to branching networks 2


\wordwikilink{http://amaznode.fladdict.net/}{Fun with amazon's recommender system}.

[amaznode.fladdict.net]
  


  \wordwikilink{http://www.ted.com/index.php/talks/blaise_aguera_y_arcas_demos_photosynth.html}{Photosynth}.


  Spreaders?

\url{http://dish.andrewsullivan.com/2014/03/20/the-conspiratorial-sort/}


  
  %% show examples of how the following works

  \textbf{Some extra thinking:}
    
     
      (
    
    
  
  

  $ \mbox{$x^k$ piece of $\left( \sum_{i'=0}^\infty V_{i'}x^{i'}\right)^j$}$




Harold Ramis

``What did you do Ray?''

It's the Stay Puft Marshmallow Man

Connect Ghostbusters to kid's showing self control
over marshmallows.




  Random network applications

  Add code to site, github.

Random network applications
  Piece on Ugander 2013a

  Types of motifs possible.






\section{Nutshell}

\begin{frame}[label=]
  \textbf{Nutshell:}

  \textbf{Optimal supply networks:}
    
     Be
    
    
    
    
    
  




  Ugander's

  Structural Diversity work.

  Beautiful.


  Add page on 
  
  ``Virality Prediction and Community Structure in Social Networks''

  \cite{weng2013a}

  Weng, Men?er, Ahn


  ``Subgraph Frequencies: Mapping the Empirical and Extremal Geography of
  Large Graph Collections''

  To add to Motifs

  Ugander: Neighborhood graphs.

  Triad space.

  Triadic closure (Rapaport)

  Suppression of certain graph types.

  Extremal graph theory.

  Razborov, flag albebras, 2007.

  Not a lot of open triads.

  \cite{ugander2013a}


  
For optimal supply networks
Metastable states in random media.  Optimal paths.
P. Jogi and Sornette, 1998, Phys Rev. E 57, 6931--6943.
 
 \cite{xie2007a}
 Geographical networks evolving with an optimal policy


%% For random networks:

\subsection{Simple analysis}









%% build course map figures

  \textbf{Weighted networks}

  %% Barrat et al. 2004
  

http://prl.aps.org/abstract/PRL/v112/i4/e048701
James P. Gleeson1,*, Jonathan A. Ward2, Kevin P. O’Sullivan1, and William T. Lee1
1MACSI, Department of Mathematics and Statistics, University of Limerick, Limerick, Ireland
2Centre for the Mathematics of Human Behaviour, Department of Mathematics and Statistics, University of Reading, Whiteknights RG6 6AH, United Kingdom
Selected for a Viewpoint in Physics Published 30 January 2014; received 31 May 2013
See accompanying Physics Synopsis
Heavy-tailed distributions of meme popularity occur naturally in a model of meme diffusion on social networks. Competition between multiple memes for the limited resource of user attention is identified as the mechanism that poises the system at criticality. The popularity growth of each meme is described by a critical branching process, and asymptotic analysis predicts power-law distributions of popularity with very heavy tails (exponent α<2, unlike preferential-attachment models), similar to those seen in empirical data.

%% Add a drawn picture for "and in detail" section

%% Get into hierarchies.

For branching networks:
  
Cite woldenberg1966a
\cite{woldenberg1966a}
Allometry, Herbert Simon!

Describe process of network growth...
Extract pictures from
\cite{glock1931a}

Precipitation of DEMS
\cite{gregory1994a}

Add page on Caldarelli 1997
``Randomly pinned landscape evolution''




add slide about galactic network
to intro
  

%% weighted networks
%% Barrat et al. 2004

  


%% far lands or bust
%% \insertvideo{teU1nRc0bso}{980}{1100}{Crafting landscapes}
%% far lands or bust
%% \insertvideo{7uBRfjlv480}{1410}{1530}{Crafting landscapes}


\insertvideo{lFg21x2sj-M}{}{}{Ant colony structure}

\insertvideo{1IugvemOyZY}{}{}{Ant colony structure}




      The \syllabuslink{\coursewebsite/docs/\courseprefix}.  



%% To do:
%% 
%% stories as theme
%% 
%% outline
%%
%% make a poster (movie like)
%% 
%% logo
%% 
%% New theme for website
%%
%% script for putting videos up
%% 
%% add more videos throughout
%% 
%% sketch outline of video shell
%% tape images on
%% 
%% Lada Adamic

%% london biking



 



  Supernatural clip:

  Clowns, S7 or S8?

  Terrifying Octopus.


  Networks are a mess.
  
  Join-the-dots gag.
  
  

  Katz Centrality



  \textbf{Maximum flow problem:}

  To add to supply networks...
  
  \wordwikilink{http://www.technologyreview.com/article/26869/}{http://www.technologyreview.com/article/26869/}


\section{Beep}

  \textbf{Where is Barry?}

    \includegraphics[width=\textwidth]{datamining-core-2006-06-27-tp-10.pdf}


  \textbf{Yes...}

  
   
    Weird
  


\subsection{Sub Beep 1}

  \textbf{Yes...}

  
   
    Weird
  


\subsection{Sub Beep 2}

  \textbf{Yes...}

  
   
    Weird
  


\section{Random\ walks\ on\ networks}

  \textbf{Yes...}

  
   
    Weird
  



\subsection{Sub Beep 1}

  \textbf{Yes...}

  
   
    Weird
  


\renewcommand{\insertlecturelogo}{
      \includegraphics[height=0.1\textheight]{seal4.pdf}
      X
}


\subsection{Sub Beep 2}

  \textbf{Yes...}

  
   
    Weird
  


\renewcommand{\insertlecturelogo}{
      \includegraphics[height=0.1\textheight]{icons-lightbulb2.pdf}
}

%% %% 
%% Next time: add in time to equilibrium, max lambda, noh2006a.pdf.
%% 
%% Also:
%% 
%% Laplacian matrix
%% 
%% Admittance matrix
%% 
%% Similar to a shifted version of the Laplacian
%% 
%% Kirchoff's laws
%% 
%% Connection to transportation problems
%% 
%% Electricity
%%   
%% 
  \textbf{Random walks on networks---basics:}

  
  
    Imagine a single random walker moving
    around on a network.
   
    At $t=0$, start walker at node $j$ and 
    take time to be discrete.
   
    \alert{Q:} What's the long term probability distribution
    for where the walker will be?
   

    Define \alertb{$p_i(t)$} as the probability
    that at time step $t$, our walker is at node $i$.
   
    We want to characterize the evolution
    of $\vec{p}(t)$.
   
    First task: connect $\vec{p}(t+1)$ to $\vec{p}(t)$.
   
    {Let's call our walker \alert{Barry}.}
   
    {Unfortunately for Barry,
      he lives on a high dimensional graph and is far from home.}
  
    {Worse still: Barry is \alertb{hopelessly drunk}.}
  


\renewcommand{\insertlecturelogo}{
  \includegraphics[height=0.1\textheight]{scrabbleletters2-tp.pdf}
}

  \textbf{Where is Barry?}

  
  
    Consider simple directed, ergodic (strongly connected) networks.
  
    As usual, represent network by \alert{adjacency matrix $A$}
    where
    $$
    \begin{array}{l}
      a_{ij}=1 \ \mbox{if $i$ has an edge leading to $j$}, \\
      a_{ij}=0 \ \mbox{otherwise.}
    \end{array}
    $$
  
    Barry is at node $i$ at time $t$ with probability $p_i(t)$.
  
    In the next time step he 
    \alertb{randomly lurches} toward one of $i$'s neighbors.
  
    Equation-wise:
    $$
      p_i(t+1) = \sum_{j=1}^{n} \frac{1}{k_i}  a_{ji} p_j(t).
    $$
    where $k_i$ is $i$'s degree.
    {Note: $k_i = \sum_{j=1}^{n} a_{ij}$.}
  


  \textbf{Where is Barry?}

  
  
    Linear algebra-based excitement:
    $
    p_i(t+1) = \sum_{j=1}^{n} a_{ji} \frac{1}{k_j} p_j(t)
    $
    is more usefully viewed as
    $$
    \vec{p}(t+1) 
    = 
    A^{\textnormal{T}} K^{-1}
    \vec{p}(t) 
    $$
    where $[K_{ij}] = [\delta_{ij} k_i]$ 
    has node degrees on the main diagonal
    and zeros everywhere else.
  
    So... we need to find the \alert{dominant eigenvalue} 
    of $A^{\textnormal{T}} K^{-1}$.
  
    Expect this eigenvalue will be 1 (doesn't make sense
    for total probability to change).
  
    The corresponding eigenvector will be the limiting
    probability distribution (or invariant measure).
  
    Extra concerns: multiplicity of eigenvalue = 1,
    and network connectedness.
  


  \textbf{Where is Barry?}

  
  
    By inspection, we see that
    $$
    \vec{p}(\infty) = \frac{1}{\sum_{i=1}^{n} k_i} \vec{k}
    $$
    satisfies
    $
    \vec{p}(\infty)
    = 
    A^{\textnormal{T}} K^{-1}
    \vec{p}(\infty)
    $
    with eigenvalue 1.
  
    We will find Barry at node $i$ with probability
    proportional to its degree $k_i$.
  
    Nice implication: probability of finding Barry travelling along
    any edge is \alert{uniform}.
  
    Diffusion in real space smooths things out.
  
    On networks, uniformity occurs on edges.
  
    So in fact, diffusion in real space is \alertb{about the edges too}
    but we just don't see that.
  


  \textbf{Other pieces:}

  
  
    Goodness: $A^{\textnormal{T}} K^{-1}$ is similar to a real symmetric matrix
    if $A = A^{\textnormal{T}}$.
  
    Consider the transformation $M = \alertb{K^{-1/2}}$:
    $$
    \alertb{K^{-1/2}}
    \alert{A^{\textnormal{T}} K^{-1}}
    \alertb{K^{1/2}}
    =
    \alertb{K^{-1/2}}
    \alert{A^{\textnormal{T}}}
    \alertb{K^{-1/2}}.
    $$
  
    Since $A^{\textnormal{T}} = A$, we 
    have 
    $$
    ({K^{-1/2}}
    {A}
    {K^{-1/2}})^{\textnormal{T}}
    =
    {K^{-1/2}}
    {A}
    {K^{-1/2}}.
    $$
  
    Upshot: $A^{\textnormal{T}} K^{-1} = A K^{-1}$ has real eigenvalues and a complete
    set of orthogonal eigenvectors.
  
    Can also show that maximum eigenvalue magnitude is indeed 1.
  
    Other goodies: next time round.
  




\end{comment}
%%  ``Experimental evidence of massive-scale emotional contagion through
%%  social networks''

  \displaypaper{kramer2014a}{1}
  
  Somewhat creeptastic.


Vaccine

Difficulty of changing opinions

\url{http://www.newyorker.com/online/blogs/mariakonnikova/2014/05/why-do-people-persist-in-believing-things-that-just-arent-true.html}




Rosencrantz and Guildenstern are Dead  
Coin flipping:

https://www.youtube.com/watch?v=NbInZ5oJ0bc



%% For scaling:
  
Sornette
How Much is the Whole Really More than the Sum of its Parts? 1 + 1 = 2.5: Superlinear Productivity in Collective ...
arxiv.org/abs/1405.4298



  Make a 2/3 movie with students from film.

  2/3 in a hospital bed.

  Hearsay, the word.

1525–35; from phrase by hear say, translation of Middle French par ouïr dire.

  Money 
  xkcd 

  Add to scaling.


%% make composite videos

%% happiness is for the weak
%% http://thecolbertreport.cc.com/videos/nocpjv/jennifer-senior?xrs=share_copy_email



%% make this work (for social contagion):
%% \insertlocalvideo{2014-04-16scrubsS1E21contagious-smile-perry-cox.mp4}{}{}{}
%% also do for john stewart




\wordwikilink{Saturn's hexagon}{http://en.wikipedia.org/wiki/Saturn\'s\_hexagon}
  

  hexnet.org

  Global Hexagon Awareness
  hexagon love


  Add slides on 1/f noise, etc.

  Saturated fat isn't a problem.
http://mobile.nytimes.com/blogs/well/2014/03/17/study-questions-fat-and-heart-disease-link/?hp=

  
  Success of bad things:

  Why Flappy Bird Had to Die
  \url{http://www.slate.com/articles/technology/gaming/2014/02/flappy_bird_online_game_why_dong_nguyen_had_to_kill_his_perfect_creation.html}


  Add Doyle 2005a
  as a refutation of 
  Barabasi network failure model
  

  Add Stradivarius to Mona Lisa story

\url{http://www.nytimes.com/2014/01/31/us/a-violinists-triumph-is-ruined-by-thieves.html?hp}

  Much of art is held in people's heads.


add milgram's video, an excerpt ...

add a little of ghostbusters ...
very start, electrocution.
  


  The role of universality and accidents not being understood.
  
  \url{https://www.sciencenews.org/blog/context/gell-mann-hartle-spin-quantum-narrative-about-reality}


  
Douglas Adams

The world is a thing of utter inordinate complexity and richness and
strangeness that is absolutely awesome. I mean the idea that such
complexity can arise not only out of such simplicity, but probably
absolutely out of nothing, is the most fabulous extraordinary
idea. And once you get some kind of inkling of how that might have
happened, it's just wonderful. And ... the opportunity to spend 70 or 80
years of your life in such a universe is time well spent as far as I
am concerned.

Response to the question "What is it about science that really gets
your blood running?" — as quoted in Richard Dawkins in his eulogy for
Adams (17 September 2001)

\url{http://en.wikiquote.org/wiki/Douglas_Adams}

If you try and take a cat apart to see how it works, the first thing
you have on your hands is a nonworking cat. Life is a level of
complexity that almost lies outside our vision; it is so far beyond
anything we have any means of understanding that we just think of it
as a different class of object, a different class of matter; 'life',
something that had a mysterious essence about it, was God given, and
that's the only explanation we had. The bombshell comes in 1859 when
Darwin publishes On the Origin of Species. It takes a long time before
we really get to grips with this and begin to understand it, because
not only does it seem incredible and thoroughly demeaning to us, but
it's yet another shock to our system to discover that not only are we
not the centre of the Universe and we're not made by anything, but we
started out as some kind of slime and got to where we are via being a
monkey. It just doesn't read well.


Life's Complexity Pyramid\cite{oltvai2002b}
  Oltvai and Barab\'{a}si

  \includegraphics[height=0.7\textheight]{oltvai2002b_fig1}
  \includegraphics[height=0.7\textheight]{oltvai2002b_fig1_caption}

  
Map of the course

Show Indiana Jones
Muppets travel by map


%% overview slides:


%% elements of babies
%% Subject: Born Wet, Human Babies Are 75 Percent Water. Then Comes Drying : Krulwich Wonders... : NPR
%% 
%% http://www.npr.org/blogs/krulwich/2013/11/26/247212488/born-wet-human-babies-are-75-percent-water-then-comes-dryin
%% g?ft=1&f=5500502



%% prereq:
%% stats, diff eqs, lineary algebra, calc
%%
%% add poincare
%%
%% bruce west's book
%% complexity figure 
%%
%% taken's theory (complexity theory)
%% 
%% coupled systems: synchronization






\begin{comment}

%% For scaling
%% turbulence

%% 1/12 = Judas fraction
%% 3 out of 4 points on the cross
%% pyramids

%% add my interview with Whitman

%% Clauset's blue whale work

record voiceover for truthicide investigation

\url{http://www.sciencemag.org/content/340/6139/1438}

Big science versus little science:
Scaling with funding:
\url{http://www.plosone.org/article/info:doi\%2F10.1371\%2Fjournal.pone.0065263}


The Origins of Scaling in Cities

%% this course will explode in ...

%% lecture logo options
%% \changelogo{.18}{boston-red-sox-tp-1.pdf}


  

\includemedia[
width=1\linewidth,height=0.5625\linewidth,
activate=pageopen,
flashvars={
}
]
{}
{http://www.thedailyshow.com/watch/tue-september-18-2007/alan-greenspan}

%% {http://www.thedailyshow.com/video/index.jhtml?videoId=102970\&title=alan-greenspan}


%% flashvars={
%%   modestbranding=0 % no YT logo in control bar
%%   &autoplay=1
%%   &fs=1
%%   &autohide=1 % controlbar autohide
%%   &showinfo=0 % no title and other info before start
%%   &rel=0
%%   &start=#2
%%   &end=#3
%%       %   &rel=0 % no related videos after end
%% }


%% http://www.thedailyshow.com/watch/tue-september-18-2007/alan-greenspan

\changelogowithlink{.18}{http://xkcd.com/793/}{xkcd-793-physicists.png}

  
Paraprosdokian:

"Figure of speech in which the latter part of a sentence or phrase is
surprising or unexpected; frequently used in a humorous situation."


1. Do not argue with an idiot. He will drag you down to his level and beat you with experience.
2. The last thing I want to do is hurt you. But it's still on my list.
3. Light travels faster than sound. This is why some people appear bright until you hear them speak.
4. If I agreed with you, we'd both be wrong.
5. We never really grow up, we only learn how to act in public.
6. War does not determine who is right - only who is left.
7. Knowledge is knowing a tomato is a fruit. Wisdom is not putting it in a fruit salad.
8. Evening news is where they begin with 'Good Evening,' and then proceed to tell you why it isn't.
9. To steal ideas from one person is plagiarism. To steal from many is research.
10. A bus station is where a bus stops. A train station is where a train stops. On my desk, I have a work station.




%% Yawning:
%% http://blogs.scientificamerican.com/thoughtful-animal/2013/10/16/searching-for-the-social-in-contagious-yawning/



%% link to fowler and christakis arxiv paper


%% role of strong ties
%% http://www.technologyreview.com/view/512951/how-strong-social-ties-hinder-the-spread-of-rumours/

%% good news travels fast
%% http://www.nytimes.com/2013/03/19/science/good-news-spreads-faster-on-twitter-and-facebook.html?ref=science


%% add tedx stuff!
%% 
%% Jake's tweeting:
%% http://research.microsoft.com/en-us/um/redmond/events/techfest2013/projects.aspx#224
%% http://www.technologyreview.com/view/512271/researchers-peek-at-the-structure-of-the-viral-internet/
%% http://www.youtube.com/watch?feature=player_embedded&v=wSwOszoHuoI#!
%% 
%% From here:
%% 
%% http://www.economist.com/news/science-and-technology/21572159-data-social-networks-are-making-social-science-more-scientific-dr-seldon-i?fsrc=scn/tw_ec/dr_seldon_i_presume
%% 
%% Politics, too, is falling to the new psychohistorians. Boleslaw Szymanski of the Rensselaer Polytechnic Institute in New York state studies how societies change their collective minds. By studying simulated networks of people he can predict the point at which a committed minority can convert almost everyone else to its way of thinking. For an idealised model, the size of this catalytic minority is just under 10\%. Tweaking the model with data from real networks such as Twitter and Facebook, he hopes, will allow these insights to be applied to the real world.
%% 
%% 
%% ugander2012a
%% http://www.pnas.org/content/109/16/5962
%% 
%% romero2011/2012a
%% 
%% Trauma:
%% http://andrewsullivan.thedailybeast.com/2013/01/trauma-is-a-contagious-disease.html
%% 
%% Add bakshy2011a
%% 
%% Competition among memes in a world with limited attention
%% L. Weng1, A. Flammini1, A. Vespignani2,3,4 & F. Menczer1
%% 
%% Everyone's an Influencer: Quantifying Influence on Twitter
%% 
%% viralspread
%% jake hofman




  ``What''

  $x = y$

  $$\int_{z=4}^{\infty} z^3 \mbox{d} z.$$

\changelogotovideo{HQe2f8L-Rkc}{0}{55}


  
  
  \faAdjust

  \faExternalLink

  \wordwikilink{http://bit.ly/VdbsWU}{\protect G\"{o}del's Theorem}  

\wordwikilink{http://www.nytimes.com/2013/08/20/nyregion/prosecutors-expect-more-arrests-in-art-fraud-scheme.html}{Prosecutors
  Are Contemplating More Arrests in \$80 Million Art Fraud Case}


\switchbackground{london-apocalypse.jpg}



\end{comment}      


\begin{comment}
  

%% example
%% \neuralreboot{FEaCfIp9qR4}{}{}{Market much?}
%% \insertvideo{FEaCfIp9qR4}{}{}{Market much?}
%%
%% \youtubevideo{FEaCfIp9qR4}{t1}{t2}


Great branching networks of the multiverse:
\url{http://en.wikipedia.org/wiki/Yggdrasil}

To insert:

Platypus

Read they're made out of meat story.

This one goes to 11.







Contagion:
\url{http://www.nature.com/srep/2013/131018/srep02980/full/srep02980.html}
Higgs Boson
Twitter

``Scaling-laws of human broadcast communication enable distinction between human, corporate and robot {T}witter users},''


Neil's learning stuff
  
http://www.wired.com/dangerroom/2011/07/physicist-i-can-predict-insurgent-attacks-thanks-to-the-red-queen/


%% add notes on moritz2005a.pdf
%% forest fires




  

What witchcraft is Facebook?

Mass hysteria

\url{http://m.theatlantic.com/health/archive/2013/09/what-witchcraft-is-facebook/279499/}

  How bad prediction is ...
  

  
\url{http://www.nytimes.com/2013/10/12/arts/music/ylviss-unlikely-hit-started-as-a-joke.html}

Tried to make a bad song.



  %% include tweets by strogatz ...

  

  
  

%% 
%% 
%% %%   
%% Carina C. Zona @cczona
%% NSA says suspicion of 1 person entitles it to data of all w/in 3
%% degrees of separation. In perspective: that's nearly every person in
%% the US
%% 
%% 
\end{comment}


\begin{comment}
  



add recent studies to list for small worlds
\url{http://www.nytimes.com/2011/11/22/technology/between-you-and-me-4-74-degrees.html?hp}

plus horvitz

The ``six degrees'' concept dates to a 1929 short story, ``Chains,''
in which Frigyes Karinthy, the Hungarian author, suggested that no one
is more than a string of six friends away from any other person.

After Milgram published his famous paper ``The Small World Problem,''
in 1967, the playwright John Guare made ``Six Degrees of Separation,''
the title of a 1990 play that explored Milgram’s premise. And that
gave rise to the parlor game Six Degrees of Kevin Bacon, in which
disparate Hollywood personalities are linked to one another. (Elvis
Presley was in ``Change of Habit'' with Edward Asner; Mr. Asner was in
``J.F.K.'' with Kevin Bacon.)
 

The Facebook paper, titled ``Four Degrees of Separation,'' notes that
Milgram posed both an optimistic interpretation of his findings and a
pessimistic one.

On one hand, it is a startling notion that reaching someone on the
other side of the world takes only a small group of social
connections. On the other hand, Milgram said, the result could also be
evidence of psychological distance: that we were actually, on average,
five ``worlds apart.''
 
 ``From this gloomier perspective,'' the new paper says, ``it is
reassuring to see that our findings show that people are in fact only
four worlds apart, and not five.''


  Scaling from Big Science

%%  \includepaperwithfigure{sugihara2009a}{2a}
  show overall front figure


 revamp network examples
 
   
 

%% put this somewhere...
%% newman paper on first mover advantage


  Add more phase transition examples to overview slides.

  Skewed distributions:

  Baby names

  \url{http://www.babynamewizard.com/voyager\#ms=false\&exact=false}




%% fix up HOT forests---real data





scaling
\url{www.plosone.org/article/info:doi/10.1371/journal.pone.0065774}


%% youtube ref
%% subtitle
%% start time
%% stop time
\neuralreboot{V6c7Vw6R33E}{30}{60}{Ron Swanson's Pyramid of
  Greatness:}


\neuralreboot{}{}{}{And now for something completely different:}


  
  Add Google+ Circles page.


  
  Clauset's measurements of power laws.



  \textbf{Emergence:}

  \textbf{Higher complexity:}
    
     
      Many system scales (or levels) \\ 
      that
      interact with each other.
    
      Potentially much harder to explain/understand.
    
  
  \textbf{Even mathematics:\cite{foote2007a}}
          
      
      \includegraphics[width=\textwidth]{kurt_godel.jpg}
      
      \wordwikilink{http://bit.ly/VdbsWU}{\protect G\"{o}del's Theorem}:\\
      we can't prove every theorem that's true \ldots
      
      \includegraphics[width=\textwidth]{hofst.pdf}
      {\small
        ``G\"{o}del, Escher, Bach''\cite{hofstadter1980a}
%%        Hofstader, 1980.
      }
      

  
    
     
      Suggests a \alertb{strong form of emergence}:
      {
        Some phenomena cannot be analytically deduced
        from elementary aspects of a system.
      }
    
  



  \textbf{Emergence:}

  \textbf{Higher complexity:}
    
     
      Many system scales (or levels) \\ 
      that
      interact with each other.
    
      Potentially much harder to explain/understand.
    
  
  \textbf{Even mathematics:\cite{foote2007a}}
          
      
              
        
        \includegraphics[width=\textwidth]{kurt_godel.jpg}
            
      \wordwikilink{http://bit.ly/VdbsWU}{\protect G\"{o}del's Theorem}:\\
      we can't prove every theorem that's true \ldots
      
              
        
        \includegraphics[width=\textwidth]{hofst.pdf}\\
            {\small
        ``G\"{o}del, Escher, Bach''\cite{hofstadter1980a}
%%        Hoftstader, 1980.
      }
      

  
    
     
      Suggests a \alertb{strong form of emergence}:
      {
        Some phenomena cannot be analytically deduced
        from elementary aspects of a system.
      }
    
  



%% ``Operation Righteous Cowboy Lightning is a go.''

%% projects


  Someone should tackle Bettencourt's latest paper.

Code up the 2-d HOT forest model

Code up the HOT network model.

Are blue whales oddly big


Global E-mail Patterns Reveal "Clash of Civilizations" | MIT Technology Review

http://www.technologyreview.com/view/512116/global-e-mail-patterns-reveal-clash-of-civilizations/


Andrea Rapisarda
ig nobel work!

amancio2013a/b.pdf

serra2012
evolution of music
``Measuring the evolution of contemporary western popular music''

%% doyle2011a

%% add krause, couzin stuff.
%% 5 to 10 leaders needed for example.
%% Crowd experiments (100, 200 people).

add simini2012a
migration


%% baby names
%% 




Small-world: NSA's call on three steps away search (everyone!)

Small-world: Put in link to Phillip Ball story.

Add an insert video showing the transformation of variables thing ...
General plus an example, two videos.


Add Clauset's method for measuring power laws

social contagion:
%% wow
%% http://www.cmswire.com/cms/customer-experience/4-steps-to-executing-a-successful-influencerdriven-wordofmouth-campaign-022232.php?utm_source=MainRSSFeed&utm_medium=Web&utm_campaign=RSS-News



%% where's george
%% http://www.fastcoexist.com/1681677/a-new-map-of-the-us-created-by-how-our-dollar-bills-move#1



\wordwikilink{http://www.nytimes.com/2013/08/20/nyregion/prosecutors-expect-more-arrests-in-art-fraud-scheme.html}{Prosecutors
  Are Contemplating More Arrests in \$80 Million Art Fraud Case}

The fame of great art has an enormous social component.


%% for youtube videos
%% includes frame, everything

%% \youtubevideo{link}{start}{stop}
%% \neuralreboot{link}{start}{stop}{title}

\neuralreboot{-8lDYrvTILc}{}{}{Secrets of the Universe will be revealed}{}{}

%% youtube parameters
%% https://developers.google.com/youtube/player_parameters

\changelogo{0.18}{MacGillivray_William_John_Dory_cut.png}
  
  \textbf{Course mascot:}
    \includegraphics[width=\textwidth]{Silvery_john_dory-tp-1.pdf}

    \alertg{What's the John Dory?}
  

\changelogo{.18}{sealie-lambie-productions.jpg}


  add figure from Godel Escher Bach

%%  \includegraphics[width=0.7\textwidth]{hofst.jpg}

  \includemedia[
  activate=pageopen,
  addresource=audio/doink-doink.mp3,
  flashvars={
    source=audio/doink-doink.mp3
    &autoPlay=1
    &showinfo=0
  },
  transparent
  ]{\color{blue}\framebox[0.4\linewidth][c]{Singing bird}}{APlayer.swf}

\begin{frame}[plain]

  \textbf{Secrets of the Universe will be revealed:}
  
  \includemedia[
  label=audio.theclang,
  width=0.001\linewidth,height=0.001\linewidth,
  activate=pageopen,
  flashvars={
    modestbranding=0 % no YT logo in control bar
    &autoplay=1
    &fs=1
    &autohide=1 % controlbar autohide
    &showinfo=0 % no title and other info before start
    &rel=0
    % &rel=0 % no related videos after end
  }
  ]
  {}
  {http://www.youtube.com/v/-8lDYrvTILc?rel=0}




%%    \includemovie[autoplay,continue,text=click here!]{10pt}{10pt}{audio/The_Clang.mpg}
%
%%    \includemovie[text=click here!,autoplay,continue]{10pt}{10pt}{audio/Law_And_Order_theme.mpg}
%

  My scientific heroes:

  Leonardo da Vinci
  
  Doc Brown

  Walter Bishop

  ``I like porcupines.  They show that God has a sense of humour.''
  

\begin{frame}[plain]

  \textbf{Secrets of the Universe will be revealed:}
  
  \includemedia[
  label=vid.spreadclip,
  width=1\linewidth,height=0.5625\linewidth,
  activate=pageopen,
  flashvars={
    modestbranding=0 % no YT logo in control bar
    &autoplay=1
    &fs=1
    &autohide=1 % controlbar autohide
    &showinfo=0 % no title and other info before start
    &rel=0
    % &rel=0 % no related videos after end
  }
  ]
  {\includegraphics[width=1\linewidth]{videos/2012/2012-11-03testing-cut.jpg}}
  {http://www.youtube.com/v/tcRudblV-eM?rel=0}





Also here:

\url{http://thautcast.com/drupal5/content/rescuing-autism-pseudo-science-part-one-simon-baron-cohen}


  Wealth distribution in NYC.
  \url{http://www.huffingtonpost.com/2013/08/16/wealth-inequality-new-york_n_3764217.html}

\end{comment}


\begin{comment}
  

%% %%   \textbf{Spreading:}
%% 
%%     
%%     
%%       Example: Spreading of buildings in the U.S.:
%%     
%% 
%%   \begin{center}
%%     \includemovie[
%%     controls=true,
%%     toolbar=true,
%%     poster=walmartspread-frame.jpg,
%%     ]{100mm}{75mm}{videos/2010/walmartspread.mp4}
%%   \end{center}
%% 
%% 


  More on spreading (religion):
  \url{http://www.bdkeller.com/2013/08/spreading-the-word/}

Add to centrality measures paper's mentioned
by John Ugander.

  Peter Sheridan Dodds @peterdodds
RT @scott_bot: History of how people illustrate the structure of
knowledge: \url{scottbot.net/HIAL/?p=38807}

Tree of knowledge stuff



HBR 
If You Want to Raise Prices, Tell a Better Story
\url{http://t.co/WnMXqP34M5}

  \textbf{Pigeons ...}
  
  Context-dependent hierarchies in pigeons

  Success:

  Fame = talk.

  Why innovations don't catch on.
\url{http://www.newyorker.com/reporting/2013/07/29/130729fa_fact_gawande?currentPage=all}

  Religion:
\url{http://www.nytimes.com/2013/07/27/technology/the-faithful-embrace-youversion-a-bible-app.html?hpw}

How to spread religion, as I have long said:
\url{http://dish.andrewsullivan.com/2013/07/27/the-good-e-book/}

Simple things first.


Add Marvel comics small world network paper:
http://arxiv.org/pdf/cond-mat/0202174v1.pdf


http://journals.ametsoc.org/doi/abs/10.1175/1520-0469(1948)005%3C0165:TDORWS%3E2.0.CO;2
THE DISTRIBUTION OF RAINDROPS WITH SIZE
J. S. Marshall and W. Mc K. Palmer


\url{http://en.wikipedia.org/wiki/Free_will_theorem}

Fairy circles:
Strong interaction between plants induces circular barren patches: fairy circles
\url{http://arxiv.org/pdf/1306.4848v1.pdf}


  
Bettencourt
\url{http://www.sciencemag.org/content/340/6139/1438}




share data sets
e.g., metabolism


  

Heap's law


clauset
whales!


  bumblebee, travelling salesman

  lihoreau2010b.pdf
  


Complexity by subtraction
http://www.eurekalert.org/pub_releases/2013-04/nesc-spa041213.php?utm_medium=referral&utm_source=t.co


Functional cartography of metabolic networks
http://www.nature.com/nature/journal/v433/n7028/abs/nature03288.html



http://www.nature.com/nature/journal/vaop/ncurrent/full/nature12071.html
Slower recovery in space before collapse of connected populations

Lei Dai,	 Kirill S. Korolev	 Jeff Gore



Add to complex networks:
Link communities reveal multiscale complexity in networks
ahn2011a

%%%

Add Radicchi's paper:
``Defining and identifying communities in networks''

%%%

Add to complex networks:

%% http://www.engadget.com/2013/01/03/inside-ups-worldport-sorting-hub/

delivery 
UPS/FEDEX
heart-like
%% http://www.youtube.com/watch?v=F3TpKvsxqts

%%% 

limits to tree height

%% http://news.sciencemag.org/sciencenow/2013/01/simple-physics-may-limit-the-siz.html

%%% 

Add to the big story:

Lawrence Krauss: Our Godless Universe is Precious
%% http://www.youtube.com/watch?v=SB5cBl2np-I&feature=player_embedded

Right!
Acci

%%%

project:
``Understanding Road Usage Patterns in Urban Areas''
Wang2012a

%%%


Contagion:

``Structural Diversity in Social Contagion'': Study from Cornell University and Facebook published in the Proceedings of the National Academy of Sciences.

%%%%%%

Lego puzzles
http://copyranter.blogspot.com/2012/03/did-german-ad-agency-blatantly-steal.html



%% \todo{Add requirement of building towards publishing a paper; either class will support existing work or give a chance to development new work; make my makerevtex4 code available}


  Universality of citation distributions

  Radicchi 2008, PNAS

  Scaling collapse example

  radicchi2008b

  lognormal


Advertising

Lego!
 
http://copyranter.blogspot.com/2012/03/did-german-ad-agency-blatantly-steal.html



Projects:

Sugihara2012a.pdf

``Detecting Causality in Complex Ecosystems''

scheffer


Projects:

Response to disasters

Carter Butts's work, networks.

Bagrow et al. 
PLoS ONE
2011

Get his slides ...




\url{http://switchboard.nrdc.org/blogs/dgunders/not_eating_away_at_our_resourc.html}

Throwing away food.  Maybe 40\% of food is uneaten.



Predicting the future

\url{http://www.foreignpolicy.com/articles/2012/11/16/predicting_the_future_is_easier_than_it_looks?page=full}




Recipes

Lada

\url{http://andrewsullivan.thedailybeast.com/2012/11/perfecting-dinner-through-data-mining.html}



\url{http://andrewsullivan.thedailybeast.com/2012/11/perfecting-dinner-through-data-mining.html}



Make a geeky, interactive map of 300, 303, linear algebra

Here be dragons.

Here be matrices.

\url{http://1.bp.blogspot.com/_V73c9Avfduk/TTBQqupLLyI/AAAAAAAAAls/GDX9x4SMsM4/s1600/TomeMap.jpg}







faces 

boxes plotting things

http://andrewsullivan.thedailybeast.com/2012/11/faces-1.html



Add list of Complex Systems Certificate 


CSYS/MATH 300 (Principles of Complex Systems)

CSYS/CS 302 (Modeling Complex Systems)

CSYS/MATH 266 (Chaos, Fractals, and Dyn. Sys)

CSYS/MATH 303 (Complex Networks)

CSYS/BIOL/CS 352 (Evolutionary Computation)

CSYS/STAT/CS 256 (Neural Computation)

CSYS/STAT/CS 355 (Statistical Pattern Recognition)

CSYS/STAT 253 (Appl Time Series \& Forecasting)
CSYS/STAT/CE 369 (Applied Geostatistics)

CSYS/CE 359 (Applied Artificial Neural Networks)

B-List (select 0-2 courses) (other special topics may apply)
CSYS/MATH 221 (Deterministic Modls Oper Rsch)

CSYS/MATH 268 (Mathematical Biology \& Ecology)

CSYS/CS 251 (Artificial Intelligence)

CSYS/CE 245 (Intelligent Transportation Sys)

CSYS/CE 226 (Civil Engr Systems Analysis)

EE/STAT 270 (Stochastic Processes)

ME 295 (Systems and Synthetic Biology)

CSYS/ME 312 (Advanced Bioengineering Systems)

CSYS/ME 350 (Multi-Scale Modeling)

PA 308 (Decision Making Models)


PA 317 (Systems Analysis and Strategic Management)
BIOL 271 (Evolution)

CSYS/PBIO 295 (Ecological \& Environmental Modeling)

PHYS 265 (Thermal Physics)

CS 395 (Evolutionary Robotics)
STAT 330 (Bayesian Statistics)
ENVS 295 (Envir. Modeling and Systems Thinking)


\changelogo{.18}{cheat-to-win-bracelet-the-onion-tp-1.pdf}


and then this!

\wordwikilink{http://www.nytimes.com/2012/10/21/sports/how-armstrongs-wall-fell-one-rider-at-a-time.html?hpw}{Armstrong's Wall of Silence Fell Rider by Rider}

But the lunch conversation between Landis and Messick would eventually
be seen as the first significant crack in Armstrong's gilded
foundation, a critical turning point in antidoping officials' quest to
penetrate the code of secrecy that endured in cycling.

  \wordwikilink{http://www.nytimes.com/2012/10/18/sports/cycling/inquiry-into-kayle-leogrande-led-to-lance-armstrongs-eventual-fall.html?hp}{`Tattooed Guy' Was Pivotal in Armstrong Case [nytimes]}

  \begin{center}
  \includegraphics[width=\textwidth]{CYCLING-articleLarge.jpg}
  \end{center}

  
  
    ``... Leogrande's doping \alertb{sparked} a series of events ...''
%%   
%%    ``Without Leogrande, who knows, the Armstrong investigation maybe never would have happened,'' 
%%    Tygart said.
  


%%  \begin{center}
%%    \includegraphics[width=0.7\textwidth]{cheat-to-win-bracelet-the-onion.jpg}
%%    \wordwikilink{http://store.theonion.com/p-5045-cheat-to-win-bracelet.aspx}{[the onion]}
%%  \end{center}


\changelogo{0.15}{lightbulb-idea-calculus-circle-tp-1.pdf}



xkcd plots:
http://jakevdp.github.com/blog/2012/10/07/xkcd-style-plots-in-matplotlib/

  Visualizing vastness:

  http://opinionator.blogs.nytimes.com/2012/10/15/visualizing-vastness/

Possible project

Malaria
Movement of people
Phone calls
http://www.sciencemag.org/content/338/6104/267


Control Centrality and Hierarchical Structure in Complex Networks

%%   We introduce the concept of control centrality to quantify the
%% ability of a single node to control a directed weighted network. We
%% calculate the distribution of control centrality for several real
%% networks and find that it is mainly determined by the network’s degree
%% distribution. We show that in a directed network without loops the
%% control centrality of a node is uniquely determined by its layer index
%% or topological position in the underlying hierarchical structure of
%% the network. Inspired by the deep relation between control centrality
%% and hierarchical structure in a general directed network, we design an
%% efficient attack strategy against the controllability of malicious
%% networks.


%% Liu Y-Y, Slotine J-J, Barabási A-L (2012) Control Centrality and Hierarchical Structure in Complex Networks. PLoS ONE 7(9): e44459.
%% \url{http://unam.us4.list-manage2.com/track/click?u=0eb0ac9b4e8565f2967a8304b&id=7d4a593e5f&e=d38efa683e}



Complex Systems Society

\url{http://www.complexssociety.eu/}



Evolution of economic division/inequality:

\url{http://www.nytimes.com/2012/10/14/opinion/sunday/the-self-destruction-of-the-1-percent.html?src=recg}



Dark social

\url{http://www.theatlantic.com/technology/archive/2012/10/dark-social-we-have-the-whole-history-of-the-web-wrong/263523/}



Evolutionary Anthropology
Ayn Rand
http://io9.com/5950256/evolutionary-anthropology-to-ayn-rand-you-fail


Add to population/facility density stuff:

\url{http://andrewsullivan.thedailybeast.com/2012/10/the-starbucks-stop-here.html}



Synchrony

\url{http://io9.com/5947112/watch-32-discordant-metronomes-achieve-synchrony-in-a-matter-of-minutes}

And now Gollum will sing the British national anthem God Save the Queen.

Add other characters to recordings.

Gollum asking questions.  Solving questions.  Dumbledore.  JFK exhorting
people to ask what they can do for linear algebra.


  \url{http://www.datagenetics.com/blog/september32012/}

  Pin codes


Zipf's law

\url{http://www.nytimes.com/2012/09/16/opinion/sunday/the-computer-as-music-critic.html?_r=1&hp}

\url{http://www.nature.com/srep/2012/120726/srep00521/full/srep00521.html}

\cite{serra2012a}


The great power of humanity: 

We copy and imitate and replicate
while we think we're freely choosing,
acting independently.

Monty Python: ``Yes, we are all individuals.''



More on randomness and optimization

barabasi2012a.pdf papadopolous2012a.pdf



Complexity framing

Find nyt articles on genome project

Ten years after the big genome breakthrough
the Times published an article
wondering what happened to all those
amazing discoveries we expected.




  spreading of mormonism:

  exactly my main thoughts on contagion:

  peer pressure is huge

  also springing the more ridiculous stuff on
  people after they join.

  start off simple.

\url{http://www.reddit.com/r/IAmA/comments/z6ufq/iama\_hardcore\_mormon\_who\_will\_tell\_you\_what\_we/}



wisdom of groups

cooperation

\url{http://arxiv.org/abs/1208.4091}

Wisdom of groups promotes cooperation in evolutionary social dilemmas



\url{http://www.nature.com/news/computational-social-science-making-the-links-1.11243}



Idea: pop up citations on slides with mouse over

  
  Project: best selling book numbers

  What are the distributions?

  \url{http://en.wikipedia.org/wiki/List\_of\_best-selling\_books}



  Scaling: add turbulence.





Networks

Social networks in movies:

\url{http://flowingdata.com/2012/08/17/character-social-networks-in-movies/}

Could be a project.


  Spiegelman:

  Scale invariance.

  \includegraphics[width=0.8\textwidth]{2011-10-10spiegelman-fractals-lessard.jpg}


cliodynamics

\url{http://www.nature.com/news/human-cycles-history-as-science-1.11078?WT.ec\_id=NEWS-20120807}


  Books to read:

  Thinking, fast and slow.
  


Library search
\url{http://jorendorff.github.com/hackday/2012/library/}





Write up riberio2012a
Cricket


Anomalous Diffusion and Long-range Correlations in the Score Evolution of the Game of Cricket

Anomalous diffusion:
ribeiro2012a.pdf

Mediocre and scarce work so far.

mukherjee2012a.pdf
mukherjee2012b.pdf



spread of technology

\url{http://www.theatlantic.com/technology/archive/2012/07/most-people-didnt-have-a-c-until-1973-and-other-strange-tech-timelines/260427/}


Filippo Radicchi's work on human performance

Olympics

\url{http://www.wired.com/wiredscience/2012/07/universal-laws-at-the-olympics/}




Changizi

Economically organized hierarchies in WordNet and the Oxford English Dictionary
\cite{changizi2008a}



Series of frames about the Monkey typing stuff.

Miller.

``Some Effects of Intermittent Silence''
\cite{miller1957a}

Hammer it to bits.

There's a recent paper to cite here.



Interesting/good/thought of this: Cure Together.

Find Patients Like You.



adamic2008a

Knowledge Sharing and Yahoo Answers: Everyone Knows Something




Why do leading biologists, economists, and others dismiss group selection?

Pinker says group selection doesn't make sense:

His argument and many responses from other scholars are 
\wordwikilink{http://edge.org/conversation/the-false-allure-of-group-selection}{here.}


 
  Pinker is wrong.
 
  Gintis is right.
 
  Dawkins was accurately portrayed on South Park.







Attack emergence of cooperation nonsense.

More here:
%% \url{http://www.pnas.org/content/early/2012/05/16/1206569109.short?rss=1&amp%3bssource=mfr}

Write this concept up?  Where to publish.  The arxiv.  A call for better models.

More money makes people less humane.


contagion

violence as disease

\url{http://andrewsullivan.thedailybeast.com/2012/07/an-epidemic-of-violence.html}

\url{http://www.thenation.com/article/168529/can-street-violence-be-fought-virus}

``more on the metaphor''

\url{http://thenewinquiry.com/essays/the-plague-years/}



Puzzles:

From Bill G.

1000 bottles of wine

10 poison detecting instruments (not mice)

a day before the big party, you hear
your nemesis has poisoned one bottle.

test work over night.

what's the largest number of wine bottles
that can be safely served?



Scaling:

Moore's Law

Eroom's Law



quarterology

\cite{economos1983a}

whales 
1/3 scaling



computational folkloristics

huge

abello2012a

Projects



add slides on;
add as projects

katifori2012a.pdf 
mileyko2012b.pdf




Add to rich-gets-richer networks stuff

Winner's don't take all: Characterizing the competition for links on the web

pennock2002a


geophysics and power laws

czechowski2001\_annotated.pdf



japanese names, power laws
miyazima2000a.pdf

beards
robinson1976a.pdf 



Picture of a jet,

fluid around a ball

Thomas Schelling




Work by Magnasco and Weitz and others
on Horton-Strahler applied to looped networks:

%% \url{http://www.plosone.org/article/info:doi%2F10.1371%2Fjournal.pone.0037994}


\cite{mileyko2012a}


geographic spreading of disease

black plague

\cite{noble1974a}


ontogenetic growth

banavar2002c

von bertalanffy

our point that 2/3 works as well.



Possible project?

Google Knowledge Network

\url{http://www.google.com/insidesearch/features/search/knowledge.html}



More on cities

Bettencourt 2010 a and b



power of networks

\url{http://comment.rsablogs.org.uk/2012/05/22/rsa-animate-power-networks/}
\url{http://www.thersa.org/events/video/animate/rsa-animate-the-power-of-networks}




Baby name faddishness

Lieberson

Berger

\url{http://www.wired.com/wiredscience/2009/05/baby-names-quantify-the-faddishness-of-fads/}




  Lies in science:

  \url{http://www.nature.com/news/beware-the-creeping-cracks-of-bias-1.10600}

It would therefore be naive to believe that systematic error is a problem for biomedicine alone. It is likely to be prevalent in any field that seeks to predict the behaviour of complex systems---economics, ecology, environmental science, epidemiology and so on. The cracks will be there, they are just harder to spot because it is harder to test research results through direct technological applications (such as drugs) and straightforward indicators of desired outcomes (such as reduced morbidity and mortality).



Search
Google's Knowledge Graph

\url{http://mashable.com/2012/05/16/google-knowledge-graph/}j


The origin of bursts and heavy tails in human dynamics

Barabasi~\cite{barabasi2005a}



Font Size Matters

\url{http://www.plosone.org/article/info\%3Adoi\%2F10.1371\%2Fjournal.pone.0036042}


  Information content of networks...

Information Transfer among Coupled Random Boolean Networks
Chiara Damiani, Stuart A. Kauffman, Roberto Serra, Marco Villani and Annamaria Colacci
\url{http://www.springerlink.com/content/lx276184111010uk/}





%% contagion
Schweitzer

contagion PLOs ONE
donations

%% \url{http://www.plosone.org/article/info:doi%2F10.1371%2Fjournal.pone.0001458}


Title: Statistical Outliers and Dragon-Kings as Bose-Condensed Droplets
Authors: V. I. Yukalov and D. Sornette
Categories: physics.soc-ph q-fin.GN quant-ph
Comments: Latex file, 16 pages, 1 figure
Journal-ref: Eur. Phys. J. Spec. Top. 205 (2012) 53-64
\\
?A theory of exceptional extreme events, characterized by their abnormal sizes
compared with the rest of the distribution, is presented. Such outliers, called
"dragon-kings", have been reported in the distribution of financial drawdowns,
city-size distributions (e.g., Paris in France and London in the UK), in
material failure, epileptic seizure intensities, and other systems. Within our
theory, the large outliers are interpreted as droplets of Bose-Einstein
condensate: the appearance of outliers is a natural consequence of the
occurrence of Bose-Einstein condensation controlled by the relative degree of
attraction, or utility, of the largest entities. For large populations, Zipf's
law is recovered (except for the dragon-king outliers). The theory thus
provides a parsimonious description of the possible coexistence of a power law
distribution of event sizes (Zipf's law) and dragon-king outliers.
%% \\ (?http://arxiv.org/abs/1205.1364?, ?37kb)


  Section on autocatalysis

  Kauffman's stuff?


Contagion

Ugander et al.

Structural diversity in social contagion.


Dirk's grand challenges:

San Miguel et al.

sanmiguel2012a


Section on Sornette's endogenous/exogeneous stuff;
Precursors

%% for 124

Fun examples of combining parts to make a whole.
Pictures

Tangrams

Then go to what linear superposition is.


\url{http://www.youtube.com/watch?v=2guKJfvq4uI&feature=related}

Egyptian Revolution


May, R.M., 1972, Will a large complex system be stable? Nature 238, 413-414.

And following.  Project for students to work on.

Also: 
cohen1985a

\cite{may1972a}


Simini et al.
Nature

Radiation model versus Gravity model.


big data

\url{http://biomedicalcomputationreview.org/content/big-data-analytics-biomedical-research}
  


Kauffman:

Dom warm thanks. Hello Peter, and hello Claire and Gary, I look forward to meeting you. A brief statement of my own interests in cancer. For
some years I have been thinking about cancer differentiation therapy. With wonderful Dr. Sui Huang, who i stole from Harvard Med School to join
me at the University of Calgary several years ago, we have taken a breast cancer cell line, individually screened 1500 FDA approved drugs and
found 16 that induce what appears to be differentiation to normal adult breast cells. Sui has just completed gene arrays on these 16 drugs with
3 time points in each treatment, to test if the 16 all induce the same changes or different changes in gene expression patterns. Obviously this
is encouraging, with a long way to go before it may be clinically relevant. ?
The central idea behind this work is an old theory, initially derived from a 1963 article by Jacob and Monod, that genetic regulatory circuits
could have multiple alternative stable or recurrent patterns of gene activities called "attractors", and that different cell types are
different attractors in the behavior of the same genetic regulatory net. Then differentiation is a perturbation or signal that "kicks" the
genetic dynamical network from one attractor to another. And cancer, with our without mutations, may often have normal cell type attractors to
which malignant cells can be induced to differentiate.
I'd be anxious to hear your lines of thought, Claire and Gary, if you'd care to share them. And I look very much forward to interacting before
and when I get to UVM next Fall.


For scaling:

\url{http://www.nature.com/nature/journal/v376/n6535/abs/376046a0.html}

Add Scaling behaviour in the dynamics of an economic index
  



\wordwikilink{http://pre.aps.org/abstract/PRE/v57/i2/p1347\_1}{Beep}


  

\wordwikilink{http://www.shirky.com/writings/powerlaw\_weblog.html}{Clay Shirky on Power laws}

Create a simple problem based on people choosing their favorite thing.

Urn models.



Thomas Schelling on creating his model of segregation
with coins and chess boards.

Some Fun, Thirty-Five Years Ago
\cite{schelling2006a}




Jeremy Lin

\wordwikilink{Beep}{http://www.nytimes.com/2012/02/20/business/media/jeremy-lin-media-hype-stumbles-on-race.html}

Fedex guy's prediction (WSJ article)

Draft paper 
Thaler

%% aside
%% offensive statements
%% http://espn.go.com/espn/story/_/id/7591778/espn-statement-offensive-jeremy-lin-comments


  \textbf{Homo Economicus}

  What makes people think like Economists?
  Evidence on Economic Cognition
  from the 
  ``Survey of Americans and Economists on the Economy''\cite{caplan2001a}




The maximum rate of mammal evolution

\cite{evans2012a}




contagion:
salem witch trials
\url{http://andrewsullivan.thedailybeast.com/2012/01/what-caused-the-salem-witch-trials.html}

individual versus group


  
\url{http://andrewsullivan.thedailybeast.com/2012/02/the-end-of-hacking.html}


\textbf{Fish, decisions, ignorance, and democracy}



Couzin 2011

West 2011

Katz 2011

\cite{couzin2011a}

\cite{west2011a}

\cite{katz2011a}




\wordwikilink{http://en.wikipedia.org/wiki/Moore's\_law}{Moore's law}

eroom's law

%% add to website, intro, outline
%% in this course, we ask why?

  size distributions

%% \wordwikilink{http://www.nytimes.com/2012/01/15/business/the-1-percent-paint-a-more-nuanced-portrait-of-the-rich.html?\_r=1&hp=&pagewanted=all}{wealth inequality example}






social contagion and imitation

danescu-niculescu-mizil2011b.pdf


cooperation

\cite{szathmary2011a}

To group or not to group?



\url{http://pre.aps.org/abstract/PRE/v81/i6/e066119}
\cite{stephens2010a}
Phys. Rev. E 81, 066119 (2010) [4 pages]
Statistical mechanics of letters in words
  


Process books found here:
  
\url{http://www.pks.mpg.de/~edugalt/physicist-language/}



  cities

  \url{http://www.theatlanticcities.com/arts-and-lifestyle/2012/01/do-rankings-affect-our-opinions-cities/883/}



The Size, Scale, and Shape of Cities
\cite{batty2008a}


\url{http://iopscience.iop.org/1742-5468/2011/07/P07013/}

A paradoxical property of the monkey book
Sebastian Bernhardsson, Seung Ki Baek and Petter Minnhagen




  
    
    Influence studies on Twitter:
    Twitter=quitter? An analysis of Twitter quit smoking social networks.
    \url{http://www.ncbi.nlm.nih.gov/pubmed/21573238}
  
    \wordwikilink{http://www.ncbi.nlm.nih.gov/pubmed/21730101}{Disease tracking on Twitter (H1N1):}
  
    \url{http://www.ncbi.nlm.nih.gov/pubmed/22134096}
  



  
%% for language.body.tex

\wordwikilink{http://en.wikipedia.org/wiki/Heaps'\_law}{Heap's law}

Maybe useful:
\url{http://www.ccs.neu.edu/home/ekanou/ISU535.09X2/Handouts/Review_Material/zipfslaw.pdf}

%% \url{http://www.plosone.org/article/info%3Adoi%2F10.1371%2Fjournal.pone.0014139}

Connect with power laws



  \textbf{Information density of languages}
  
%% add paper showing languages of different cadences 
%% have the same amount of information
%% http://www.time.com/time/health/article/0,8599,2091477,00.html?xid=fblike
  
  

  \textbf{Reading speed}
  
%% http://persquaremile.com/2011/12/21/which-reads-faster-chinese-or-english/
  
  


  Aesthetic differences.

http://www.wired.com/wiredscience/2011/12/how-does-the-brain-perceive-art/


For bio congation

Add work by Neil and Sornette's paper

\cite{zhao2010b.pdf}


\section{Primordial slides}

\section{Language efficiency}


  Piantadoso
  

  \textbf{HOT theory}
  %% present continuum version from carlson1999a

  \textbf{Follow theoretical argument presented in:}
    Carlson and Doyle,\cite{carlson1999a}\\
    \alertg{``Highly optimized tolerance: A mechanism for power laws in design systems,''}\\
    Phys. Rev. E, 1999, \textbf{60}, pp.\ 1412--1427.
  

  \textbf{The set up:}
  
  
    Consider: 
    
     
      A spatial region $\Omega$.
     
      $P(\vec{x})$ = Probability of event starting at $\vec{x}$.
     
      Event starting at $\vec{x}$ is of size $A(\vec{x})$.
     
      A constrained resource distributed as $R(\vec{x})$
      with 
      $
      \int_{\Omega} R(\vec{x}) \dee{\vec{x}} = 
      $
    
  
    
  
  

  




  

  \includegraphics[width=0.7\textwidth]{carlson1999a_tab1.pdf}




%%  \textbf{First Returns}

  \textbf{For random walks in 1-$d$:}
  \includegraphics[width=\textwidth]{figrandomwalk_firstreturn_noname.pdf}
  
  
    A \alertg{return} to origin can only happen when $t = 2n$.
  
    In example above, returns occur at $t=8$, 10, and 14.
  
    Call $P_{\textrm{fr}(2n)$ the probability of \alertg{first return} at $t=2n$.
   
    Probability calculation $\equiv$ Counting problem \\
    (combinatorics/statistical mechanics).
  
    \alertb{Idea:} Transform first return problem into an easier return problem.
%%      $$P_{\textrm{fr}(2n) = 2 Pr(x_{t} \ge 1, t=1,\ldots,2n-1, \ \mbox{and} \  x_{2n} =0). $$
  
  



%%  \textbf{First Returns}
  
    \includegraphics[width=.95\textwidth]{figrandomwalk_firstreturn2_noname.pdf}%
    \includegraphics[width=.95\textwidth]{figrandomwalk_firstreturn3_noname.pdf}
    %%    \includegraphics[width=\textwidth]{figrandomwalk_firstreturn4_noname.pdf}
    
    
      Can assume drunkard first lurches to $x=1$.
     
      Observe walk first returning at $t=16$ stays at or above $x=1$ for $1 \le t \le 15$
      (dashed \alert{red} line).
     
      Now want walks that can return many times to $x=1$.
     
      $P_{\textrm{fr}(2n) = $ \\
      $2\cdot\frac{1}{2}Pr(x_{t} \ge 1, 1 \le t \le 2n-1, \ \mbox{and} \  x_1 = x_{2n-1} = 1) $
     
      The $\frac{1}{2}$ accounts for $x_{2n}=2$ instead of 0.
     
      The $2$ accounts for drunkards that first lurch to $x=-1$.
    
  


  \textbf{Counting first returns:}

  \textbf{Approach:}
    
    
      Move to counting numbers of walks.
    
      Return to probability at end.
    
      Define 
      $N(i,j,t)$ as the \# of possible walks between $x=i$ and $x=j$ taking $t$ steps.
     
      Consider \alert{all paths} starting at $x=1$ and ending at $x=1$ after $t=2n-2$ steps.
     
      \alertb{Idea:} If we can compute the number of walks that hit $x=0$ at least once, then we can
      subtract this from the total number to find the ones that maintain $x \ge 1$.
     
      Call walks that drop below $x=1$ \alertb{excluded walks}.
     
      We'll use a method of images to identify these excluded walks.
    
  


%%  \textbf{First Returns:}

  \textbf{Examples of excluded walks:}
    \begin{center}
      \includegraphics[width=.4\textwidth]{figrandomwalk_firstreturn5_noname.pdf}
      \includegraphics[width=.4\textwidth]{figrandomwalk_firstreturn6_noname.pdf}
    \end{center}
  

    \textbf{Key observation for excluded walks:}
      
        
        For any path starting at $x$$=$$1$ that hits 0,
        there is a unique matching path starting at $x$$=$$-1$.
         
        Matching path first mirrors and then tracks after first reaching $x$$=$$0$.
      
        \# of $t$-step paths starting and ending at $x$$=$$1$
        and hitting $x$$=$$0$ at least once\\
        {= \# of $t$-step paths starting at $x$$=$$-$1 and ending at $x$$=$$1$}
        {= $N(-1,1,t)$\\}
      
        So \alertb{$N_{\textrm{first\ return}(2n) = N(1,1,2n-2) - N(-1,1,2n-2)$}
      
    



%% %%   \textbf{First Returns:}
%% 
%%   %%     
%%     
%%       \includegraphics[width=\textwidth]{figrandomwalk_firstreturn5_noname.pdf}\\
%%       \bigskip
%%       \includegraphics[width=\textwidth]{figrandomwalk_firstreturn6_noname.pdf}
%%     
%%     
%%     \textbf{Key observation:}
%%       
%%       
%%         \# of $t$-step paths starting and ending at $x=1$
%%         and hitting $x=0$ at least once.\\
%%         {= \# of $t$-step paths starting at $x=-1$ and ending at $x=1$\\}
%%         {= $N(-1,1,t)$\\}
%%       
%%         So \alertb{$N_{\textrm{first\ return}(2n) = N(1,1,2n-2) - N(-1,1,2n-2)$}
%%         
%%         For any path starting at $x=1$ that hits 0,
%%         there is a unique matching path starting at $x=-1$.
%%          
%%         Matching path first mirrors and then tracks.
%%       
%%     
%%   %% 
%% 






Algorithms.

superconductivity

Add boyd

http://www.nytimes.com/2011/08/16/science/16butterfly.html?src=rechp




  \textbf{Reductionism}

  Great first comment here:
  http://scholarlykitchen.sspnet.org/2011/07/26/what-happens-if-science-becomes-a-low-yield-activity/

  Use Arbesman's unfortunate (but well done) paper
  as a set up.







%% %%   \textbf{Spreading:}
%% 
%%     
%%     
%%       Example: Spreading of buildings in the U.S.:
%%     
%% 
%%   \begin{center}
%%     \includemovie[
%%     controls=true,
%%     toolbar=true,
%%     poster=walmartspread-frame.jpg,
%%     ]{100mm}{75mm}{videos/2010/walmartspread.mp4}
%%   \end{center}
%% 
%% 




  
  Who killed 2/3?

  Antimortem injuries.

  Show autopsy of 2/3.



    Cities

    Peter Sheridan Dodds, Theoretical Biology's Buzzkill
    ac
    %% make this into law and order

    %% changizi's lego scaling

    more idiocy
%%    \url{http://www.radiolab.org/blogs/radiolab-blogland/2013/jan/22/krulwich-wonders-nature-has-formula-tells-us-when-its-time-die/}

%%    \url{http://www.nytimes.com/1999/01/12/science/of-mice-and-elephants-a-matter-of-scale.html?pagewanted=all}

    really improve these slides
    statistics section
    presentation and criticism of west
    poiseuille flow, diagrams
    womersley

    add reich

    add prothero 1999
    add damuth

    add murray's law?

  

      Contagion:
    Ugander2012a
    Facebook adoption

    Networks:
    Ugander2013a
    Subnetworks

  

  \textbf{Quarters, quarters everywhere:}
    
     
      ``Population density and body size in mammals''\cite{damuth1981a}\\
      John Damuth, Nature, 1981
      \includegraphics[width=\textwidth]{damuth1981a_tab1}
    
      Mean of the reliable regressions: -0.68.
    
      Overall regression (Fig.\ 1) holds up.
    
  

  
%%  Need 1992 papers



%%   %%     \textbf{Examples of non-disease spreading:}
%% 
%%     \textbf{Interesting infections:}
%%       
%%       
%%         \wordwikilink{http://www.youtube.com/watch?v=EGzHBtoVvpc}{Spreading of buildings in the US...}
%%         \begin{center}
%%           \includemovie[
%%           controls=true,
%%           toolbar=true,
%%           poster=videos/2010/walmartspread-frame.jpg,
%%           ]{64mm}{48mm}{videos/2010/walmartspread.mp4}
%%         \end{center}
%%        \wordwikilink{http://www.cnnbcvideo.com/?nid=VWB8OWHr.GqH2kYkPxOMwTQ1NDIxODA-}{Viral get-out-the-vote video.}
%%       
%%     
%% 
%%   


  %% add page/cites to papers on SARS
  %% show how models worked




  %% 1 page each:
  %% vespignani's gleam
  %% brockmann's dollar bills
  %% marta's paper
  %% simini paper


  %% For projects

  %% dragon kings

  %% doyle2011a

  %% add krause, couzin stuff.
  %% 5 to 10 leaders needed for example.
  %% Crowd experiments (100, 200 people).

  %% add simini2012a
  %% migration


  %% baby names



      \textbf{Universality?}

    
      
      
        Dealing with the $k>1$ case:
        $$
        n_k
        =
        n_{k-1}
        \frac{A_{k-1}}{\mu + A_k}
        {
          =
          n_{k-1}
          \frac{A_{k-1}}{A_k}
          \frac{1}{1 + \frac{\mu}{A_k}}
          $$
        }
                  
          $$
          =
          \alert{n_{k-2}
            \frac{A_{k-2}}{A_{k-1}}
            \frac{1}{1 + \frac{\mu}{A_{k-1}}}}
          \frac{A_{k-1}}{A_k}
          \frac{1}{1 + \frac{\mu}{A_k}}
          $$
          
          $$
          =
          n_{k-2}
          \frac{A_{k-2}}{\cancel{A_{k_1}}}
          \frac{1}{1 + \frac{\mu}{A_{k-1}}}
          \frac{\cancel{A_{k-1}}}{A_k}
          \frac{1}{1 + \frac{\mu}{A_k}}
          $$
                {
          $$
          =
          n_{1}
          \frac{A_{1}}{A_k}
          \prod_{j=2}%% {k}
          \frac{1}{1 + \frac{\mu}{A_j}}
          $$
        }
        {
          $$
          =
          n_{1}
          \frac{A_{1}}{A_k}
          \left(1 + \frac{\mu}{A_1}\right)
          \prod_{j=\alert{1}}%% {k}
          \frac{1}{1 + \frac{\mu}{A_j}}
          $$
        }
        {
          $$
          =
          \alert{\frac{\mu}{A_k}}
          \prod_{j=1}^{k}
          \frac{1}{1 + \frac{\mu}{A_j}}
          \mbox{\ since $n_1 = \mu/(\mu+A_1)$}
          $$
        }
      
    

  
      \textbf{Universality?}

    
      
      
        Time for pure excitement: Find \alert{asymptotic behavior} 
        of $n_k$ given $A_k \rightarrow k$ as $k \rightarrow \infty$.
      
        For large $k$:
        $$
        n_k = 
        \frac{\mu}{A_k}
        \prod_{j=1}^{k}
        \frac{1}{1 + \frac{\mu}{A_j}}
        {
          =
          \frac{\mu}{A_k}
          \prod_{j=1}^{k}
          \frac{A_j}{A_j + \mu}
        }
        $$
        {
          $$
          =
          \frac{\mu}{\alert{\cancel{A_k}}}
          \frac{A_1}{(A_1+\mu)}
          \frac{A_2}{(A_2+\mu)}
          \cdots
          \frac{k-1}{(k-1+\mu)}
          \frac{\alert{\cancel{k}}}{(k+\mu)}
          $$
        }
        $$
        {
          \propto
          \frac{\Gamma(k)}{\Gamma(k + \mu + 1)}
        }
        {
          \sim
          \frac{\sqrt{2\pi} k^{k+1/2} e^{-k}}
          {\sqrt{2\pi} (k+\mu+1)^{k+\mu+1+1/2}e^{-(k+\mu+1)}}
        }
        $$
        $$
        {
          \alert{ \propto k^{-\mu-1} }
        }
        $$
      
        Since $\mu$ depends on $A_k$, \alert{details matter...}
      
    

  
      \textbf{Universality?}

    
      
      
        Now we need to find $\mu$.
      
        Our assumption again:
        $
        \alertb{A = \mu t = \sum_{k=1}^{\infty} N_k(t) A_k}
        $
      
        Since $N_k = n_k t$, we have the simplification
        $
        \alertb{
          \mu = \sum_{k=1}^{\infty} n_k A_k}
        $
      
        Now subsitute in our expression for $n_k$:
                  
          $$
          \mu = 
          \sum_{k=1}^{\infty} 
          \alert{
            \frac{\mu}{A_k}
            \prod_{j=1}^{k}
            \frac{1}{1 + \frac{\mu}{A_j}}
          }
          A_k
          $$
          
          $$
          \mu = 
          \sum_{k=1}^{\infty} 
          \frac{\mu}{\alert{\cancel{A_k}}}
          \prod_{j=1}^{k}
          \frac{1}{1 + \frac{\mu}{A_j}}
          \alert{\cancel{A_k}}
          $$
                
          $$
          \alertb{1 \cancel{\mu}} = 
          \sum_{k=1}^{\infty} 
          \frac{\alertb{\cancel{\mu}}}{\alert{\cancel{A_k}}}
          \prod_{j=1}^{k}
          \frac{1}{1 + \frac{\mu}{A_j}}
          \alert{\cancel{A_k}}
          $$
              
        Closed form expression for $\mu$.
      
        We can solve for $\mu$ in some cases.
      
        Our assumption that $A = \mu t$ is okay.
      
    

  
      \textbf{Universality?}

    
      
      
        Turns out we can adjust $A_k$ and tune $\gamma$
        to be anywhere in $[2,\infty)$.
      
        $\gamma = 2$ is the lower limit since
        $$
        \mu = 
        \sum_{k=1}^{\infty} A_k n_k 
        \sim
        \sum_{k=1}^{\infty} k n_k
        $$
        must be finite.
      
        Let's now look at a specific example of $A_k$
        to see this range of $\gamma$ is possible.
      
    

  


      \textbf{Homo Economicus}

    What makes people think like Economists?
    Evidence on Economic Cognition
    from the 
    ``Survey of Americans and Economists on the Economy''\cite{caplan2001a}

  




      
    Into every generation a Slayer is born: one girl in all the world, a
    Chosen One. She alone will wield the strength and skill to fight the
    vampires, demons, and the forces of darkness; to stop the spread of
    their evil and the swell of their numbers. She is the Slayer.

  

  %%   %%   
  %%   \includemedia[
  %%   width=0.6\linewidth,height=0.45\linewidth,
  %%   activate=pageopen,
  %%   flashvars={
  %%     modestbranding=1 % no YT logo in control bar
  %%     &autohide=1       % controlbar autohide
  %%     &showinfo=0       % no title and other info before start
  %%     &rel=0            % no related videos after end
  %%   },
  %%   url                % Flash loaded from URL
  %%   ]{}{http://www.youtube.com/v/Mdc3o7wOwNA?rel=0}  
  %%   
  %% %%   http://www.youtube.com/watch?v=agzGlbRKzqw
  %%   
  %% 
  
    Excellent stupidity:

    http://economix.blogs.nytimes.com/2010/04/20/a-tale-of-many-cities/

  


      \textbf{Phase diagrams}

    \includegraphics[width=0.9\textwidth]{Phase-diag.jpg}

    \medskip

    Qualitatively distinct macro states.

  

  
    blasius2009a

    Zipf chess

    http://physics.aps.org/articles/v2/97

    \url{http://iopscience.iop.org/0295-5075/97/6/68002/article}

    \url{http://www.wired.com/wiredscience/2012/04/network-science-of-the-game-of-go/}

    Add Chess 

  



  
    \amazonbook{gleick2011a}
  

  
    \syllabuslink{\coursewebsite/docs/\courseprefix}

    \posterlink{\coursewebsite/docs/\courseprefix}

    % \reversewords{Try doing this without Perl!}

  
      \textbf{These slides brought to you by:}

    \includegraphics[width=\textwidth]{sealie-lambie-productions002.jpg}

  



Test vidoe:

\includemedia[
width=0.6\linewidth,height=0.45\linewidth,
activate=pageopen,
flashvars={
modestbranding=1 % no YT logo in control bar
&autohide=1 % controlbar autohide
&showinfo=0 % no title and other info before start
&rel=0 % no related videos after end
&start=5
&end=10
}
]{}{http://www.youtube.com/v/Mdc3o7wOwNA?rel=0}

\end{comment}
