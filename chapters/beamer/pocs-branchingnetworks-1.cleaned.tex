\section{Introduction}

\begin{frame}[label=]
  \textbf{Introduction}

  \textbf{Branching networks are useful things:}
    
     Fundamental to material \alert{supply and collection}
     \alert{Supply:} From one source to many sinks in 2- or 3-d.
     \alert{Collection:} From many sources to one sink in 2- or 3-d.
     Typically observe hierarchical, recursive self-similar structure
    
  

  \textbf{Examples:}
    
     River networks (our focus)
     Cardiovascular networks
     Plants
     Evolutionary trees
     Organizations (only in theory...)
    
  


\begin{frame}[label=]
  \textbf{Branching networks are everywhere...}

  \begin{center}
    \includegraphics[width=0.7\textwidth]{hydrosheds_amazon_large-tp-1.pdf}\\
    {\tiny \wordwikilink{http://hydrosheds.cr.usgs.gov/}{http://hydrosheds.cr.usgs.gov/}}
  \end{center}


\begin{frame}[label=]
  \textbf{Branching networks are everywhere...}

  \begin{center}
    \includegraphics[width=0.7\textwidth]{800px-Applebox.jpg}\\
    {\tiny \wordwikilink{http://en.wikipedia.org/wiki/Image:Applebox.JPG}{http://en.wikipedia.org/wiki/Image:Applebox.JPG}}
  \end{center}


\insertvideo{4DW-Dxzj7xQ}{}{}{Shaw and Magnasco's beautiful erosion simulations}


%% \begin{frame}[label=]
%%   \textbf{A beautiful simulation of erosion:}
%% 
%%   \begin{center}
%%     \includemovie[
%%     controls=true,
%%     toolbar=true,
%%     poster=videos/2008-09-23erosion-frame.jpg,
%%     ]{84mm}{63mm}{videos/2008-09-23erosion.mov}\\
%%     {\tiny Bruce Shaw (LDEO, Columbia) and Marcelo Magnasco (Rockefeller)}
%%   \end{center}
%% 
%% 
%%Networks are intrinsic to a broad spectrum of 
%%complex phenomena in the world around us:
%%thoughts and memory emerge from 
%%the interconnection of neurons in the brain,
%%nutrients and waste
%%are transported through the cardiovascular system,
%%and social and business networks link people.
%%River networks stand as an archetypal example
%%of branching networks, an important sub-class 
%%of all network structures.
%%Of significant physical interest in and of themselves,
%%river networks thus also provide an opportunity 
%%to develop results which are 
%%extendable to branching networks in general.
%%To this end, this thesis carries out a thorough examination
%%of river network geometry.  
%%The work combines
%%analytic results, numerical simulations of
%%simple models and measurements of 
%%real river networks.
%%We focus on scaling laws which are
%%central to the description of river networks.
%%Starting from a few simple assumptions
%%about network architecture, we derive
%%all known scaling laws showing that
%%only two scaling exponents are independent.
%%Having thus simplified the description
%%of networks we pursue the precise measurement
%%of real network structure and
%%the further refining of our descriptive tools.
%%We address the key issue of 
%%universality, the possibility that 
%%scaling exponents of river
%%networks take on specific values
%%independent of region. 
%%We find that deviations from
%%scaling are significant enough
%%to preclude exact, definitive measurements.
%%Importantly, geology matters as 
%%the externality of basin shape
%%is shown to be part of the reason
%%for these deviations.
%%This implies that theories that do not 
%%incorporate boundary conditions
%%are unable to produce realistic
%%river network structures.
%%We also extend a number of scaling
%%laws to incorporate fluctuations
%%about simple scaling.
%%Going further than this,
%%we find we are able to identify
%%joint probability distributions
%%that underlie these scaling laws.
%%We generalize a well-known
%%description of the size and
%%number of network components
%%as well as a description of
%%network architecture, how
%%these network components fit together.
%%Both of these generalizations
%%demonstrate that 
%%the spatial distribution
%%of network components is random and,
%%in this sense, we obtain
%%the most basic level of
%%network description.


\subsection{Definitions}

\begin{frame}[label=]
  \textbf{Geomorphological networks}

  \textbf{Definitions}
    
     \alert{Drainage basin} for a point $p$ is the 
      complete region of land from which overland flow drains through $p$.
     Definition most sensible for a point in a stream.
     \alert{Recursive structure:} Basins contain basins and so on.
     In principle, a drainage basin is defined at every point
      on a landscape.
     On flat hillslopes, drainage basins are effectively linear.
     We treat subsurface and surface flow as following
      the gradient of the surface.
     Okay for large-scale networks...
    
  

\begin{frame}[label=]
  \textbf{Basic basin quantities: $a$, $l$, $L_\parallel$, $L_\perp$:}

      
    \includegraphics[width=\textwidth]{basin.pdf}
    
    
     
      \alert{$a$} = drainage basin area
     
      \alert{$\msl$} = length of longest (main) stream 
      (which may be fractal)
     
      \alert{$L = L_\parallel$} = longitudinal length of basin
     
      \alert{$L = L_\perp$} = width of basin
    

  
\note{
Drainage area is fairly well defined.

Different definitions of $L_\parellel$ are possible.
One would be to draw a straight line from the outlet
of the basin such that the basin is divided into
two halves of equal area.  Another would be to
take the point furthest from the outlet, and connect
the longitudinal line from that point to the outlet.
In all cases, the width line would be at right angles
and the width would be given by the smallest rectangle
one could draw.  Different reasonable definitions
should give the same overall results.

Similarly, different definitions of $l$ are also possible.
One is to take the longest stream as measured from the outlet;
another is to move up through the network, at each fork
following the stream belonging to the basin with
the largest drainage area.
}

\subsection{Allometry}

\begin{frame}[label=]
  \textbf{Allometry}


  \bigskip

      
    
    
    
      \alertb{Isometry:} 
      dimensions scale linearly with each other.
    
    \includegraphics[height=0.25\textheight]{iso_tree-tp-5.pdf}
    
    
          
      
      
        \alertb{Allometry:} dimensions scale nonlinearly.
      
      \includegraphics[height=0.25\textheight]{allo_tree-tp-5.pdf}
        
  


%% \begin{frame}[label=]
%%   \textbf{Note on notation:}
%%   
%%   
%%    The $l$ used here is a lower-case ell (and not the letter I for Icicle).
%%     I'll see what I can do about the font.
%%   
%% 


\begin{frame}[label=]
  \textbf{Basin allometry}

      
    \includegraphics[width=\textwidth]{basin.pdf}
    
    \textbf{Allometric relationships:}
      
       
        $$\msl \propto a^h$$
       
        $$\msl \propto L^d$$
       Combine above:
        $$ a \propto  L^{d/h} \equiv L^{D}$$
      
    
  
\begin{frame}[label=]
  \textbf{`Laws'}

  
     Hack's law (1957)\cite{hack1957a}:  
    $$ \alert{\boxed{\msl \propto a^h}}$$
    $$ \mbox{reportedly} \ 0.5 < h < 0.7 $$
     Scaling of main stream length with basin size:
    $$ \alert{\boxed{\msl \propto L_\parallel^d}} $$
    $$ \mbox{reportedly} \ 1.0 < d < 1.1 $$
     Basin allometry:
    $$ \alert{\boxed{L_\parallel \propto a^{h/d} \equiv a^{1/D}}}$$
    $$D < 2 \rightarrow \mbox{basins elongate}.$$
  


\subsection{Laws}

\begin{frame}[label=]
  \textbf{There are a few more `laws':\cite{dodds1999a}}

  \rowcolors[]{1}{blue!20}{blue!10} 
%  \rowcolors[\hline]{3}{green!25}{yellow!50}
%  \begin{tabular}{!{\vrule}|rl!{\vrule}} 
  \begin{tabular}{rl}
    \alert{Relation:} & \alert{Name or description:}  \\
    & \\
    $T_{k} = T_1 (R_T)^{k}$ & {Tokunaga's law}  \\
    $\msl \sim L^{{d}}$ & {self-affinity of single channels}  \\
    $n_{\om}/n_{\om+1} = R_n$ & Horton's law of stream numbers  \\
    $\bar{\msl}_{\om+1}/\bar{\msl}_{\om} = R_{\msl}$ & Horton's law 
    of main stream lengths \\
    $\bar{a}_{\om+1}/\bar{a}_{\om} = R_a$ & Horton's law of basin areas \\
    $\bar{\okell}_{\om+1}/\bar{\okell}_{\om} = R_{\okell}$ & Horton's law 
    of stream segment lengths  \\
    $L_\perp \sim L^H$ & scaling of basin widths \\
    $P(a) \sim a^{-\tau}$ & probability of basin areas \\
    $P(\msl)\sim \msl^{-\gamma}$ & probability of stream lengths \\
    $\msl \sim a^h$ & Hack's law \\
    $a \sim L^D$ & scaling of basin areas \\
    $\Lambda \sim a^\beta$ & Langbein's law \\
    $\lambda \sim L^\varphi$ & variation of Langbein's law\\
  \end{tabular}


\begin{frame}[label=]
  \textbf{Reported parameter values:\cite{dodds1999a}}

  \rowcolors[]{1}{blue!20}{blue!10} 
  \begin{center}
    \begin{tabular}{rl}
      \alert{Parameter:}    & \alert{Real networks:}  \\
      &     \\
      $R_n$               & 3.0--5.0 \\
      $R_a$               & 3.0--6.0 \\
      $R_\msl=R_T$               & 1.5--3.0 \\
      $T_1$               & 1.0--1.5 \\
      ${d}$               & $1.1 \pm 0.01$ \\
      $D$                 & $1.8 \pm 0.1$ \\
      $h$                 & 0.50--0.70 \\
      $\tau$              & $1.43 \pm 0.05$ \\
      $\gamma$            & $1.8 \pm 0.1$ \\
      $H$                 & 0.75--0.80 \\
      $\beta$             & 0.50--0.70  \\
      $\varphi$           & $1.05 \pm 0.05$ \\
    \end{tabular}
  \end{center}


\begin{frame}[label=]
  \textbf{Kind of a mess...}

  \textbf{Order of business:}
    
     Find out how these relationships are connected.
     Determine most fundamental description.
     Explain origins of these parameter values  \\
      \bigskip
      {For (3): \alert{Many attempts: not yet sorted out...}}
    
  


\section{Stream\ Ordering}

\begin{frame}[label=]
  \textbf{Stream Ordering:}

  \textbf{Method for describing network architecture:}
    
    
      Introduced by Horton (1945)\cite{horton1945a}
     
      Modified by Strahler (1957)\cite{strahler1952a}
     
      Term: Horton-Strahler Stream Ordering\cite{rodriguez-iturbe1997a}
    
      Can be seen as \alert{iterative trimming} of a network.
    
  


\begin{frame}[label=]
  \textbf{Stream Ordering:}

  \textbf{Some definitions:}
    
     A \alert{channel head} is a point in landscape
      where flow becomes focused enough to form a stream.
     A \alert{source stream} is defined as the 
      stream that reaches from a channel head
      to a junction with another stream.
     Roughly analogous to capillary vessels.
     Use symbol $\omega=1, 2, 3, ...$ for stream order.
    
    
  


\begin{frame}[label=]
  \textbf{Stream Ordering:}

  \begin{columns}[b]
    
    
    \includegraphics[width=\textwidth]{network1}
    
    
    \includegraphics[width=\textwidth]{network2}
    
    
    \includegraphics[width=\textwidth]{network3}
    
    
    Label all \alert{source streams} as \alert{order $\omega=1$} and remove.
    Label all \alert{new} source streams as \alert{order $\omega=2$} and remove.
    Repeat until one stream is left (order = $\Omega$)
    Basin is said to be of the order of the last stream removed.
    Example above is a basin of order $\Omega=3$.
  


\begin{frame}[label=]
  \textbf{Stream Ordering---A large example:}

  \begin{center}
    \includegraphics[angle=0,width=.8\textwidth]{figorder_paths_mispi10}  
  \end{center}


\begin{frame}[label=]
  \textbf{Stream Ordering:}
  
  \textbf{Another way to define ordering:}
    
     As before, label all \alert{source streams} as \alert{order $\omega=1$}.
     Follow all labelled streams downstream
     Whenever two streams of the same order ($\omega$) meet, the resulting stream
      has order incremented by 1 ($\omega+1$).
    
          
      
       If streams of different orders $\omega_1$ and $\omega_2$ meet,
        then the resultant stream has order equal to the largest of the two.
       Simple rule:
        $$
        \omega_3 = \max(\omega_1,\omega_2) + \delta_{\omega_1,\omega_2}
        $$
       {\small where $\delta$ is the Kronecker delta.}
      
      
      \includegraphics[angle=0,width=\textwidth]{figorder_paths_mispi10}  
    
%     Simple rule:
%      $$
%      \omega_3 = \max(\omega_1,\omega_2) + \delta_{\omega_1,\omega_2}
%      $$
%      where $\delta$ is the Kronecker delta.
%    
    
  


\begin{frame}[label=]
  \textbf{Stream Ordering:}  

  \textbf{One problem:}
    
     Resolution of data messes with ordering
     Micro-description changes (e.g., order of a basin may increase) 
     ... but relationships based on ordering appear to be robust to resolution changes.
    
  
  

\begin{frame}[label=]
  \textbf{Stream Ordering:}

  \textbf{Utility:}
    
     Stream ordering helpfully discretizes a network.
     Goal: understand \alert{network architecture}
    
  


\begin{frame}[label=]
  \textbf{Stream Ordering:}


  \textbf{Resultant definitions:}
    
     
      A basin of order $\Omega$ has 
      \alert{$n_\omega$} streams (or sub-basins) of order $\omega$.
      
       $n_\omega > n_{\omega+1}$
      
     
      An order $\omega$ basin has \alert{area $a_\omega$}.
     
      An order $\omega$ basin has a \alert{main stream length $\msl_\omega$}.
      
      An order $\omega$ basin has a \alert{stream segment length $\okell_\omega$}
      
       an order $\omega$ stream segment is only that part of the stream which is actually of order $\omega$
       an order $\omega$ stream segment runs from the 
        basin outlet up to the junction of two order $\omega-1$ streams
      
    
  


\section{Horton's\ Laws}

\begin{frame}[label=]
  \textbf{Horton's laws}

  \textbf{Self-similarity of river networks}
    
    
      First quantified by Horton (1945)\cite{horton1945a}, 
      expanded by Schumm (1956)\cite{schumm1956a}
    
  

  \textbf{Three laws:}
    
     Horton's law of stream numbers:
      $$
      \alert{\boxed{n_{\om}/n_{\om+1} = R_n > 1 }}
      $$
     
      Horton's law of stream lengths:
      $$
      \alert{\boxed{\bar{\msl}_{\om+1}/\bar{\msl}_{\om} = R_\msl > 1}}
      $$
     
      Horton's law of basin areas:
      $$
      \alert{\boxed{\bar{a}_{\om+1}/\bar{a}_{\om} = R_a > 1}}
      $$
    
  

\begin{frame}[label=]
  \textbf{Horton's laws}

  \textbf{Horton's Ratios:}
    
     So... Horton's laws are defined by three ratios: 
      $$R_n, \ R_\msl, \ \mbox{and} \ R_a.$$
     Horton's laws describe \alert{exponential decay or growth}:
      \begin{align*}
        n_\om & = n_{\om-1}/R_n \\
        {& = n_{\om-2}/R_n^{\ 2} \\}
        {& \vdots \\}
        {& = n_{1}/R_n^{\ \om-1} \\}
        {& = n_{1} e^{-(\om-1) \ln{R_n}} \\}
      \end{align*}
    
  


\begin{frame}[label=]
  \textbf{Horton's laws}

  \textbf{Similar story for area and length:}
    
    
      $$
      \bar{a}_\om = \bar{a}_1 e^{(\om-1)\ln R_a}
      $$
    
      $$
      \bar{\msl}_\om = \bar{\msl}_1 e^{(\om-1)\ln R_\msl}
      $$
     As stream order increases, \alert{number drops} and
      \alert{area and length increase}.
    
  


  
\begin{frame}[label=]
  \textbf{Horton's laws}

  \textbf{A few more things:}
    
     Horton's laws are laws of averages.
     Averaging for number is \alert{across} basins.
     Averaging for stream lengths and areas is \alert{within} basins.
     Horton's ratios go a long way to defining a branching network...
     But we need one other piece of information... 
    
  



\begin{frame}[label=]
  \textbf{Horton's laws}

  \textbf{A bonus law:}
    
     
      Horton's law of stream segment lengths:
      $$
      \alert{\boxed{\bar{\okell}_{\om+1}/\bar{\okell}_{\om} = R_\okell > 1}}
      $$
     
      Can show that $R_\okell = R_\msl$.
    
      \insertassignmentquestion{02}{2}{3}
    
  


\begin{frame}[label=]
  \textbf{Horton's laws in the real world:}

  \begin{center}
    \includegraphics[angle=0,width=.35\textwidth]{fignalomega_mispi10_002_noname} 
%%    \includegraphics[angle=90,width=.45\textwidth]{fignalomega_mispi10}
    \includegraphics[angle=0,width=.45\textwidth]{fignalomega_nile} \\
    \includegraphics[angle=0,width=.45\textwidth]{fignalomega_amazon} 
  \end{center}



\begin{frame}[label=]
  \textbf{Horton's laws-at-large}
  
  \textbf{Blood networks:}
    
    
      Horton's laws hold for sections
      of cardiovascular networks
    
      Measuring such networks is tricky and messy...
    
      Vessel diameters obey an analogous Horton's law.
    
  


  \textbf{Data from real blood networks}

%% ignore vein data
  {\small
    \begin{center}
    \begin{tabular}{c|ccc|cc|c}
      Network & $R_n$ & $R_r^{-1}$ & $R_\ell^{-1}$ & $-\frac{\ln{R_r}}{\ln{R_n}}$ & 
      $-\frac{\ln R_\ell}{\ln{R_n}}$  & $\alpha$ \\
      \hline
      & & & & & & \\
      West \etal\      & --   & --   & --   & 1/2  & 1/3  & 3/4   \\
      & & & & & & \\
%      Dimensional analysis               & --   & --   & --   & 1/2  & 1/2  & 2/3   \\
      rat (PAT)           & 2.76 & 1.58 & 1.60 & 0.45 & 0.46 & 0.73  \\
      & & & & & & \\
      cat (PAT)           & 3.67 & 1.71 & 1.78 & 0.41 & 0.44 & 0.79  \\
      {\tiny (Turcotte \etal\cite{turcotte1998a})}
      & & & & & & \\
      & & & & & & \\
      dog (PAT)           & 3.69 & 1.67 & 1.52 & 0.39 & 0.32 & 0.90  \\
%%      dog (PVT)           & 3.76 & 1.70 & 1.56 & 0.40 & 0.34 & 0.88  \\
      & & & & & & \\
      pig (LCX)           & 3.57 & 1.89 & 2.20 & 0.50 & 0.62 & 0.62  \\
      pig (RCA)           & 3.50 & 1.81 & 2.12 & 0.47 & 0.60 & 0.65  \\
      pig (LAD)           & 3.51 & 1.84 & 2.02 & 0.49 & 0.56 & 0.65  \\
%%      pig (TV)            & 3.05 & 1.73 & 1.61 & 0.49 & 0.43 & 0.71  \\
%%      pig (SV)            & 3.37 & 1.72 & 1.77 & 0.45 & 0.47 & 0.73  \\ 
      & & & & & & \\
      human (PAT)         & 3.03 & 1.60 & 1.49 & 0.42 & 0.36 & 0.83  \\
%%      human (PVT)         & 3.30 & 1.68 & 1.68 & 0.43 & 0.43 & 0.77  \\
      human (PAT)         & 3.36 & 1.56 & 1.49 & 0.37 & 0.33 & 0.94  \\
%%      human (PVT)         & 3.33 & 1.58 & 1.50 & 0.38 & 0.34 & 0.91  \\ 
    \end{tabular}
    \end{center}
    }


\begin{frame}[label=]
  \textbf{Horton's laws}

  \textbf{Observations:}
    
     Horton's ratios vary:
      \begin{center}
        \rowcolors[]{1}{blue!20}{blue!10} 
        \begin{tabular}{rl}
          $R_n$               & 3.0--5.0 \\
          $R_a$               & 3.0--6.0 \\
          $R_\msl$               & 1.5--3.0 \\
        \end{tabular}
      \end{center}
     No accepted explanation for these values.
     Horton's laws tell us how quantities vary
      from level to level ...
     ... but they don't explain how networks
      are structured.
    
  


\section{Tokunaga's\ Law}

\begin{frame}[label=]
  \textbf{Tokunaga's law}

  \textbf{Delving deeper into network architecture:}
    
     Tokunaga (1968) identified a clearer picture of network structure\cite{tokunaga1966a,tokunaga1978a,tokunaga1984a}
     As per Horton-Strahler, use \alert{stream ordering}.
     \alert{Focus:} describe how streams of different orders connect to each other.
     Tokunaga's law is also a law of averages.
    
  



%% ??? need a picture here

\begin{frame}[label=]
  \textbf{Network Architecture}

  \textbf{Definition:}
    
     
      $ \alert{\Tmunu} = $
      the average number of \alert{side streams} of \alert{order $\nu$}
      that enter as tributaries to 
      streams of \alert{order $\mu$}
     $\mu$, $\nu$ = 1, 2, 3, \ldots
     $\mu \ge \nu+1$
     Recall each stream segment of order $\mu$ is `generated' by
      two streams of order $\mu-1$
    These generating streams are not considered side streams.
    
  


\begin{frame}[label=]
  \textbf{Network Architecture}

  \textbf{Tokunaga's law}
    
    
      Property 1: Scale independence---depends only on difference between orders:
      {
        $$ \alert{ \Tmunu  = T_{\mu-\nu}} $$
      }
    
      Property 2: Number of side streams grows exponentially with difference in orders:
      {
        $$ \alert{\Tmunu= T_1 (R_T)^{\mu-\nu-1}} $$
      }
    
      We usually write Tokunaga's law as:
      $$ \alert{\boxed{T_k = T_1 (R_T)^{k-1}}} \mbox{\ \ where $R_T \simeq 2$} $$
      .
    
  


\begin{frame}[label=]
  \textbf{Tokunaga's law---an example:}

      
    $$ T_1 \simeq 2 $$
    $$ R_T \simeq 4 $$
    
    \includegraphics[width=0.8\textwidth]{network1}
  


\begin{frame}[label=]
  \textbf{The Mississippi}

  \textbf{A Tokunaga graph:}
    \includegraphics[width=0.9\textwidth]{figtok_mispi10_3_noname-tp-3}    
  


\section{Nutshell}

\begin{frame}[label=]
  \textbf{Nutshell:}

  \textbf{Branching networks I:}
    
    
      Show remarkable \alert{self-similarity} over many scales.
    
      There are many interrelated scaling laws.
    
      Horton-Strahler \alertb{Stream ordering} gives one useful way of 
      getting at the architecture of branching networks.
    
      \alert{Horton's laws} reveal self-similarity.
    
      Horton's laws can be misinterpreted as
      suggesting a pure hierarchy.
    
      \alert{Tokunaga's laws} neatly describe network architecture.
    
      Branching networks exhibit a mixed hierarchical structure.
    
      Horton and Tokunaga can be connected analytically (next up).
    
  


\insertvideo{teU1nRc0bso}{980}{1100}{Crafting landscapes}

