% simple disease spreading
% general disease spreading

\section{Basic\ Contagion\ Models}

\begin{frame}
  \frametitle{Contagion models}

  \begin{block}{Some large questions concerning network contagion:}
    \begin{enumerate}
    \item<2-> 
      For a given \alertb{spreading mechanism}
      on a given network,
      what's the \alert{probability}
      that there will be \alertb{global spreading}?
    \item<3-> If spreading does take off, how far will it go?
    \item<4-> 
      How do the \alert{details} of the \alertb{network}
      affect the outcome?
    \item<5-> 
      How do the \alert{details} of the \alertb{spreading mechanism}
      affect the outcome?
    \item<6-> 
      What if the \alertb{seed} is one or many nodes?
    \end{enumerate}
  \end{block}

  \begin{itemize}
  \item<7-> 
    \alert{Next up}: We'll look at some fundamental
    kinds of spreading on generalized random networks.
  \end{itemize}

\end{frame}

\begin{frame}
  \frametitle{Spreading mechanisms}

  \begin{columns}
    \column{0.5\textwidth}
    \includegraphics[angle=-90,width=\textwidth]{rn_spread03}
    \column{0.5\textwidth}
    \begin{itemize}
    \item<1->
      \alert{General spreading mechanism}:\\
      State of node $i$ depends on
      history of $i$ and $i$'s neighbors' states.
    \item<2-> 
      \alert{Doses} of entity may be stochastic and history-dependent.
    \item<3-> 
      May have \alertb{multiple, interacting entities} spreading at once.
    \end{itemize}
  \end{columns}
\end{frame}


\begin{frame}
  \frametitle{Spreading on Random Networks}

  \begin{itemize}
  \item<1-> 
    For random networks, we know local structure is pure branching.
  \item<2-> 
    Successful spreading is $\therefore$ contingent on 
    \alert{single edges} infecting nodes.
      \begin{columns}
        \column{0.1\textwidth}
        \column{0.35\textwidth}
        \begin{overprint}
          \onslide<1-2 | handout: 0 | trans:0>
          \onslide<3->
          \alertb{Success}\\
          \includegraphics[angle=-90,width=\textwidth]{rn_spread01}
        \end{overprint}
        \column{0.1\textwidth}
        \column{0.35\textwidth}
        \begin{overprint}
          \onslide<1-2 | handout: 0 | trans:0>
          \onslide<3->
          \alertb{Failure:}\\
          \includegraphics[angle=-90,width=\textwidth]{rn_spread02}
        \end{overprint}
        \column{0.1\textwidth}
      \end{columns}
  \item<4-> 
    Focus on \alert{binary} case with edges and nodes
    either infected or not.
  \item<5->
    \alert{First big question:} for a given network
    and contagion process, can global spreading from
    a single seed occur?
  \end{itemize}

\end{frame}

\section{Global\ spreading\ condition}

\begin{frame}
  \frametitle{Global spreading condition}

  \begin{itemize}
  \item<1->
    We need to find:\cite{dodds2011b}\\
    $\alertb{\gainratio}$ = the average \# of infected
    edges that one random infected edge brings about.
  \item<1->
    Call $\alertb{\gainratio}$ the \alertb{gain ratio}.
  \item<2-> 
    Define \alertb{$\infprob_{k1}$} as the probability that
    a node of degree $k$ is infected by
    a single infected edge.
  \item<3->
    $$
    \gainratio
    =
    \sum_{k=0}^{\infty}
    \uncover<3->{
      \underbrace{
        \frac{kP_k}{\tavg{k}}
      }_{
        \mbox{ \scriptsize
          \begin{tabular}{l}
            prob. of\\
            connecting to \\
            a degree $k$ node\\
          \end{tabular}
        }
      }
    }
    \uncover<4->{
      \bullet
      \underbrace{
        (k-1)
      }_{
        \mbox{\scriptsize
          \begin{tabular}{l}
            \# outgoing \\
            infected \\
            edges
          \end{tabular}
        }
      }
    }
    \uncover<5->{
      \bullet
      \underbrace{
        \infprob_{k1}
      }_{
        \mbox{\scriptsize
          \begin{tabular}{l}
            Prob. of \\
            infection
          \end{tabular}
        }
      }
    }
    $$
    $$
    \uncover<6->{
      +
      \sum_{k=0}^{\infty}
      \overbrace{
        \frac{kP_k}{\tavg{k}}
      }
    }
    \uncover<7->{
      \bullet
      \underbrace{0}_{
        \mbox{\scriptsize
          \begin{tabular}{l}
            \# outgoing \\
            infected \\
            edges
          \end{tabular}
        }
      }
    }
    \uncover<8->{
      \bullet
      \underbrace{
        (1-\infprob_{k1})
      }_{
        \mbox{\scriptsize
          \begin{tabular}{l}
            Prob. of \\
            no infection
          \end{tabular}
        }
      }
    }
    $$      
      \end{itemize}
    \end{frame}

\begin{frame}
  \frametitle{Global spreading condition}

  \begin{itemize}
  \item<1-> 
    Our global spreading condition is then:
    $$
    \boxed{
      \alertb{
        \gainratio
        =
        \sum_{k=0}^{\infty}
        \frac{kP_k}{\tavg{k}}
        \bullet
        (k-1)
        \bullet
        \infprob_{k1}
        > 1.
        }
      }
    $$
  \item<2->
    \alert{Case 1:}
    \uncover<3->{
      If $\infprob_{k1}=1$ 
    }
    \uncover<4->{
      then
      $$
      \gainratio = 
        \sum_{k=0}^{\infty}
        \frac{kP_k}{\tavg{k}}
        \bullet
        (k-1)
        =
        \frac{\tavg{k(k-1)}}{\tavg{k}} > 1.
      $$
    }
  \item<5->
    \alert{Good:} This is just our giant component condition again.
  \end{itemize}
\end{frame}

\begin{frame}
  \frametitle{Global spreading condition}

  \begin{itemize}
  \item<1-> 
    \alert{Case 2:}
    \uncover<2->{
      If $\infprob_{k1}=\beta<1$ 
    }
    \uncover<3->{
      then
      $$
      \gainratio = 
      \sum_{k=0}^{\infty}
      \frac{kP_k}{\tavg{k}}
      \bullet
      (k-1)
      \bullet
      \beta > 1.
      $$
    }
  \item<4-> 
    A fraction (1-$\beta$) of edges do not transmit 
    infection.
  \item<5-> 
    Analogous phase transition to giant component case
    but \alertb{critical value} of $\tavg{k}$ is \alertb{increased}.
  \item<6-> 
    Aka \wordwikilink{http://en.wikipedia.org/wiki/Percolation\_theory}{bond percolation}.
  \item<7-> 
    Resulting degree distribution $\tilde{P}_k$:
    $$
    \tilde{P}_k
    =
    \beta^k
    \sum_{i=k}^{\infty}
    \binom{i}{k}
    (1-\beta)^{i-k}
    P_i.
    $$
    \insertassignmentquestionsoft{07}{7}
  \item<8-> 
    We can show $F_{\tilde{P}}(x) = F_{P}(\beta x + 1 - \beta)$.
  \end{itemize}

\end{frame}

\section{Social\ Contagion\ Models}

\begin{frame}
  \frametitle{Global spreading condition}

  \begin{itemize}
  \item<1->
    \alert{Cases 3, 4, 5, ...:}
    \uncover<2->{
      Now allow $\infprob_{k1}$ to depend on $k$
    }
  \item<3->
    \alertb{Asymmetry}: Transmission along an edge depends on
    node's degree at other end.
  \item<4->
    Possibility: $\infprob_{k1}$ increases with $k$...
    \uncover<5->{\alert{unlikely}}.
  \item<6->
    Possibility: $\infprob_{k1}$ is not monotonic in $k$...
    \uncover<7->{\alert{unlikely}}.
  \item<8->
    Possibility: $\infprob_{k1}$ decreases with $k$...
    \uncover<8->{\alert{hmmm}}.
  \item<9->
    $\infprob_{k1} \searrow$ is a plausible representation
    of a simple kind of social contagion.
  \item<10->
    \alert{The story:}\\
    More well connected people are harder to influence.
  \end{itemize}

\end{frame}

\begin{frame}
  \frametitle{Global spreading condition}

  \begin{itemize}
  \item<1-> 
    \alert{Example:} $\infprob_{k1}=1/k$.
  \item<2-> 
    $$
    \gainratio
    =
    \sum_{k=\alert{1}}^{\infty}
    \frac{kP_k}{\tavg{k}}
    \bullet
    (k-1)
    \bullet
    \infprob_{k1}
    \uncover<3->{
      =
      \sum_{k=1}^{\infty}
      (k-1)
      \bullet
      \frac{kP_k}{\tavg{k}}
      \bullet
      \frac{1}{k}
    }
    $$
    $$
    \uncover<4->{
      =
      \sum_{k=1}^{\infty}
      \frac{P_k}{\tavg{k}}
      \bullet
      (k-1)
    }
    \uncover<5->{
      =
      1 - \frac{1-P_0}{\tavg{k}}
    }
    $$
  \item<7->
    Since $\gainratio$ is always less than $1$, no spreading
    can occur for this mechanism.
  \item<8->
    Decay of $\infprob_{k1}$ is too fast.
  \item<9->
    Result is independent of degree distribution.
  \end{itemize}

\end{frame}

\begin{frame}
  \frametitle{Global spreading condition}

  \begin{itemize}
  \item<1-> 
    \alert{Example:} $\infprob_{k1}=H(\frac{1}{k}-\phi)$\\
    where $0<\alert{\phi} \le 1$ is a \alert{threshold}
    and $H$ is the \wordwikilink{http://en.wikipedia.org/wiki/Heaviside_step_function}{Heaviside function}.
  \item<2-> 
    Infection only occurs for nodes with \alertb{low} degree.
  \item<3-> 
    Call these nodes \alert{vulnerables}:\\ 
    they flip
    when \alertb{only one} of their friends flips.
  \item<4-> 
    $$
    \gainratio
    =
    \sum_{k=\alert{1}}^{\infty}
    \frac{kP_k}{\tavg{k}}
    \bullet
    (k-1)
    \bullet
    \infprob_{k1}
    \uncover<5->{
      =
      \sum_{k=1}^{\infty}
      \frac{kP_k}{\tavg{k}}
      \bullet
      (k-1)
      \bullet
      H\left(
        \frac{1}{k}-\phi
      \right)
    }
    $$
    $$
    \uncover<6->{
      =
      \sum_{k=1}^{\alert{\lfloor \frac{1}{\phi} \rfloor}}
      (k-1)
      \bullet
      \frac{kP_k}{\tavg{k}}
    \mbox{\quad where $\lfloor \cdot \rfloor$ means floor.}
    }
    $$
  \end{itemize}
\end{frame}

\begin{frame}
  \frametitle{Global spreading condition}

  \begin{itemize}
  \item<1->
    The uniform threshold model global spreading condition:
    $$
    \gainratio
    =
    \sum_{k=1}^{{\lfloor \frac{1}{\phi} \rfloor}}
    (k-1)
    \bullet
    \frac{kP_k}
    {\tavg{k}} 
    > 1.
    $$
  \item<2-> 
    As \alert{$\phi \rightarrow 1$}, all nodes become resilient and $r \rightarrow 0$.
  \item<3-> 
    As \alert{$\phi \rightarrow 0$}, all nodes become vulnerable and the contagion
    condition matches up with the giant component condition.
  \item<4->
    \alert{Key}: If we fix $\phi$ and then vary $\tavg{k}$, we
    may see \alertb{two} phase transitions.
  \item<5-> Added to our standard giant component transition,
    we will see a cut off in spreading as nodes become more connected.
  \end{itemize}

\end{frame}

\begin{frame}
  Virtual contagion:
  \wordwikilink{http://en.wikipedia.org/wiki/Corrupted_Blood_incident}{Corrupted Blood},
  a 2005 virtual plague in World of Warcraft:
  
  \begin{center}
    \includegraphics[height=0.7\textheight]{WoW_Corrupted_Blood_Plague.jpg}
  \end{center}

\end{frame}

\subsection{Network\ version}

\begin{frame}
  \frametitle{Social Contagion}

  \begin{block}<1->{Some important models (recap from CSYS 300)}
    \begin{itemize}
    \item<1-> Tipping models---Schelling (1971)\cite{schelling1971a,schelling1973a,schelling1978a}
      \begin{itemize}
      \item<2->
        Simulation on checker boards.
      \item<3->
        Idea of thresholds.
      \end{itemize}
    \item<4-> Threshold models---Granovetter (1978)\cite{granovetter1978a}
    \item<5-> Herding models---Bikhchandani et al. (1992)\cite{bikhchandani1992a,bikhchandani1998a}
      \begin{itemize}
      \item<6->
        Social learning theory, Informational cascades,...
      \end{itemize}
    \end{itemize}
  \end{block}

\end{frame}

\begin{frame}
  \frametitle{Threshold model on a network}

  Original work:

  \bigskip

  \alert{``A simple model of global cascades on random networks''}\\
  D. J. Watts.  Proc. Natl. Acad. Sci., 2002\cite{watts2002a}

  \bigskip

  \begin{itemize}
  \item<2-> Mean field Granovetter model $\rightarrow$ network model
  \item<3-> Individuals now have a limited view of the world
  \end{itemize}

\end{frame}

\begin{frame}
  \frametitle{Threshold model on a network}

  \begin{itemize}
  \item<1-> Interactions between individuals 
    now represented by a network
  \item<2-> Network is \alert{sparse}
  \item<3-> Individual $i$ has $k_i$ contacts
  \item<4-> Influence on each link is \alert{reciprocal} and of \alert{unit weight}
  \item<5-> Each individual $i$ has a fixed threshold $\phi_i$
  \item<6-> Individuals repeatedly poll contacts on network
  \item<7-> Synchronous, discrete time updating 
  \item<8-> Individual $i$ becomes active when\\
    number of active contacts $a_i \ge \phi_i k_i$
  \item<9-> Activation is permanent (SI)
  \end{itemize}

\end{frame}

%%  \begin{block}<1->{Word-of-mouth contagion:}
%%  \end{block}


\begin{frame}
  \frametitle{Threshold model on a network}

  \begin{overprint}
    \onslide<1| handout: 0 | trans: 0>
    \setlength\fboxsep{0pt}
    \setlength\fboxrule{0.5pt}
    \fbox{\includegraphics[angle=0,width=1\textwidth]{contagioncondition3a}}%
    \onslide<2| handout: 0 | trans: 0>
    \setlength\fboxsep{0pt}
    \setlength\fboxrule{0.5pt}
    \fbox{\includegraphics[angle=0,width=1\textwidth]{contagioncondition3b}}%
    \onslide<3| handout: 1 | trans: 1>
    \setlength\fboxsep{0pt}
    \setlength\fboxrule{0.5pt}
    \fbox{\includegraphics[angle=0,width=1\textwidth]{contagioncondition3c}}%
  \end{overprint}

  \begin{itemize}
  \item All nodes have threshold $\phi=0.2$.
  \end{itemize}


%% previous work
%% definition of vulnerables
%% global condition

\end{frame}


\begin{frame}
  \frametitle{The most gullible}

  \begin{block}<1->{Vulnerables:}
    \begin{itemize}
    \item<2-> Recall definition: individuals who can be activated by
    just one contact being active are \alert{vulnerables}.
    \item<3-> The vulnerability condition for node $i$:
      $1/k_i \ge \phi_i$.
    \item<4->
      Means \# contacts  $k_{i} \le \lfloor 1/\phi_i \rfloor$.
    \item<5->
      \alert{Key:} For global spreading events (cascades) on random networks, must have a
      \tc{blue}{\textit{global component of vulnerables}}\cite{watts2002a}
    \item<6-> For a uniform threshold $\phi$, our global spreading condition
      tells us when such a component exists:
      $$
      \gainratio
      =
      \sum_{k=1}^{{\lfloor \frac{1}{\phi} \rfloor}}
      \frac{kP_k}{\tavg{k}} 
      \bullet
      (k-1)
      > 1.
      $$
    \end{itemize}
  \end{block}

\end{frame}

\begin{frame}
  \frametitle{Example random network structure:}

  \begin{columns}
    \column{0.65\textwidth}
    \includegraphics[width=\textwidth]{2011-04-04random-network-contagion-sketch_3a-tp-5.pdf}
    \column{0.35\textwidth}
    \begin{itemize}
    \item 
      $\Omega_{\textnormal{crit}}$ = critical mass = global vulnerable component
    \item 
      $\Omega_{\textnormal{trig}}$ = triggering component
    \item 
      $\Omega_{\textnormal{final}}$ = potential extent of spread
    \item 
      $\Omega$ = entire network
    \end{itemize}
  \end{columns}
  \bigskip
  $$
  \Omega_{\textnormal{crit}} 
  \subset
  \Omega_{\textnormal{trig}};
  \
  \Omega_{\textnormal{crit}} 
  \subset
  \Omega_{\textnormal{final}};
  \
  \mbox{and}
  \
  \Omega_{\textnormal{trig}},
  \Omega_{\textnormal{final}} 
  \subset
  \Omega.
  $$
\end{frame}


%%
%%   \begin{frame}
%%     \frametitle{Cascade window}
%%     
%%     \begin{block}<1->{When does a global cluster of vulnerables exist?}
%%       \begin{itemize}
%%       \item<2-> $z$ = average number of contacts per individual.
%%       \end{itemize}
%%       \begin{enumerate}
%%       \item<3-> 
%%         \tc{blue}{Low $z$:} No cascades in poorly connected networks.\\
%%         No global clusters of any kind.
%%       \item<4-> 
%%         \tc{blue}{High $z$:} Giant component exists but not enough vulnerables.
%%       \item<5-> 
%%         \tc{blue}{Intermediate $z$:} Global cluster of vulnerables exists.\\
%%         Cascades are possible in \tc{red}{``Cascade window.''}
%%       \end{enumerate}
%%     \end{block}
%%
%%  \end{frame}

%% previous work
%% explanation of cascade window


%% previous work
%% cascade window 1
%% figure of basic cascade window outline

\begin{frame}
  \frametitle{Global spreading events on random networks}

  \begin{columns}
    \column{0.5\textwidth}
    \setlength\fboxsep{0pt}
    \setlength\fboxrule{0.5pt}
    \fbox{\includegraphics<1->[width=\textwidth]{watts2002a_fig2b}}\\
    \small{( n.b., $z = \tavg{k}$)}
    \column{0.5\textwidth}
    \begin{itemize}
    \item<1->
      \alert{Top curve:} final fraction infected if successful.
    \item<3->
      \alert{Middle curve:} chance of starting a global spreading event (cascade).
    \item<2-> 
      \alert{Bottom curve:} fractional size of vulnerable subcomponent.\cite{watts2002a}
    \end{itemize}
  \end{columns}

  \begin{itemize}
  \item<4-> 
    Global spreading events occur only if size of vulnerable subcomponent $>0$.\\
  \item<5-> 
    System is robust-yet-fragile just below upper boundary\cite{carlson1999a,carlson2000a,sornette2003a}
  \item<6-> 
    `Ignorance' facilitates spreading.
  \end{itemize}

\end{frame}

\begin{frame}
  \frametitle{Cascades on random networks}

  \begin{columns}
    \column{0.5\textwidth}
    \includegraphics[width=\textwidth]{2011-04-04random-network-contagion-sketch_3c-tp-5.pdf}
    \begin{itemize}
    \item 
      Above lower phase transition
    \end{itemize}
    \column{0.5\textwidth}
    \includegraphics[width=\textwidth]{2011-04-04random-network-contagion-sketch_3b-tp-5.pdf}
    \begin{itemize}
    \item 
      Just below upper phase transition
    \end{itemize}
  \end{columns}

\end{frame}


\begin{frame}
  \frametitle{Cascades on random networks}

    \begin{columns}
      \column{0.5\textwidth}
      \setlength\fboxsep{0pt}
      \setlength\fboxrule{0.5pt}
      \fbox{\includegraphics<1->[width=\textwidth]{watts2002a_fig2a}}\\
      \small{( n.b., $z = \tavg{k}$)}
      \column{0.5\textwidth}
      \begin{itemize}
      \item<1-> Time taken for cascade to spread through network.\cite{watts2002a}
      \item<2-> Two phase transitions.
      \end{itemize}
    \end{columns}

    \begin{itemize}
    \item<3-> Largest vulnerable component = \alert{critical mass}.
    \item<4-> Now have endogenous mechanism for spreading from
      an individual to the critical mass and then beyond.
    \end{itemize}

\end{frame}

\begin{frame}
  \frametitle{Cascade window for random networks}

  \begin{center}
    \setlength\fboxsep{0pt}
    \setlength\fboxrule{0.5pt}
    \fbox{\includegraphics<1->[width=.6\textwidth]{figtransitions_random01_noname}}\\
  \end{center}
  \small{( n.b., $z = \tavg{k}$)}
  \begin{itemize}
  \item<1-> Outline of cascade window for random networks. 
  \end{itemize}

\end{frame}

\begin{frame}
  \frametitle{Cascade window for random networks}

  \includegraphics[angle=-90,width=\textwidth]{figcascadewindow_cut4.pdf}

\end{frame}



\subsection{All-to-all\ networks}

\begin{frame}
  \frametitle{Social Contagion}
  
  \begin{block}<1->{Granovetter's Threshold model---recap}
    \begin{columns}
      \column{0.4\textwidth}
      \includegraphics[width=\textwidth]{figthreshold_eg3_noname}
      \column{0.6\textwidth}
      \begin{itemize}
      \item<1-> Assumes deterministic response functions
      \item<2-> $\phi_\ast$ = threshold of an individual.
      \item<3-> $f(\phi_\ast)$ = distribution of thresholds in a population.
      \item<4-> $F(\phi_\ast)$ = cumulative distribution = $\int_{\phi_\ast'=0}^{\phi_\ast} f(\phi_\ast') \dee{\phi_\ast'}$
      \item<5-> $\phi_t$ = fraction of people `rioting' at time step $t$.
      \end{itemize}
    \end{columns}
  \end{block}
  
\end{frame}

\begin{frame}
  \frametitle{Social Sciences---Threshold models}
 
   \begin{itemize}
   \item<1->
   At time $t+1$, fraction rioting
   = fraction with $\phi_\ast \le \phi_t$.
   \item<2->
   \[ \phi_{t+1} = \int_{0}^{\phi_t} f(\phi_\ast) \dee{\phi_\ast}
   = \left. F(\phi_\ast) \right|_{0}^{\phi_t} = F(\phi_t) \]
   \item<3->
   $\Rightarrow$ Iterative maps of the unit interval $[0, 1]$.
   \end{itemize}

\end{frame}

\begin{frame}
  \frametitle{Social Sciences---Threshold models}

  Action based on perceived behavior of others.

  \includegraphics[width=1\textwidth]{figthreshold_noname}

  \begin{itemize}
  \item<1-> Two states: S and I
  \item<1-> Recover now possible (SIS)
  \item<1-> $\phi$ = fraction of contacts `on' (e.g., rioting)
  \item<2-> Discrete time, synchronous update (strong assumption!)
  \item<3-> This is a \alert{Critical mass model}
  \end{itemize}

\end{frame}


%% \begin{frame}
%%   \frametitle{Social Sciences---Threshold models}
%% 
%%   \includegraphics[width=.45\textwidth]{figthreshold3_noname}
%%   \includegraphics[width=.45\textwidth]{figthresholdF3b_noname}
%% 
%%   \begin{itemize}
%%   \item Critical mass model
%%   \end{itemize}
%% 
%% \end{frame}

\begin{frame}
  \frametitle{Social Sciences---Threshold models}

  \includegraphics[width=.45\textwidth]{figthreshold2_noname}
  \includegraphics[width=.45\textwidth]{figthresholdF2b_noname}\\

  \begin{itemize}
  \item Example of single stable state model
  \end{itemize}

\end{frame}


\begin{frame}
  \frametitle{Social Sciences---Threshold models}

  \begin{block}<1->{Implications for collective action theory:}
    \begin{enumerate}
    \item<2-> Collective uniformity $\not\Rightarrow$ individual uniformity
    \item<3-> Small individual changes $\Rightarrow$ large global changes
    \end{enumerate}
  \end{block}

  \begin{block}<4->{Next:}
    \begin{itemize}
    \item<4-> Connect mean-field model to network model.
    \item<5-> Single seed for network model: $1/N \rightarrow 0$.
    \item<6-> Comparison between network and mean-field model
      sensible for vanishing seed size for the latter.
    \end{itemize}
  \end{block}

\end{frame}


\begin{frame}
  \frametitle{All-to-all versus random networks}
  \centering
  \includegraphics[height=.8\textheight]{figcascwind5_noname}\\

  
\end{frame}


%% %%%%%%%%%%%%%%
%%   %% initiators %
%% %%%%%%%%%%%%%%
%% 
%%   %% cascade window
%%   %% comparison between different types of initiators
%%   %% 
%% %%
%% \begin{frame}
%%   \frametitle{cascade initiators, $\phi=0.18$}
%%   Activate random individuals:\\
%%   \centering
%%   \includegraphics[width=0.62\textwidth]{figtest_nw_threshold_cwi04c_noname}\\
%%   Cascade initiators for $k_{\textnormal{init}}=1$, 2, 3, 4, 6, and 9.
%%   %% 
%%   %% 
%% 
%%   %% cascade window
%%   %% comparison between different types of initiators
%% 
%%   %%  
%% \end{frame}
%% %%
%% \begin{frame}
%%   \frametitle{cascade initiators, $\phi=0.18$}
%%   \centering
%%   \includegraphics[width=0.65\textwidth]{figtest_nw_threshold_cwi04i_noname}\\
%%   Averagely connected nodes versus nodes in top 10 percent.
%%   
%%   
%%   
%%   
%%   %% average degree of everyone versus initiators
%%   
%%   \centering
%% \end{frame}
%% 
%% \begin{frame}
%%   \frametitle{cascade initiators---average degree}
%%   \includegraphics[width=0.6\textwidth]{figtest_network_threshold2_19x1e_noname}\\
%%   \tc{blue}{dashed line:} mean degree of all individuals with $k>0$.\\
%%   \tc{blue}{solid line:} mean degree of cascade initiators.
%%   
%%   
%% \end{frame}
%% 
%% 
%% %% distribution of initiator degrees
%% 
%% %%
%% %% \begin{frame}
%% %%    \frametitle{cascade initiators---degree distributions}
%% %%   \[
%% %%   \begin{array}{l}
%% %%     P_{k,\textnormal{init}}  = [1 - (1-S)^k] \cdot P_k\\
%% %%     \\
%% %%     \qquad = P({\textnormal{cascade}} | k) \cdot P_k \\
%% %%     \\
%% %%     \\
%% %%     S = \mbox{size of} \\
%% %%     \qquad \mbox{vulnerable cluster} \\
%% %%   \end{array}
%% %%   \ \ \raisebox{-5cm}{
%% %%     \includegraphics[width=0.5\textwidth]{figtest_nw_thr2_06fh7_noname}
%% %%   }
%% %%   \]
%% %% 
%% %% 
%% %% 
%% %% 
%% %%   \end{frame}
%% %%
%% %% \begin{frame}
%% %%    \frametitle{cost/benefit analysis}
%% %%   Sampling individuals $\propto$ number sampled $n$.\\
%% %%   Triggering relatively very costly $\rightarrow$ trigger one individual only.\\
%% %%   Choose most connected individual from $n$ samples.
%% %%   \includegraphics[width=0.45\textwidth]{figtest_nw_thr2_06fcost3_noname}
%% %%   \includegraphics[width=0.45\textwidth]{figtest_nw_thr2_06fcost4_noname}\\
%% %%   $\bullet$ $n \nearrow$ as cost $\searrow$.\\
%% %%   $\bullet$ Chosen individual's degree increases slowly with $n$.
%% %%   
%% %% 
%% %% 
%% 
%% 
%% %% %% mean degree of early adopters
%% %% %% + way disease spreads
%% %% 
%% %%   \end{frame}
%% %%
%% 
%% \begin{frame}
%%   \frametitle{Early Adopters---Mean degree + Rate of adoption}
%% 
%%   \raisebox{-4cm}{    
%%     \includegraphics[width=0.62\textwidth]{figtest_nw_thr2_06eh2_6_noname}
%%   }
%%   \begin{tabular}{l}
%%     \tc{red}{---} vulnerables  \\
%%     \tc{blue}{---} non-vulnerables \\
%%     --- total
%%   \end{tabular}
%% 
%% \end{frame} 
%% 
%% %% % SIR comparison
%% %% 
%% %%
%% 
%% \begin{frame}
%%   \frametitle{Comparison to disease spreading models}
%%   
%%   \raisebox{-4cm}{    
%%     \includegraphics[width=0.45\textwidth]{figtest_nw_SIR_01e2_noname}
%%   }
%%   \begin{tabular}{l}
%%     SIR model on random graph \\
%%     \\
%%     early adopters \textit{always}\\
%%     above average \\
%%   \end{tabular}
%% 
%%   Probability of connecting
%%   to a $k$ degree node $\propto k P_k$.
%%   
%% \end{frame}

%% \begin{frame}
%%   \frametitle{Early adopters---degree distributions}
%% 
%%   \begin{tabular}{ccccc}
%%     \includegraphics[width=0.2\textwidth]{figtest_nw_thr2_06e_1_mod_noname}&
%%     \includegraphics[width=0.2\textwidth]{figtest_nw_thr2_06e_2_mod_noname}&
%%     \includegraphics[width=0.2\textwidth]{figtest_nw_thr2_06e_3_mod_noname}&
%%     \includegraphics[width=0.2\textwidth]{figtest_nw_thr2_06e_4_mod_noname}\\
%%     \includegraphics[width=0.2\textwidth]{figtest_nw_thr2_06e_5_mod_noname}&
%%     \includegraphics[width=0.2\textwidth]{figtest_nw_thr2_06e_6_mod_noname}&
%%     \includegraphics[width=0.2\textwidth]{figtest_nw_thr2_06e_7_mod_noname}&
%%     \includegraphics[width=0.2\textwidth]{figtest_nw_thr2_06e_8_mod_noname}\\
%%     \includegraphics[width=0.2\textwidth]{figtest_nw_thr2_06e_9_mod_noname}&
%%     \includegraphics[width=0.2\textwidth]{figtest_nw_thr2_06e_10_mod_noname}&
%%     \includegraphics[width=0.2\textwidth]{figtest_nw_thr2_06e_11_mod_noname}&
%%     \includegraphics[width=0.2\textwidth]{figtest_nw_thr2_06e_12_mod_noname}\\
%% %%    \includegraphics{figtest_nw_thr2_06e_13_mod_noname.ps,width=0.2\textwidth}&
%% %%    \includegraphics{figtest_nw_thr2_06e_14_mod_noname.ps,width=0.2\textwidth}&
%% %%    \includegraphics{figtest_nw_thr2_06e_15_mod_noname.ps,width=0.2\textwidth}
%% %%    & 
%%   \end{tabular}
%% $$P_{k,t} \mbox{\ versus\ } k$$
%% 
%% %% 
%%    \end{frame}
%%
%% 
%% \begin{frame}
%%     \frametitle{The multiplier effect}
%%   
%%     \centering
%%     \includegraphics[width=0.9\textwidth]{fignw_threshold_gamma_multiplier13_avg_noname}
%%     
%%     Gamma distributed degrees (skewed)
%% %    \includegraphics[width=0.48\textwidth]{fignw_threshold_ramify_multiplier20_21comb3mod_noname}
%% %% 
%% %% 
%% 
%% \end{frame}
%% 
%% %%%%%%%%%%%%%%%
%% %% 1c  unconnected, room for a general model
%% %%    rumours don't spread like SIR models
%% %%    diseases aren't like threshold models
%% 
%% 
%% %%%%%%%%%%%%%% extensions
%% %%%%%%%%%%%%%% threshold model
%% %%%%%%%%%%%%%% 1. group structure
%% 
%% \begin{frame}
%%   \frametitle{Special subnetworks can act as triggers}
%% 
%%   \includegraphics[width=0.8\textwidth]{betheladder5}
%%   \begin{itemize}
%%   \item $\phi=1/3$ for all nodes
%%   \end{itemize}
%%   
%%   
%% \end{frame}

\neuralreboot{LMiYjkG4onM}{}{}{Pangolin happiness:}

\section{Theory}

\begin{frame}
  \frametitle{Threshold contagion on random networks}

  \begin{block}<1->{\alert{Three key pieces} to describe analytically:}
    \begin{enumerate}
    \item<2->
      The fractional size of the largest subcomponent of vulnerable nodes, 
      \alertb{$\Svuln$}.
    \item<3->
      The chance of starting a global spreading event, \alertb{$\Ptrig = \Strig$}.
    \item<4->
      The expected final size of any successful spread, \alertb{$S$}.
      \begin{itemize}
      \item<5-> n.b., the distribution of $S$ is almost always bimodal.
      \end{itemize}

    \end{enumerate}
  \end{block}

\end{frame}

\begin{frame}
  \frametitle{Example random network structure:}

  \begin{columns}
    \column{0.65\textwidth}
    \includegraphics[width=\textwidth]{2011-04-04random-network-contagion-sketch_3a-tp-5.pdf}
    \column{0.35\textwidth}
    \begin{itemize}
    \item 
      $\Omega_{\textnormal{crit}}$ = $\Omega_{\textnormal{vuln}}$ = critical mass = global vulnerable component
    \item 
      $\Omega_{\textnormal{trig}}$ = triggering component
    \item 
      $\Omega_{\textnormal{final}}$ = potential extent of spread
    \item 
      $\Omega$ = entire network
    \end{itemize}
  \end{columns}
  \bigskip
  $$
  \Omega_{\textnormal{crit}} 
  \subset
  \Omega_{\textnormal{trig}};
  \
  \Omega_{\textnormal{crit}} 
  \subset
  \Omega_{\textnormal{final}};
  \
  \mbox{and}
  \
  \Omega_{\textnormal{trig}},
  \Omega_{\textnormal{final}} 
  \subset
  \Omega.
  $$
\end{frame}

\subsection{Spreading\ possibility}

\begin{frame}
  \frametitle{Threshold contagion on random networks}

  \begin{itemize}
  \item<1-> 
    \alert{First goal:} 
    Find the largest component of vulnerable nodes.
  \item<2-> 
    Recall that for finding the giant component's size, 
    we had to solve:
    $$
    {
      F_{\pi}(x)
      =
      x F_{P}
      \left(
        F_{\rho} (x)
      \right)
    }
    \mbox{\ \  and \ }
    {
      F_{\rho}(x)
      =
      x F_{R}
      \left(
        F_{\rho} (x)
      \right)
    }
    $$
  \item<3-> 
    We'll find a similar result for 
    the subset of nodes that are vulnerable.
  \item<4-> 
    This is a node-based percolation problem.
  \item<5-> 
    For a general monotonic threshold distribution \alert{$f(\phi)$},
    a degree $k$ node is vulnerable with probability
    $$
    \infprob_{k1} = \int_{0}^{1/k} f(\phi) \dee{\phi}.
    $$
  \end{itemize}

\end{frame}

\begin{frame}
  \frametitle{Threshold contagion on random networks}

  \begin{itemize}
  \item<1->
    Everything now revolves around the \alert{modified} generating function:
    $$
    F_P^{(\vuln)}(x) 
    = 
    \sum_{k=0}^{\infty}
    \infprob_{k1}
    P_k
    x^k.
    $$
  \item<2->
    Generating function for friends-of-friends distribution is
    related in same way as before:
    $$
    F_R^{(\vuln)}(x) 
    = 
    \frac{
      \diff{}{x}{F}_P^{(\vuln)}(x)}
    {\diff{}{x} {F}_P^{(\vuln)}(x)|_{x=1}}.
    $$
  \end{itemize}

\end{frame}

\begin{frame}
  \frametitle{Threshold contagion on random networks}

  \begin{itemize}
  \item<1-> Functional relations for component size g.f.'s
    are almost the same...
    $$
    \uncover<2->{
      {F}_{\pi}^{(\vuln)}(x)
      =
    }
    \uncover<3->{
      \underbrace{
        1-{F}_P^{(\vuln)}(1)
      }_{
        \mbox{
          \scriptsize
          \begin{tabular}{l}
            central node \\
            is not \\
            vulnerable
          \end{tabular}
        }
      }
      +
    }
    \uncover<2->{
      x {F}_{P}^{(\vuln)}
      \left(
        {F}_{\rho}^{(\vuln)} (x)
      \right)
    }
    $$
    $$
    \uncover<4->{
      {F}_{\rho}^{(\vuln)}(x)
      =
    }
    \uncover<5->{
      \underbrace{
        1-{F}_R^{(\vuln)}(1)
      }_{
        \mbox{
          \scriptsize
          \begin{tabular}{l}
            first node \\
            is not \\
            vulnerable
          \end{tabular}
        }
      }
      +
    }
    \uncover<4->{
      x {F}_{R}^{(\vuln)}
      \left(
        {F}_{\rho}^{(\vuln)} (x)
      \right)
    }
    $$
  \item<6->
    Can now solve as before to find $S_{\textnormal{vuln}} = 1 - F_\pi^{(\vuln)}(1)$.
  \end{itemize}

\end{frame}

\neuralreboot{jofNR_WkoCE}{60}{120}{Vulpine vocalization:}

\subsection{Spreading\ probability}

\begin{frame}
  \frametitle{Threshold contagion on random networks}

  \begin{itemize}
  \item<1-> \alert{Second goal:}
    Find probability of triggering largest vulnerable 
    component.
  \item<2-> 
    Assumption is \alert{first node} is \alertb{randomly chosen}.
  \item<3->
    \alertb{Same set up} as for vulnerable component except
    now we don't care if the initial node is vulnerable or not:
    $$
    {F}_{\pi}^{(\textnormal{trig)}}(x)
    =
    x \alert{{F}_{P}}
    \left(
      {F}_{\rho}^{(\vuln)} (x)
    \right)
    $$
    $$
    {F}_{\rho}^{(\vuln)}(x)
    =
    1-{F}_R^{(\vuln)}(1)
    +
    x {{F}_{R}^{(\vuln)}}
    \left(
      {F}_{\rho}^{(\vuln)} (x)
    \right)
    $$
  \item<4->
    Solve as before to find $\Ptrig =  \Strig = 1 - F_\pi^{(\textnormal{trig)}}(1)$.
  \end{itemize}
\end{frame}

\begin{frame}[plain]
  \includegraphics[width=\textwidth]{2014-03-17advice-animals-cheezburger.jpg}
\end{frame}

\subsection{Final\ size}

\begin{frame}
  \frametitle{Threshold contagion on random networks}

  \begin{itemize}
  \item<1->
    \alert{Third goal:} Find expected fractional size of spread.
  \item<2->
    Not obvious even for uniform threshold problem.
  \item<3->
    Difficulty is in figuring out if and when
    nodes that need $\ge 2$ hits switch on.
  \item<4->
    Problem \alertb{solved} for infinite seed case by Gleeson and Cahalane:\\
    ``Seed size strongly affects cascades on random networks,'' 
    Phys.\ Rev.\ E, 2007.\cite{gleeson2007a}
  \item<5->
    Developed further by Gleeson
    in ``Cascades on correlated and modular random networks,'' 
    Phys.\ Rev.\ E, 2008.\cite{gleeson2008a}
  \end{itemize}

\end{frame}

\begin{frame}
  \frametitle{Expected size of spread}

  \begin{block}{Idea:}
    \begin{itemize}
    \item<1-> 
      Randomly turn on a fraction $\phi_0$ of nodes at time $t=0$
    \item<2-> 
      Capitalize on local branching network structure of random
      networks (again)
    \item<3-> 
      Now think about what must happen for
      a specific node $i$ to become active at time $t$:
    \item<4->[$\bullet$]
      \alertb{$t=0$:} $i$ is one of the seeds (prob = $\phi_0$)
    \item<5->[$\bullet$]
      \alertb{$t=1$:} $i$ was not a seed but enough of $i$'s friends switched
      on at time $t=0$ so that $i$'s threshold is now exceeded.
    \item<6->[$\bullet$] 
      \alertb{$t=2$:} enough of $i$'s friends and friends-of-friends switched
      on at time $t=0$ so that $i$'s threshold is now exceeded.
    \item<7->[$\bullet$] 
      \alertb{$t=n$:} enough nodes within $n$ hops of $i$ 
      switched on at $t=0$ and their effects have propagated to reach $i$.
    \end{itemize}
  \end{block}

\end{frame}


\begin{frame}
  \frametitle{Expected size of spread}

  \includegraphics<1>[angle=-90,width=\textwidth]{figfullspread.pdf}
  \includegraphics<2| handout: 0 | trans: 0>[angle=-90,width=\textwidth]{figfullspread2.pdf}
  \includegraphics<3| handout: 0 | trans: 0>[angle=-90,width=\textwidth]{figfullspread3.pdf}
  \includegraphics<4| handout: 0 | trans: 0>[angle=-90,width=\textwidth]{figfullspread4.pdf}
  \includegraphics<5| handout: 0 | trans: 0>[angle=-90,width=\textwidth]{figfullspread5.pdf}

\end{frame}

\begin{frame}
  \frametitle{Expected size of spread}

  \includegraphics<1| handout: 0 | trans: 0>[angle=-90,width=\textwidth]{figfullspread_circ1.pdf}
  \includegraphics<2| handout: 0 | trans: 0>[angle=-90,width=\textwidth]{figfullspread_circ2.pdf}
  \includegraphics<3| handout: 0 | trans: 0>[angle=-90,width=\textwidth]{figfullspread_circ3.pdf}
  \includegraphics<4| handout: 0 | trans: 0>[angle=-90,width=\textwidth]{figfullspread_circ4.pdf}
  \includegraphics<5>[angle=-90,width=\textwidth]{figfullspread_circ5.pdf}

\end{frame}


\begin{frame}
  \frametitle{Expected size of spread}

  \begin{block}{Notes:}
    \begin{itemize}
    \item<1-> 
      Calculations presume
      nodes do not become inactive (strong restriction, liftable)
    \item<2->
      Not just for threshold model---works
      for a wide range of contagion processes.
    \item<3->
      We can analytically determine the entire time evolution,
      not just the final size.
    \item<4->
      We can in fact determine \\
      $\Prob$(node of degree $k$ switches on at time $t$).
    \item<5->
      Even more, we can compute:
      $\Prob$(specific node $i$ switches on at time $t$).
    \item<6->
      Asynchronous updating can be handled too.
    \end{itemize}
  \end{block}

\end{frame}


\begin{frame}
  \frametitle{Expected size of spread}

  \begin{block}<1->{Pleasantness:}
    \begin{itemize}
    \item<1-> 
      \alertb{Taking off from a single seed story} is about \alert{expansion} away from
      a node.
    \item<2-> 
      \alertb{Extent of spreading story} is about \alert{contraction} at
      a node.
    \end{itemize}
    \includegraphics[width=\textwidth]{contraction-expansion-tp-10}
  \end{block}

\end{frame}

\begin{frame}
  \frametitle{Expected size of spread}

  \begin{itemize}
  \item<1->
    \alert{Notation:}
    $ \phi_{k,t} = 
    \Prob(\mbox{a degree $k$ node is active at time $t$}) $.
  \item<2->
    \alert{Notation:}
    $\infprob_{k j} = \Prob$ (a degree $k$ node becomes active
    if $j$ neighbors are active).
  \item<3-> 
    Our starting point: $ \phi_{k,0} = \phi_0$.
  \item<4->
    $ 
    \alertb{\binom{k}{j}
      \phi_0^{\, j}
      (1-\phi_0)^{k-j} }
    $ 
    =
    $\Prob$ ($j$ of a degree $k$ node's neighbors were seeded at time $t=0$).
  \item<5-> 
    Probability a degree $k$ node was a seed at $t=0$ is $\alert{\phi_0}$ (as above).
  \item<6-> 
    Probability a degree $k$ node was not a seed at $t=0$ is $\alert{(1-\phi_0)}$.
  \item<7-> 
    Combining everything, we have:
    $$ 
    \phi_{k,1}
    = 
    \alert{\phi_{0}}
    + 
    \alert{(1-\phi_{0})}
    \sum_{j=0}^{k}
    \alertb{\binom{k}{j}
    \phi_0^{\, j}
    (1-\phi_0)^{k-j} }
  \infprob_{k j}.
    $$
  \end{itemize}

\end{frame}

\begin{frame}
  \frametitle{Expected size of spread}

  \begin{itemize}
  \item<1->
    For general $t$, we need to know
    the probability an edge coming into a degree $k$ node
    at time $t$ is active.
  \item<2->
    \alert{Notation:} call this probability $\theta_t$.
  \item<3->
    We already know $\theta_0 = \phi_0$.
  \item<4->
    Story analogous to $t=1$ case.  For specific node $i$:
    $$
    \phi_{i,t+1}
    = 
    \alert{\phi_{0}}
    + 
    \alert{(1-\phi_{0})}
    \sum_{j=0}^{k_i}
    \alertb{\binom{k_i}{j}
    \theta_t^{\, j}
    (1-\theta_t)^{k_i-j}}
    \infprob_{k_i j}.
    $$
  \item<5->
    Average over all nodes with degree $k$ to obtain expression for $\phi_{t+1}$:
    $$
    \phi_{t+1}
    = 
    \alert{\phi_{0}}
    + 
    \alert{(1-\phi_{0})}
    \sum_{k=0}^{\infty} P_k 
    \sum_{j=0}^{k}
    \alertb{\binom{k}{j}
    \theta_t^{\, j}
    (1-\theta_t)^{k-j}}
    \infprob_{kj}.
    $$
  \item<6->
    So we need to compute $\theta_t$...  \visible<7>{\alertb{massive excitement...}}
  \end{itemize}
  
\end{frame}

\begin{frame}
  \frametitle{Expected size of spread}
  
  \begin{block}{First connect $\theta_0$ to $\theta_1$:}
    \begin{itemize}
    \item<1->
      $
      \theta_{1}
      =
      \phi_0 +
      $
      $$
      (1-\phi_0)
      \sum_{k=1}^{\infty}
      \alert{\frac{k P_k}{\tavg{k}}}
      \alertb{\sum_{j=0}^{k-1}}
      \binom{k-1}{j}
      \theta_{0}^{\ j}
      (1-\theta_{0})^{k-1-j}
      \infprob_{kj}
      $$
    \item<1->
      $ \alert{\frac{kP_k}{\tavg{k}}} = Q_k$ = $\Prob$ (edge connects to a degree $k$ node).
    \item<1->
      \alertb{$\sum_{j=0}^{k-1}$} piece gives $\Prob$ (degree node $k$ activates
      if $j$ of its $k-1$ incoming neighbors are active).
    \item<1->
      $\phi_0$ and $(1-\phi_0)$ terms account for state of node at time $t=0$.
    \item<2->
      See this all generalizes to give $\theta_{t+1}$ in terms of $\theta_{t}$...
    \end{itemize}
  \end{block}
\end{frame}

\begin{frame}
  \frametitle{Expected size of spread}
  
  \begin{block}{Two pieces: edges first, and then nodes}
    \begin{enumerate}
    \item<1->
      $
      \theta_{t+1}
      =
      \underbrace{\phi_0}_{\alertb{\mbox{exogenous}}} 
      $
      $$
      +
      (1-\phi_0)
      \underbrace{
      \sum_{k=1}^{\infty}
      \frac{k P_k}{\tavg{k}}
      \sum_{j=0}^{k-1}
      \binom{k-1}{j}
      \theta_{t}^{\ j}
      (1-\theta_{t})^{k-1-j}
      \infprob_{kj}
      }_{\alertb{\mbox{social effects}}}
      $$
      with $\theta_0 = \phi_0$.
    \item<1->
      $ 
      \phi_{t+1}
      = 
      $
      $$
      \underbrace{\phi_0}_{\alertb{\mbox{exogenous}}} 
      + 
      (1-\phi_{0})
      \underbrace{
      \sum_{k=0}^{\infty}
      P_k
      \sum_{j=0}^{k}
      \binom{k}{j}
      \theta_t^{\, j}
      (1-\theta_t)^{k-j} 
      \infprob_{kj}
      }_{\alertb{\mbox{social effects}}}.
      $$
    \end{enumerate}
  \end{block}

%% not true:
%%   \begin{itemize}
%%   \item<2-> 
%%     \alert{Observe:} $\theta_{t+1}$ is an increasing function of $\theta_{t}$
%%   \item<3-> 
%%     $\Rightarrow$ Stable attracting fixed point(s) must exist: $\theta_t \rightarrow \theta_\infty$
%%   \end{itemize}

\end{frame}


\begin{frame}
  \frametitle{Comparison between theory and simulations}

  \begin{columns}
    \column{0.5\textwidth}
    \includegraphics[width=\textwidth]{gleeson2007a_fig1.pdf}
    
    {\small From Gleeson and Cahalane\cite{gleeson2007a}}
    \column{0.5\textwidth}
    \begin{itemize}
    \item 
      Pure random networks with simple threshold responses
    \item 
      $R$ = uniform threshold (our $\phi_\ast$);
      $z$ = average degree; $\rho = \phi$; $q = \theta$; $N=10^5$.
    \item 
      $\phi_0 = 10^{-3}$, $0.5 \times 10^{-2}$,
      and 
      $10^{-2}$.
    \item
      Cascade window is for $\phi_0 = 10^{-2}$ case.
    \item
      Sensible expansion of cascade window as $\phi_0$ increases.
    \end{itemize}
  \end{columns}
\end{frame}

\begin{frame}
  \frametitle{Notes:}

    \begin{itemize}
    \item<1->
      Retrieve cascade condition for 
      spreading from a single seed in limit $\phi_0 \rightarrow 0$.
    \item<2-> 
      Depends on map $\theta_{t+1} = G(\theta_{t};\phi_0)$.
    \item<3-> 
      First: if self-starters are present, some activation is assured:
      $$
      G(0;\phi_0) = 
      \sum_{k=1}^{\infty} 
      \frac{kP_k}{\tavg{k}}
      \bullet
      \infprob_{k0} > 0.
      $$
      meaning $\infprob_{k0}>0$ for at least one value of $k \ge 1$.
    \item<4-> 
      If $\theta=0$ is a fixed point of $G$ (i.e., $G(0;\phi_0) = 0$)
      then spreading occurs if
      $$
      G'(0;\phi_0) = 
      \sum_{k=0}^{\infty} 
      \frac{k P_k}{\tavg{k}}
      \bullet
      (k-1) 
      \bullet
      \infprob_{k1} > 1.
      $$
      \insertassignmentquestionsoft{08}{8}
    \end{itemize}
\end{frame}

\begin{frame}
  \frametitle{Notes:}

  \begin{block}{In words:}    
    \begin{itemize}
    \item<1-> 
      If $G(0;\phi_0) > 0$, spreading must occur because
      some nodes turn on for free.
    \item<2-> 
      If $G$ has an \alert{unstable fixed point} at $\theta = 0$,
      then cascades are also always possible.
    \end{itemize}
  \end{block}

  \begin{block}<3->{Non-vanishing seed case:}
    \begin{itemize}
    \item<3-> 
      Cascade condition is more complicated for
      $\phi_0 > 0$.
    \item<4-> 
      If $G$ has a \alert{stable fixed point} at $\theta = 0$,
      and an \alert{unstable fixed point} for some $0 < \theta_\ast < 1$,
      then for $\theta_0  > \theta_\ast$, spreading takes off.
    \item<5->
      Tricky point: $G$ depends on $\phi_0$, so as we change
      $\phi_0$, we also change $G$.
    \end{itemize}
  \end{block}

\end{frame}

\begin{frame}
  \frametitle{General fixed point story:}

  \begin{columns}
    \column{0.33\textwidth}
    \includegraphics[angle=0,width=\textwidth]{figGfunction01.pdf}
    \column{0.33\textwidth}
    \includegraphics[angle=0,width=\textwidth]{figGfunction02.pdf}
    \column{0.33\textwidth}
    \includegraphics[angle=0,width=\textwidth]{figGfunction03.pdf}
  \end{columns}

  \begin{itemize}
  \item<1-> 
    Given $\theta_0 (= \phi_0)$, $\theta_\infty$ will be 
    the nearest stable fixed point, either above or below.
  \item<2-> 
    n.b., adjacent fixed points must have opposite stability types.
  \item<3-> 
    \alert{Important:}
    Actual form of $G$ depends on $\phi_0$.  
  \item<4->
    So choice of $\phi_0$ dictates both $G$ and starting
    point---can't start anywhere for a given $G$.
  \end{itemize}

\end{frame}

\begin{frame}
  \frametitle{Comparison between theory and simulations}

  \begin{columns}
    \column{0.5\textwidth}
    \includegraphics[width=\textwidth]{gleeson2007a_fig2.pdf}

    {\small From Gleeson and Cahalane\cite{gleeson2007a}}
    \column{0.5\textwidth}
    \begin{itemize}
    \item<1-> 
      Now allow thresholds to be distributed
      according to a Gaussian with mean $R$.
    \item<1-> 
      $R$ = \alert{0.2}, \alertb{0.362}, and 0.38; $\sigma = 0.2$.
    \item<2->
      $\phi_0 = 0$ but some nodes have thresholds $\le 0$
      so effectively $\phi_0 > 0$.
    \item<3->
      Now see a (nasty) discontinuous phase transition
      for low $\tavg{k}$.
    \end{itemize}
  \end{columns}
\end{frame}

\begin{frame}
  \frametitle{Comparison between theory and simulations}

  \begin{columns}
    \column{0.5\textwidth}
    \includegraphics[width=\textwidth]{gleeson2007a_fig3.pdf}

    {\small From Gleeson and Cahalane\cite{gleeson2007a}}
    \column{0.5\textwidth}
    \begin{itemize}
    \item<1-> 
      Plots of stability points for $\theta_{t+1} = G(\theta_t; \phi_0)$.
    \item<1-> 
      n.b.: 0 is not a fixed point here: $\theta_0 = 0$ always takes off.
    \item<1-> 
      Top to bottom: $R$ = 0.35, 0.371, and 0.375.
    \item<2-> 
      n.b.: higher values of $\theta_0$ for (b) and (c)
      lead to higher fixed points of $G$.
    \item<3-> 
      Saddle node bifurcations 
      appear and merge (b and c).
    \end{itemize}
  \end{columns}

\end{frame}

\begin{frame}
  \frametitle{Spreadarama}

  \begin{block}{Bridging to single seed case:}
    \begin{itemize}
    \item<1-> 
      Consider largest vulnerable component
      as initial set of seeds.
    \item<2->
      Not quite right as spreading must move
      through vulnerables.
    \item<3->
      But we can usefully think of the vulnerable
      component as activating at time $t=0$
      because order doesn't matter.
    \item<4->
      Rebuild $\phi_t$ and $\theta_t$ expressions...
    \end{itemize}
  \end{block}

\end{frame}

\begin{frame}
  \frametitle{Spreadarama}

  \begin{block}{Two pieces modified for single seed:}
    \begin{enumerate}
    \item<1->
      $
      \theta_{t+1}
      =
      \thetavuln 
      \ +
      $
      $$
      (1-\thetavuln)
      \sum_{k=1}^{\infty}
      \frac{k P_k}{\tavg{k}}
      \sum_{j=0}^{k-1}
      \binom{k-1}{j}
      \theta_{t}^{\ j}
      (1-\theta_{t})^{k-1-j}
      \infprob_{kj}
      $$
      with $\theta_0 = \thetavuln$ = $\Prob$ an edge leads
      to the giant vulnerable component (if it exists).
    \item<1->
      $ 
      \phi_{t+1}
      = 
      \Svuln
      \ + 
      $
      $$
      (1-\Svuln)
      \sum_{k=0}^{\infty}
      P_k
      \sum_{j=0}^{k}
      \binom{k}{j}
      \theta_t^{\, j}
      (1-\theta_t)^{k-j} 
      \infprob_{kj}.
      $$
    \end{enumerate}
  \end{block}

\end{frame}


\begin{frame}
  \frametitle{Time-dependent solutions}

  \begin{block}<1->{Synchronous update}
    \begin{itemize}
    \item<2-> Done: Evolution of $\phi_t$ and $\theta_t$
      given exactly by the maps we have
      derived.
    \end{itemize}
  \end{block}

  \begin{block}<3->{Asynchronous updates}
    \begin{itemize}
    \item<3-> 
      Update nodes with probability $\alpha$.
    \item<4-> 
      As $\alpha \rightarrow 0$, updates become
      effectively independent.
    \item<5-> 
      Now can talk about $\phi(t)$ and $\theta(t)$.
    \end{itemize}
  \end{block}

\end{frame}

\begin{comment}
  
\section{Appendix}

\begin{frame}
  \frametitle{Threshold contagion on random networks}

  \begin{itemize}
  \item<1-> 
    \alert{First goal:} 
    Find the largest component of vulnerable nodes.
  \item<2-> 
    Recall that for finding the giant component's size, 
    we had to solve:
    $$
    {
      F_{\pi}(x)
      =
      x F_{P}
      \left(
        F_{\rho} (x)
      \right)
    }
    \mbox{\ \  and \ }
    {
      F_{\rho}(x)
      =
      x F_{R}
      \left(
        F_{\rho} (x)
      \right)
    }
    $$
  \item<3-> 
    We'll find a similar result for 
    the subset of nodes that are vulnerable.
  \item<4-> 
    This is a node-based percolation problem.
  \item<5-> 
    For a general monotonic threshold distribution \alert{$f(\phi)$},
    a degree $k$ node is vulnerable with probability
    $$
    \infprob_{k1} = \int_{0}^{1/k} f(\phi) \dee{\phi}.
    $$
  \end{itemize}

\end{frame}

\begin{frame}
  \frametitle{Threshold contagion on random networks}

  \begin{itemize}
  \item<1->
    Everything now revolves around the \alert{modified} generating function:
    $$
    F_P^{(\vuln)}(x) 
    = 
    \sum_{k=0}^{\infty}
    \infprob_{k1}
    P_k
    x^k.
    $$
  \item<2->
    Generating function for friends-of-friends distribution is
    related in same way as before:
    $$
    F_R^{(\vuln)}(x) 
    = 
    \frac{
      \diff{}{x}{F}_P^{(\vuln)}(x)}
    {\diff{}{x} {F}_P^{(\vuln)}(x)|_{x=1}}.
    $$
  \end{itemize}

\end{frame}

\begin{frame}
  \frametitle{Threshold contagion on random networks}

  \begin{itemize}
  \item<1-> Functional relations for component size g.f.'s
    are almost the same...
    $$
    \uncover<2->{
      {F}_{\pi}^{(\vuln)}(x)
      =
    }
    \uncover<3->{
      \underbrace{
        1-{F}_P^{(\vuln)}(1)
      }_{
        \mbox{
          \scriptsize
          \begin{tabular}{l}
            central node \\
            is not \\
            vulnerable
          \end{tabular}
        }
      }
      +
    }
    \uncover<2->{
      x {F}_{P}^{(\vuln)}
      \left(
        {F}_{\rho}^{(\vuln)} (x)
      \right)
    }
    $$
    $$
    \uncover<4->{
      {F}_{\rho}^{(\vuln)}(x)
      =
    }
    \uncover<5->{
      \underbrace{
        1-{F}_R^{(\vuln)}(1)
      }_{
        \mbox{
          \scriptsize
          \begin{tabular}{l}
            first node \\
            is not \\
            vulnerable
          \end{tabular}
        }
      }
      +
    }
    \uncover<4->{
      x {F}_{R}^{(\vuln)}
      \left(
        {F}_{\rho}^{(\vuln)} (x)
      \right)
    }
    $$
  \item<6->
    Can now solve as before to find $S_{\textnormal{vuln}} = 1 - F_\pi^{(\vuln)}(1)$.
  \end{itemize}

\end{frame}

\begin{frame}
  \frametitle{Threshold contagion on random networks}

  \begin{itemize}
  \item<1-> \alert{Second goal:}
    Find probability of triggering largest vulnerable 
    component.
  \item<2-> 
    Assumption is \alert{first node} is \alertb{randomly chosen}.
  \item<3->
    \alertb{Same set up} as for vulnerable component except
    now we don't care if the initial node is vulnerable or not:
    $$
    {F}_{\pi}^{(\textnormal{trig)}}(x)
    =
    x \alert{{F}_{P}}
    \left(
      {F}_{\rho}^{(\vuln)} (x)
    \right)
    $$
    $$
    {F}_{\rho}^{(\vuln)}(x)
    =
    1-{F}_R^{(\vuln)}(1)
    +
    x {{F}_{R}^{(\vuln)}}
    \left(
      {F}_{\rho}^{(\vuln)} (x)
    \right)
    $$
  \item<4->
    Solve as before to find $\Ptrig =  \Strig = 1 - F_\pi^{(\textnormal{trig)}}(1)$.
  \end{itemize}
\end{frame}

\end{comment}
