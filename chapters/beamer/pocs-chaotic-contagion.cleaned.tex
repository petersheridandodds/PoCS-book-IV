\section{Chaotic\ Contagion}

  \textbf{Chaotic Contagion on Networks:}

  
    \displaypaper{dodds2013a}{1}
  

  
    \displaypaper{harris2013a}{1}
  



\subsection{Chaos}

  \textbf{Chaotic contagion:}

  
     What if individual response functions are not monotonic?
    Consider a simple deterministic version:
          
      
       Node $i$ has an \alertb{`activation threshold'} $\phi_{i,1}$
      [] \ldots and a \alertb{`de-activation threshold'} $\phi_{i,2}$
       Nodes like to imitate but only up to a limit---they don't
        want to be like everyone else.
      
      
      \includegraphics[width=\textwidth]{figupdownrfn_noname_A} 
      



  \includegraphics[width=\textwidth]{figupdownrfn_network05_noname.pdf} 


  \textbf{Chaotic contagion}

  \textbf{Definition of the tent map:}
    $$ 
    F(x) = 
    \left\{
      \begin{array}{l}
        rx \mbox{\ for $0 \le x \le \frac{1}{2}$}, \\
        r(1-x) \mbox{\ for $\frac{1}{2} \le x \le 1$}. \\
      \end{array}
    \right.
    $$
    
     The usual business: 
      look at how
      $F$ iteratively maps the unit interval $[0,1]$.
    
  


  \textbf{The tent map}

  Effect of increasing $r$ from 1 to 2.
      
    \includegraphics[width=\textwidth]{figtentmap7_noname} 
    
    \includegraphics[width=\textwidth]{figtentmap6_noname} 
    
    \includegraphics[width=\textwidth]{figtentmap8_noname} 
  
      
    
          
      \includegraphics[width=\textwidth]{fignetwork_ud_meanfield02_pdd_noname}  
      
%      \alert{$\rightpointleft$ Orbit diagram} \\
%      \alert{$\HandCuffLeft$ Orbit diagram} \\
      \textbf{Orbit diagram:}
        Chaotic behavior increases
        as map slope $r$ is increased.
      
      

  \textbf{Chaotic behavior}

  Take $r=2$ case:
  \begin{center}
    \includegraphics[width=0.32\textwidth]{figtentmap8_noname} 
  \end{center}

  
   What happens if nodes have limited information?
   As before, allow interactions to take place on a sparse random network.
   Vary average degree $z=\tavg{k}$, a measure of information
  


  \textbf{Two population examples:}

  \includegraphics[width=\textwidth]{figupdownrfn_noname} 
    
  
     
      Randomly select $(\phi_{i,1},\phi_{i,2})$ from
      gray regions shown in plots B and C.
    
      Insets show composite response function
      averaged over population.
    
      We'll consider plot C's example: \alertb{the tent map}.
  


  \textbf{Invariant densities---stochastic response functions}

  \centering
  \begin{tabular}{cc}
    \includegraphics[width=0.48\textwidth]{fignetwork_ud_purerandom_seq14c1_1_1_noname} &
    \includegraphics[width=0.48\textwidth]{fignetwork_ud_purerandom_seq14c2_1_1_noname} \\
    activation time series & activation density \\
  \end{tabular}


  \textbf{Invariant densities---stochastic response functions}
  \begin{tabular}{llll}
    \includegraphics[width=0.2\textwidth]{fignetwork_ud_purerandom_seq14c1_1_1_noname} &
    \includegraphics[width=0.2\textwidth]{fignetwork_ud_purerandom_seq14c2_1_1_noname} &
    \includegraphics[width=0.2\textwidth]{fignetwork_ud_purerandom_seq14c1_4_2_noname} &
    \includegraphics[width=0.2\textwidth]{fignetwork_ud_purerandom_seq14c2_4_2_noname} \\
    \\
    \includegraphics[width=0.2\textwidth]{fignetwork_ud_purerandom_seq14c1_13_6_noname} &
    \includegraphics[width=0.2\textwidth]{fignetwork_ud_purerandom_seq14c2_13_6_noname} &
    \includegraphics[width=0.2\textwidth]{fignetwork_ud_purerandom_seq14c1_4_8_noname} &
    \includegraphics[width=0.2\textwidth]{fignetwork_ud_purerandom_seq14c2_4_8_noname} \\
    \\
    \includegraphics[width=0.2\textwidth]{fignetwork_ud_purerandom_seq14c1_2_9_noname} &
    \includegraphics[width=0.2\textwidth]{fignetwork_ud_purerandom_seq14c2_2_9_noname} &
    \includegraphics[width=0.2\textwidth]{fignetwork_ud_purerandom_seq14c1_13_12_noname} &
    \includegraphics[width=0.2\textwidth]{fignetwork_ud_purerandom_seq14c2_13_12_noname} \\
  \end{tabular}



  \textbf{Invariant densities---deterministic response functions for one specific network with $\tavg{k}=18$}
  \begin{tabular}{ll}
    \includegraphics[width=0.2\textwidth]{fignetwork_ud_purerandom_seq16_1_1_1_noname}%
    \includegraphics[width=0.2\textwidth]{fignetwork_ud_purerandom_seq16_2_1_1_noname} &
    \qquad    \includegraphics[width=0.2\textwidth]{fignetwork_ud_purerandom_seq16_1_18_1_noname}%
    \includegraphics[width=0.2\textwidth]{fignetwork_ud_purerandom_seq16_2_18_1_noname} \\
    \\
    \includegraphics[width=0.2\textwidth]{fignetwork_ud_purerandom_seq16_1_5_1_noname}%
    \includegraphics[width=0.2\textwidth]{fignetwork_ud_purerandom_seq16_2_5_1_noname} &
    \qquad    \includegraphics[width=0.2\textwidth]{fignetwork_ud_purerandom_seq16_1_8_1_noname}%
    \includegraphics[width=0.2\textwidth]{fignetwork_ud_purerandom_seq16_2_8_1_noname} \\
    \\
    \includegraphics[width=0.2\textwidth]{fignetwork_ud_purerandom_seq16_1_9_1_noname}%
    \includegraphics[width=0.2\textwidth]{fignetwork_ud_purerandom_seq16_2_9_1_noname} &
    \qquad    \includegraphics[width=0.2\textwidth]{fignetwork_ud_purerandom_seq16_1_10_1_noname}%
    \includegraphics[width=0.2\textwidth]{fignetwork_ud_purerandom_seq16_2_10_1_noname} \\
  \end{tabular}

  %%      Examples of time series for one realized random network with $\avg{k}=18$ and
  %%      varying randomly chosen initial seeds (I have excluded events that did not take off).
  %%      Very interesting---sometimes the dynamics hold on for at least 2000 cycles but many
  %%      times they collapse to one or two period orbits.


  \textbf{Invariant densities---stochastic response functions}
  \begin{tabular}{llll}
    \includegraphics[width=0.22\textwidth]{fignetwork_ud_purerandom_seq14d1_1_1_noname} &
    \includegraphics[width=0.22\textwidth]{fignetwork_ud_purerandom_seq14d2_1_1_noname} &
    \includegraphics[width=0.22\textwidth]{fignetwork_ud_purerandom_seq14e1_1_1_noname} &
    \includegraphics[width=0.22\textwidth]{fignetwork_ud_purerandom_seq14e2_1_1_noname} \\
  \end{tabular}
  Trying out higher values of $\tavg{k}$\ldots
        
%      Examples of time series for single initial seeds and stochastic tent map response functions.
%      As $z$ increases, the invariant density spreads out as would expect (for the fully mixed
%      version, the invariant density is uniform, $\rho(\phi)=1$.
%      Again, $N=10^4$ and the histograms of cascade size $s$ are derived from $s(t)$ for $t \ge 100$.


  \textbf{Invariant densities---deterministic response functions}
  \begin{tabular}{llll}
    \includegraphics[width=0.22\textwidth]{fignetwork_ud_purerandom_seq13d1_1_1_noname} &
      \includegraphics[width=0.22\textwidth]{fignetwork_ud_purerandom_seq13d2_1_1_noname} &
      \includegraphics[width=0.22\textwidth]{fignetwork_ud_purerandom_seq13e1_1_1_noname} &
      \includegraphics[width=0.22\textwidth]{fignetwork_ud_purerandom_seq13e2_1_1_noname} \\
      \includegraphics[width=0.22\textwidth]{fignetwork_ud_purerandom_seq13f1_1_1_noname} &
      \includegraphics[width=0.22\textwidth]{fignetwork_ud_purerandom_seq13f2_1_1_noname} &
      \\
  \end{tabular}
  Trying out higher values of $\tavg{k}$\ldots

%      Same as Fig.~\ref{fig:network_ud_purerandom_seq13de} but back to deterministic
%      up/down response functions.
%      Again, $N=10^4$ and the histograms of cascade size $s$ are derived from $s(t)$ for $t \ge 100$.
%      The third plot shows that even for $\tavg{k}=100$, the dynamics can collapse.



  \textbf{Connectivity leads to chaos:}

      
    \includegraphics[width=\textwidth]{fignetwork_ud_meanfield02_pdd_noname}\\
    \includegraphics[width=\textwidth]{fignetwork_ud_meanfield04_pdd_noname} 
    
    Stochastic response functions:
    \includegraphics[width=\textwidth]{fignetwork_ud_purerandom_seq14_pdd_noname}  
    



  
  \textbf{Bifurcation diagram: Asynchronous updating}

      
    \includegraphics[width=\textwidth]{fignetwork_ud_meanfield02_pdd_noname} \\
    \includegraphics[width=\textwidth]{fignetwork_ud_meanfield04_pdd_noname} 
    
    \includegraphics[width=\textwidth]{fignetwork_ud_meanfield06_pdd_noname}  
    

  \textbf{Bifurcation diagram: Asynchronous updating}

    \includegraphics[height=0.9\textheight]{harris2013a_fig3}  
  



%%%%  \textbf{Chaotic behavior---the tent map}
%%
%%  \includegraphics[width=0.3\textwidth]{figtentmap5_noname} 
%%  \includegraphics[width=0.3\textwidth]{figtentmap4_noname} 
%%  \includegraphics[width=0.3\textwidth]{figtentmap3_noname} 
%%
%%  1, 2, and 5-fold self-composed tent maps.
%%
%%%%
%%%%  \textbf{Standard tent map bifurcation diagram}
%%  \centering
%%  \includegraphics[width=0.7\textwidth]{fignetwork_ud_meanfield02_pdd_noname}  
%%
%%%%
%%%%  \textbf{Tent map bifurcation diagram with $\tavg{k}$ as parameter---stochastic}
%%  \centering
%%  \includegraphics[width=0.7\textwidth]{fignetwork_ud_purerandom_seq14_pdd_noname}  
%%
%%%%
%%%%  \textbf{Tent map bifurcation diagram with $\tavg{k}$ as parameter---deterministic}
%%  \centering
%%  \includegraphics[width=0.7\textwidth]{fignetwork_ud_purerandom_seq13_pdd_noname}  
%%
%%%%
%%%%  \textbf{Standard logistic map bifurcation diagram}
%%  \centering
%%  \includegraphics[width=0.7\textwidth]{fignetwork_ud_meanfield04_pdd_noname}  
%%
%%%%
%%
%%
%%%%  \textbf{Invariant densities---deterministic response functions}
%%  \begin{tabular}{llll}
%%      \includegraphics[width=0.22\textwidth]{fignetwork_ud_purerandom_seq13c1_1_1_noname} &
%%      \includegraphics[width=0.22\textwidth]{fignetwork_ud_purerandom_seq13c2_1_1_noname} &
%%      \includegraphics[width=0.22\textwidth]{fignetwork_ud_purerandom_seq13c1_5_2_noname} &
%%      \includegraphics[width=0.22\textwidth]{fignetwork_ud_purerandom_seq13c2_5_2_noname} \\
%%      \includegraphics[width=0.22\textwidth]{fignetwork_ud_purerandom_seq13c1_1_4_noname} &
%%      \includegraphics[width=0.22\textwidth]{fignetwork_ud_purerandom_seq13c2_1_4_noname} &
%%      \includegraphics[width=0.22\textwidth]{fignetwork_ud_purerandom_seq13c1_5_7_noname} &
%%      \includegraphics[width=0.22\textwidth]{fignetwork_ud_purerandom_seq13c2_5_7_noname} \\
%%      \includegraphics[width=0.22\textwidth]{fignetwork_ud_purerandom_seq13c1_1_10_noname} &
%%      \includegraphics[width=0.22\textwidth]{fignetwork_ud_purerandom_seq13c2_1_10_noname} &
%%      \includegraphics[width=0.22\textwidth]{fignetwork_ud_purerandom_seq13c1_4_12_noname} &
%%      \includegraphics[width=0.22\textwidth]{fignetwork_ud_purerandom_seq13c2_4_12_noname} \\
%%  \end{tabular}
%%
%%
%%


%%%%%%%%%%%%%% extensions
%%%%%%%%%%%%%% threshold model
%%%%%%%%%%%%%% 2. non-monotonic response functions

  
    \youtubevideo{7JHrZyyq870}{}{} 
    \small 
    How the bifurcation diagram changes with increasing average degree
    $\tavg{k}$
    as a function of the synchronicity parameter $\alpha$
    for the stochastic response (tent map) case.
  

  
    \youtubevideo{_zwK6poIBvc}{}{}
    \small
    How the bifurcation diagram changes with increasing $\alpha$, the
    synchronicity parameter as a function of average degree $\tavg{k}$
    for the stochastic response (tent map) case.
  

  
    \youtubevideo{3bo4fzp4Snw}{}{}
    \small
    LIC dynamics on a fixed graph with a
    shared stochastic (tent map) response function. Average degree =
    6, update synchronicity parameter $\alpha$ = 1. The macroscopic
    behavior is period-1, plus noisy fluctuations.
  

  
    \youtubevideo{7UCuIa_ktmw}{}{} 
    \small 
    LIC dynamics on a fixed
    graph with a shared stochastic (tent map) response
    function. Average degree = 11, update synchronicity parameter
    $\alpha = 1$. The macroscopic behavior is period-2, plus noisy
    fluctuations.
  

  
    \youtubevideo{oWKt8Zj1Ccw}{}{}
    \small
    LIC dynamics on a fixed graph with a
    shared stochastic (tent map) response function. $\avg{k} = 30$,
    update synchronicity parameter $\alpha = 1$.
    The macroscopic behavior is
    chaotic.
  

  
    \youtubevideo{AfhUlkIOiOU}{}{}
    \small
    LIC dynamics on a fixed graph with
    fixed, deterministic response functions. Average degree = 30,
    update synchronicity parameter $\alpha$ = 1. Shown are nodes
    which continue changing (703/1000) after the transient chaotic
    behavior has "collapsed."
  

  
    \youtubevideo{ZwY0hTstJ2M}{}{}
    \small
    LIC dynamics on a fixed graph with fixed, deterministic response
    functions. Average degree = 30, update synchronicity parameter
    $\alpha$ = 1. The dynamics exhibit transient chaotic behavior before
    collapsing to a fixed point.
  

  
    \youtubevideo{YDhjmFyBSn4}{}{}
    \small
    LIC dynamics on a fixed graph with fixed, deterministic response
    functions. Average degree = 17, update synchronicity parameter
    $\alpha$ = 1. The dynamics exhibit transient chaotic behavior before
    collapsing to a period-4 orbit.
  


\begin{frame}[plain]

  \includegraphics[width=1.2\textwidth]{fignetwork_ud_purerandom_orbitdiagrams006_noname.pdf}



\begin{frame}[plain]
  
  \includegraphics[width=1.2\textwidth]{fignetwork_ud_purerandom_seq77_collectedgoodies003_noname.pdf}




