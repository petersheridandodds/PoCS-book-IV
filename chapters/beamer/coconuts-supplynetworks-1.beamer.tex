%%\section{Introduction}

\section{Optimal\  transportation}

\begin{frame}
  \frametitle{Optimal supply networks}

  \begin{block}<1->{What's the best way to distribute stuff?}
    \begin{itemize}
    \item<2-> Stuff = medical services, energy, people, \ldots
    \item<3-> \alert{Some} fundamental network problems:
      \begin{enumerate}
      \item<4-> Distribute stuff from a \alert{single source} to
        \alert{many sinks}
      \item<5-> Distribute stuff from \alert{many sources} to
        many sinks
      \item<6-> \alert{Redistribute} stuff between nodes that
        are both sources and sinks
      \end{enumerate}
    \item<7-> Supply and Collection are equivalent problems
    \end{itemize}
  \end{block}
  
\end{frame}


\begin{frame}
  \frametitle{Single source optimal supply}

  \begin{block}{Basic question for distribution/supply networks:}
    \begin{itemize}
    \item<1->
      How does flow behave given cost:
      $$
      C 
      = 
      \sum_{j} I_j^{\, \gamma} Z_j
      $$
      where \\
      \alertb{$
      I_j 
      $
      = current on link $j$}\\
      and\\
      \alertb{$Z_j$ = link $j$'s impedance}?
    \item<2->
      Example: $\gamma=2$ for electrical networks.
    \end{itemize}
  \end{block}
  
\end{frame}


\begin{frame}
  \frametitle{Single source optimal supply}

  \includegraphics[width=\textwidth]{bohn2007a_fig2}
  
  \begin{itemize}
  \item[(a)]
    $\gamma > 1$: \alert{Braided} (bulk) flow 
  \item[(b)]
    $\gamma < 1$:
    Local minimum: \alert{Branching} flow
  \item[(c)] 
    $\gamma < 1$:
    Global minimum: \alert{Branching} flow
    
  \end{itemize}

  \medskip
  
  {\small From Bohn and Magnasco\cite{bohn2007a}}

  {\small See also Banavar et al.\cite{banavar2000a}}

\end{frame}

\begin{frame}
  \frametitle{Single source optimal supply}

  Optimal paths related to transport (Monge) problems:

  \includegraphics[width=0.48\textwidth]{xia2003a_fig1.pdf}
  \includegraphics[width=0.48\textwidth]{xia2003a_fig6.pdf}

  Xia (2003)\cite{xia2003a}

\end{frame}

\begin{frame}
  \frametitle{Growing networks:}

  \begin{center}
    \includegraphics[width=0.75\textwidth]{xia2007a_fig1.pdf}
  \end{center}

%%  \includegraphics[width=0.49\textwidth]{xia2007a_fig2}

  Xia (2007)\cite{xia2007a}

\end{frame}

\begin{frame}
  \frametitle{Growing networks:}

  \begin{center}
    \includegraphics[width=0.75\textwidth]{xia2007a_fig3.pdf}
  \end{center}

%%  \includegraphics[width=0.49\textwidth]{xia2007a_fig4}

  Xia (2007)\cite{xia2007a}

\end{frame}

\begin{frame}
  \frametitle{Single source optimal supply}

  \begin{block}{An immensely controversial issue...}
    \begin{itemize}
    \item<1->
      The form of river networks and blood networks:
      optimal or not?\cite{west1997a,banavar1999a,dodds2001a,dodds2010a}
    \end{itemize}
  \end{block}

  \begin{block}<2->{Two observations:}
    \begin{itemize}
    \item<2->
      Self-similar networks appear everywhere in nature
      for single source supply/single sink collection.
    \item<3->
      Real networks \alertb{differ} in \alert{details of scaling}
      but reasonably \alertb{agree} in \alert{scaling relations}.
    \end{itemize}
  \end{block}

\end{frame}


\begin{frame}
  \frametitle{River network models}

  \begin{block}<1->{Optimality:}
    \begin{itemize}
    \item<1->Optimal channel networks\cite{rodriguez-iturbe1997a}
    \item<1->Thermodynamic analogy\cite{scheidegger1991a}
    \end{itemize}
  \end{block}
  \uncover<2->{versus...}
  \begin{block}<2->{Randomness:}
    \begin{itemize}
    \item<2->Scheidegger's directed random networks
    \item<2->Undirected random networks
    \end{itemize}
  \end{block}
  
\end{frame}

\section{Optimal\ branching}

\begin{frame}

  \showtarotcards{0.35}{
%%  \dealnewtarotcard{0.35}{
    john-dory,
    overview,
    complex-networks,
    random-networks,
    scale-free-networks,
    small-world-networks,
    theory-six-degrees,
    landscapes-of-forking-paths,
    networks-of-blood,
    trees-of-reality,
    orders-of-streams,
    laws-of-branching,
    unknown-mechanism,
    law-of-optimal-forks,
}

\end{frame}

\subsection{Murray's\ law}

\begin{frame}
  \frametitle{Optimization---Murray's law}

  \begin{columns}
    \column{0.45\textwidth}
    \includegraphics[width=\textwidth]{murrays-law-tp-10}
    \column{0.55\textwidth}
    \begin{itemize}
    \item<1-> Murray's law (1926) connects branch radii at forks:\cite{murray1926a,murray1926b,murray1927a,labarbera1990a,thompson1961a}
      $$ \boxed{ \alert{r_0^{3} = r_1^{3} + r_2^{3}}} $$
      where $r_0$ = radius of main branch,
      and $r_1$ and $r_2$ are radii of sub-branches.
    \end{itemize}
  \end{columns}
    \begin{itemize}
    \item<2-> Holds up well for outer branchings of blood networks.
    \item<3-> Also found to hold for trees\cite{murray1927a,mcculloh2003a,mcculloh2004a}.
    \item<4-> See D'Arcy Thompson's ``On Growth and Form'' for background inspiration\cite{thompson1952a,thompson1961a}.
    \end{itemize}

\end{frame}


%% \begin{frame}
%%   \frametitle{Optimization approaches}
%% 
%%   \begin{block}{Cardiovascular networks:}
%%     \begin{itemize}
%%     \item<1-> Murray's law (1926) connects branch radii at forks:\cite{murray1926a,murray1926b,murray1927a,labarbera1990a,thompson1961a}
%%       $$ \boxed{ \alert{r_0^{3} = r_1^{3} + r_2^{3}}} $$
%%       where $r_0$ = radius of main branch\\
%%       and $r_1$ and $r_2$ are radii of sub-branches.
%%     \item<1-> See D'Arcy Thompson's ``On Growth and Form'' for background inspiration\cite{thompson1952a,thompson1961a}.
%%     \item<2-> Calculation assumes 
%%       \wordwikilink{http://en.wikipedia.org/wiki/Hagen-Poiseuille_equation}{Poiseuille flow}.
%%     \item<3-> Holds up well for outer branchings of blood networks.
%%     \item<4-> Also found to hold for trees\cite{murray1927a,mcculloh2003a,mcculloh2004a}.
%%     \item<5-> Use hydraulic equivalent of Ohm's law:
%%       $$
%%       \Delta p = \Phi Z  \Leftrightarrow V = IR
%%       $$
%%       where $\Delta p$ = pressure difference, $\Phi$ = flux.
%%     \end{itemize}
%%   \end{block}
%% \end{frame}

\begin{frame}
%%  \frametitle{Optimization---Murray's law}

    \begin{itemize}
    \item<1-> Use hydraulic equivalent of Ohm's law:
      $$
      \Delta p = \Phi Z  \Leftrightarrow V = IR
      $$
      where $\Delta p$ = pressure difference, $\Phi$ = flux.
      \begin{columns}
        \column{0.4\textwidth}
        \includegraphics[width=\textwidth]{poiseuille-flow-tp-10}
        \column{0.6\textwidth}
        \begin{itemize}
        \item<2->
          Fluid mechanics: 
          \wordwikilink{http://en.wikipedia.org/wiki/Hagen-Poiseuille_equation}{Poiseuille impedance}
          for smooth 
          \wordwikilink{http://en.wikipedia.org/wiki/Hagen-Poiseuille_equation}{Poiseuille flow}
          in a tube
          of radius $r$ and length $\ell$:
          $$ Z = \frac{8\eta \ell}{\pi r^4} $$
        \end{itemize}
      \end{columns}
    \item<2->
      $\eta$ = 
      \wordwikilink{http://en.wikipedia.org/wiki/Dynamic_viscosity}{dynamic viscosity}
      (units: $ML^{-1}T^{-1}$).
    \item<3-> 
      Power required to overcome impedance: 
      $$ P_{\textnormal{drag}} = \Phi \Delta p  = \Phi^2 Z. $$
    \item<4-> 
      Also have rate of energy expenditure in maintaining blood
      given metabolic constant $c$:
      $$ P_{\textnormal{metabolic}} = c r^2 \ell  $$
    \end{itemize}

\end{frame}

\begin{frame}
  \frametitle{Optimization---Murray's law}

  \begin{block}<1->{Aside on $P_{\textnormal{drag}}$}
  \begin{itemize}
  \item<2-> 
    Work done = $F \cdot d$ = energy transferred by force $F$
  \item<3-> 
    Power = $P$ = rate work is done = $F \cdot v$
  \item<4-> $\Delta p$ = Force per unit area
  \item<5-> $\Phi$ = Volume per unit time \\ = cross-sectional area $\cdot$ velocity
  \item<6-> So $\Phi \Delta p$ = Force $\cdot$ velocity
  \end{itemize}
  \end{block}

\end{frame}


\begin{frame}
  \frametitle{Optimization---Murray's law}

  \begin{block}{Murray's law:}
    \begin{itemize}
    \item<1-> Total power (cost):
      $$ 
      P = P_{\textnormal{drag}} + P_{\textnormal{metabolic}}
      \uncover<2->{=
      \Phi^2 \frac{8\eta \alert{\ell}}{\pi \alert{r^4}}
      + c \alert{r^2 \ell}}
      $$
    \item<3-> Observe power increases linearly with $\ell$
    \item<4-> But $r$'s effect is nonlinear: 
      \begin{itemize}
      \item<5->  
        increasing $r$
        makes flow easier \alert{but increases metabolic cost} (as $r^2$)
      \item<6->
        decreasing $r$
        decrease metabolic cost \alert{but impedance goes up} (as $r^{-4}$)
      \end{itemize}
    \end{itemize}
  \end{block}

\end{frame}


\begin{frame}
  \frametitle{Optimization---Murray's law}

  \begin{block}{Murray's law:}
    \begin{itemize}
    \item<1-> Minimize $P$ with respect to $r$:
      $$
      \partialdiff{P}{r}
      = 
      \partialdiff{}{r} 
      \left( 
        \Phi^2 \frac{8\eta {\ell}}{\pi {r^4}}
      + c {r^2 \ell}
      \right)
    $$
      \uncover<2->{
        $$
        = 
        -4 \Phi^2 \frac{8\eta {\ell}}{\pi {r^5}}
        + c {2r \ell}
        \uncover<3->{\alert{=0}}
      $$
      \item<4-> Rearrange/cancel/slap:
        $$
        \alert{\Phi^2} = \frac{c \pi r^6}{16 \eta} \uncover<5->{= k^2 \alert{r^6}}
        $$
        \uncover<5->{where $k$ = constant.}
        
      }
    \end{itemize}
  \end{block}

\end{frame}


\begin{frame}
  \frametitle{Optimization---Murray's law}

  \begin{block}{Murray's law:}
    \begin{itemize}
    \item<1-> So we now have:
      $$
      \Phi = k r^3 
      $$
    \item<2-> 
      Flow rates at each branching have to add up
      (else our organism is in serious trouble...):
      $$
      \Phi_0 = \Phi_1 + \Phi_2
      $$
      where again 0 refers to the  main branch and 1 and 2 refers
      to the offspring branches
    \item<3->
      All of this means we have a groovy cube-law:
      $$ 
      \boxed{\alert{r_0^3 = r_1^3 + r_2^3}}
      $$
      
    \end{itemize}
  \end{block}

\end{frame}

\subsection{Murray\ meets\ Tokunaga}


\begin{frame}
  \frametitle{Optimization}

  \begin{block}{Murray meets Tokunaga:}
    \begin{itemize}
    \item<1-> 
      $\Phi_\om$ = volume rate of flow into an order
      $\om$ vessel segment
    \item<2-> 
      Tokunaga picture:
      $$ 
      \Phi_\om
      = 
      2 \Phi_{\om - 1}
      +
      \sum_{k=1}^{\om-1}
      T_k
      \Phi_{\om-k}
      $$
    \item<3->
      Using $\phi_\om = k r_{\om}^{3}$
      $$
      r_\om^3
      = 
      2 r_{\om - 1}^{3}
      +
      \sum_{k=1}^{\om-1}
      T_k
      r_{\om-k}^{3}
      $$
    \item<4->
      Find Horton ratio for vessel radius $R_r = r_{\om}/r_{\om-1}$...
    \end{itemize}
  \end{block}

\end{frame}


\begin{frame}
  \frametitle{Optimization}

  \begin{block}{Murray meets Tokunaga:}
    \begin{itemize}
    \item<1-> 
      Find $R_r^{\, 3}$ satisfies same equation as $R_n$ and $R_v$\\
      ($v$ is for volume):
      $$
      \boxed{ \alert{R_r^3 = R_n = R_v} }
      $$
    \item<2->
      Is there more we could do here to constrain the Horton
      ratios and Tokunaga constants?
    \end{itemize}
  \end{block}

\end{frame}


\begin{frame}
  \frametitle{Optimization}

  \begin{block}{Murray meets Tokunaga:}
    \begin{itemize}
    \item<1-> 
      Isometry: $V_\om \propto \ell_\om^{\, 3}$
    \item<2->
      Gives 
      $$\boxed{\alert{R_\ell^3 = R_v = R_n}}$$
    \item<3-> 
      We need one more constraint...
    \item<4->
      West et al (1997)\cite{west1997a} achieve similar
      results following Horton's laws.
    \item<5->
      So does Turcotte et al. (1998)\cite{turcotte1998a}
      using Tokunaga (sort of).
    \end{itemize}
  \end{block}

\end{frame}

