%% clauset paper on sports



%% burstiness

%%   Explore Sugarscape.



%%%%%%%%%%%%%%%%%%%%%%%%%%%%%%%%%%%%%
%% very clear summary slide 
%%%%%%%%%%%%%%%%%%%%%%%%%%%%%%%%%%%%%

%% student to redo barabasi fragility?



\section{The\ Plan}

%% \subsection{Requirements}

  \textbf{Semester projects}

  \textbf{Requirements:}
    
     
      3 minute introduction to project (5th week).
     
      5-10 minute final presentation.
    
      Report: $\ge$ 5 pages (single space), journal-style
    
  

  \textbf{Goals:}
    
     
      Understand, critique, and communicate published work.
     
      Seed research papers or help papers along.
    
  


%% \subsection{Narrative\ hierarchy}

  \textbf{Narrative hierarchy}

  \textbf{Presenting at many scales:}
    
     
      1 to 3 word encapsulation, a soundbite,
     
      a sentence/title,
     
      a few sentences,
     
      a paragraph,
     
      a short paper,
     
      a long paper,
     
      $\ldots$
    
  


\section{Suggestions\ for\ Projects}



  \textbf{Twitter---living in the now:}

  
    \includegraphics[width=.8\textwidth]{figtwitter_daily_anewfreq_spesh_thetruth002_breakfast_lunch_dinner_noname.pdf}
    
     
      Research opportunity: be involved in our socio-info-algorithmo-econo-geo-technico-physical systems research group
      studying Twitter and other wordful large data sets.
    
  


      \textbf{topics:}

    
    
        
      Develop and elaborate an \alert{online experiment}
      to study some aspect of \alert{social phenomena}
     e.g., 
      collective search, 
      cooperation, 
      cheating, 
      influence, 
      creation,
      decision-making, 
      etc.
     Part of the PLAY project.
    
    

  

  \textbf{topics:}
  
  
    Rummage round in the 
    \wordwikilink{http://www.uvm.edu/~cmplxsys/?Page=newsevents/readings.php\&SM=newsevents/\_newseventsmenu.html}{papers}
    we've covered
    in our weekly Complex Systems Reading Group at UVM.
  




  \textbf{topics:}

  
            
      \includegraphics[width=\textwidth]{gonzalez2008a_fig1ab.pdf}
        
      \includegraphics[width=\textwidth]{brockmann2006a_fig1b.pdf}
      
       
        Study movement and interactions of people.
       
        Brockmann \etal\cite{brockmann2006a} ``Where's George'' study.
       
        Barabasi's group: tracking movement
        via cell phones\cite{gonzalez2008a}.
      
      







  \textbf{topics:}

  
          
      
      \includegraphics[width=\textwidth]{liu2011a_fig5}
      
      \includegraphics[width=\textwidth]{liu2011a_fig4}\\
      ``Controllability of complex networks''\cite{liu2011a}
      Liu et al., Nature 2011.
      
      



  \textbf{Sociotechnical phenomena---Foldit:}

  
  \includegraphics[width=.9\textwidth]{cooper2010a_fig1.pdf}
  
   
    \alertg{``Predicting protein structures with a multiplayer
      online game.''}
    Cooper et al., Nature, 2010.\cite{cooper2010a}
   
    Also: \wordwikilink{http://www.zooniverse.org}{zooniverse}, 
    \wordwikilink{http://www.gwap.com/gwap/gamesPreview/espgame/}{ESP game},
    \wordwikilink{http://www.captcha.net/}{captchas}.
  
  



  \textbf{topics:}

  
  Explore ``Catastrophic cascade of failures in interdependent networks''\cite{buldyrev2010a}.
  Buldyrev et al., Nature 2010.

  \includegraphics[width=0.8\textwidth]{buldyrev2010a_fig1.pdf}
  



  \textbf{The madness of modern geography:}

  
    \includegraphics[width=0.48\textwidth]{92_big01.jpg}
    \includegraphics[width=0.48\textwidth]{92_big02.jpg}
    
    
      Explore distances between points on the Earth 
      as travel times.
    
      See Jonathan Harris's work 
      \wordwikilink{http://www.number27.org/assets/work/extras/maps/traveltime/index.html}{here} 
      and
      \wordwikilink{http://www.visualcomplexity.com/vc/project\_details.cfm?id=92\&index=5\&domain=Transportation\%20Networks}{here}.
    
  



  \textbf{topics:}

  
  
   
    Explore general theories on \alertb{system robustness}.
  
    Are there \alert{universal signatures} that presage system failure?
  
    See \alertb{``Early-warning signals for critical transitions''}\\
    Scheffer et al., Nature 2009.\cite{scheffer2009a}
  
    ``Although predicting
    such critical points before they are reached is extremely difficult,
    work in different scientific fields is now suggesting the existence of
    generic early-warning signals that may indicate for a wide class of
    systems if a critical threshold is approaching.''
   
    Later in class: Doyle et al., robust-yet-fragile systems
  
  

  %%
  %%  
  %%    \alert{Abstract:} Complex dynamical systems, ranging from ecosystems to financial
  %%    markets and the climate, can have tipping points at which a sudden
  %%    shift to a contrasting dynamical regime may occur. 
  %%
  %%     See 
  %%  


  \textbf{topics:}

  
  
   
    Study the human disease and disease gene networks (Goh \etal, 2007):
  
  \includegraphics[width=\textwidth]{goh2007a_fig2a}
  


  \textbf{topics:}

  \textbf{Explore and critique Fowler and Christakis et al. work on
      social contagion of:}
          
      \includegraphics[width=\textwidth]{cacioppo2009a_fig1}
      
      
       
        Obesity\cite{christakis2007a}
      
        Smoking cessation\cite{christakis2008a}
      
        Happiness\cite{fowler2008a}
      
        Loneliness\cite{cacioppo2009a}
      
      

  \textbf{One of many questions:}
    How does the (very) sparse sampling
    of a real social network affect their findings?
  



  \textbf{Culturomics:}

  \small{``Quantitative analysis of culture using millions of
    digitized books'' by Michel et al., Science, 2011\cite{michel2011a}}

  \includegraphics[width=0.45\textwidth]{michel2011a_fig3a.pdf} 
  \includegraphics[width=0.45\textwidth]{michel2011a_fig3e.pdf} \\
  \includegraphics[width=0.45\textwidth]{michel2011a_fig3f.pdf}
  \includegraphics[width=0.35\textwidth]{michel2011a_fig4f.pdf}

  {\small
    \wordwikilink{http://www.culturomics.org/}{http://www.culturomics.org/}\\
    \wordwikilink{http://ngrams.googlelabs.com/}{Google Books ngram viewer}
  }



  \textbf{topics:}

  \textbf{The problem of missing data in networks:}
    
     
      Clauset et al. (2008)\\
      ``Hierarchical structure and the prediction of missing links in networks''\cite{clauset2008a}
     
      Kossinets (2006)\\
      ``Effects of missing data in social networks''\cite{kossinets2006b}
     
      Much more ...
    
  


  %% The following article has just been published in the Journal of Social
  %% Structure
  %% (http://www.cmu.edu/joss/content/articles/volume10/huisman.pdf)
  %% 
  %% 
  %% Huisman, M. (2009). Imputation of missing network data: Some simple
  %% procedures.
  %% 
  %% 
  %% Analysis of social network data is often hampered by non-response and
  %% missing data. Recent studies show the negative effects of missing actors
  %% and ties on the structural properties of social networks. This means
  %% that the results of social network analyses can be severely biased if
  %% missing ties were ignored and only complete cases were analyzed. To
  %% overcome the problems created by missing data, several treatment methods
  %% are proposed in the literature: model-based methods within the framework
  %% of exponential random graph models, and imputation methods. In this
  %% paper we focus on the latter group of methods, and investigate the use
  %% of some simple imputation procedures to handle missing network data. The
  %% results of a simulation study show that ignoring the missing data can
  %% have large negative effects on structural properties of the network.
  %% Missing data treatment based on simple imputation procedures, however,
  %% does also have large negative effects and simple imputations can only
  %% successfully correct for non-response in a few specific situations.
  %% 
  %% 
  %% 
  %% Dr Garry Robins
  %% School of Behavioural Science
  %% University of Melbourne
  %% Victoria 3010
  %% Australia
  %% 
  %% Editor
  %% Journal of Social Structure
  %% Web: http://www.cmu.edu/joss/index.html



      \textbf{topics:}

    
    
        
      Explore ``self-similarity of complex networks''\cite{song2005a,song2006a}\\
      First work by Song \etal, Nature, 2005.
    
      See accompanying comment by Strogatz\cite{strogatz2005a}
    
      See also ``Coarse-graining and self-dissimilarity of complex networks'' by
      Itzkovitz et al.\cite{itzkovitz2005a}
    
    \includegraphics[width=0.49\textwidth]{song2005a_fig1a}
    \includegraphics[width=0.49\textwidth]{song2005a_fig1b}
    

  

      \textbf{topics:}

    \textbf{Related papers:}
      
      
        ``Origins of fractality in the growth of complex networks''\\
        Song et al. (2006a)\cite{song2006a}
      
        ``Skeleton and Fractal Scaling in Complex Networks''\\
        Go et al. (2006a)\cite{goh2006a}
      
        ``Complex Networks Renormalization: Flows and Fixed Points''\\
        Radicchi et al. (2008a)\cite{radicchi2008a}
      
    
    
  
      \textbf{topics:}

    
    
     
      Explore patterns, designed and undesigned, of cities
      and suburbs.
    

    \begin{center}
      \includegraphics[height=0.6\textheight]{lr_sprawl-custom1.jpg}
    \end{center}
    

  
      \textbf{topics:}

    
      \small
      ``Looking at Gielen's work, it's tempting to propose a new branch of the
      human sciences: geometric sociology, a study of nothing but the shapes
      our inhabited spaces make. Its research agenda would ask why these
      forms, angles and geometries emerge so consistently, from prehistoric
      settlements to the fringes of exurbia. Are sites like these an
      aesthetic pursuit, a mathematical accident, a calculated bending of
      property lines based on glitches in the local planning code or an
      emergent combination of all these factors? Or are they the expression
      of something buried deep in human culture and the unconscious,
      something only visible from high above?''
    

    \small
    \wordwikilink{http://opinionator.blogs.nytimes.com/2010/09/17/the-geometry-of-sprawl/}{http://opinionator.blogs.nytimes/..../the-geometry-of-sprawl/}
    
  


  %%   %%   \textbf{topics:}
  %% 
  %%   
  %%   
  %%     Statistics: Study Peter Hoff's (and
  %%     others') work on \alert{latent variables}.  
  %%    \alertb{Idea:} explain connection pattern in
  %%     a network through hidden individual or dyadic variables
  %%   
  %%     Method has been
  %%     applied to the study of international relations networks.
  %%   
  %% 
  %% 
      \textbf{topics:}

    
    
     
      Study collective creativity arising out of social interactions
     
      Productivity, wealth, creativity, disease, etc. appear to increase superlinearly with population
     
      Start with Bettencourt et al.'s (2007)
      ``Growth, innovation, scaling, and the pace of life in
      cities''\cite{bettencourt2007a}
     
      Dig into Bettencourt (2013)
      ``The Origins of Scaling in Cities''\cite{bettencourt2007a}
    
    

  
      \textbf{topics:}

          
      \includegraphics[width=\textwidth]{bohorquez2009a_figS2}
      
      
      
      
        Physics/Society---\alert{Wars:} Study work that
        started with Lewis Richardson's ``Variation of the frequency of
        fatal quarrels with magnitude'' in 1949.
      
        Specifically explore Clauset et al. 
        and Johnson et al.'s work\cite{clauset2007b,johnson2006a,bohorquez2009a}
        on terrorist attacks and civil wars
      
        Richardson bonus: Britain's coastline, turbulence, weather prediction, ...
      
      
    
  

      \textbf{topics:}

    
          
      
       
        Study Hidalgo et al.'s ``The Product Space Conditions the Development of Nations''\cite{hidalgo2007a}
       
        How do products depend on each other, and how does this network evolve?
       
        How do countries depend on
        each other for water, energy, people (immigration), investments?
            
      
      \includegraphics[width=\textwidth]{spacelabelslegends.pdf}
        

  
      \textbf{topics:}

    
      
       
        Explore \wordwikilink{http://en.wikipedia.org/wiki/Dunbar\%27s_number}{Dunbar's number}
      
        See
        \wordwikilink{http://www.lifewithalacrity.com/2004/03/the_dunbar_numb.html}{here}
        and 
        \wordwikilink{http://www.lifewithalacrity.com/2005/03/dunbar_altruist.html}{here}
        for some food for thought regarding large-scale online games and Dunbar's number.
        [\wordwikilink{http://www.lifewithalacrity.com}{http://www.lifewithalacrity.com}]
       
        Recent work:
        ``Network scaling reveals consistent fractal pattern in hierarchical mammalian societies''
        Hill et al.\ (2008)\cite{hill2008a}.
      
    

  
  \textbf{Study networks and creativity:}

      
    \includegraphics[width=\textwidth]{guimera2005b_fig2}
    
    
    
      Guimer\`{a} et al., Science 2005:\cite{guimera2005b}
      ``Team Assembly Mechanisms Determine Collaboration Network Structure and Team Performance''
     
      Broadway musical industry
     
      Scientific collaboration in Social Psychology, Economics, Ecology, and Astronomy.
    
  


  %%   %%   \textbf{topics:}
  %% 
  %%   
  %%   
  %%     Investigate and review Cybernetics, a
  %%     forerunner to Complex Systems.
  %%   
  %% 
  %%   %% 
  %%   %%   \textbf{topics:}
  %% 
  %%   
  %%   
  %%     Read and review Herbert Simon's ``Sciences
  %%     of the Artificial'' (or more Simon's work more generally).
  %%   
  %% 
  %%   %% 
  %%   %%   \textbf{topics:}
  %% 
  %%   
  %%   
  %%     Investigate the life and work of 
  %%     \wordwikilink{http://en.wikipedia.org/wiki/Frank_Harary}{Frank Harary}, 
  %%     graph theory champion.
  %%   
  %% 
  %%   %% 
  %%   %%   \textbf{topics:}
  %% 
  %%   
  %%   
  %%     Investigate and report on General Systems
  %%     Theory.
  %%   
  %% 
  %% 

  %%   %%   \textbf{topics:}
  %% 
  %%   
  %%    Study \alert{bipartite networks}: structure and dynamics
  %%    Rich and interesting both mathematically
  %%     and practically speaking.
  %%   
  %% 
  %% 
      \textbf{topics:}

    
    
     Study scientific collaboration networks.
     Mounds of data + good models.
     See seminal work by De Solla Price\cite{price1965a}.\\
      plus modern work by Redner, Newman, \etal
     We will study some of this in class...
    
    

  
      \textbf{topics:}
      
    
    
     
      Study Kearns et al.'s experimental studies
      of people solving classical graph theory problems\cite{kearns2006a}
     
      ``An Experimental Study of the Coloring Problem on Human Subject Networks''
      (Possibly) Run some of these experiments for our class.
    
    

  




  \textbf{topics:}

  
  
   
    Study \alert{collective tagging} (or folksonomy)
   
    e.g., \href{http://del.icio.us}{del.icio.us}, \href{http://www.flickr.com}{flickr}
   
    See work by Bernardo Huberman et al. at HP labs.
  
  



  \textbf{topics:}

  
  
  
    Study games (as in game theory) on
    networks.  
  
    For cooperation: Review Martin Nowak's piece in Science,
    ``Five rules for the evolution of cooperation.''\cite{nowak2006a}
    and related works.
   Much work to explore: voter models, contagion-type models, etc.
  
  


  \textbf{topics:}

  
  
  
    \alertb{Semantic networks}: explore word-word
    connection networks generated by linking semantically related words.
  
    Also: Networks based on morphological or phonetic similarity.
   
    More general: Explore \alertb{language evolution}
  
    One paper to start with: ``The small world of human language''
    by Ferrer i Cancho and Sol\'{e}\cite{ferrericancho2001a}
   
    Study spreading of
    neologisms.
   
    Examine new words relative to existing words---is there 
    a pattern?  Phonetic and morphological similarities.
   
    \alert{Crazy:} Can new words be predicted?
   
    Use Google Books n-grams as a data source.
  
  


%% %%   \textbf{topics:}
%% 
%%   
%%   
%%     Investigate \alert{Service Science}, which doesn't
%%     sound very good but IBM believes will be bigger than computer
%%     science.
%%   
%%     \alert{Definition:} ``Service Science, Management, and
%%     Engineering (SSME) is an interdisciplinary approach to the study,
%%     design, and implementation of service systems---complex systems in
%%     which specific arrangements of people and technologies take actions
%%     that provide value for others.''
%%     \begin{overprint} 
%%        
%%        
%%       \includegraphics[width=.07\textwidth]{wikipedia.jpg}
%%     \end{overprint} 
%%   
%% 
%% 
%% %%   \textbf{topics:}
%% 
%%   
%%   
%%     Investigate \alert{safety codes} (building, fire,
%%     etc.).  
%%   
%%     What kind of relational networks do safety codes form?  How have they
%%     evolved?
%%   
%% 
%% 

  \textbf{topics:}

    
    
     
      Explore proposed measures of system
      complexity.  
    
    

  
  
  
    Study Stuart Kauffman's \alert{$nk$ boolean
      networks} which model regulatory gene networks\cite{kauffman1993a}
  
  


  \textbf{topics:}

  
    
    
      Critically explore Bejan's Constructal Theory.
    
      See Bejan's book ``Shape and Structure, from Engineering to Nature.''\cite{bejan2000a}
    
      Bejan asks why we see branching network flow structures so often in
      Nature---trees, rivers, etc.
    
  

  
    
    
      Read and critique ``Historical Dynamics:
      Why States Rise and Fall'' by Peter Turchin.\cite{turchin2003a}
    
      Can history be explained by differential equations?:
      \wordwikilink{http://www.eeb.uconn.edu/people/turchin/Clio.htm}{Clyodynamics},
    
      Construct a working version of \wordwikilink{http://en.wikipedia.org/wiki/Psychohistory\_(fictional)}{Psychohistory}.
    
      \wordwikilink{http://en.wikipedia.org/wiki/Big\_History}{``Big History''}
    
      Arbesman: ``The life-spans of Empires''\cite{arbesman2011a}
    
      Also see 
      \wordwikilink{http://www.eeb.uconn.edu/people/turchin/SEC.htm}{``Secular Cycles''}.
    
  


  \textbf{topics:}

  
  
  
    Explore work by Doyle, Alderson, et al. 
    as well as Pastor-Satorras et al. on the structure 
    of the \alertb{Internet(s)}.
  
  





%% %%   \textbf{topics:}
%% 
%%   
%%   
%%     Investigate and review Cybernetics, a
%%     forerunner to Complex Systems.
%%   
%% 
%% 
%% %%   \textbf{topics:}
%% 
%%   
%%   
%%     Read and review Herbert Simon's ``Sciences
%%     of the Artificial'' (or more Simon's work more generally).
%%   
%% 
%% 
%% %%   \textbf{topics:}
%% 
%%   
%%   
%%     Investigate and report on General Systems
%%     Theory.
%%   
%% 
%% 

  \textbf{topics:}

  
  
  
    Review: Study Castronova's and others'
    work on massive multiplayer online games.  
    How do social networks
    form in these games?\cite{castronova2005a}
  
    See work by Johnson et al.\ on gang formation
    in the real world and in World of Warcraft (really!).
  
  


%% %%   \textbf{topics:}
%% 
%%   
%%   
%%     Study Michael Kearns and others' work on
%%     Cobot.  Very cool.  
%%    See \url{http://cobot.research.att.com/}.
%%   
%% 
%% %%   \textbf{topics:}
%% 
%%   
%%    
%%     Study Kearns et al.'s experimental studies
%%     of people solving classical graph theory problems\cite{kearns2006a}
%%    
%%     ``An Experimental Study of the Coloring Problem on Human Subject Networks''
%%     (Possibly) Run some of these experiments for our class.
%%   
%% 
%% 



%% %%   \textbf{topics:}
%% 
%%   
%%   
%%     Biology: Study leaf network patterns.
%%       
%%     Key on very interesting work by Xia.
%%   
%%     Classic Monge problem: how to move stuff
%%     from one place to another.
%%    
%%     Bulk flow versus network flow.
%%   
%% 
%% 

%% %%   \textbf{topics:}
%% 
%%   
%%   
%%     Biology: Study spider webs.
%%   
%% 
%% 
  \textbf{topics:}

  \textbf{Social networks:}
  
  
    Study social networks as revealed
    by email patterns, Facebook connections, tweets, etc.
  
    ``Empirical analysis of evolving social networks''
    Kossinets and Watts, Science, Vol 311, 88-90, 2006.\cite{kossinets2006a}
  
    ``Inferring friendship network structure by using mobile phone data''
    Eagle, et al., PNAS, 2009.
  
    ``Community Structure in Online Collegiate Social Networks''\\
    Traud et al., 2008.\\
    \wordwikilink{http://arxiv.org/abs/0809.0690}{http://arxiv.org/abs/0809.0690}
  
  



      \textbf{topics:}

    \textbf{Vague/Large:}
      
      
        Study amazon's recommender
        networks.
        \includegraphics[width=0.9\textwidth]{beedlebard.pdf}

        See work by Sornette \etal.
      
        Vague/Large:

        Study Netflix's open data
        (movies and people form a bipartite graph).
      
    




  \textbf{topics:}

  \textbf{Vague/Large:}
  
  
    Study how the Wikipedia's
    content is interconnected.

    \bigskip
    \includegraphics[width=0.3\textwidth]{wikipedia-tp-3.pdf}
  
  




  \textbf{topics:}

  \textbf{More Vague/Large:}
    
    
      How do countries depend on
      each other for water, energy, people (immigration), investments?
      %% cite barabasi paper
    
      How is the media connected?
      Who copies whom?
    
      (Problem: Need to be able to measure interactions.)
    
      Investigate memetics, the `science' of memes.
     
      \wordwikilink{http://memetracker.org/}{http://memetracker.org/}
    
      Sport...
    
  


  \textbf{Voting}
 
  \textbf{Score-based voting versus rank-based voting:}
    
     
      Balinski and Laraki\cite{balinski2007a}\\
      \alertb{``A theory of measuring, electing, and ranking''}\\
      Proc. Natl. Acad. Sci., pp. 8720--8725 (2007)
    
    
  



  \textbf{topics:}

  \textbf{More Vague/Large:}
    
    
      How does \alert{advertising} work
      collectively?  
     
      Does one car manufacturers' ads
      indirectly help other car manufacturers?
    
      Ads for junk food versus fruits and vegetables.
     
      Ads for cars versus bikes versus walking.
    
  


%% %%   \textbf{topics:}
%% 
%%   
%%   
%%     Vague/Large:
%% 
%%  Study social network
%%     evolution in Second Life.
%%   
%% 
%% 
  \textbf{topics:}

  \textbf{More Vague/Large:}
  
  
    Study spreading of anything
    where influence can be measured (very hard).
  
    Study any interesting micro-macro story to do
    with evolution, biology, ethics, religion, history, food,
    international relations, \ldots
  
    Data is key.
  
  




%% web stuff:
%% amazon book linkages
%% del.icio.us

%% percolation on networks?

%% cooperation

%% any data set where influence is clearly measured

%% games on networks

%% ------


                                %+ rinaldo's paper on impedance, whether or not networks
                                %will have flow in loops

                                %+ european paper on search in networks (star versus distributed)
                                %
                                %+ newman's work:
%%  good delivery
%%  friends of friends
%%  random networks

                                %+ uri alon: motifs

                                %+ small worlds

                                %+ barabasi---scale free networks

                                %+ river networks

                                %+ cardiovascular networks---3/4 stuff

                                %+ pstar stuff

                                %+ kleinberg
                                %+ search in networks






%% web stuff:
%% amazon book linkages
%% del.icio.us

%% percolation on networks?

%% cooperation

%% any data set where influence is clearly measured

%% games on networks

%% ------


%% + rinaldo's paper on impedance, whether or not networks
%% will have flow in loops

%% + european paper on search in networks (star versus distributed)
%% 
%% + newman's work:
%%  good delivery
%%  friends of friends
%%  random networks

%% + uri alon: motifs

%% + small worlds

%% + barabasi---scale free networks

%% + river networks

%% + cardiovascular networks---3/4 stuff

%% + pstar stuff

%% + kleinberg
%% + search in networks

\section{Archive}

  \textbf{topics:}

  
      
    
    
      Study \wordwikilink{http://en.wikipedia.org/wiki/Phyllotaxis}{phyllotaxis}, how plants grow new
      buds and branches.  
    
      Some delightful mathematics appears involving
      the Fibonacci series.
    
      Excellent work to start with:
      ``Phyllotaxis as a Dynamical Self Organizing Process: Parts I, II, and III''
      by Douady and Couder\cite{douady1996a,douady1996b,douady1996c}
    
    
    \includegraphics[width=\textwidth]{Helianthus_whorl_mod.jpg}\\
    {\tiny\wordwikilink{http://andbug.blogspot.com/2009/02/phyllotaxis-01.html}{http://andbug.blogspot.com/}}\\
    \includegraphics[width=.5\textwidth]{Opposite.png}
    \includegraphics[width=.5\textwidth]{Alternate.png}\\
    {\tiny\wordwikilink{http://en.wikipedia.org/wiki/Phyllotaxis}{Wikipedia}}
    

