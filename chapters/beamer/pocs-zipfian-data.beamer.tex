\section{Zipf\ in\ brief}

\begin{frame}
  \frametitle{George Kingsley Zipf:}

  \begin{block}{In brief:}
    \begin{itemize}
    \item<1-> 
      \wordwikilink{http://en.wikipedia.org/wiki/George_Kingsley_Zipf}{Zipf} (1902--1950) was a linguist at Harvard, specializing
      in Chinese languages.
    \item<2->
      Unusual passion for statistical analysis of texts.
    \item<3->
      Studied human behavior much more generally...
    \end{itemize}
  \end{block}

  \begin{block}<4->{Zipf's masterwork:}
    \begin{itemize}
    \item 
    ``Human Behavior and the Principle of Least Effort''\\
    Addison-Wesley, 1949\\
    Cambridge, MA\cite{zipf1949a}
    \end{itemize}
  \end{block}
  
  \begin{block}{}
  \begin{itemize}
  \item<5-> \alertb{Bonus field of study:} 
    \wordwikilink{http://en.wiktionary.org/wiki/glottometrics}{Glottometrics.}
  \item<6-> \alertb{Bonus `word' word:} \wordwikilink{http://en.wiktionary.org/wiki/glossolalia}{Glossolalia.}
  \end{itemize}
  \end{block}

\end{frame}

\begin{frame}
  \frametitle{Human Behavior/Principle of Least Effort:}

  \begin{block}{From the Preface---}
    Nearly twenty-five years ago it occurred to me that we might gain
    considerable insight into the mainsprings of human behavior
    if we viewed it purely as a natural phenomenon like everything
    else in the universe, ...
  \end{block}

  \begin{block}<2->{And---}
%%    The present book reports ... some fundamental principles that
%%    seem to govern important aspects of our behavior, both as
%%    individuals and as members of social groups.
    ... the expressed purpose of this book is to establish
    \alertb{The Principle of Least Effort}
    as the primary principle 
    that governs our entire individual and collective behavior ...
  \end{block}

\end{frame}

\begin{frame}
  \frametitle{The Principle of Least Effort:}

  \begin{block}{Zipf's framing (p.\ 1):}
    ``... a person in solving his immediate problems
    will view these against the background of his probable
    future problems \textit{as estimated by himself}.''

    \smallskip

    \uncover<2->{
      ``... he will strive ... to minimize the \textit{total work}
      that he must expend in solving \textit{both}
      his immediate problems \textit{and} his 
      probable future problems.''
    }

    \smallskip

    \uncover<3->{
      ``[he will strive to] minimize the \textit{\alert{probable average rate of his work-expenditure}}...''
    }

  \end{block}

\end{frame}

\begin{frame}
  \frametitle{Rampaging research}

  \begin{block}{Within Human Behavior and the Principle of Least Effort:}
    \begin{columns}
      \column{0.5\textwidth}
      \begin{itemize}
      \item<1-> City sizes
      \item<1-> \# retail stores in cities
      \item<1-> \# services (barber shops, beauty parlors, cleaning, ...)
      \item<1-> \# people in occupations
      \item<1-> \# one-way trips in cars and trucks vs. distance
      \end{itemize}
      \column{0.5\textwidth}
      \begin{itemize}
      \item<1-> \# new items by dateline
      \item<1-> weight moved between cities by rail
      \item<1-> \# telephone messages between cities
      \item<1-> \# people moving vs. distance
      \item<1-> \# marriages vs. distance
      \end{itemize}
    \end{columns}
  \end{block}

  \begin{block}{}
  \begin{itemize}
  \item<2-> 
    Observed \alertb{general dependency of `interactions'}
    between \alert{cities $A$ and $B$} on \alertb{$P_A P_B/D_{AB}$} where
    $P_A$ and $P_B$ are population size and 
    $D_{AB}$ is distance between $A$ and $B$.
    \visible<3->{$\Rightarrow$ \alertb{`Gravity Law.'}}
  \end{itemize}
  \end{block}

\end{frame}

\section{Zipfian\ empirics}

\begin{frame}
  \frametitle{Zipfian empirics:}

  \begin{itemize}
  \item \alertb{vocabulary balance}: $f \sim r^{\, -1} \rightarrow r\cdot f \sim \mbox{constant}$\\
    ($f$ = frequency, $r$ = rank).
  \end{itemize}

  \centering
  \includegraphics[angle=-2,height=0.7\textheight]{zipf1949a_p024.pdf}

\end{frame}

\begin{frame}
  \frametitle{Zipfian empirics:}

  \begin{itemize}
  \item $f \sim r^{-1}$ for word frequency:
  \end{itemize}

    \centering
    \includegraphics[angle=1,height=0.7\textheight]{zipf1949a_p025.pdf}

\end{frame}

\begin{frame}
  \frametitle{Zipf's basic idea:}

  \begin{block}{Forces of Unification and Diversification:}
  \begin{itemize}
  \item<1-> Easiest for the speaker to use just one word.
    \begin{itemize}
    \item<5-> \visible<4->{\alert{Encoding is simple} but \alertb{decoding is hard}}
    \end{itemize}
  \item<2-> Zipf uses the analogy of tools: \alertb{one tool for all tasks}.
  \item[]
  \item<3-> Optimal for listener if all pieces of information
    correspond to different words (or morphemes).
  \item<4-> Analogy: a specialized tool for every task.
    \begin{itemize}
    \item<6-> \visible<5->{\alert{Decoding is simple} but \alertb{encoding is hard}}
    \end{itemize}
  \item[]
  \item<7-> Zipf thereby argues for a tension that should lead
    to an uneven distribution of word usage.
  \item<8-> No formal theory beyond this...
    (more later\cite{ferrericancho2003a})
  \end{itemize}
  \end{block}

\end{frame}


\begin{frame}
  \frametitle{Zipfian empirics:}
  
  \begin{itemize}
  \item Number of meanings \alertb{$m_r \propto f_r^{1/2}$} where
    $r$ is rank and $f_r$ is frequency.
  \end{itemize}

  \centering
  \includegraphics[angle=-1,height=0.6\textheight]{zipf1949a_p030.pdf}

\end{frame}

\begin{frame}
  \frametitle{Zipfian empirics:}

  \begin{itemize}
  \item Article length in the Encyclopedia Britannica:
  \end{itemize}
  \centering
  \includegraphics[angle=0,height=0.7\textheight]{zipf1949a_p177.pdf}
  \begin{itemize}
  \item (?) slope of $-3/5$ corresponds to $\gamma=5/3$.
%%  \item $\alpha \simeq 3/5$ corresponds to $\gamma=1+1/\alpha \simeq 2$.
  \end{itemize}

\end{frame}



\begin{frame}
  \frametitle{Zipfian empirics:}

  \begin{itemize}
  \item Population size of districts:
  \end{itemize}
  \centering
  \includegraphics[angle=0,height=0.7\textheight]{zipf1949a_p375.pdf}
  \begin{itemize}
  \item $\alpha=1$ corresponds to $\gamma=1+1/\alpha=2$.
  \end{itemize}

\end{frame}

\begin{frame}
  \frametitle{Zipfian empirics:}

  \begin{itemize}
  \item Number of employees in organizations
  \end{itemize}
  \begin{center}
    \includegraphics[angle=0,height=0.7\textheight]{zipf1949a_p384.pdf}
  \end{center}
  \begin{itemize}
  \item $\alpha=2/3$ corresponds to $\gamma=1+1/\alpha=5/2$.
  \end{itemize}

\end{frame}


\begin{frame}
  \frametitle{Zipfian empirics:}

  \begin{itemize}
  \item 
    \# news items as a function of population $P_2$ of location
    in the Chicago Tribune 
  \item
    $D$ = distance, $P_1$ = Chicago's population
  \item
    Solid line = +1 exponent.
  \end{itemize}
  \begin{center}
    \includegraphics[angle=1,height=0.65\textheight]{zipf1949a_p388.pdf}
  \end{center}

\end{frame}

\begin{frame}
  \frametitle{Zipfian empirics:}

  \begin{itemize}
  \item 
    \# obituaries in the New York Times for locations
    with population $P_2$.
  \item
    $D$ = distance, $P_1$ = New York's population
  \item
    Solid line = +1 exponent.
  \end{itemize}
  \begin{center}
    \includegraphics[angle=0,height=0.65\textheight]{zipf1949a_p389.pdf}
  \end{center}

\end{frame}


\begin{frame}
  \frametitle{Zipfian empirics:}

  \begin{itemize}
  \item 
    Movement of stuff between cities
  \item
    $D$ = distance, $P_1$ and $P_2$ = city populations.
  \item
    Solid line = +1 exponent.
  \end{itemize}

  \centering
  \includegraphics[angle=0,height=0.65\textheight]{zipf1949a_p393.pdf}

\end{frame}

\begin{frame}
  \frametitle{Zipfian empirics:}

  \begin{itemize}
  \item 
    Length of trip versus frequency of trip.
  \item
    Solid line = -1/2 exponent corresponds to $\gamma = 2$.
  \end{itemize}
  \includegraphics[angle=0,height=0.7\textheight]{zipf1949a_p401.pdf}

\end{frame}

\begin{frame}
  \frametitle{Zipfian empirics:}

  \begin{itemize}
  \item The probability of marriage?
  \item $\gamma = 1$?
  \end{itemize}
  \begin{center}
    \includegraphics[angle=0,height=0.7\textheight]{zipf1949a_p407.pdf}
  \end{center}

\end{frame}

\begin{frame}[plain]
  %% \frametitle{}


  \begin{block}{
      \small
      Comment \#60 in \wordwikilink{http://opinionator.blogs.nytimes.com/2009/05/19/math-and-the-city}{Math and the City}
      by Strogatz, NYT:}

    \includegraphics[width=1.2\textwidth]{2009-05-19comment-on-zipf.pdf}

%%    George Kingsley Zipf was my teacher at Harvard…He had given a class project where we were to see if Chemical Companies when ranked by the number of different chemicles they produced, followed his Law of Least Effort. I missed turning in my assignment due to the accidental death of my father….When I returned from the funeral I was given a message to call Dr. Zipf immediately. I did and when I explained why I was late turning in the data. He said, “Well, your father’s gone and I (Zipf) have no pipeline to God. I expect the data will be on my desk tomorrow morning!”…..My mother, sister and extended family spread huge books of trade magazines on the kitchen and dining room tables and furiously went to work….We worked until late in the night and finished the project…..I drove to Harvard the next morning and angrily gave the hundreds of ‘three by five cards’ to Zipf. All he said was, “Thank you.” Years later, I wondered whether his’meaness’ had really been his way of helping me and my family to take our minds of our grief that day and concentrate on finishing my assignment. In my youth I thought not, but now as I approach 80, I like to think his seemingly hurtful attitude was really an act of kindness,,,,,
%%
%%    — Jim Terry
  \end{block}
\end{frame}

\begin{frame}
  \frametitle{Recent Zipf action:}

  \begin{block}{}
  \begin{columns}
    \column{0.5\textwidth}
    \includegraphics[width=\textwidth]{gonzalez2008a_fig2d.pdf}
    \column{0.5\textwidth}
    \begin{itemize}
    \item 
      Probability of people being in certain locations
      follows a Zipfish law...
    \item 
      From Gonz\`{a}lez et al., Nature (2008)\\
      \alertb{``Understanding individual human mobility patterns''}\cite{gonzalez2008a}
    \end{itemize}
  \end{columns}
  \end{block}

    Bonus: 
    \wordwikilink{http://www.youtube.com/watch?v=PG55vPOh64U}{Marta's talk}
    at \newline
    \wordwikilink{http://www.uvm.edu/~tedxuvm/?Page=archive/2011/default.php}{UVM's 2011 TEDx event ``Big Data, Big Stories.''}


\end{frame}

\section{Yet\ more\ Zipfian\ Empirics}

\begin{frame}
  \frametitle{Zipfian empirics:}

    \centering
  \includegraphics[angle=0,height=0.7\textheight]{zipf1949a_p033.pdf}

\end{frame}

\begin{frame}
  \frametitle{Zipfian empirics:}

    \centering
  \includegraphics[angle=0,width=\textwidth]{zipf1949a_p034.pdf}

\end{frame}

\begin{frame}
  \frametitle{Zipfian empirics:}

    \centering
  \includegraphics[angle=0,height=0.7\textheight]{zipf1949a_p042.pdf}

\end{frame}

\begin{frame}
  \frametitle{Zipfian empirics:}

    \centering
  \includegraphics[angle=0,height=0.7\textheight]{zipf1949a_p043.pdf}

\end{frame}

\begin{frame}
  \frametitle{Zipfian empirics:}

    \centering
  \includegraphics[angle=0,height=0.7\textheight]{zipf1949a_p045.pdf}

\end{frame}

\begin{frame}
  \frametitle{Zipfian empirics:}

    \centering
  \includegraphics[angle=0,height=0.7\textheight]{zipf1949a_p046.pdf}

\end{frame}

\begin{frame}
  \frametitle{Zipfian empirics:}

    \centering
  \includegraphics[angle=0,width=\textwidth]{zipf1949a_p064.pdf}

\end{frame}

\begin{frame}
  \frametitle{Zipfian empirics:}

    \centering
  \includegraphics[angle=0,height=0.7\textheight]{zipf1949a_p080.pdf}

\end{frame}

\begin{frame}
  \frametitle{Zipfian empirics:}

    \centering
  \includegraphics[angle=0,height=0.7\textheight]{zipf1949a_p093.pdf}

\end{frame}

\begin{frame}
  \frametitle{Zipfian empirics:}

    \centering
  \includegraphics[angle=0,height=0.7\textheight]{zipf1949a_p094.pdf}

\end{frame}

\begin{frame}
  \frametitle{Zipfian empirics:}

    \centering
  \includegraphics[angle=0,width=\textwidth]{zipf1949a_p096.pdf}

\end{frame}

\begin{frame}
  \frametitle{Zipfian empirics:}

  \centering
  \includegraphics[angle=0,height=0.7\textheight]{zipf1949a_p111.pdf}

\end{frame}

\begin{frame}
  \frametitle{Zipfian empirics (p.\ 176):}

  \begin{itemize}
  \item Article length in the Encylopedia Brittanica
  \end{itemize}
  \centering
  \includegraphics[angle=-2,height=0.7\textheight]{zipf1949a_p176.pdf}

\end{frame}

\begin{frame}
  \frametitle{Zipfian empirics:}

  \centering
  \includegraphics[angle=0,height=0.7\textheight]{zipf1949a_p232.pdf}

\end{frame}

\begin{frame}
  \frametitle{Zipfian empirics:}

  \centering
  \includegraphics[angle=0,height=0.7\textheight]{zipf1949a_p385.pdf}

\end{frame}

\begin{frame}
  \frametitle{Zipfian empirics:}

  \centering
  \includegraphics[angle=0,height=0.7\textheight]{zipf1949a_p392.pdf}

\end{frame}

\begin{frame}
  \frametitle{Zipfian empirics:}

  \begin{itemize}
  \item \# species per genera:
  \end{itemize}
  \centering
  \includegraphics[angle=0,height=0.7\textheight]{zipf1949a_p233.pdf}
  \begin{itemize}
  \item $\alpha=1$ corresponds to $\gamma=1+1/\alpha=2$.
  \end{itemize}

\end{frame}

\begin{frame}
  \frametitle{Zipfian empirics:}

  \centering
  \includegraphics[angle=0,height=0.7\textheight]{zipf1949a_p408.pdf}

\end{frame}
