%% Add spreading of phonemes around the world

%% Add \cite{ferrericancho2003a}

%% Add Heap's law

%% http://en.wikipedia.org/wiki/Heaps'_law

%% Linyuan Lue, Zi-Ke Zhang and Tao Zhou, Zipf's law leads to heaps' law: analyzing their relation in finite-size systems, PLoS ONE 5, e14139, 2010 [html] [doi]


%% add a section on the remnants of other languages
%% the massive impacts of some

%% tuesday, wednesday, thursday

%% other: minutes, seconds, degrees: ancient and universal
%% ways of measuring 


%% add paper showing languages of different cadences 
%% have the same amount of information
%% http://www.time.com/time/health/article/0,8599,2091477,00.html?xid=fblike
%% http://persquaremile.com/2011/12/21/which-reads-faster-chinese-or-english/


%% Add Petersen 2011a

%% Add Culturomics

%% Piantadosi

%% gell-mann2011a

%% spreading as a function of phoneme diversity
%% new zealand dude


%% lieberman2007a
\section{Irregular\ verbs}

\begin{frame}
  \frametitle{Irregular verbs}

  \begin{block}{Cleaning up English:}
    \alert{``Quantifying the evolutionary dynamics of language''}\cite{lieberman2007a}\\
    Lieberman et al., Nature, Vol 449, 713-716, 2007.
  \end{block}

  \begin{columns}
    \column{0.4\textwidth}
    \includegraphics[width=\textwidth]{NatureEvolutionofLanguageCover.pdf}
    \column{0.6\textwidth}
    \begin{block}{}
      \begin{itemize}
      \item<1-> 
        Exploration of how verbs with irregular 
        conjugation gradually become regular over time.
      \item<1-> 
        Comparison of verb behavior in Old, Middle, and Modern English.
      \end{itemize}
    \end{block}
  \end{columns}

\end{frame}

\begin{frame}
  \frametitle{Irregular verbs}


  \begin{block}{}
    \includegraphics[width=0.9\textwidth]{lieberman2007a_fig1a}

    \begin{itemize}
    \item Universal tendency towards regular conjugation
    \item Rare verbs tend to be regular in the first place
    \end{itemize}
  \end{block}

\end{frame}

\begin{frame}
  \frametitle{Irregular verbs}

  \begin{block}{}
  \includegraphics[width=\textwidth]{lieberman2007a_fig1b}

  \begin{itemize}
  \item<1->
    Rates are relative.
  \item<2->
    The \alertb{more common} a verb is, the \alertb{more resilient}
    it is to change.
  \end{itemize}
  \end{block}

\end{frame}

\begin{frame}[plain]
  \frametitle{Irregular verbs}

  \begin{block}{}
    \includegraphics[width=1.2\textwidth]{lieberman2007a_tab1}

    \begin{itemize}
    \item \alert{Red} = regularized
    \item Estimates of half-life for regularization ($\propto f^{1/2}$)
    \end{itemize}
  \end{block}

\end{frame}

\begin{frame}
  \frametitle{Irregular verbs}

  \begin{block}{}
  \includegraphics[width=\textwidth]{lieberman2007a_fig2a}

  \begin{itemize}
  \item 
    `Wed' is next to go.
  \item 
    -ed is the winning rule...
  \item<+->
    But `snuck' is 
    \wordwikilink{http://books.google.com/ngrams/graph?content=snuck\%2Csneaked\&year_start=1800\&year_end=2000\&corpus=0\&smoothing=3}{sneaking up on sneaked.}\cite{michel2010a}
  \end{itemize}
  \end{block}

\end{frame}

%% \begin{frame}
%%   \frametitle{Irregular verbs}
%% 
%%   \begin{block}{}
%%   \includegraphics[width=\textwidth]{lieberman2007a_fig2b}
%% 
%%   \begin{itemize}
%%   \item Regularization rate $\propto$ word frequency$^{-1/2}$
%%   \item Half life $\propto$ word frequency$^{1/2}$
%%   \end{itemize}
%%   \end{block}
%% 
%% \end{frame}

\begin{frame}
  \frametitle{Irregular verbs}

  \begin{block}{}
  \includegraphics[width=.95\textwidth]{lieberman2007a_fig3}

  \begin{itemize}
  \item Projecting back in time to proto-Zipf story of many tools.
  \end{itemize}
  \end{block}

\end{frame}

\section{Word\ lifespans}

\section{Meanings}

\begin{frame}
  \frametitle{Word meanings}

  \begin{block}{Preliminary findings on word frequency and number of meanings}
    \begin{itemize}
    \item 
      Corpus: 10,000 most frequent words from Project Gutenberg
    \item 
      \# meanings for each word estimated using \wordwikilink{http://www.dictionary.com}{dictionary.com}
    \item 
      Friends: perl, regular expressions, wget.
    \end{itemize}
  \end{block}

\end{frame}

\begin{frame}
  \frametitle{Word meanings}

  \begin{block}{}
    \begin{tabular}{ll}
      \textbf{A.} & 
      \textbf{B.} \\
      \includegraphics[width=0.47\textwidth]{figwordfreq01_noname.pdf} & 
      \includegraphics[width=0.49\textwidth]{figwordmeaning03c_noname.pdf}
      %%          \includegraphics[width=0.49\textwidth]{figwordmeaning03b_noname.pdf}\\
    \end{tabular}

    \textbf{A.} 
    Word frequency versus rank, 
    slope $\alpha \sim -1.2$ corresponds to
    to a frequency distribution with $\gamma \sim 1.8$.\\
    \textbf{B.} 
    Relationship between average number of meanings and 
    average frequency (bins are by rank, with
    each circle representing 500 words).  Slope of 1/3 lower than
    Zipf's 1/2\cite{zipf1949a}.
  \end{block}

\end{frame}


\begin{frame}
  \frametitle{Word meanings}

  \begin{block}{}
    \begin{tabular}{ll}
      \textbf{A.} & 
      \textbf{B.} \\
      \includegraphics[width=0.49\textwidth]{figwordmeaning02_noname.pdf} &
      \includegraphics[width=0.46\textwidth]{figwordmeaning03_noname.pdf} \\
    \end{tabular}
    \begin{itemize}
    \item Meaning number as a function of word rank.
    \item The three exponents combine within error:
      $ 1.2 \times 1/3 = 0.4 \simeq 0.45.$
    \end{itemize}
  \end{block}

\end{frame}


\begin{frame}
  \frametitle{Word meanings}

  \begin{block}{}
    \begin{tabular}{ll}
      \textbf{A.} & 
      \textbf{B.} \\
      \includegraphics[width=0.49\textwidth]{figwordmeaning26b_noname.pdf} & 
      \includegraphics[width=0.49\textwidth]{figwordmeaning25b_noname.pdf} \\
    \end{tabular}

    \begin{itemize}
    \item 
      Scaling collapse for meaning number distribution
    \item 
      Each curve corresponds to approximately 500 words
      group according to rank (1--500, 501--1000, ...).
    \item  
      With normalization
      $$
      P(n_m) = f^{-1/3}
      G \left( f^{-1/3} n_m \right).
      $$
    \end{itemize}
  \end{block}

\end{frame}


\begin{frame}
  \frametitle{Word meanings}

  \begin{block}{Further work:}
    \begin{itemize}
    \item<1-> Check these scalings again
    \item<2-> Explore alternate data sources
    \item<3-> Think about
      why meaning number might scale with frequency.
    \item<4-> May be an information theoretic story.
    \item<5-> If we add context, we may be able to
      use a modified version of Simon's approach\cite{simon1955a}
    \item<6->
      The city story here would be that there may
      be many cities and towns with the same
      name (e.g., Springfield) with an uneven distribution in populations.
    \end{itemize}
  \end{block}

\end{frame}


