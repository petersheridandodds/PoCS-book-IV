\section{Overview}

  \textbf{Overview}

  \textbf{The basic idea/problem/motivation/history:}
    
     
      Organizations as information exchange entities.
     
      Catastrophe recovery.
     
      Solving ambiguous, ill-defined problems.
    
      Robustness as `optimal' design feature.
    
  

  \textbf{A model of organizational networks:}
    
    
      Network construction algorithm.
    
      Task specification.
    
      Message routing algorithm.
    
  

  \textbf{Results:}
    
    
      Performance measures.
    
  


\subsection{Toyota}

%%   Toyota-Aisin example.
%%  eye-sheen
%% 
%%  1997
%%  brake valve part, only supplier
%%  4 hours supply (just in time)
%%  
%%  14000 cars per day
%%  
%%  months of delay predicted
%%  6 months before new machines would arrive
%%  
%%  recovered in about 5 days
%%  
%%  sewing machine maker with no experience in car parts
%%  spent about 500 man hours refitting a milling machine
%%  to produce 40 valves a day
%%  36 suppliers
%%  150 subcontractors
%%  50 supply lines
%%  
%%  result of strengths of ties across a hierarchy
%% 
%%  wrote manuals on how to deal with such a crisis
%%  
%% scary for other companies

  \textbf{February, 1997:}

  \textbf{Aisin (eye-sheen), maker of brake valve parts for Toyota, burns to ground.\cite{nishiguchi2000a}}
    
    
      4 hours supply (``just in time'').
    
      14,000 cars per day $\rightarrow$ 0 cars per day.
    
      6 months before new machines would arrive.
     
      {
        Recovered in 5 days.}
    
  

  
   
    For more see Nishiguchi and Beaudet\cite{nishiguchi2000a}\\
    ``Fractal Design: Self-organizing Links in Supply Chain''\\
    in ``Knowledge Creation: A New Source of Value''
  
  


  \textbf{February, 1997:}

  \textbf{Some details:}
    
    
      36 suppliers, 150 subcontractors
     
      50 supply lines
     
      Sewing machine maker with no experience in car parts
      spent about 500 man hours refitting a milling machine
      to produce 40 valves a day.
    
      Recovery depended on horizontal links which
      arguably provided:
      
       
        robustness
      
        searchability
      
    
  


  \textbf{Some things fall apart:}
  \includegraphics[width=\textwidth]{lehmann-brothers-bag.jpg}

  \includegraphics[width=\textwidth]{lehmann-brothers-bag-zoom.jpg}

  \textbf{Rebirth:}
  \includegraphics[width=\textwidth]{despicable_me_bank_of_evil_formerly_lehman_brothers.png}

\subsection{Ambiguous\ problems}

  \textbf{Motivation}

  \textbf{Recovery from catastrophe involves solving problems that are:}
    
     
      Unanticipated,
     
      Unprecedented,
    
      Ambiguous (nothing is obvious),
     
      Distributed (knowledge/people/resources),
     
      Limited by existing resources,
     
      Critical for survival.
    
  

  \textbf{Frame:}
    
     
      Collective solving of ambiguous problems
    
  


  \textbf{Motivation}

  \textbf{Ambiguity:}
    
    
      Question much less answer is not well understood.
    
      Back and forth search process rephrases question.
    
      Leads to iterative process of query reformulation.
    
      Ambiguous tasks are inherently not decomposable.
    
      How do individuals collectively work
      on an ambiguous organization-scale problem?
    
      How do we define ambiguity?
    



  \textbf{Let's modelify:}

  \textbf{Modeling ambiguous problems is hard\ldots}
    
    
      Model response instead\ldots
     
      Individuals need novel information and must communicate with others
      outside of their usual contacts.
      
      Creative search is intrinsically inefficient.
    
  

  \textbf{Focus on robustness:}
    
     
      Avoidance of individual failures.
    
      Survival of organization even when failures do occur.
    
   


\subsection{Models\ of\ organizations:}

  \textbf{Why organizations exist:}

  \textbf{\wordwikilink{http://en.wikipedia.org/wiki/Ronald\_Coase}{Ronald Coase}, 1937, ``The Nature of the Firm''\cite{coase1937a}}
    
     
      Notion of \wordwikilink{http://en.wikipedia.org/wiki/Transaction\_cost}{Transaction Costs}.
     
      More efficient for individuals to cooperate outside of the market.
    
  

  \begin{center}
    \includegraphics[width=0.8\textwidth]{coase}
  \end{center}


  \textbf{Real organizations---Extremes}

  \textbf{Hierarchy:}
    
     
      Maximum efficiency,
     
      Suited to static environment,
     
      Brittle.
    
  

  \textbf{Market:}
    
    
      Resilient,
    
      Suited to rapidly changing environment,
    
      Requires costless interactions.
    
  


  \textbf{Real organizations\ldots}

  \textbf{But real, complex organizations are in the middle\ldots}
    \begin{center}
      \includegraphics[width=\textwidth]{orgnetworkrange}
    \end{center}
    
    
      \alertb{``Heterarchies''} (D. Stark, 1999)\cite{stark1999a}
    
  



  \textbf{Organizations as efficient hierarchies}

  
    
    
      Economics: \alertb{Organizations $\equiv$ Hierarchies.}
    
      e.g., Radner (1993)\cite{radner1993a}, Van Zandt (1998)\cite{vanzandt1998a}
    
      Hierarchies performing associative operations:
      \begin{center}
        \includegraphics[width=0.8\textwidth]{associativenet}
      \end{center}
    
  


  \textbf{Optimal network topologies for local search}

  %% queueing, point of collapse,
  %% average search time + congestion,
  %% simulated annealing

  \textbf{Guimer\`{a} et al., 2002\cite{guimera2002b}}
  \begin{center}
    \includegraphics[width=0.45\textwidth]{mb}
    \includegraphics[width=0.45\textwidth]{md}
    
    
      Parallel search and congestion.
    
      Queueing and network collapse.
     
      Exploration of random search mechanisms.
    
  \end{center}
  


  \textbf{Optimal network topologies for local search}

      
    \includegraphics[width=\textwidth]{r1}
    
    
    
      Betweenness: $\beta$.
    
      Polarization: 
      $$\pi = \frac{\max \beta}{\tavg{\beta}}-1.$$
    
      $L$ = number of links.
    
  
  \bigskip


  
   
    Goal: minimize average search time.
   
    Few searches $\Rightarrow$ hub-and-spoke network.
   
    Many searches $\Rightarrow$ decentralized network.
  


%% %%  \textbf{Organizations as efficient hierarchies}
%%
%%  Goh?
%%
%%
\section{Modelification}

\subsection{Goals}

  \textbf{Desirable organizational qualities:}

  
    
    
      Low cost (requiring few links).
    
      Scalability.
    
      Ease of construction---existence is plausible.
    
      Searchability.
     
      \alert{`Ultra-robustness'}:
      
      [I]
        \alertb{Congestion robustness}\\
        (Resilience to failure due to information exchange);
      [II] 
        \alertb{Connectivity robustness}\\
        (Recoverability in the event of failure).
      
    
  



%%%%  \textbf{Searchability}
%%
%%  Guimer\`{a} \textit{et al.}, 2002
%%
%%  Optimal network topologies for searching
%%  using only local information.
%%
%%  \hspace{3ex}
%%  Low cost searches $\Rightarrow$ hub-based networks.
%%
%%  \hspace{3ex}
%%  High cost searches $\Rightarrow$ featureless networks.
%%  
%%
  \textbf{Searchability}

  \textbf{Small world problem:}
    
    
      Can individuals pass a message
      to a target individual using only personal connections?
    
      Yes, large scale networks searchable 
      if nodes have \alertb{identities}.
    
      ``Identity and Search in Social Networks,''
      Watts, Dodds, \& Newman, 2002.\cite{watts2002b}
    
  



%%%%%%%%%%%
%% model
%%%%%%%%%%%

\subsection{Model}

  \textbf{Model}

  \textbf{Organizational network robustness:}
    ``Information exchange and the robustness of organizational networks,''\\
    Dodds, Watts, and Sabel, 2003.\cite{dodds2003c}\\
    Proc. Natl. Acad. Sci., edited by 
    \wordwikilink{http://en.wikipedia.org/wiki/Harrison\_White}{Harrison White}
  

  \textbf{Formal organizational structure:}
    
    
      \alertb{Underlying hierarchy:}
      
      
        branching ratio $b$
      
        depth $L$
      
        $N = (b^L-1)/(b-1)$ nodes
      
        $N-1$ links
      
    
      \alertb{Additional informal ties:}
      
       
        Choose $m$ links according to a
        two parameter probability distribution
       
        $ 0 \le m \le (N-1)(N-2)/2 $
      
    
    
  





  \textbf{Model---underlying hierarchy}

  \textbf{Model---formal structure:}
    \begin{center}
      \includegraphics[width=0.75\textwidth]{networkvariation8}
    \end{center}
    $$ 
    b=3, \quad  L=3, \quad N=13
    $$
  


%%%%  \textbf{Model---underlying hierarchy}
%%  \begin{center}
%%    \includegraphics[width=0.75\textwidth]{networkvar1}
%%  \end{center}
%%  \Large \[ b=3, \quad  L=4, \quad N=40\]
%% 
  \textbf{Model---addition of links}

  \textbf{Team-based networks ($m=12$):}
    \begin{center}
      \includegraphics[width=0.75\textwidth]{networkvariation10}
    \end{center}
  
    

  \textbf{Model---addition of links}

  \textbf{Random networks ($m=12$):}
    \begin{center}
      \includegraphics[width=0.75\textwidth]{networkvariation11}
    \end{center}
  


  \textbf{Model---addition of links}

  \textbf{Random interdivisional networks ($m=6$):}
    \begin{center}
      \includegraphics[width=0.75\textwidth]{networkvariation12}
    \end{center}
  


  \textbf{Model---addition of links}

  \textbf{Core-periphery networks ($m=6$):}
    \begin{center}
      \includegraphics[width=0.75\textwidth]{networkvariation13}
    \end{center}
  


  \textbf{Model---addition of links}

  \textbf{Multiscale networks $(m=12)$:}
    \begin{center}
      \includegraphics[width=0.75\textwidth]{networkvariation14}
    \end{center}
  



  \textbf{Model---construction}

  \begin{center}
    \includegraphics[width=\textwidth]{linkaddition3}
  \end{center}


  \textbf{Model---construction}

  
    
     
      Link addition probability:
      $$
      P(D,d_1,d_2) 
      \propto 
      e^{-D/\lambda} e^{-f(d_1,d_2)/\zeta}
      $$
     
      First choose $(D,d_1,d_2)$.
     
      Randomly choose $(y,x_1,x_2)$ given $(D,d_1,d_2)$.
     
      Choose links without replacement.
    
  


  \textbf{Model---construction}

  \textbf{Requirements for $f(d_1,d_2)$:}
    
     
      $f \geq 0$ for $d_1+d_2 \geq 2$
    
      $f$ increases monotonically with $d_1$, $d_2$.
    
      $f(d_1,d_2) = f(d_2,d_1)$.
    
      $f$ is maximized when $d_1=d_2$.
    
  

  \textbf{Simple function satisfying 1--4:}
    $$
    f(d_1,d_2) = (d_1^2 + d_2^2-2)^{1/2}
    $$
    $$
    \Rightarrow
    P(y,x_1,x_2) \propto e^{-D/\lambda} e^{-(d_1^2 + d_2^2-2)^{1/2}/\zeta} 
    $$  
  


%%%%  \textbf{Model---limiting cases}
%%
%%\vfill
%%
%%  \hspace{3ex}
%%  $\lambda$ large, $\zeta=0$: team-based networks
%%
%%  \hspace{3ex}
%%  $\lambda$ large, $\zeta$ large: random networks
%%
%%  \hspace{3ex}
%%  $\lambda=0$, $\zeta$ large: random interdivisional networks
%%
%%  \hspace{3ex}
%%  $\lambda$ small, $\zeta$ small: core-periphery networks
%%
%%  \hspace{3ex}
%%  $\lambda$, $\zeta$ intermediate: multiscale networks
%%
%%\vfill
%%
  \textbf{Model---limiting cases}

  \vfill
  \begin{center}
    \includegraphics[width=\textwidth]{networkspace}
  \end{center}
  \vfill



%%%%%%%%%%%%%
%% testing %%
%%%%%%%%%%%%%

\subsection{Testing}

  \textbf{Message passing pattern}

  
  
    Each of $T$ time steps,  each node generates a message with probability $\mu$.
   
    Recipient of message chosen based on distance from sender.
   
    $$
    P(\mbox{recipient at distance}\ d) \propto e^{-d/\xi}.
    $$
  
  
   
    $\xi$ = measure of uncertainty;
   
    $\xi=0$: local message passing;
    
    $\xi=\infty$: random message passing.
  

  \textbf{Message passing pattern:}

  \textbf{Distance $d_{12}$ between two nodes $x_1$ and $x_2$:}
    \bigskip
           
      \includegraphics[width=\textwidth]{linkaddition3}
       
      $$
      d_{12} = \max(d_1,d_2) =3
      $$
        \bigskip
        
     
      Measure unchanged with presence of informal ties.
    
  

  \textbf{Message passing pattern}

  \textbf{Simple message routing algorithm:} 
    
    
      Look ahead one step:
      always choose neighbor closest to recipient node.
    
      \alertb{Pseudo-global knowledge:}
      
       
        Nodes understand hierarchy.
       
        Nodes know only local informal ties.
      
    
  



  \textbf{Message passing pattern}

  \textbf{Interpretations:}
    
    
      Sender knows specific recipient.
    
      Sender requires certain kind of recipient.
    
      Sender seeks specific information but recipient unknown.
    
      Sender has a problem but information/recipient unknown.
    

  


  \textbf{Message passing pattern}

  \textbf{Performance:}
  
  
    Measure Congestion Centrality $\rho_i$,
    fraction of messages passing through node $i$.
  
    Similar to betweenness centrality.
    
    However: depends on 
    
    
      Search algorithm;
    
      Task specification ($\mu$, $\xi$).
    
    
    Congestion robustness comes from
    minimizing $\rho_{\textnormal{max}}$.
  
  



%% %%   \textbf{Message passing pattern}
%% 
%%   Two message routing algorithms: $C_1$ and $C_2$ search.
%% 
%%   $C_1$: Look ahead one step---choose neighbor closest to recipient node.
%% 
%%   $C_2$: Look ahead two steps---choose neighbor with neighbor closest to recipient node.
%% 
%%   Nodes understand hierarchy but know only local informal ties.
%% 
%%   Examine $C_1$ only.
%% 
%% 
\subsection{Results}

%%%%%%%%%%%%%
%% results %%
%%%%%%%%%%%%%

  \textbf{Performance testing:}

  \textbf{Parameter settings (unless varying):}
    
    
      Underlying hierarchy:
      $b=5$, $L=6$, $N=3096$;
    
      Number of informal ties:
      $m=N$.
    
      Link addition algorithm:
      $\lambda=\zeta=0.5$.
    
      Message passing:
      $\xi=1$, $\mu=10/N$, $T=1000$.
    
  


  \textbf{Results---congestion robustness}
  \begin{center}
    \includegraphics[width=0.75\textwidth]{figmsdomain_rhomax30_1234_5C_noname}
    %%    \includegraphics[width=0.65\textwidth]{figmsdomain_rhomax30_1234_5C_noname}
  \end{center}

%% %%   \textbf{Results}
%%   \begin{center}
%%     \includegraphics[width=0.5\textwidth]{figmsdomain_rhomax30_1234_3_3d_noname}
%%   \end{center}
%% 
  \textbf{Results---varying number of links added:}

      
    \includegraphics[width=\textwidth]{figzl_linkadd_rhomax2_32_noname}
    
    
    [] 
        {\Large$\diamond$}=TB
    []
        $\bigtriangledown$=R
    []
        $\bigtriangleup$=RID 
    []
        {\small$\bigcirc$}=CP
    []
        $\Box$=MS 
    
  

  \textbf{Results---varying message passing pattern}

      
    \includegraphics[width=\textwidth]{figzl_rhomax_xi_31_noname}
    
    
    [] 
        {\Large$\diamond$}=TB
    []
        $\bigtriangledown$=R
    []
        $\bigtriangleup$=RID 
    []
        {\small$\bigcirc$}=CP
    []
        $\Box$=MS 
    
  


  \textbf{Results---Maximum firm size}

  
  
    Congestion may increase with size
    of network.
   
    Fix rate of message passing ($\mu$)
    and
    Message pattern ($\xi$).
  
    Fix branching ratio of hierarchy  and add more levels.
   
    Individuals have limited capacity 
    $\Rightarrow$ limit to firm size.
  



  \textbf{Results---Scalability}

      
    \includegraphics[width=\textwidth]{figrhomaxNscaling33_6lin_noname}
    
    
    [] 
        {\Large$\diamond$}=TB
    []
        $\bigtriangledown$=R
    []
        $\bigtriangleup$=RID 
    []
        {\small$\bigcirc$}=CP
    []
        $\Box$=MS 
    
  

  \textbf{Results---Scalability}

      
    \includegraphics[width=\textwidth]{figrhomaxNscaling33_10log2_noname}
    
    
    [] 
        {\Large$\diamond$}=TB
    []
        $\bigtriangledown$=R
    []
        $\bigtriangleup$=RID 
    []
        {\small$\bigcirc$}=CP
    []
        $\Box$=MS 
    
  


  \textbf{Connectivity Robustness}

  \textbf{Inducing catastrophic failure:}
    
    
      Remove $N_r$ nodes and measure relative size
      of largest component $C = S/(N-N_r)$.
     
      Four deletion sequences:
      
       
        Top-down;
       
        Random;
       
        Hub;
       
        Cascading failure.
      
    
      Results largely independent of sequence.
    
  
  

  \textbf{Results---Connectivity Robustness}

      
    \includegraphics[width=\textwidth]{figrobustness51_mod}
    
    
    [] 
        {\Large$\diamond$}=TB
    []
        $\bigtriangledown$=R
    []
        $\bigtriangleup$=RID 
    []
        {\small$\bigcirc$}=CP
    []
        $\Box$=MS 
    
  
    


  \textbf{Summary of results}

  \small
  \begin{tabular}{l|lll}
    Feature & Congestion  & Connectivity & Scalability \\ 
    & Robustness & Robustness &  \\\hline
    \\
    Core-periphery & good & average &average \\
    \\
    Random & poor & good & poor \\
    \\
    Rand. Interdivisional & poor & good & poor \\
    \\
    Team-based & poor & poor & poor\\
    \\
    \alertb{Multiscale} & \alertb{good} & \alertb{good} &\alertb{good} \\
  \end{tabular}
  

\section{Conclusion}

  \textbf{Conclusary moments}

  \textbf{Multi-scale networks:}
    
    
      Possess good Congestion Robustness and
      Connectivity Robustness $\Rightarrow$ Ultra-robust;
    
      Scalable;
    
      Relatively insensitive to parameter choice;
    
    
    
      Above suggests existence of multi-scale structure is plausible.
    
  
  

  \textbf{Conclusary moments}

  
    
    
      Foregoing is an attempt to model what organizations
      might look like beyond simple hierarchies (2003).
    
      Possible work: develop `bottom up' model of organizational
      networks based on social search, identity 
      (emergent searchability).
    
      Balance of \alertb{generalists versus specialists}---how many
      middle managers does an organization need?
    
      Still a need for data on real organizations\ldots
    
  






