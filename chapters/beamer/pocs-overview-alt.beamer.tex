%%%%%%%%%%%%%%%%%%%%%%%%%%%%%%%%%%%%%
% very clear summary slide 
%%%%%%%%%%%%%%%%%%%%%%%%%%%%%%%%%%%%%

\section{Plan}

\begin{frame}
  \frametitle{Something of a plan:}

  \begin{itemize}
  \item<1-> \alert{Lecture 1:} Overview; Background
  \item<2-> \alert{Lecture 2:} Random, Scale-free, \\
    \qquad \qquad and Small-World networks
  \item<3-> \alert{Lecture 3:} Models of Contagion
  \item<4-> \alert{Lecture 4:} Transportation networks; \\
    \qquad \qquad Discovering structure
  \end{itemize}

\end{frame}

\begin{frame}
  \frametitle{Exciting details regarding these slides:}

  \begin{itemize}
  \item<1->
    Three versions (all in pdf): 
    \begin{enumerate}
    \item 
      Presentation,
    \item
      Flat Presentation,
    \item
      Handout (2x2).
    \end{enumerate}
  \item<2->
    Presentation versions are \alertb{navigable} 
    and hyperlinks are \alertb{clickable}.
  \item<3->
    Web links look 
    \wordwikilink{http://www.google.com}{like this}. 
  \item<4->
    References in slides link to full citation at end.\cite{anderson1972a}
  \item<5->
    Citations contain links to papers in pdf (if available).
  \item<6->
    50 hours of lectures $\rightarrow$ 5 hours.
  \item<7->
    Brought to you by a concoction of \LaTeX, Beamer, and perl.
  \end{itemize}
\end{frame}

%%%%%%%%%%%%%%%%%%%%%%%%%%%%%%%%%%%%%
% basic definitions
%%%%%%%%%%%%%%%%%%%%%%%%%%%%%%%%%%%%%

\section{Basic definitions}

\begin{frame}
  \frametitle{Basic definitions}

  \begin{block}{Complex System---Some ingredients:}
    \begin{itemize}
    \item<2->
      Distributed system of many interrelated parts
    \item<3->
      No centralized control 
    \item<4->
      Nonlinear relationships
    \item<5->
      Existence of feedback loops
    \item<6->
      Complex systems are open (out of equilibrium)
    \item<7->
      Presence of Memory
    \item<8->
      Modular (nested)/multiscale structure
    \item<9->
      Opaque boundaries
    \item<10->
      Emergence---`More is Different'\cite{anderson1972a}
    \end{itemize}
  \end{block}
  \mbox{}
  \hfill
  \includegraphics[width=.07\textwidth]{wikipedia.jpg}

\end{frame}

\begin{frame}
  \frametitle{Basic definitions}

% dictionary definition

  \begin{block}{{Complex:} (Latin = with + fold/weave (com + plex))}
    \alert{Adjective}
    \begin{itemize}
    \item Made up of multiple parts; intricate or detailed.
    \item Not simple or straightforward.
    \end{itemize}
  \end{block}
\mbox{} \hfill \includegraphics[width=.07\textwidth]{wikipedia.jpg}

\end{frame}

\begin{frame}
  \includegraphics[width=\textwidth]{network_dictionary_cut.pdf}
\end{frame}

\begin{frame}
  \frametitle{Thesaurus deliciousness:}

  \begin{center}
    \includegraphics[width=0.9\textwidth]{network_thesaurus_cut.pdf}
  \end{center}

\end{frame}




%% \begin{frame}
%%   \frametitle{Basic definitions}
%% 
%%   \alert{Network:} (net + work, 1500's)
%%   \hfill
%%   \includegraphics[width=.07\textwidth]{wikipedia.jpg}
%% 
%%   \begin{block}<2->{\alert{Noun:}}
%%     \begin{enumerate}
%%     \item<2-> Any interconnected group or system
%%     \item<3-> Multiple computers and other devices connected together to share information
%%     \end{enumerate}
%%   \end{block}
%% 
%%   \begin{block}<4->{\alert{Verb:}}
%%     \begin{enumerate}
%%     \item<4-> To interact socially for the purpose of getting connections or personal advancement
%%     \item<5-> To connect two or more computers or other computerized devices
%%     \end{enumerate}
%%   \end{block}
%% 
%% \end{frame}

\begin{frame}
  \frametitle{Ancestry:}

  From Keith Briggs's excellent
  \wordwikilink{http://keithbriggs.info/network.html}{etymological investigation:}

  \medskip

  \begin{columns}
    \column{0.05\textwidth}
    \column{0.4\textwidth}
    \begin{itemize}
    \item<1-> 
      Opus reticulatum:
    \item<1-> 
      A Latin origin?
    \end{itemize}
    \column{0.55\textwidth}
    \includegraphics[width=\textwidth]{opus_reticulatum.jpg}\\
    {\tiny [http://serialconsign.com/2007/11/we-put-net-network]}
  \end{columns}

\end{frame}

\begin{frame}
  \frametitle{Ancestry:}

  \begin{block}<1->{First known use: Geneva Bible, 1560}
    `And thou shalt make unto it a grate like networke of brass (Exodus xxvii 4).'
  \end{block}

  \begin{block}<2->{From the OED via Briggs:}
    \begin{itemize}
    \item<2-> 
      1658--: reticulate structures in animals
    \item<3-> 
      1839--: rivers and canals
    \item<4-> 
      1869--: railways
    \item<5-> 
      1883--: distribution network of electrical cables
    \item<6-> 
      1914--: wireless broadcasting networks
    \end{itemize}
  \end{block}

\end{frame}


\begin{frame}
  \frametitle{Ancestry:}

  \begin{block}{Net and Work are venerable old words:}
    \begin{itemize}
    \item
      \alert{`Net'} first used to mean spider web 
      {\small (King {\AE}lfr\'{e}d, 888)}.
    \item
      \alert{`Work'} appear to have long meant purposeful action.
    \end{itemize}
  \end{block}

  \begin{columns}
    \column{0.5\textwidth}
    \includegraphics[width=\textwidth]{briggs2005a_fig1}
    \column{0.5\textwidth}
    \includegraphics[width=\textwidth]{briggs2005a_fig2}
  \end{columns}

  \begin{itemize}
  \item<2->
    `Network' = something built
    based on the idea of natural, flexible lattice or web.
  \item<3-> 
    c.f., ironwork, stonework, fretwork.
  \end{itemize}

\end{frame}

\begin{frame}
  \frametitle{Key Observation:}

  \begin{itemize}
  \item <1->
    Many \alert{complex systems}\\ 
    can be viewed as \alert{complex networks}\\
    of physical or abstract interactions.
  \item <2->
    Opens door to mathematical and numerical analysis.
  \item <3-> 
    Dominant approach of last decade of 
    a \alertb{theoretical-physics/stat-mechish} flavor.
  \item <4-> 
    Mindboggling amount of work published 
    on complex networks since 1998...
  \item <5-> 
    ... largely due to your typical theoretical physicist:
    \begin{overprint}
      \onslide<1-5 | handout:0 | trans: 0>
      \onslide<6- | handout:1 | trans: 1>
      \smallskip
      \begin{columns}
        \column{0.3\textwidth}
        \includegraphics[width=\textwidth]{piranha3.jpg}
        \column{0.7\textwidth}
        \begin{itemize}
        \item \textit{Piranha physicus}
        \item<7-> Hunt in packs.
        \item<8-> Feast on new and interesting ideas \\
          {\small (see chaos, cellular automata, ...)}
        \end{itemize}
      \end{columns}
    \end{overprint}
  \end{itemize}

\end{frame}

%%%%%%%%%%%%%%%%%%%%%%%%%%%%%%%%%%%%%
% popularity
%%%%%%%%%%%%%%%%%%%%%%%%%%%%%%%%%%%%%

\begin{frame}
  \frametitle{Popularity (according to ISI)}

  \begin{block}<1->{``Collective dynamics of `small-world' networks''\cite{watts1998a}}
    \begin{itemize}
    \item 
      Watts and Strogatz\\
      Nature, 1998
    \item 
      \alert{$\approx 3752$} citations {\tiny(as of June 5, 2009)}\\
    \item 
      Over 1100 citations in 2008 alone.
    \end{itemize}
  \end{block}

  \begin{block}<1->{``Emergence of scaling in random networks''\cite{barabasi1999a}}
    \begin{itemize}
    \item 
      Barab\'{a}si and Albert\\
      Science, 1999
    \item 
      \alert{$\approx 3860$} citations {\tiny(as of June 5, 2009)}
    \item 
      Over 1100 citations in 2008 alone.
    \end{itemize}
  \end{block}
\end{frame}

\section{Popularity according to books}

\begin{frame}
  \frametitle{Popularity according to books:}

  \showbook{tippingpoint.jpg}
  {The Tipping Point: How Little Things can make a Big Difference}
  {Malcolm Gladwell\cite{gladwell2000a}}

  \bigskip

  \showbook{nexus.jpg}
  {Nexus: Small Worlds and the Groundbreaking Science of Networks}
  {Mark Buchanan}

\end{frame}

\begin{frame}
  \frametitle{Popularity according to books:}

  \showbook{linked.jpg}
  {Linked: How Everything Is Connected to Everything Else and What It Means}
  {Albert-Laszlo Barab\'{a}si}

  \bigskip

  \showbook{sixdegrees.jpg}
  {Six Degrees: The Science of a Connected Age}
  {Duncan Watts\cite{watts2003a}}

\end{frame}

%% \begin{frame}
%%   \frametitle{Books}
%% 
%% \showbook{handbookgraphs.jpg}
%% {Handbook of Graphs and Networks}
%% {editors: Stefan Bornholdt and H. G. Schuster\cite{bornholdt2003a}}
%% 
%% \bigskip
%% 
%% \showbook{evolutionofnetworks.jpg}
%% {Evolution of Networks}
%% {S. N. Dorogovtsev and J. F. F. Mendes\cite{dorogovtsev2003a}}
%% 
%% \end{frame}
%% 
%% \begin{frame}
%%   \frametitle{Books}
%% 
%% \showbook{socialnetworkanalysis.jpg}
%% {Social Network Analysis}
%% {Stanley Wasserman and Kathleen Faust\cite{wasserman1994a}}
%% 
%% \bigskip
%% 
%% \showbook{inthebeatofaheart.jpg}
%% {In the Beat of a Heart: Life, Energy, and the Unity of Nature}
%% {John Whitfield}
%% 
%% \end{frame}

\begin{frame}
  \frametitle{Numerous others:}
  
  \small
  
    \begin{itemize}
    \item 
      \alertb{Complex Social Networks}---F. Vega-Redondo\cite{vega-redondo2007a}
    \item 
      \alertb{Fractal River Basins: Chance and Self-Organization}---I. Rodr\'{\i}guez-Iturbe and A. Rinaldo\cite{rodriguez-iturbe1997a}
    \item 
      \alertb{Random Graph Dynamics}---R. Durette
    \item 
      \alertb{Scale-Free Networks}---Guido Caldarelli
    \item 
      \alertb{Evolution and Structure of the Internet: A Statistical Physics Approach}---Romu Pastor-Satorras and Alessandro Vespignani
    \item 
      \alertb{Complex Graphs and Networks}---Fan Chung
    \item 
      \alertb{Social Network Analysis}---Stanley Wasserman and Kathleen Faust
    \item 
      \alertb{Handbook of Graphs and Networks}---Eds: Stefan Bornholdt and H. G. Schuster\cite{bornholdt2003a}
    \item 
      \alertb{Evolution of Networks}---S. N. Dorogovtsev and J. F. F. Mendes\cite{dorogovtsev2003a}
    \end{itemize}

\end{frame}

%%%%%%%%%%%%%%%%%%%%%%%%%
%% Observations
%%%%%%%%%%%%%%%%%%%%%%%%%

\begin{frame}
  \frametitle{More observations}

  \begin{itemize}
  \item<1->
    But surely \alert{networks aren't new}...
  \item<2->
    Graph theory is well established...
  \item<3->
    Study of social networks started in the 1930's...
  \item<4->
    So why all this `new' research on networks?
  \item<5->
    \alert{Answer:} \alertb{Oodles of Easily Accessible Data.}
  \item<6->
    We can now inform (alas) our theories \\
    with a much more measurable reality.$^\ast$
  \item<7-> 
    A worthy goal: establish \alertb{mechanistic explanations}.

    \medskip
    \visible<8>{
    {\small 
      $\mbox{}^\ast$\textit{If this is upsetting, maybe string theory is for you...}}
  }
  \end{itemize}
\end{frame}

\begin{frame}
  \frametitle{More observations}

  \begin{itemize}
  \item<1->
    \alertb{Web-scale} data sets can be overly \alert{exciting}.
  \end{itemize}
  
  \begin{block}<2->{Witness:}
    \begin{itemize}
    \item<2->
      The End of Theory: The Data Deluge Makes the Scientific Theory Obsolete (Anderson, Wired)
      \wikilink{http://www.wired.com/science/discoveries/magazine/16-07/pb\_theory\#}
    \item<3->
      ``The Unreasonable Effectiveness of Data,''\\ Halevy et al.\cite{halevy2009a}.
    \end{itemize}
  \end{block}

  \begin{block}<4->{But:}
  \begin{itemize}
  \item<4-> 
    For scientists, description is only part of the battle.
  \item<5-> 
    We still need to \alertb{understand}.
  \end{itemize}
  \end{block}

\end{frame}


%%%%%%%%%%%%%%%%%%%%%%%%
%% Basic definitions
%%%%%%%%%%%%%%%%%%%%%%%%

\begin{frame}
  \frametitle{Super Basic definitions}

  \begin{block}<1->{\alert{Nodes} = A collection of entities 
      which have properties that
      are somehow related to each other}
    \begin{itemize}
    \item <2-> 
      e.g., people, forks in rivers, proteins, webpages, organisms,...
    \end{itemize}
  \end{block}

  \begin{block}<3->{\alert{Links} = Connections between nodes}
    \begin{itemize}
    \item<4->
      \alert{Links} may be directed or undirected.
    \item<5->
      \alert{Links} may be binary or weighted.
    \end{itemize}
  \end{block}

  \uncover<6->{
    Other spiffing words: vertices and edges.
  }

\end{frame}

%% \begin{frame}
%%   \frametitle{Basic definitions}
%% 
%%   \begin{block}{\alert{Links} = Connections between nodes}
%%     \begin{itemize}
%%     \item<2->
%%       \alert{links}
%%       \begin{itemize}
%%       \item <3->
%%       may be real and fixed (rivers),
%%       \item <4->
%%       real and dynamic (airline routes), 
%%       \item <5->
%%       abstract with physical impact (hyperlinks),
%%       \item <6->
%%       or purely abstract (semantic connections between concepts).
%%       \end{itemize}
%%     \item<7->
%%       \alert{Links} may be directed or undirected.
%%     \item<8->
%%       \alert{Links} may be binary or weighted.
%%     \end{itemize}
%%   \end{block}
%% 
%% \end{frame}

\begin{frame}
  \frametitle{Super Basic definitions}

  \begin{block}{\alert{Node degree} = Number of links per node}
    \begin{itemize}
    \item<2-> Notation: Node $i$'s degree = $k_i$.
    \item<3-> $k_i$ = 0,1,2,\ldots.
    \item<4-> Notation: the average degree of a network = $\avg{k}$ \\
      \visible<5->{(and sometimes $z$)}
    \item<6->
      Connection between number of edges $m$ and average degree:
      $$
      \tavg{k} = \frac{2m}{N}.
      $$
    \item<7->
      \alertb{Defn:} ${\cal N}_i$ = the set of $i$'s $k_i$ neighbors
    \end{itemize}
  \end{block}

\end{frame}

\begin{frame}
  \frametitle{Super Basic definitions}

  \begin{block}{Adjacency matrix:}
    \begin{itemize}
    \item<1->
      We represent a directed network by a 
      matrix $A$ with link weight $a_{ij}$ for nodes $i$ and $j$
      in entry $(i,j)$.
    \item<2->
      e.g.,
      $$
      A = \left[
        \begin{array}{ccccc}
          0 & 1 & 1 & 1 & 0\\
          0 & 0 & 1 & 0 & 1\\
          1 & 0 & 0 & 0 & 0 \\
          0 & 1 & 0 & 0 & 1 \\
          0 & 1 & 0 & 1 & 0 \\
        \end{array}
      \right]
      $$
    \item<3->
      (n.b., for numerical work, we 
      always use sparse matrices.)
    \end{itemize}
  \end{block}

\end{frame}


%%%%%%%%%%%%%%%%%%%%%%%%%%%%%%%%%%%%%
% examples
%%%%%%%%%%%%%%%%%%%%%%%%%%%%%%%%%%%%%

\section{Examples of Complex Networks}

\begin{frame}
  \frametitle{Examples}

  \begin{block}{So what passes for a complex network?}
    \begin{itemize}
    \item<2-> Complex networks are \alert{large} (in node number)
    \item<3-> Complex networks are \alert{sparse} (low edge to node ratio)
    \item<4-> Complex networks are usually \alert{dynamic} and \alert{evolving}
    \item<5-> Complex networks can be social, economic, natural, informational, abstract, ...
    \end{itemize}
  \end{block}

\end{frame}


\begin{frame}
  \frametitle{Examples}

  \begin{block}<1->{Physical networks}
    \begin{columns}
      \column {0.5\textwidth}
        \begin{itemize}
        \item<1-> River networks
        \item<2-> Neural networks
        \item<3-> Trees and leaves
        \item<4-> Blood networks
        \end{itemize}

      \column{0.5\textwidth}
        \begin{itemize}
        \item<5-> The Internet
        \item<6-> Road networks
        \item<7-> Power grids
        \end{itemize}

    \end{columns}
  \end{block}

  \medskip

  \begin{columns}[t]
    \column{0.3\textwidth}
      \includegraphics<5->[height=.28\textheight]{internet_opte_com.png} 
%%      {\centering \tiny (\url{opte.com})}
    \column{0.4\textwidth}
    \includegraphics<1->[height=.28\textheight]{Rivierescr.jpg}
    \column{0.3\textwidth}
    \includegraphics<3->[height=.28\textheight]{BoucleSach_imacr.jpg} 
  \end{columns}

  \begin{itemize}
  \item<8> \alert{Distribution} (branching) vs.\ \alert{redistribution} (cyclical)
  \end{itemize}

\end{frame}

\begin{frame}
  \frametitle{Examples}

  \begin{columns}
    \column{0.4\textwidth}
    \begin{block}{Interaction networks}
      \begin{itemize}
      \item<1->  The Blogosphere
      \item<2->  Biochemical networks
      \item<3->  Gene-protein networks
      \item<4->  Food webs: who eats whom
      \item<5->  The World Wide Web (?)
      \item<6->  Airline networks
      \item<7->  Call networks (AT\&T)
      \item<8->  The Media
      \item<9->  Paper citations
      \end{itemize}
    \end{block}
    \column{0.6\textwidth}
    \includegraphics[width=\textwidth]{datamining-core-2006-06-27.png}\\
    {\tiny \wordwikilink{http://datamining.typepad.com}{datamining.typepad.com}}
  \end{columns}

\end{frame}

\begin{frame}
  \frametitle{Examples}

  \begin{columns}
    \column{0.4\textwidth}
    \begin{block}{Interaction networks: social networks}
      \begin{itemize}
      \item<1-> Snogging
      \item<2-> Friendships
      \item<3-> Acquaintances
      \item<4-> Boards and directors
      \item<5-> Organizations %% formal and informal ties
      \item<6-> 
        \wordwikilink{http://www.myspace.com}{myspace.com}, 
        \wordwikilink{http://www.facebook.com}{facebook.com}
      \end{itemize}
    \end{block}
    \column{0.6\textwidth}
    \includegraphics[width=\textwidth]{bearman_sex_network.jpg}\\
    {\tiny (Bearman \etal, 2004)} 
  \end{columns}

  \begin{itemize}
  \item<7->
  `Remotely sensed' by:
  email activity, 
  instant messaging, 
  phone logs \uncover<8->{\alert{(*cough*)}}.
  \end{itemize}

\end{frame}

\begin{frame}
  \frametitle{Examples}

    \includegraphics[width=\textwidth]{bearman_sex_network.jpg}\\

\end{frame}

\begin{frame}
  \frametitle{Examples}

  \begin{block}{Relational networks}
    \begin{itemize}
    \item<1-> Consumer purchases \\
      \visible<2->{(Wal-Mart: $\approx 1 \ \mbox{petabyte} \ = 10^{15} \ \mbox{bytes}$)}
    \item<3-> Thesauri: Networks of words generated by meanings
    \item<4-> Knowledge/Databases/Ideas
    \item<5-> Metadata---Tagging:
      \wordwikilink{http://del.icio.us}{del.icio.us},
      \wordwikilink{http://www.flickr.com}{flickr}
    \end{itemize}
  \end{block}
  \begin{overprint}
    \onslide<5-| handout:1| trans:1>
    \includegraphics{delicious.pdf}
  \end{overprint}

\end{frame}

\begin{frame}
  \frametitle{Clickworthy Science:}

    \includegraphics[height=0.75\textheight]{bollen2009a_fig5.pdf}\\
    {\small Bollen et al.\cite{bollen2009a}}

\end{frame}

%%%%%%%%%%%%%%%%%%%%%%%%%%%%%%%%%%%%%
% properties
%%%%%%%%%%%%%%%%%%%%%%%%%%%%%%%%%%%%%

\section{Properties of Complex Networks}

\begin{frame}
  \frametitle{}

  \begin{block}<1->{A notable feature of large-scale networks:}
    \begin{itemize}
    \item<2->
      Graphical renderings are often just a big mess.
      \begin{overprint}
        \onslide<1-2 | handout: 0 | trans:0>
        \onslide<3->
        \begin{columns}
          \column{0.4\textwidth}
          \includegraphics[height=\textwidth]{nw_purerandom_graphviz01_10}
          \column{0.6\textwidth}
          \begin{itemize}
          \item[] 
            $\Leftarrow$ Typical hairball
          \item 
            number of nodes $N$ = 500
          \item 
            number of edges $m$ = 1000
          \item 
            average degree $\tavg{k}$ = 4
          \end{itemize}
        \end{columns}
      \end{overprint}
    \item<4->
      And even when renderings somehow look good:\\
      \visible<5->{
      \alertb{``That is a very graphic analogy which aids 
      understanding wonderfully while being,
      strictly speaking, wrong in every possible way''} \\
      {\small
      said Ponder [Stibbons]
      ---\textit{Making Money}, T. Pratchett.
      }
      }
    \item<6-> 
      We need to extract \alert{digestible, meaningful aspects}.
    \end{itemize}
  \end{block}

\end{frame}

\begin{frame}
  \frametitle{Properties}

  \begin{block}{Some key features of real complex networks:}
  \begin{columns}
    \column{0.1\textwidth}
    \column{0.4\textwidth}
    \begin{itemize}
    \item Degree distribution
    \item Assortativity
    \item Homophily
    \item Clustering
    \item Motifs
    \item Modularity
    \end{itemize}
    \column{0.4\textwidth}
    \begin{itemize}
    \item Concurrency
    \item Hierarchical scaling
    \item Network distances
    \item Centrality
    \item Efficiency
    \item Robustness
    \end{itemize}
    \column{0.1\textwidth}
  \end{columns}
  \end{block}

  \begin{itemize}
  \item<1-> Coevolution of network \alertb{structure}
    \\ and \alertb{processes} on networks.
  \end{itemize}

\end{frame}

\begin{frame}
  \frametitle{Properties}

  \begin{block}<1->{1. Degree distribution $P_k$}
    \begin{itemize}
    \item<2->
      $P_k$ is the probability that a randomly selected
      node has degree $k$
    \item<3->
      \alertb{Big deal:} Form of $P_k$ key to network's behavior
    \item<4->
      \alert{ex 1:}
      \erdosrenyi\ random networks have a Poisson
      distribution:
      $$ P_k = e^{-\tavg{k}} \tavg{k}^k / k! $$
    \item<5->
      \alert{ex 2:}
      \alertb{``Scale-free'' networks:}
      $P_k \propto k^{-\gamma}$ $\Rightarrow$ `hubs'
    \item<5->
      We'll come back to this business soon...
    \end{itemize}
  \end{block}

\end{frame}

%% \begin{frame}
%%   \frametitle{Properties}
%% 
%%   \begin{block}<1->{1. Degree distribution $P_k$}
%%     \begin{itemize}
%%     \item<1->
%%       \alert{ex 2:}
%%       \alert{``Scale-free'' networks:}
%%       $P_k \propto k^{-\gamma}$ $\Rightarrow$ `hubs'
%%     \item<2->
%%       Link cost controls skew
%%     \item<3->
%%       Hubs may facilitate or impede contagion
%%     \end{itemize}
%%   \end{block}
%%  
%% \end{frame}


\begin{frame}
  \frametitle{Properties}

  \begin{block}{2. Assortativity/3. Homophily:}
    \begin{itemize}
    \item<1-> Social networks: \wordwikilink{http://en.wikipedia.org/wiki/Homophily}{Homophily} = birds of a feather
    \item<2-> e.g., degree is standard property for sorting:\\
      measure degree-degree correlations.
    \item<3->
      \alert{Assortative} network:\cite{newman2002a} 
      similar degree nodes connecting to each other.\\
      \begin{itemize}
      \item<5->
      \visible<5->{\textit{Often \alertb{social}: company directors, coauthors, actors.}}
      \end{itemize}
    \item<4->
      \alert{Disassortative} network: high degree nodes connecting to low degree nodes.\\
      \begin{itemize}
      \item<6-> 
      \visible<6->{\textit{Often \alertb{techological} or \alertb{biological}: 
        Internet, protein interactions, neural networks, food webs.}}
      \end{itemize}
    \end{itemize}
  \end{block}

\end{frame}

\begin{frame}
  \frametitle{Properties}

  \begin{block}{4. Clustering:}
    \begin{itemize}
    \item<2-> Your friends tend to know each other.
    \item<3-> Two measures:
      $$ C_1 = \avg{\frac{\sum_{j_1 j_2 \in {\cal N}_i} a_{j_1 j_2}}{k_i(k_i-1)/2}}_{i} 
      \mbox{\ due to Watts \& Strogatz\cite{watts1998a}}
      $$  
      $$ C_2 = \frac{3 \times \rm\#triangles}{\rm \#triples} 
      \mbox{\ due to Newman\cite{newman2003a}}
      $$ 
    \item<4-> $C_1$ is the \alert{average fraction} 
      of 
      \alertb{pairs of neighbors} who are \alertb{connected}.
    \item<5->
      Interpret $C_2$ as probability two of a node's friends
      know each other.
    \end{itemize}
  \end{block}

\end{frame}

%% \begin{frame}
%%   \frametitle{Properties---First clustering measure:}
%% 
%%   \begin{itemize}
%%   \item<1-> $C_1$ is the \alert{average fraction} 
%%     of 
%%     \alertb{pairs of neighbors} who are \alertb{connected}.
%%   \item<2-> Fraction of pairs of neighbors who are connected is
%%     $$ \frac{\sum_{j_1 j_2 \in {\cal N}_i} a_{j_1 j_2}}{k_i(k_i-1)/2} $$
%%     where
%%     $k_i$ is node $i$'s degree, and 
%%     ${\cal N}_i$ is the set of $i$'s neighbors.
%%   \item<3->
%%     Averaging over all nodes, we have
%%     $$ C_1 = \frac{1}{n}{\sum_{i=1}^{n}\frac{\sum_{j_1 j_2 \in {\cal N}_i} a_{j_1 j_2}}{k_i(k_i-1)/2}} 
%%     \visible<5->{ = \avg{\frac{\sum_{j_1 j_2 \in {\cal N}_i} a_{j_1 j_2}}{k_i(k_i-1)/2}}_{i} }$$  
%%   \item<6-> For sparse networks, $C_1$ tends to discount
%%     highly connected nodes.
%%   \end{itemize}
%% 
%% \end{frame}
%% 
%% 
%% \begin{frame}
%%   \frametitle{Properties---Triples and triangles:}
%% 
%%   \begin{itemize}
%%   \item<1->
%%     \alertb{Defn:} Nodes $i_1$, $i_2$, and $i_3$ form a \alert{triple}
%%     around $i_1$ if $i_1$ is connected to $i_2$ and $i_3$.
%%   \item<2->
%%     \alertb{Defn:} Nodes $i_1$, $i_2$, and $i_3$ form a \alert{triangle}
%%     if each pair of nodes is connected
%%   \item<3-> 
%%    The definition
%%     $$ C_2 = \frac{3 \times \rm\#triangles}{\rm \#triples} $$ 
%%     measures the fraction of \alertb{closed triples}.
%%   \item<4-> 
%%     The \alert{`3'} appears because for each triangle, \\
%%     we have 3 closed triples.
%%   \item<5->
%%     Interpret $C_2$ as probability two of a node's friends
%%     know each other.
%%   \item<6-> Social Network Analysis (SNA): fraction of
%%     \alertb{transitive triples}.
%%   \item<7-> In general, \alert{$C_1 \ne C_2$}.
%%   \end{itemize}
%% 
%% \end{frame}

\begin{frame}
  \frametitle{Properties}

  \begin{block}{5. Motifs:}
  \begin{itemize}
  \item<1-> 
    Small, recurring functional subnetworks 
  \item<2->
    e.g., Feed Forward Loop:
    \begin{overprint}
      \onslide<2-| handout:1| trans:1>
      \begin{center}
        \includegraphics[width=0.45\textwidth]{feedforwardloop}%
      \end{center}
    \end{overprint}
    Shen-Orr, Uri Alon, \etal\cite{shen-orr2002a}
  %% , Wiggins \etal
  \end{itemize}
  \end{block}

\end{frame}

\begin{frame}
  \frametitle{Properties}

  \begin{block}{6. modularity:}
    \begin{center}
      \begin{tabular}{c}
        \includegraphics[height=0.6\textheight]{ncaa_annotated}\\
        Clauset \etal, 2006\cite{clauset2006a}: NCAA football
      \end{tabular}
    \end{center}
  \end{block}

\end{frame}

\begin{frame}
  \frametitle{Properties}

  \begin{block}{7. Concurrency:}
    \begin{itemize}
    \item<1-> 
      Transmission of a contagious element
      only occurs during contact\cite{kretzschmar1996a}
    \item<2-> 
      Rather obvious but easily missed in a simple model
    \item<3-> 
      Dynamic property---static networks are not enough
    \item<4-> 
      Knowledge of previous contacts crucial
    \item<5-> 
      \alert{Beware} cumulated network data!
    \end{itemize}
  \end{block}

\end{frame}

\begin{frame}
  \frametitle{Properties}

  \begin{block}{8. Horton-Strahler stream ordering:}
    \begin{itemize}
    \item<1-> Metrics for branching networks:
      \begin{itemize}
      \item<2-> 
        Method for ordering streams hierarchically
      \item<3->
        Reveals fractal nature of natural branching networks
      \item<4->
        Hierarchy is not pure but mixed (Tokunaga).\cite{tokunaga1966a,dodds1999a}
      \item<5->
        Major examples: rivers and blood networks.
        %%      Number: $R_n = N_{\omega}/N_{\omega+1}$ 
        %%      \item<4->
        %%      Segment length: $R_l = \tavg{l_{\omega+1}}/\tavg{l_{\omega}}$ 
        %%      \item<5->
        %%      Area/Volume: $R_a = \tavg{a_{\omega+1}}/\tavg{a_{\omega}}$ 
      \end{itemize}
    \end{itemize}
  \end{block}
  \begin{overprint}
    \onslide<1-| handout:1| trans:1>
    \begin{center}
      \includegraphics[height=0.4\textheight]{network1a}%
      \includegraphics[height=0.4\textheight]{network2b} 
      \includegraphics[height=0.3\textheight]{network3c}%
    \end{center}
  \end{overprint}
  \begin{itemize}
  \item<6-> \alertb{Beautifully described} but \alert{poorly explained}.
  \end{itemize}

\end{frame}

\begin{frame}
  \frametitle{Properties}

  \begin{block}<1->{9. Network distances:}
    \begin{block}<2->{\alert{(a) shortest path length $d_{ij}$:}}
      \begin{itemize}
      \item <3->Fewest number of steps between nodes $i$ and $j$.      
      \item <4->(Also called the chemical distance between $i$ and $j$.)
      \end{itemize}
    \end{block}
    \begin{block}<5->{\alert{(b) average path length $\tavg{d_{ij}}$:}}
      \begin{itemize}
      \item <6-> Average shortest path length in whole network.
      \item <7-> 
        Good algorithms exist for calculation.
      \item <8->
        Weighted links can be accommodated.
      \end{itemize}
    \end{block}

  \end{block}

\end{frame}

\begin{frame}
  \frametitle{Properties}

  \begin{block}{9. Network distances:}
    \begin{block}<1->{\alert{(c) Network diameter $d_{\rm max}$:}}
      \begin{itemize}
      \item<2->
        Maximum shortest path length in network.
      \end{itemize}
    \end{block}
    \begin{block}<3->{\alert{(d) Closeness $d_{\rm cl} = [\sum_{ij} d_{ij}^{\ -1} / \binom{n}{2}]^{-1}$:}}
      \begin{itemize}
      \item<4-> 
        Average `distance' between any two nodes.
      \item<5->
        Closeness handles disconnected networks ($d_{ij}=\infty$)
      \item<6->
        $d_{\rm cl} = \infty$ only when all nodes are isolated.
%%      \item<7->
%%        Closeness perhaps compresses too much into one number.
      \end{itemize}
    \end{block}
  \end{block}

\end{frame}

\begin{frame}
  \frametitle{Properties}

  \begin{block}{10. Centrality:}
    \begin{itemize}
    \item<2-> Many such measures of a node's `importance.'  
    \item<3-> \alert{ex 1:} Degree centrality: $k_i$.
    \item<4-> \alert{ex 2:} Node $i$'s betweenness \\
      = fraction of shortest paths that pass through $i$.
    \item<5-> \alert{ex 3:} Edge $\ell$'s betweenness \\
      = fraction of shortest paths that travel along $\ell$.
    \item<6-> \alert{ex 4:} Recursive centrality: Hubs and Authorities
      (Jon Kleinberg\cite{kleinberg1998a})
    \end{itemize}
    
  \end{block}

\end{frame}




\section{Nutshell}

\begin{frame}[label=]
  \frametitle{Nutshell:}

  \begin{block}{Overview Key Points:}
    \begin{itemize}
    \item<1->
      The field of complex networks came into
      existence in the late 1990s.
    \item<2->
      Explosion of papers and interest since 1998/99.
    \item<3->
      Hardened up much thinking about complex systems.
    \item<4->
      Specific focus on networks that are 
      \alert{large-scale}, 
      \alertb{sparse}, 
      \alert{natural} or \alert{man-made}, 
      \alertb{evolving} and \alertb{dynamic}, 
      and 
      (crucially) \alert{measurable}.
    \item<5->
      Three main (blurred) categories: 
      \begin{enumerate}
      \item 
      \alert{Physical} (e.g., river networks),
      \item 
      \alert{Interactional} (e.g., social networks),
      \item 
      \alert{Abstract} (e.g., thesauri).
      \end{enumerate}
    \end{itemize}
    
  \end{block}

\end{frame}

\begin{frame}[label=]
  \frametitle{Nutshell:}

  \begin{block}{Overview Key Points (cont.):}
    \begin{itemize}
    \item<1->
      Obvious connections with the vast
      extant field of graph theory.
    \item<2->
      But focus on dynamics is more of a physics/stat-mech/comp-sci
      flavor.
    \item<3->
      Two main areas of focus:
      \begin{enumerate}
      \item 
        \alertb{Description:} Characterizing very large networks
      \item
        \alertb{Explanation:} Micro story $\Rightarrow$ Macro features
      \end{enumerate}
    \item<4->
      Some essential structural aspects are understood: degree distribution, clustering,
      assortativity, group structure, overall structure,...
    \item<5->
      Still much work to be done, especially with respect to dynamics...
    \end{itemize}
    
  \end{block}

\end{frame}
