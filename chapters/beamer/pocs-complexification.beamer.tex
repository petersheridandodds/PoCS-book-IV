
\begin{frame}
  \showtarotcards{0.20}{
    overview,
    golden-age-of-reductionism,
    manifesto,
    scaling,
    power-law-size-distributions,
    variable-transformation,
    random-walks,
    rich-get-richer,
    efficient-language,
    law-of-first-digits,
    surprise-of-being-robust-yet-fragile,
    misreading-of-the-book-of-sand,
    tales-of-tails,
    data-poor,
    pouring-data,
    emergence-of-structure,
    emergence-of-destruction,
    emergence-of-thinking,
    emergence-of-stories,
    mechanics-statistical,
    thresholds-of-percolation,
    complex-networks,
    web-elements,
    random-networks,
    strangeness-of-friends,
    small-world-networks,
    theory-six-degrees,
    scale-free-networks,
    confusion-of-contagions,
    disease-of-random-mixing,
    unending-pandemics,
    thresholds-of-the-mind,
    social-wild,
    social-contagion,
    fame,
    contagious-stories,
    paths-of-universality,
%    dependent-paths,
%    end,
}                                                
\end{frame}                                      

\section{Universality}

\begin{frame}
  \small

  \frametitle{Limits to what's possible:}

  \begin{block}{\wordwikilink{http://en.wikipedia.org/wiki/Universality\_(dynamical\_systems)}{Universality}:}
    \begin{itemize}
    \item<+->
      The property that the macroscopic aspects of a system do not depend
      sensitively on the system's details.
    \item<+->
      Key figure:
      \wordwikilink{http://en.wikipedia.org/wiki/Leo\_Kadanoff}{Leo Kadanoff}
    \item<+->
      Kadanoff's retrospective: ``Innovations in Statistics Physics''~\cite{kadanoff2014a}
    \end{itemize}
  \end{block}

  \begin{block}<+->{Examples:}
    \begin{itemize}
    \item<+-> 
      The Central Limit Theorem:
      $$
      P(x; \mu, \sigma) \dee{x} = 
      \frac{1}{\sqrt{2\pi}\sigma}
      e^{-(x-\mu)^2/2\sigma^2} \dee{x}.
      $$
    \item<+-> 
      Navier Stokes equation for fluids.
    \item<+-> 
      Nature of phase transitions in statistical mechanics.
    \end{itemize}
  \end{block}

\end{frame}

\begin{frame}
  \frametitle{Universality}

  \begin{block}{}
    \begin{itemize}
    \item<+-> 
      Sometimes \alert{details don't matter too much}.
    \item<+-> 
      \alertb{Many-to-one mapping} from micro to macro
    \item<+-> 
      Suggests not all possible behaviors are available 
      \qquad at higher levels of complexity.
    \item<+-> 
      Universality means some things are fated.
    \end{itemize}
  \end{block}

  \begin{block}<4->{Large questions:}
    \begin{itemize}
    \item<5->
      How universal is universality?
    \item<6->
      What are the possible long-time states (attractors) for a universe?
    \end{itemize}
  \end{block}
    

\end{frame}


\begin{frame}
  \frametitle{Fluid mechanics}

  \begin{block}{}
  \begin{itemize}
  \item<+-> 
    Fluid mechanics = One of the great successes of 
    understanding complex systems.
  \item<+-> 
    Navier-Stokes equations: micro-macro system evolution.
  \item<+-> 
    The big three: Experiment + Theory + Simulations.
  \item<+-> 
    Works for many very different `fluids':
    \begin{itemize}
    \item 
      the atmosphere,
    \item 
      oceans,
    \item 
      blood,
    \item 
      galaxies,
    \item 
      the earth's mantle \ldots
    \item<+->
      \alert{and ball bearings on lattices \ldots ?}
    \end{itemize}
  \end{itemize}
  \end{block}

\end{frame}


\begin{frame}
  \frametitle{Lattice gas models}

  \begin{block}{Collision rules in 2-d on a hexagonal lattice:}
  \begin{center}
    \includegraphics[width=0.7\textwidth]{lattice_gas_rules_example.jpg}
  \end{center}
  \end{block}

  \begin{block}{}
  \begin{itemize}
  \item<2->
    Lattice matters \ldots
  \item<3-> 
    No `good' lattice in 3-d.
  \item<4->
    Upshot: play with `particles' of a system to obtain new or specific macro behaviours.
  \end{itemize}
  \end{block}


\end{frame}

\begin{frame}
  \frametitle{Hexagons---\wordwikilink{http://en.wikipedia.org/wiki/Honeycomb}{Honeycomb:}}

  \begin{center}
    \includegraphics[height=0.65\textheight]{Honey_comb.jpg}
  \end{center}

  \begin{itemize}
  \item<1->
    Orchestrated?  Or an accident of bees working hard?
  \item<2-> 
    See ``On Growth and Form'' by 
    \wordwikilink{http://en.wikipedia.org/wiki/D'Arcy\_Wentworth\_Thompson}{D'Arcy Wentworth Thompson}.\cite{thompson1952a,thompson1961a}
  \end{itemize}
  

\end{frame}


\begin{frame}
  \frametitle{Hexagons---\wordwikilink{http://en.wikipedia.org/wiki/Giant's\_Causeway}{Giant's Causeway:}}

  \includegraphics[width=\textwidth]{giantscauseway_mist_medium.jpg}

  {\tiny \url{http://newdesktopwallpapers.info}}

%% from http://newdesktopwallpapers.info/Ireland%20Wallpapers/slides/Giant's%20Causeway,%20County%20Antrim,%20Ireland.html

\end{frame}


\begin{frame}

  \frametitle{Hexagons---\wordwikilink{http://en.wikipedia.org/wiki/Giant's\_Causeway}{Giant's Causeway:}}

  \includegraphics[width=\textwidth]{giantscauseway_hex_medium.jpg}

  {\tiny \url{http://www.physics.utoronto.ca/}}

%% from http://www.physics.utoronto.ca/news_repository/u-of-t-scientists-solve-mystery-of-giants-causeway-with-kitchen-materials/image/image_view_fullscreen

\end{frame}

\begin{frame}
  \frametitle{Saturn has a hexagon:}

  \begin{center}
    \includegraphics[width=0.7\textwidth]{Saturn_N_polar_hexagon_W00077335.jpg}
  %% source: http://en.wikipedia.org/wiki/Saturn's_hexagon
  \end{center}

  \begin{itemize}
  \item 
    \wordwikilink{http://en.wikipedia.org/wiki/Saturn's\_hexagon}{One side is longer than Earth's diameter}
  \end{itemize}

\end{frame}

\begin{frame}

  \frametitle{Hexagons run amok:}

  \begin{columns}
    \column{0.3\textwidth}
    \includegraphics[width=\textwidth]{340px-Graphen.jpg}\\
    \bigskip
    \includegraphics[width=\textwidth]{screenie-bluesbrothers-fence.jpg}
    \column{0.7\textwidth}
    \begin{block}{}
      \begin{itemize}
      \item 
        \wordwikilink{http://en.wikipedia.org/wiki/Graphene}{Graphene}:
        single layer of carbon molecules
        in a perfect hexagonal lattice (super strong).
      \item 
        \wordwikilink{http://en.wikipedia.org/wiki/Chicken\_wire}{Chicken wire} \ldots
      \end{itemize}
    \end{block}
  \end{columns}

\end{frame}

\begin{frame}

  \begin{block}{Triumph of the Hexagon}
    \youtubevideo{xyY0ymMYXPo}{}{}
    
    From the remarkable 
    \wordwikilink{http://hexnet.org}{Hexnet.org}, 
    the Global Hexagonal Awareness Resource Center.
  \end{block}
  
\end{frame}

\begin{frame}
  \showtarotcards{0.20}{
    overview,
    golden-age-of-reductionism,
    manifesto,
    scaling,
    power-law-size-distributions,
    variable-transformation,
    random-walks,
    rich-get-richer,
    efficient-language,
    law-of-first-digits,
    surprise-of-being-robust-yet-fragile,
    misreading-of-the-book-of-sand,
    tales-of-tails,
    data-poor,
    pouring-data,
    emergence-of-structure,
    emergence-of-destruction,
    emergence-of-thinking,
    emergence-of-stories,
    mechanics-statistical,
    thresholds-of-percolation,
    complex-networks,
    web-elements,
    random-networks,
    strangeness-of-friends,
    small-world-networks,
    theory-six-degrees,
    scale-free-networks,
    confusion-of-contagions,
    disease-of-random-mixing,
    unending-pandemics,
    thresholds-of-the-mind,
    social-wild,
    social-contagion,
    fame,
    contagious-stories,
    paths-of-universality,
    dependent-paths,
%    end,
}                                                
\end{frame}                                      

\section{Symmetry\ Breaking}

\begin{frame}
  \frametitle{Symmetry Breaking}

  \begin{block}{}
    \displaypaper{anderson1972a}{2}
    
    \bigskip
    
    \begin{columns}
      \column{0.02\textwidth}
      \column{0.33\textwidth}
      \includegraphics[width=\textwidth]{501px-Andersonphoto.jpg}
      \column{0.65\textwidth}
      \begin{itemize}
      \item<2-> 
        \wordwikilink{http://en.wikipedia.org/wiki/Philip_Warren_Anderson}{Anderson}
        argues against idea that the only real scientists are
        those working on the fundamental laws.
      \item<3->
        Symmetry breaking $\rightarrow$ different laws/rules at
        different scales \ldots
      \end{itemize}
    \end{columns}

    \bigskip

    \visible<4->{2006 study: \wordwikilink{http://physicsworld.com/cws/article/news/25623}{``most creative physicist in the world''}}
  \end{block}

\end{frame}

\begin{frame}
  \frametitle{Symmetry Breaking}

  \begin{block}{\alertg{``Elementary entities of science X obey the laws of science Y''}}
    \begin{columns}[t] 
      \begin{column}{0.45\textwidth} 
        \begin{itemize}
        \item X 
        \item solid state or many-body physics 
        \item chemistry \\ \mbox{}
        \item molecular biology 
        \item cell biology
        \item[$\vdots$]
        \item psychology
        \item social sciences
        \end{itemize}
      \end{column} 
      \begin{column}{0.45\textwidth} 
        \begin{itemize}
        \item Y
        \item elementary particle physics 
        \item solid state many-body physics
        \item chemistry 
        \item molecular biology 
        \item[$\vdots$]
        \item physiology 
        \item psychology
        \end{itemize}
      \end{column} 
    \end{columns} 
  \end{block}

\end{frame}

\begin{frame}
  \frametitle{Symmetry Breaking}

  \begin{block}{Anderson:}
    \begin{itemize}
    \item<+->
      \ [the more we know about] ``fundamental laws, the less
      relevance they seem to have to the very real problems
      of the rest of science.''
    \item<+->
      \alertb{Scale} and \alertb{complexity}
      thwart the constructionist hypothesis.
    \item<+->
      Accidents of history 
      and \wordwikilink{http://en.wikipedia.org/wiki/Path_dependence}{path dependence}
      matter.
    \end{itemize}
  \end{block}

\end{frame}

\begin{frame}
  \frametitle{Symmetry Breaking}

  \begin{block}{}
    \displayamazonbook{sornette2006a}
    \begin{itemize}
    \item<1-> 
      Page 291--292 of Sornette\cite{sornette2003a}:\\
      Renormalization $\equiv$ Anderson's hierarchy.
    \item<2->
      But Anderson's hierarchy is not a simple one: the rules change.
    \item<3->
      Crucial dichotomy between evolving systems
      following stochastic paths that lead
      to \\
      (a) \alertb{inevitable} 
      \alertb{or} 
      (b) \alertb{particular}
      destinations (states).
    \end{itemize}
  \end{block}


\end{frame}


%% more ????
\begin{frame}
  \frametitle{More is different:}

  \includegraphics[width=\textwidth]{xkcd435-complexity.png}\\
  \wordwikilink{http://xkcd.com/435/}{http://xkcd.com/435/}

\end{frame}

\section{The\ Big\ Theory}

\begin{frame}

  \frametitle{A real science of complexity:}

  \begin{block}<1->{A real theory of \sout{everything} anything:}
    \begin{enumerate}
    \item<2->
      Is not just about the ridiculously small stuff \ldots
    \item<3->
      It's about the increase of complexity
    \end{enumerate}
  \end{block}

  \medskip

  \begin{block}{}
  \uncover<4->{
    \begin{columns}
      \column{0.1\textwidth}
      \column{0.4\textwidth}
      Symmetry breaking/
      Accidents of history
      \column{0.05\textwidth}
      vs.
      \column{0.05\textwidth}
      \column{0.4\textwidth}
      Universality
    \end{columns}
  }
  \end{block}

  \medskip

  \begin{block}{}
  \begin{itemize}
  \item<5-> 
    Second law of thermodynamics: \alertb{we're toast in the long run}.
  \item<6-> 
    So how likely is the local complexification of structure we enjoy?
  \item<7-> 
    How likely are the Big Transitions?
  \end{itemize}
  \end{block}

\end{frame}

\begin{frame}
  \frametitle{Why complexify?}

  \begin{block}{}
    \displaypaper{arthur1993b}{1}
    \begin{itemize}
    \item<+-> 
      Argues that evolution toward increased performance brings a
      ratcheting cycle
      of complexification and simplification.
    \item<+-> 
      Jet engine replaced the complex piston engine and then itself
      became more complex.
    \item<+-> 
      Complexification $\equiv$ evolution of algorithms?
    \item<+->  
      Differential equations and stories $\subset$ Algorithms.
    \item<+->  
      Life is a loaded word: The Search for Extraterrestrial Algorithms (SETA)?
    \end{itemize}
  \end{block}
  
\end{frame}

\begin{frame}
  \frametitle{Why complexify?}

  \begin{block}{Driving complexity's trajectory:}
    \begin{itemize}
    \item 
      Big Bang
    \item 
      Randomness leads to replicating structures;
    \item 
      Biological evolution;
    \item 
      Sociocultural evolution;
    \item 
      Technological evolution;
    \item 
      Sociotechnological evolution.
    \end{itemize}
  \end{block}

\end{frame}


\begin{frame}
  \frametitle{Complexification---the Big Transitions:}

  \begin{block}{}
  \begin{columns}
    \column{0.32\textwidth}
    \begin{itemize}
    \item<+->
      Big Bang.
    \item<+-> 
      Big Randomness.
    \item<+-> 
      %% galactic networks
      %% stars and planets
      Big Structure.
    \item<+-> 
      Big Replicate.
    \item<+-> 
      Big Life.
    \item<+-> 
      Big Evolve.
    \end{itemize}
    \column{0.29\textwidth}
    \begin{itemize}
    \item<+-> 
      Big Word.
    \item<+-> 
      Big Story.
    \item<+-> 
      Big Number.
    \item<+-> 
      Big Farm.
    \item<+-> 
      Big God.
    \item<+-> 
      Big Make.
    \item<+-> 
      Big City.
    \item<+-> 
      Big Culture.
    \end{itemize}
    \column{0.41\textwidth}
    \begin{itemize}
    \item<+->
      Big Science.
    \item<+-> 
      Big Data.
    \item<+-> 
      Big Information.
    \item<+-> 
      Big Algorithm.
    \item<+-> 
      Big Connection.
    \item<+-> 
      Big Social.
    \item<+-> 
      Big Awareness.
    \item<+-> 
      Big Spread.
    \item<+->
      Big \ldots ?
    \end{itemize}
  \end{columns}
  \end{block}

\end{frame}


\begin{frame}
%% three levels of complexity
%%  \frametitle{}

%%  \includegraphics[width=\textwidth]{2011-07-15complexity-framing-cropped.jpg}
  \includegraphics[width=\textwidth]{2011-07-15complexity-framing-cropped-sharp-tp-3.pdf}
%%  \includegraphics[width=\textwidth]{2011-01-18complexity-framing-sketch-tp-cropped.pdf}

\end{frame}

\begin{frame}
  \small
 
  \begin{block}<1->{}
    \begin{columns}
      \column{0.03\textwidth}
      \column{0.2\textwidth}
      \includegraphics[width=\textwidth]{xkcd-904-sports.png}\\
      {\tiny
      \wordwikilink{http://xkcd.com/904/}{http://xkcd.com/904/}}
      \includegraphics[width=\textwidth]{2014-11-15narrative-hierarchy-sketches-stories_001_base_1200px-tp-5.png}\\
      \includegraphics[width=\textwidth]{2014-11-15narrative-hierarchy-sketches-stories_001_broken-story_1200px-tp-5.png}\\
      \includegraphics[width=\textwidth]{2014-11-15narrative-hierarchy-sketches-stories_001_broken_1200px-tp-5.png}\\
      \column{0.8\textwidth}
      \begin{itemize}
%%      \item<1-> 
%%        Sociotechnical algorithms for measuring/predicting decisions, contagion, demographics, weather, \ldots
      \item<1->
        \wordwikilink{http://nautil.us/issue/5/fame/homo-narrativus-and-the-trouble-with-fame}{Homo narrativus}---we run on stories.
      \item<1->
        Extraction of metaphors, frames, narratives, and stories from large-scale text.
      \item<1-> 
        \wordwikilink{http://www.uvm.edu/~pdodds/fama/2015/06/04/the-narrative-hierarchystories-and-storytelling-on-all-scales/}{The narrative hierarchy: Scalability of stories}.
      \item<1-> 
        Adjacent narratives, mistruths, and conspiracy theories.
      \item<1-> 
        The taxonomy of human stories.
      \end{itemize}
      \centering
      \includegraphics[width=0.2\textwidth]{2015-09-28adjacent-stories001-tp-5.png}
      \includegraphics[width=0.2\textwidth]{2015-09-28adjacent-stories003-tp-5.png}
      \includegraphics[width=0.2\textwidth]{2015-09-28adjacent-stories004-tp-5.png}
      \includegraphics[width=0.2\textwidth]{2015-09-28adjacent-stories002-tp-5.png}\\
      \includegraphics[width=0.2\textwidth]{2015-09-28adjacent-stories006-tp-5.png}
      \includegraphics[width=0.2\textwidth]{2015-09-28adjacent-stories008-tp-5.png}
      \includegraphics[width=0.2\textwidth]{2015-09-28adjacent-stories010-tp-5.png}
      \includegraphics[width=0.2\textwidth]{2015-09-28adjacent-stories011-tp-5.png}
    \end{columns}
  \end{block}

\end{frame}


\begin{frame}
  \small
  %% \frametitle{\wordwikilink{http://en.wikipedia.org/wiki/Terry\_Pratchett}{Pratchett} on stories:}
  \frametitle{\wordwikilink{http://en.wikipedia.org/wiki/Terry\_Pratchett}{(Sir Terry) Pratchett's}
    \wordwikilink{http://wiki.lspace.org/wiki/Narrativium}{Narrativium}:
    %% and 
    %% \wordwikilink{http://wiki.lspace.org/wiki/Narrative\_Causality}{Narrative Causality}:
  }

  \begin{columns}
    \column{0.3\textwidth}
    \includegraphics[width=\textwidth]{4463841109_f3dbc05754_pratchett.jpg}
    \column{0.7\textwidth}
    \begin{block}<+->{}
      \begin{itemize}
      \item<+-> 
        ``The most common element on the disc, although not
        included in the list of the standard five: earth, fire, air,
        water and surprise. It ensures that everything runs properly
        as a story.''
      \item<+->
        ``A little narrativium goes a long way: the simpler the story,
        the better you understand it. Storytelling is the opposite of
        reductionism: 26 letters and some rules of grammar are no story
        at all.''
      \end{itemize}
    \end{block}
  \end{columns}

  \begin{block}<+->{}
    \begin{itemize}
    \item
      ``Heroes only win when outnumbered, and things which have a
      one-in-a-million chance of succeeding often do so.''
    \end{itemize}
  \end{block}

\end{frame}

\begin{frame}
  \frametitle{\wordwikilink{http://exp.lore.com/post/40411963108/kurt-vonneguts-classic-lecture-on-the-shapes-of-stories}{Kurt Vonnegut on the shapes of stories:}}

  \includegraphics[height=0.9\textheight]{tumblr_mft5lpRiy01r2qa6go1_1280_1.jpg}

\end{frame}

\begin{frame}
  \frametitle{\wordwikilink{http://exp.lore.com/post/40411963108/kurt-vonneguts-classic-lecture-on-the-shapes-of-stories}{Kurt Vonnegut on the shapes of stories:}}

  \includegraphics[height=0.9\textheight]{tumblr_mft5lpRiy01r2qa6go1_1280_2.jpg}
  
\end{frame}

\insertvideo{oP3c1h8v2ZQ}{}{}{Kurt Vonnegut on the shapes of stories:}

\begin{frame}

  \begin{block}{
      \wordwikilink{http://hedonometer.org/books.html}{Online, interactive
        Emotional Shapes of Stories} for 10,000+ books:
    }
    \centering
    \includegraphics[width=\textwidth]{2014-09-15frankenstein.png}
  \end{block}
  
\end{frame}


%% \begin{frame}
%%   \frametitle{Ian Stewart and Jack Cohen:}
%% 
%%   \begin{block}{}
%%     Ian Stewart and Jack Cohen are the source of the coinage Pan
%%     narrans, of which they say We are not Homo sapiens, Wise Man. We are
%%     the third chimpanzee. What distinguishes us from the ordinary
%%     chimpanzee Pan troglodytes and the bonobo chimpanzee Pan paniscus, is
%%     something far more subtle than our enormous brain, three times as
%%     large as theirs in proportion to body weight. It is what that brain
%%     makes possible. And the most significant contribution that our large
%%     brain made to our approach to the universe was to endow us with the
%%     power of story. 
%%     \visible<2->{We are \alertb{Pan narrans}, the storytelling ape.}
%%   \end{block}
%%   
%% \end{frame}


\begin{frame}
  \showtarotcards{0.20}{
    overview,
    golden-age-of-reductionism,
    manifesto,
    scaling,
    power-law-size-distributions,
    variable-transformation,
    random-walks,
    rich-get-richer,
    efficient-language,
    law-of-first-digits,
    surprise-of-being-robust-yet-fragile,
    misreading-of-the-book-of-sand,
    tales-of-tails,
    data-poor,
    pouring-data,
    emergence-of-structure,
    emergence-of-destruction,
    emergence-of-thinking,
    emergence-of-stories,
    mechanics-statistical,
    thresholds-of-percolation,
    complex-networks,
    web-elements,
    random-networks,
    strangeness-of-friends,
    small-world-networks,
    theory-six-degrees,
    scale-free-networks,
    confusion-of-contagions,
    disease-of-random-mixing,
    unending-pandemics,
    thresholds-of-the-mind,
    social-wild,
    social-contagion,
    fame,
    contagious-stories,
    paths-of-universality,
    dependent-paths,
    end,
}                                                
\end{frame}                                      


\section{Final\ words}

\begin{frame}
  \small

  \frametitle{The absolute basics:}

  \begin{block}<+->{Modern basic science in three steps:}
    \begin{enumerate}
    \item<+->
      Find interesting/meaningful/important phenomena,
      optionally involving spectacular amounts of data.
    \item<+->
      Describe what you see.
    \item<+->
      Explain it.
    \end{enumerate}
  \end{block}

  \begin{block}<+->{}
    \alertg{Unlocks our (limited) ability to:}
    Create, predict, and control.
    \end{block}

  \begin{block}<+->{}
    And be good people: \alertg{Share.}
  \end{block}

  \begin{block}<+->{}
    \alertg{Beware your assumptions:}
    Don't use tools/models because they're there,
    or because everyone else does \ldots
  \end{block}

\end{frame}


\section{For\ your\ consideration}

\begin{frame}
  \small
  \frametitle{This is a thing that could be next:}

  \begin{block}{}
    \begin{columns}
      \column{0.02\textwidth}
      \column{0.28\textwidth}
      CoNKs: The PoCS\\ strikes back:
      \smallskip
      \includegraphics[width=\textwidth]{networksvox-icon.png}\\
      \smallskip
      CSYS/MATH 303: \\
      \wordwikilink{http://www.uvm.edu/~pdodds/teaching/courses/303/}{Complex
        Networks}\\
      \wordwikilink{https://twitter.com/@networksvox)}{@networksvox}
      \column{0.70\textwidth}
      \begin{itemize}
      \item<2-> 
        Branching networks (rivers, cardiovascular systems).
      \item<2-> 
        Optimal (re)distribution networks (hospitals, coffee shops, airlines, post, Internet).
      \item<2-> 
        Structure detection for complex systems.
      \item<2-> 
        Moar Contagion.
      \item<2-> 
        Random networks-arama.
      \item<2-> 
        Distributed Search.
      \item<2-> 
        Organizational networks.
      \item<2-> 
        Deeper investigations of scale-free networks.
      \item<3-> 
        and more \ldots
      \end{itemize}
    \end{columns}
  \end{block}

\end{frame}

\begin{comment}
  
\section{Flotsam}

\begin{frame}

  \frametitle{Homo narrativus---What's the Story?:}

  \begin{columns}
    \column{0.5\textwidth}
    \includegraphics[width=\textwidth]{xkcd-904-sports.png}\\
    \wordwikilink{http://xkcd.com/904/}{http://xkcd.com/904/}
    \column{0.5\textwidth}
    \begin{itemize}
    \item<+->
      Mechanisms = 
      Evolution equations, 
      algorithms,
      stories, \ldots
    \item<+->
      Rollover zing:
      ``Also, all financial analysis.
      And, more directly, D\&D.''
    \end{itemize}
  \end{columns}

\end{frame}

\end{comment}
