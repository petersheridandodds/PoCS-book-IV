%% Intro:

%% I'm going to talk about two things:
%% what i'll call the paradox of fame, of superstars,
%% and the mystery of social contagion, of superspreading,
%% and i'll give you the secrets to these things, all for free.

%% Closing

%% Everything I say will be likely completely obvious or completely wrong.



%% people need to own their spreading
%% thresholds


%%%%%%%%%%%%%%

%% add a video of people saying spread
%% i told all my friends
%% replace facebook emblem

%% big scale, big fail?
%% 
%% One picture/figure per slide.
%% One tag line.

%% Everything I say will be likely completely obvious or completely wrong.

%%%%%%

\begin{frame}
  \frametitle{The most famous painting in the world:}

  \begin{center}
    \includegraphics[height=0.5\textheight]{monalisa.pdf}%
  \end{center}

%%   \begin{itemize}
%%   \item<1-> 
%%     Painted $\sim$ 1500.
%%   \item<2->
%%     Fame only took off in the 1900s.
%%   \end{itemize}

  %% that's a long time to sort out our collective view
  %% of just how great this painting is
  %% many, many other examples of objects that are incredibly
  %% more famous than those in second place

\end{frame}


\begin{frame}
  \frametitle{The dismal predictive powers of editors \ldots...}

  \begin{center}
    \includegraphics[width=0.9\textwidth]{lego-harry-potter-annees-1-a-4-playstation-portable-psp-007-tp-1.pdf}
%%    \includegraphics[width=0.9\textwidth]{hp1_posters.jpg}
    %%    \includegraphics[height=0.5\textheight]{Dumbledore_and_Elder_Wand.jpg}
  \end{center}

  \alertg{Twelve \ldots}
  %% saved by an eight year old.}
  %% granddaughter of the head of bloombsbury

  %% i've had to pretend to be a house elf
  %% who sometimes has conversations with golum

  %% come back to this at the end

  %%  \includegraphics[width=.05\textwidth]{wikipedia-tp-3.pdf}

\end{frame}

\begin{frame}
  \frametitle{The completely unpredicted fall\\ of Eastern Europe:}

  \includegraphics[height=0.7\textheight]{berlinwall.jpg}

  {
    \small
    Timur Kuran:\cite{kuran1991a} 
    ``Now Out of Never: The Element of Surprise in the East European Revolution of 1989''
  }

\end{frame}



\begin{frame}
  \frametitle{We understand bushfire stories:}

  \begin{overprint}
    \onslide<1-3 | handout=0 | trans=0>
    \includegraphics[height=0.45\textheight]{r488645_6299881_australian_bushfire_abc.jpg}
    \onslide<4- |  handout=1 | trans=1>
    \includegraphics[height=0.45\textheight]{r488645_6299881_australian_bushfire_abc.jpg}
    \includegraphics[height=0.45\textheight]{Thetippingpoint.jpg}
  \end{overprint}

  \begin{enumerate}
  \item<1-> 
    Sparks start fires.
  \item<2->
    System properties control a fire's spread.
   \item<3->[]
   \item<3-> 
     But we make two mistakes about \alertb{Social Fires}...
   \end{enumerate}

  %% i come from a very large island that's often partly on fire and partly under water.
  %% 
  %% we would never go back and find the match and say this is what caused a devastating bushfire
  %% we know it's a system level property of the forest or bush that means it's on
  %% the edge; any match would do.
  %% 
  %% we can see how forest fires work.  it's straightforward.
  %% dry conditions, dry timber, dead material on the ground, ready to go
  %% 
  %% for social fires, we make two mistakes
  %% 
  %%   \item<2-> Because of properties of special individuals?
  %%   \item<3-> Or system level properties?
  %%   \item<4-> Is the match that lights the fire important?
  %%   \item<5-> Yes.  But only because we are narrative-making machines...
  %%   \item<6-> We like to think things happened for reasons...
  %%   \item<7-> Reasons for success are usually ascribed to intrinsic properties (e.g., Mona Lisa)
  %%   \item<8-> System/group properties harder to understand
  %%   \item<9-> Always good to examine what is said before and after the fact...

  %% \item<8-> \alert{}: Global versus local
  %%   System-level propertives versus individual properties
  %%   \alert{Difficult to understand}  \alert{Easy to understand}

\end{frame}

\changelecturelogo{.18}{cheat-to-win-bracelet-the-onion-tp-1.pdf}

\begin{frame}
  \frametitle{How we want to understand:}
%% and then this!
%% http://www.nytimes.com/2012/10/21/sports/how-armstrongs-wall-fell-one-rider-at-a-time.html?hpw

  \wordwikilink{http://www.nytimes.com/2012/10/18/sports/cycling/inquiry-into-kayle-leogrande-led-to-lance-armstrongs-eventual-fall.html?hp}{`Tattooed Guy' Was Pivotal in Armstrong Case [nytimes]}

  \begin{center}
  \includegraphics[width=\textwidth]{CYCLING-articleLarge.jpg}
  \end{center}

  \begin{itemize}
  \item<1->
    ``... Leogrande's doping \alertb{sparked} a series of events ...''
%%  \item<2-> 
%%    ``Without Leogrande, who knows, the Armstrong investigation maybe never would have happened,'' 
%%    Tygart said.
  \end{itemize}


%%  \begin{center}
%%    \includegraphics[width=0.7\textwidth]{cheat-to-win-bracelet-the-onion.jpg}
%%    \wordwikilink{http://store.theonion.com/p-5045-cheat-to-win-bracelet.aspx}{[the onion]}
%%  \end{center}

\end{frame}

\changelecturelogo{0.15}{icons-lightbulb-tp.pdf}


\begin{frame}

  \frametitle{Reason 1---We are Homo Narrativus:}

  %% we like to tell stories:

  \begin{center}
    \includegraphics[height=0.8\textheight]{xkcd-904-sports-tp-1.pdf}\\
  \end{center}

  \small
  \wordwikilink{http://xkcd.com/904/}{http://xkcd.com/904/}

%%    We are story telling machines.

\end{frame}


\begin{frame}

  \frametitle{Reason 2---``We are all individuals'':}

  \includegraphics[width=\textwidth]{life_magazine_monty_python_crowd_of_brian_desktop_1920x1200_wallpaper-148835.png}
%%  \bigskip
%%  \bigskip
%%  \uncover<1->{
%%    \alertg{Distributed, network minds are hard for us to intuit ...}
%%  }

  %% we don't have natural instincts, frames or metaphors for groups
  %% we're very happy when we can at least pretend that one person is in charge
  %% and the group is operating as some kind of single body

%%  \alertg{We have far less intuition for groups of distributed, networked minds}

\end{frame}


\begin{frame}

  %% lyrebird
  \begin{block}{Reason 3---We are spectacular imitators:}
    \youtubevideo{VjE0Kdfos4Y}{}{}
  \end{block}

  \bigskip

  {\small \alertg{BBC/David Attenborough.}}

\end{frame}


%% \begin{frame}
%% \frametitle{Social Contagion}
%% 
%% \begin{block}<1->{Why do things spread?}
%%   \begin{itemize}
%%   \item<2-> Because of properties of special individuals?
%%   \item<3-> Or system level properties?
%%   \item<4-> Is the match that lights the fire important?
%%   \item<5-> Yes.  But only because we are narrative-making machines...
%%   \item<6-> We like to think things happened for reasons...
%%   \item<7-> Reasons for success are usually ascribed to intrinsic properties (e.g., Mona Lisa)
%%   \item<8-> System/group properties harder to understand
%%   \item<9-> Always good to examine what is said before and after the fact...
%%   \end{itemize}
%% \end{block}
%% 
%% \begin{block}<1->{The big deal here}
%%   \begin{columns}
%%     \begin{column}{0.45\textwidth}
%%       \visible<1->{Global properties} \\
%%       
%%     \end{column}
%%     \begin{column}{0.45\textwidth}
%%       Local \\
%%     \end{column}
%%   \end{columns}
%% \end{block}
%% 
%% \begin{itemize}
%% \item<8-> \alert{}: Global versus local
%%   System-level propertives versus individual properties
%%   \alert{Difficult to understand}  \alert{Easy to understand}
%% \end{itemize}
%% 
%% \end{frame}
%% 
%% 
%% \begin{frame}
%%   Afterwards, it's all obvious (a terrible word).
%% \end{frame}

\section{Superstars}

\begin{frame}
  \frametitle{Mistake 1:\\ Success is due to intrinsic properties}

  \begin{overprint}
    \onslide<1 | handout=1 | trans=1>
    \begin{center}
      \includegraphics[height=0.5\textheight]{monalisa.pdf}%
    \end{center}
    \onslide<2 | handout=0 | trans=0>
    \begin{center}
      \includegraphics[height=0.5\textheight,width=.33\textwidth]{null.pdf}%
    \end{center}
    \onslide<3 | handout=0 | trans=0>
    \begin{center}
      \includegraphics[height=0.5\textheight]{monalisasimpson.jpg}
    \end{center}
    \onslide<4 | handout=0 | trans=0>
    \begin{center}
      \includegraphics[height=0.5\textheight]{s_monalisa.jpg}
    \end{center}
    \onslide<5 | handout=0 | trans=0>
    \begin{center}
      \bigskip
      \includegraphics[height=0.1\textheight]{monalisa.pdf}\\%
      {\tiny it's disappointingly small}
    \end{center}
    \end{overprint}

  %% it's so small! only 30 x 21''

    \bigskip

    \uncover<1->{
      See \wordwikilink{http://www.amazon.com/Becoming-Mona-Lisa-Donald-Sassoon/dp/0156027119}{``Becoming Mona Lisa'' by David Sassoon}
    }
  
%%   \item<2-> 
%%     Not the world's greatest painting from the start...
%%   \item<3->
%%     Escalation through theft, vandalism, \visible<4->{\alert{parody}, ...}
%%   \begin{itemize}
%%   \item<1->
%%  \end{itemize}
%%
%% what about michelangelo or raphael, both pretty solid with a brush?
%% 
%% many other works were copied more in the 1800s
%% really only in the late 1800s and early 1900s that it takes off
%% if it's the greatest painting ever, how come we didn't figure
%% that out straight away
%% some of leonardo's mates should have said, my word, that is just amazing!
%% 
%% monet made sunlight roar out of the wall by painting some hay stacks
  
\end{frame}


%% \begin{frame}
%%   \frametitle{Dominance hierarchies}
%% 
%%   Chase et al. (2002): \alert{``Individual differences versus social dynamics in the formation of animal dominance hierarchies''}\cite{chase2002a}
%% 
%%   The aggressive female Metriaclima zebra:
%%   \begin{center}
%%   \includegraphics<1->[width=0.5\textwidth]{maylandia_lombardoi_wiki.jpg}
%%   \includegraphics[width=.07\textwidth]{wikipedia.jpg}
%%   \end{center}
%% 
%%   Pecking orders for fish... 
%%   
%% \end{frame}
%% 
%% \begin{frame}
%%   \frametitle{Dominance hierarchies}
%% 
%%   \begin{itemize}
%%   \item Fish forget---changing of dominance hierarchies:
%%   \end{itemize}
%% 
%%   \includegraphics<1->[width=0.48\textwidth]{chase2002afig1_1}%
%%   \includegraphics<1->[width=0.48\textwidth]{chase2002afig1_2}%
%% 
%%   \begin{itemize}
%%   \item<2->22 observations: about 3/4 of the time, hierarchy changed
%%   \end{itemize}
%%   
%% \end{frame}


\begin{frame}
%%   \frametitle{Music Lab Experiment}

  \begin{columns}
    \column{0.15\textwidth}
    \column{0.5\textwidth}
    \includegraphics[width=\textwidth]{MusicLab_mainlogo}%%
    \column{0.3\textwidth}
    \begin{tabular}{ll}
      48 songs \\
      30k participants\\    % teenagers
    \end{tabular}
  \end{columns}

  \medskip

  \begin{columns}
    \column{0.48\textwidth}
    Exp 1--- weak social \\
    \includegraphics[width=\textwidth]{ml_info-v1-original} 
    \column{0.48\textwidth}
    Exp. 2---strong social\\
    \includegraphics[width=\textwidth]{ml_info-v2-original} 
  \end{columns}

%%   \begin{columns}
%%     \column{0.45\textwidth}
%%     \includegraphics[width=\textwidth]{instructions-tp.pdf}
%%     \column{0.55\textwidth}
%%     \alertb{9 parallel worlds:\\ 1 independent, 8 social} \\
%%   \end{columns}

    \medskip
    {\small
      \alertg{``An experimental study of inequality and unpredictability in an artificial cultural market,''}\cite{salganik2006a}
      Salganik et al., Science, 2006.
    }

\end{frame}

\begin{frame}
%%  \frametitle{Music Lab Experiment}
  \frametitle{Resolving the paradox:}

  \begin{center}
    \includegraphics[width=.6\textwidth]{salganik2006a_fig3d.pdf}
  \end{center}

  {\Large
    Increased social awareness: \\
    \alertb{Stronger inequality} + \alertb{Less predictability}.
  }
\end{frame}

%% \begin{frame}
%%    \frametitle{Music Lab Experiment}
%% 
%%   \includegraphics[width=0.45\textwidth]{ml_ms_noinfo_info_v1}
%%   \includegraphics[width=0.45\textwidth]{ml_ms_noinfo_info_v2}
%%   
%%   \begin{itemize}
%%   \item Variability in final number of downloads.
%%   \end{itemize}
%%   
%% \end{frame}

%% \begin{frame}
%%   \frametitle{Music Lab Experiment}
%%   \centering
%%   \includegraphics[width=0.45\textwidth]{ml_compare_gini_v1v2_unordered}
%% 
%%   \begin{itemize}
%%   \item Inequality as measured by Gini coefficient:
%%     $$
%%     G= \frac{1}{(2N_{\textrm{s}-1)} \sum_{i=1}^{N_{\textrm{s}} \sum_{j=1}^{N_{\textrm{s}} | m_i - m_j |    
%%     $$
%%   \end{itemize}
%% 
%% \end{frame}
%% 
%% \begin{frame}
%%   \frametitle{Music Lab Experiment}
%%   \centering
%%   \includegraphics[width=0.45\textwidth]{ml_compare_unpred_v1v2}
%% 
%%   \begin{itemize}
%%   \item  Unpredictability
%%     $$ 
%%     U = 
%%     \frac{1}
%%     %% {N_{\textrm{s}(N_{\textrm{w}-1)N_{\textrm{w}/2}
%%     {N_{\textrm{s}\binom{N_{\textrm{w}}{2}}
%%     \sum_{i=1}^{N_{\textrm{s}} 
%%     \sum_{j=1}^{N_{\textrm{w}} 
%%     \sum_{k=j+1}^{N_{\textrm{w}} 
%%     | m_{i,j} - m_{i,k} |  
%%     $$
%%   \end{itemize}
%% 
%% \end{frame}

%% \begin{frame}
%%   \frametitle{Music Lab Experiment}
%% 
%%   \begin{block}{Sensible result:}
%%   \begin{itemize}
%%   \item<1-> Stronger social signal leads to \alert{greater following and greater inequality}.
%%   \end{itemize}
%%   \end{block}
%% 
%%   \begin{block}<2->{Peculiar result:}
%%   \begin{itemize}
%%   \item<3-> Stronger social signal leads to greater \alert{unpredictability}.
%%   \end{itemize}
%%   \end{block}
%% 
%%   \begin{block}<4->{Very peculiar observation:}
%%   \begin{itemize}
%%   \item<5-> The most unequal distributions would suggest the greatest
%%     variation in underlying `quality.'
%%   \item<6-> But success may be due to social construction through \alert{following}.
%%     \visible<7->{\alert{(so let's tell a story...\cite{sunstein2006a,taleb2007a}})}
%%   \end{itemize}
%%   \end{block}

%%   Stronger social signal $\Rightarrow$ \\
%%   Stronger inequality and less predictability.


\begin{frame}
  \frametitle{Payola/Deceptive advertising hurts us all:}

  \begin{center}
    \includegraphics[width=0.9\textwidth]{ml_pair1_34}
    %%   \includegraphics[width=0.49\textwidth]{ml_pair2_34}
  \end{center}

\end{frame}


\section{Superspreading}


\begin{frame}
  \frametitle{Mistake 2:\\ Seeing success is `due to social' and wanting to say
    `all your interactions are belong to us'}

  \includegraphics[width=\textwidth]{facebook_logo.jpg}

%%   Buzzagent
%% Away
%%   video of spreading madness

  %% zuckerburg

  %% much harder problem of understanding what will spread now
  %% half of all advertising money is wasted, we're just not sure
  %% which half

\end{frame}

\begin{frame}
  \frametitle{The hypodermic model of influence:}

  \begin{center}
    \includegraphics[width=0.8\textwidth]{influence_model_hypodermic}    
  \end{center}


\end{frame}

\begin{frame}
  \frametitle{The two step model of influence:\cite{katz1955a}}

  \begin{center}
    \includegraphics[width=0.8\textwidth]{influence_model_twostep}    
  \end{center}

\end{frame}

\begin{frame}
  \frametitle{The network model of influence:}

  \begin{center}
    \includegraphics[width=0.8\textwidth]{influence_model_general}    
  \end{center}

\end{frame}

\begin{frame}
  \frametitle{The network model of influence:}

  \begin{columns}
    \column{0.4\textwidth}
    \includegraphics[width=\textwidth]{influence_model_general}    
    \column{0.6\textwidth}
    \begin{block}{How superspreading works:}
      Many interconnected, average, trusting people\\
      must benefit from both\\
      \alertb{receiving} and \alertb{sharing} a message\\
      far from its source.
    \end{block}
  \end{columns}

  %% average influentials

  %% social wild!

  %% always a system story like fires
  %% not always the most connected people that will lead
  %% to a bigger fire; matters where they are in the social network
  %% groups adopting a new behaviour together can greatly accelerate spreading

  \bigskip
  \bigskip

  \small{
    ``Influentials, Networks, and Public Opinion Formation''\cite{watts2007a}\\
    Watts and Dodds, J. Cons. Res., 2007.
  }

\end{frame}

\begin{frame}

  %% the Internet is a series of pipes that facilitates
  %% the sharing of cat pictures

  %% extremes help us understand slower and deeper spreading

  \frametitle{Things that spread quickly:}

    \includegraphics[height=0.3\textheight]{2000px-Rubiks_cube-tp-1.pdf}
    \ \includegraphics[height=0.3\textheight]{tattoo-cross}
    \ \includegraphics[height=0.3\textheight]{enhanced-buzz-7600-1349980225-3.jpg}

    \bigskip

    \begin{center}
      \includegraphics[width=1.0\textwidth]{2012-10-16buzzfeed-categories.pdf}
    \end{center}

    \uncover<1->{
      {\huge
        + News ...
      }
    }

    \vfill 

    \wordwikilink{http://www.buzzfeed.com}{buzzfeed.com}:


\end{frame}

%% \begin{frame}
%%   \frametitle{LOL + cute + fail + wtf:}
%% 
%%   %% the fail hedgehog
%%   
%%   \begin{center}
%%     \includegraphics[width=0.8\textwidth]{2012-10-16buzzfeed-fail-cropped-tp-2.pdf}
%% 
%%     %%    \includegraphics[width=1.0\textwidth]{2012-10-16buzzfeed-categories.pdf}
%%   \end{center}
%% 
%% \end{frame}

%% \begin{frame}
%% 
%%   \includegraphics[width=1.0\textwidth]{2012-10-16buzzfeed-homepage-cropped.pdf}
%% 
%% \end{frame}

%% \begin{frame}
%%   \frametitle{The whole lolcats thing:}
%% 
%%   \includegraphics[width=\textwidth]{enhanced-buzz-7600-1349980225-3.jpg}
%% \end{frame}
%% 
%% \begin{frame}
%%   \frametitle{Some things really stick:}
%% 
%%   \begin{center}
%%     \includegraphics[height=0.8\textheight]{tattoo-cross}
%%   \end{center}
%% 
%% \end{frame}


%% \begin{frame}
%%   \frametitle{wtf + geeky + omg:}
%% 
%%   \begin{center}
%%     \includegraphics[height=0.8\textheight]{2000px-Rubiks_cube-tp-1.pdf}
%%   \end{center}
%%   
%% \includegraphics[width=.05\textwidth]{wikipedia-tp-3.pdf}
%% 
%% \end{frame}








%% \begin{frame}
%%   \frametitle{Social Contagion}
%% 
%%   \begin{center}
%%     \includegraphics[height=0.8\textheight]{sheeple.png}\\
%%     \wordwikilink{http://xkcd.com/610/}{http://xkcd.com/610/}
%%   \end{center}
%% 
%% \end{frame}




%% \begin{frame}
%%   \frametitle{Social Contagion}
%% 
%%   \includegraphics[width=0.9\textwidth]{044thepumafad_640}
%% 
%% \end{frame}
  
%% \begin{frame}
%%   \frametitle{Social Contagion}
%% 
%%   \begin{block}<+->{Two focuses for us}
%%     \begin{itemize}
%%     \item<+-> Widespread media influence
%%     \item<+-> Word-of-mouth influence
%%     \end{itemize}
%%   \end{block}
%% 
%%   \begin{block}<+->{We need to understand influence}
%%     \begin{itemize}
%%     \item<+-> Who influences whom?  \visible<+->{Very hard to measure...}
%%     \item<+-> What kinds of influence response functions are there?
%%     \item<+-> Are some individuals super influencers?\\
%%       \visible<+->{Highly popularized by Gladwell\cite{gladwell2000a} as `connectors'}
%%     \item<+-> The infectious idea of opinion leaders (Katz and Lazarsfeld)\cite{katz1955a}
%%     \end{itemize}
%%   \end{block}
%% \end{frame}



%% \begin{frame}
%%   \frametitle{Threshold model on a network}
%% 
%%   \begin{block}{}
%%     \begin{center}
%%       \includegraphics<1 | handout:0| trans:0>[angle=-90,width=1\textwidth]{contagioncondition3a}%
%%       \includegraphics<2 | handout:0| trans:0>[angle=-90,width=1\textwidth]{contagioncondition3b}%
%%       \includegraphics<3>[angle=-90,width=1\textwidth]{contagioncondition3c}%
%%     \end{center}
%% 
%%     \begin{itemize}
%%     \item All nodes have threshold $\phi=0.2$.
%%     \end{itemize}
%%   \end{block}
%% 
%% 
%% %% previous work
%% %% definition of vulnerables
%% %% global condition
%% 
%% \end{frame}


\begin{frame}

%%  \frametitle{Dark spreading:}

%% Influence is hard to measure.

  \begin{center}
  \includegraphics[width=1\textwidth]{2012-10-15darksocial-tp-1p5.pdf}

  \medskip

  \tiny \wordwikilink{http://bit.ly/RAaWhl}{Dark Social: We Have the Whole History of the Web Wrong} [The Atlantic]
  \end{center}

  %% Madrigal at the Atlantic

\end{frame}

\begin{frame}
  \frametitle{A completely made up pie chart:}

  \begin{center}
    
    \includegraphics[width=\textwidth]{2012-10-16real-dark-social-paper53-cropped-tp-3}
  \end{center}

\end{frame}


%% sketching of how real success works
%% Clip: The Secret to fame

\insertvideo{_7RHngSRltk}{}{}{How to make things spread:}

%% \begin{frame}
%%   
%%   \frametitle{Shareworthy Content is King:}
%% 
%%   %% content must be about sticking and sharing
%%   %% out in the wild
%% 
%%   \Large
%%   \begin{enumerate}
%%   \item<1->
%%     Build entities/messages/stories that have intrinsic and social value
%%     out in the wild.
%%     %% blackberry versus iphone
%%   \item<2->
%%     \alertb{Advertise but lay off the social interactions.}
%%   \item<3->
%%     Just keep trying.
%%     %%  (at least 13 times).
%%   \end{enumerate}
%% 
%% \end{frame}
 
\begin{comment}

\begin{frame}
  \url{http://www.theatlantic.com/technology/archive/2012/10/dark-social-we-have-the-whole-history-of-the-web-wrong/263523/}

  ``The only real way to optimize for social spread is in the nature of
  the content itself. There's no way to game email or people's instant
  messages. There's no power users you can contact. There's no
  algorithms to understand. This is pure social, uncut.''
\end{frame}

%%  Loop back to Harry Potter.
%%
%%  Picture?
%%  But remember, your novel really may be terrible
%%  and may never take off in any decent society.


  
\begin{frame}
  \frametitle{Social contagion}

  \begin{block}{Summary}
    \begin{itemize}

    \item<1-> \alert{`Influential vulnerables'} are key to spread.
    \item<2-> Early adopters are mostly vulnerables.
    \item<3-> Vulnerable nodes important but not necessary.
    \item<4-> Groups may greatly facilitate spread.
    \item<5-> Seems that cascade condition is a global one.
    \item<6-> Most extreme/unexpected cascades occur in highly connected networks 
    \item<7-> `Influentials' are posterior constructs.\\
    \item<8-> Many potential influentials exist.
    \end{itemize}
  \end{block}
  
\end{frame}

%% \begin{frame}
%%   \frametitle{Summary}
%% 
%%   \begin{itemize}
%% %  \item<1-> Cascade initiators not greatly above average.
%% %  \item<2-> Average initiators easier to find and influence.
%% %  \item<4-> Early adopters may be above or below average.
%%   \end{itemize}
%% 
%% \end{frame}

\begin{frame}
  \frametitle{Social contagion}

  \begin{block}{Implications}
  \begin{itemize}
  \item<1->
    Focus on \alertb{the influential vulnerables}.
  \item<2->
    Create entities that can be transmitted successfully
    through many individuals rather than broadcast from one `influential.'
  \item<3->
    Only \alertb{simple ideas} can spread by word-of-mouth.\\
    \qquad (Idea of opinion leaders spreads well...)
  \item<4->
    Want enough individuals who will adopt and display.
  \item<5->
    Displaying can be \alertb{passive} = free (yo-yo's, fashion),\\
    or \alertb{active} = harder to achieve (political messages).
  \item<6->
    Entities can be novel or designed to combine with others,
    e.g. block another one.
  \end{itemize}
  \end{block}

  %% don't mess with people's trust

\end{frame}



%% \begin{frame}
%% %%  \frametitle{\wordwikilink{http://en.wikipedia.org/wiki/Terry\_Pratchett}{Pratchett} on stories:}
%%   \frametitle{\wordwikilink{http://en.wikipedia.org/wiki/Terry\_Pratchett}{(Sir Terry) Pratchett's}
%%     \wordwikilink{http://wiki.lspace.org/wiki/Narrativium}{Narrativium}:
%% %%    and 
%% %%    \wordwikilink{http://wiki.lspace.org/wiki/Narrative\_Causality}{Narrative Causality}:
%%   }
%% 
%%   \begin{columns}
%%     \column{0.3\textwidth}
%%     \includegraphics[width=\textwidth]{4463841109_f3dbc05754_pratchett.jpg}
%%     \column{0.7\textwidth}
%%     \begin{block}<+->{}
%%         \begin{itemize}
%%         \item<+-> 
%%           ``The most common element on the disc, although not
%%           included in the list of the standard five: earth, fire, air,
%%           water and surprise. It ensures that everything runs properly
%%           as a story.''
%%         \item<+->
%%           ``A little narrativium goes a long way: the simpler the story,
%%           the better you understand it. Storytelling is the opposite of
%%           reductionism: 26 letters and some rules of grammar are no story
%%           at all.''
%%         \end{itemize}
%%       \end{block}
%%     \end{columns}
%% 
%%     \begin{block}<+->{}
%%       \begin{itemize}
%%       \item
%%         ``Heroes only win when outnumbered, and things which have a
%%         one-in-a-million chance of succeeding often do so.''
%%       \end{itemize}
%%     \end{block}
%% 
%% \end{frame}



%% \begin{frame}
%% 
%%   End with a picture of a turtle.
%% 
%% \end{frame}





Dark social: semaphore

Use of `like.'

Superstars dominate but are unpredictable.  Why?

Section on imitation:
Lyrebird?

Copying: My accent.  
How I would like to speak, how I actually speak.

%% big deal
%% i am giving people something outside of myself
%% it's nothing to do with me
%% i am just the medium

%% really work on the soundbites
%% put them on the slides
%% two, three, four words

Monty Python

Closing remark: Never give up.  Keep trying.

Confederacy of Dunces.

We make two kinds of mistakes about success,
of different levels of sophistication.

1. We think it's only about the thing itself.
``Content is King.''

2. Once we see that the social element
matters, that luck matters, we try to capitalize
by manipulating social interactions.
Amway, certain religions, BuzzAgent, Facebook.

Give examples of ads where people mention
how they tell all their friends.

The end story is that ``Content is King'' but 
with an understanding of the power of 
distributed average people in social networks
and the lack of power of supposedly key individuals.

Part of the regal content has to be socially based.

I can haz cheezburger.
Buzzfeed has capitalized on this.

Jokes, surprises, 

%% http://www.youtube.com/watch?feature=player_embedded&v=vWjvP8PZSP4#!

And news.

Example of a great joke.
A visual joke.

People have to both enjoy receiving, seeing, 
experiencing the thing spreading.  
An idea, an experience, a way of talking,
And then they have to be motivated to share.
That they will gain some benefit, some
enjoyment

Say something about stories and language.

All of this can be used for good and bad.
Language can be used for good and bad.


Over and over, we tell a story after the fact
that explains everything.

We're storytellers, teleology rules our minds.



Title:
How to become famous.
Fame

Why is fame so impossible to predict?
Why fame is impossible to predict.
Why success is impossible to predict.

Why things take off.

Why is global success so unpredictable?
Making sense of influence, social contagion, marketing, and stories.

Social imitation is everywhere.  But there's a weird limit. 
If you start dressing exactly like your best friend,
they you're a psycho.  Emulation is okay

%%%%%%%%%%%%%%

The spreading of language use.
How we don't notice language.
Like.
Anglicisms into the US.
Chalk and cheese.


Create summary first.

Need some boom, boom, boom.

Why advertisers and publishers fail.
Discounting of social.

No such thing as fate.

Why was Harry Potter so hard to predict?
How could publishers not see this?

Musiclab
Social element of success.

Influentials
Social element of spreading.

The enormous power of imitation,
cultural replication.

Sociotechnical.

Contagion


Why is the idea of influentials so influential?

1. Because we much more readily understand individuals
than groups.

2. We take hold of the influential idea because
we can then see how to pay for it.

Homo narrativus.

There is some understanding that social contagion matters 
but the most prominent reactions again have been,
I believe, poorly constructed.  Facebook acts as
some strange character at a party who knows (or wants
to know) what everyone has been doing and bought,
and then broadcasts to everyone else.  It's just unnatural.

Sharing has to remain organic.

The solution: make `better' products.

Content is King.

Make things that people will want to share.

Things that are forced will not go beyond two
steps into the social network.  Away from
the source, the message has to carry itself.

For successful spreading via word-of-moth to work, 
there are two main requirements:

1. People must somehow benefit from hearing
the message, and then must want to pass it on.

2. The message must not degrade or change.

Jokes make great examples.

(Think of a great joke.)

  
\begin{frame}
  \url{http://www.theatlantic.com/technology/archive/2012/10/dark-social-we-have-the-whole-history-of-the-web-wrong/263523/}

  ``The only real way to optimize for social spread is in the nature of
  the content itself. There's no way to game email or people's instant
  messages. There's no power users you can contact. There's no
  algorithms to understand. This is pure social, uncut.''
\end{frame}

\end{comment}

