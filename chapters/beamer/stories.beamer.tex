%% Stories:

%% http://nautil.us/blog/why-natural-selection-became-darwins-fittest-metaphor

\section{Narrativium}

\begin{frame}
  \small
  %% \frametitle{\wordwikilink{http://en.wikipedia.org/wiki/Terry\_Pratchett}{Pratchett} on stories:}
  \frametitle{\wordwikilink{http://en.wikipedia.org/wiki/Terry\_Pratchett}{(Sir Terry) Pratchett's}
    \wordwikilink{http://wiki.lspace.org/wiki/Narrativium}{Narrativium}:
    %% and 
    %% \wordwikilink{http://wiki.lspace.org/wiki/Narrative\_Causality}{Narrative Causality}:
  }

  \begin{columns}
    \column{0.3\textwidth}
    \includegraphics[width=\textwidth]{4463841109_f3dbc05754_pratchett.jpg}
    \column{0.7\textwidth}
    \begin{block}<+->{}
      \begin{itemize}
      \item<+-> 
        ``The most common element on the disc, although not
        included in the list of the standard five: earth, fire, air,
        water and surprise. It ensures that everything runs properly
        as a story.''
      \item<+->
        ``A little narrativium goes a long way: the simpler the story,
        the better you understand it. Storytelling is the opposite of
        reductionism: 26 letters and some rules of grammar are no story
        at all.''
      \end{itemize}
    \end{block}
  \end{columns}

  \begin{block}<+->{}
    \begin{itemize}
    \item
      ``Heroes only win when outnumbered, and things which have a
      one-in-a-million chance of succeeding often do so.''
    \end{itemize}
  \end{block}

\end{frame}

\begin{frame}
  \footnotesize
  
  \begin{block}{
      \wordwikilink{https://aeon.co/essays/why-story-is-used-to-explain-symphonies-and-sport-matches-alike}
      {The story trap} 
      by Philip Ball, 2015-11-12}

    \begin{center}
      \includegraphics[width=0.6\textwidth]{2015-11-12philip-ball-story-trap-header.jpg}
    \end{center}

    \begin{itemize}
    \item
      ``We use neat stories to explain everything from sports matches
      to symphonies. Is it time to leave the nursery of the mind?''
    \item
      ``\ldots we might wonder if the ultimate intelligibility of the
      universe will be determined not so much by the capacity of our minds
      to formulate the appropriate concepts and equations, but by whether we
      can find a meaningful story to tell about it.''
    \end{itemize}

  \end{block}

\end{frame}

\begin{frame}

  \begin{block}<+->{Competing storytelling organizations:}
    \begin{itemize}
    \item 
      News.
    \item 
      Art.
    \item 
      Music industry.
    \item 
      Books, magazines.
    \item 
      Movie studios, Netflix, HBO, Disney.
    \item 
      Video Games.
    \item 
      Social media: Facebook, Medium, Tumblr, blogs.
    \end{itemize}
  \end{block}

  \begin{block}<+->{Cultural products from \wordwikilink{http://pantheon.media.mit.edu/methods}{Pantheon}:}
    \begin{itemize}
    \item 
      Writers, artists, movie directors, video game directors.
    \end{itemize}
  \end{block}
  
\end{frame}

\begin{frame}
  
 
  \begin{block}<1->{Understanding the Sociotechnocene---Stories:}
    \begin{columns}
      \column{0.02\textwidth}
      \column{0.20\textwidth}
      \includegraphics[width=\textwidth]{xkcd-904-sports.png}\\
      {\tiny
      \wordwikilink{http://xkcd.com/904/}{xkcd.com/904/}}
      \includegraphics[width=\textwidth]{2014-11-15narrative-hierarchy-sketches-stories_001_base_1200px-tp-5.png}\\
      \includegraphics[width=\textwidth]{2014-11-15narrative-hierarchy-sketches-stories_001_broken-story_1200px-tp-5.png}\\
      \includegraphics[width=\textwidth]{2014-11-15narrative-hierarchy-sketches-stories_001_broken_1200px-tp-5.png}\\
      \column{0.77\textwidth}
      \begin{itemize}
      \item<+-> 
        Perhaps: A true science of stories.
      \item<+->
        Claim: \wordwikilink{http://nautil.us/issue/5/fame/homo-narrativus-and-the-trouble-with-fame}{Homo narrativus}---we run on stories.
      \item<+-> 
        Claim:
        \wordwikilink{http://www.uvm.edu/~pdodds/fama/2015/06/04/the-narrative-hierarchystories-and-storytelling-on-all-scales/}{The
          narrative hierarchy and the Scalability of stories}.
      \item<+->
        Research: Extraction of metaphors, frames, narratives, and stories from large-scale text.
      \item<+-> 
        Research: The taxonomy of human stories.
      \item<+-> 
        Harness: Sociotechnical algorithms for measuring/predicting decisions, contagion, demographics, weather, \ldots
      \end{itemize}
    \end{columns}
  \end{block}

\end{frame}

\begin{frame}

  \begin{itemize}
  \item<1-> 
    \wordwikilink{https://twitter.com/adjacentstories/status/725683772092485632}{Adjacent
      narratives}---why mistruths and 
    \wordwikilink{http://www.theweek.co.uk/conspiracy-theories/62926/the-top-ten-conspiracy-theories-in-the-world}{conspiracy theories} exist and flourish:
  \end{itemize}
  \centering
  \includegraphics[width=0.3\textwidth]{2015-09-28adjacent-stories008-tp-5.png}
  \qquad\qquad
  \includegraphics[width=0.3\textwidth]{2015-09-28adjacent-stories003-tp-5.png}\\
  \includegraphics[width=0.3\textwidth]{2015-09-28adjacent-stories004-tp-5.png}
  \qquad\qquad
  \includegraphics[width=0.3\textwidth]{2015-09-28adjacent-stories002-tp-5.png}\\
  \includegraphics[width=0.3\textwidth]{2015-09-28adjacent-stories006-tp-5.png}
  \qquad\qquad
  \includegraphics[width=0.3\textwidth]{2015-09-28adjacent-stories011-tp-5.png}\\
  \includegraphics[width=0.3\textwidth]{2015-09-28adjacent-stories001-tp-5.png}
  \qquad\qquad
  \includegraphics[width=0.3\textwidth]{2015-09-28adjacent-stories010-tp-5.png}

\end{frame}

\begin{frame}

  \begin{block}{
      \wordwikilink{http://content.time.com/time/specials/packages/article/0,28804,1860871\_1860876_1860992,00.html}{1999
        Gallup poll:}
    }
    \begin{itemize}
    \item 
      6\% of Americans said the lunar landings were fake.
    \item 
      5\% were undecided.
    \end{itemize}
  \end{block}


  \begin{block}{Video replay:}
    \youtubevideo{RMINSD7MmT4}{}{}
  \end{block}

\end{frame}

\section{Power}

\begin{frame}

  %% build out this story; play an excerpt
  
  \begin{block}{Story Wars:}
    \begin{columns}
      \column{0.02\textwidth}
      \column{0.48\textwidth}
      \includegraphics[width=\textwidth]{2016-03-25on-the-media-isis-hostage-new-narrative.png}
      \column{0.02\textwidth}
      \column{0.48\textwidth}
      \begin{itemize}
      \item 
        \wordwikilink{https://fr.wikipedia.org/wiki/Nicolas\_Hénin}{Nicholas
          H\'{e}nin}, French Journalist, 
        \wordwikilink{http://www.wnyc.org/story/former-isis-hostage-we-need-new-narrative/}{held captive for 10 months}.
      \end{itemize}
    \end{columns}
  \end{block}

\end{frame}

\begin{frame}
  \footnotesize

  \begin{block}{From the end of the interview:}
    \uncover<+->{
    NICOLAS HENIN: No, it was just like in a movie. And, by the way,
    even the people going to Syria, joining ISIS in Syria to fight,
    even these people see himself as movie characters.
    }
    \uncover<+->{
    They play their own movie. 
    }
    \uncover<+->{
    This is why I think that the most powerful way to fight ISIS
    are not bombs. 
    }
    \uncover<+->{
    It is to kill the narrative. 
    }
    \uncover<+->{
    We have to write another movie. 
    }
    \uncover<+->{
      We have to build other heroes. 
    }
    \uncover<+->{
      And this is why I believe that
      the French are making big mistakes in the ways they, they fight ISIS.
    }

    \medskip
    \uncover<+->{
    We created, for instance, accounts on the social media named “Stop
    Jihadism,” and this is [BLEEP], like they did not understand
    anything. 
    }
    \uncover<+->{
      And I did understand why we are so bad. 
    }
    \uncover<+->{
      It's just because in
      France we don't know how to write TV series properly.
    }

    \medskip
    \uncover<+->{
    [BROOKE LAUGHS]
    }

    \medskip
    \uncover<+->{
      Just because we have no imagination, we cannot just tell beautiful
      stories, create beautiful characters, beautiful heroes. 
    }

    \medskip
    \uncover<+->{
      And this is what we have to do because in our world, in our societies what do
      people want? 
    }
    \uncover<+->{
      They want to become heroes. 
    }
    \uncover<+->{
      They want to be famous. 
    }
    \uncover<+->{
      They want to be, to be recognized.
    }
  \end{block}
  
\end{frame}

\begin{frame}

  \begin{block}<+->{The American Dream = Rags to Riches}
    
    \begin{itemize}
    \item<+->
      The story that anyone can become King or Queen.
    \item<+->
      Story of individual, not the collective.
    \item<+->
      But we know about fame and success:\\
      \uncover<+->{
        The presence of outsized fame in a social system
        means social imitation is a driver of value.
      }
    \item<+->
      Stories of societies can only hold if they
      have been and remain believable.
    \end{itemize}

  \end{block}
  
\end{frame}

\begin{frame}

  \begin{block}<+->{\wordwikilink{http://www.nytimes.com/2016/04/29/opinion/if-not-trump-what.html}{If
        not Trump, what?}, David Brooks, New York Times:}
    \uncover<+->{
    ``We'll probably need a new national story. 
    }
    \uncover<+->{
    Up until now, America's story has been some version of the rags-to-riches story, 
    }
    \uncover<+->{
    the lone individual who rises from the
    }
    \uncover<+->{
    bottom through pluck and work. 
    }
    \uncover<+->{
    But that story isn't working for people anymore, 
    }
    \uncover<+->{
    especially for people who think the system is rigged.''
    }

    \medskip
    \uncover<+->{
      ``I don't know what the new national story will be, 
    }
    \uncover<+->{
    but maybe it will be less individualistic and more redemptive. 
    }
    \uncover<+->{
    Maybe it will be a story about communities that heal those who
    }
    \uncover<+->{
    suffer 
    from addiction, broken homes, trauma, prison and loss, 
    }
    \uncover<+->{
      a story of those who triumph over the isolation, 
      social instability and dislocation so common today.''
    }
  \end{block}

\end{frame}

\begin{frame}

  \begin{block}<+->{Claim: Stories must have real substance to endure}
    \begin{itemize}
    \item<+->
      Enormous disasters: Fabrications of real experiences.
      \begin{itemize}
      \item<+-> 
        Plain old making stuff up:
        \wordwikilink{https://en.wikipedia.org/wiki/A_Million\_Little\_Pieces}{A
          million little pieces} 
        \uncover<+->{
          ...  Oprah will get you.
        }
      \item<+->
        Wikipedia's has a list of
        \wordwikilink{https://en.wikipedia.org/wiki/Fake_memoirs}{famous fake memoirs}.
      \item<+->
        Expansive plagiarism:
        \wordwikilink{https://en.wikipedia.org/wiki/How\_Opal\_Mehta\_Got\_Kissed,\_Got\_Wild,\_and\_Got\_a\_Life}{How
          Opal Mehta Got Kissed, Got Wild, and Got a Life}. \\
        \uncover<+->{
          \#kudos
        }
      \item<+->
        Self-plagiarism and more standard badness:
        \wordwikilink{https://en.wikipedia.org/wiki/Jonah\_Lehrer}{Jonah
          Lehrer}. \\
        \uncover<+->{
          Amazingly: Made up Bob Dylan quotes.
        }
      \item<+->
        Lance Armstrong.
        \uncover<+->{
          Also got to meet Oprah.
        }
      \end{itemize}
    \item<+->
      Enormous power: Fiction that speaks to real experiences.
    \end{itemize}
  \end{block}

\end{frame}

\section{Shapes}

\insertvideo{oP3c1h8v2ZQ}{}{}{Kurt Vonnegut on the shapes of stories:}

\begin{frame}
  \frametitle{\wordwikilink{http://exp.lore.com/post/40411963108/kurt-vonneguts-classic-lecture-on-the-shapes-of-stories}{Kurt Vonnegut on the shapes of stories:}}

  \includegraphics[height=0.9\textheight]{tumblr_mft5lpRiy01r2qa6go1_1280_1.jpg}

\end{frame}

\begin{frame}
  \frametitle{\wordwikilink{http://exp.lore.com/post/40411963108/kurt-vonneguts-classic-lecture-on-the-shapes-of-stories}{Kurt Vonnegut on the shapes of stories:}}

  \includegraphics[height=0.9\textheight]{tumblr_mft5lpRiy01r2qa6go1_1280_2.jpg}
  
\end{frame}

\begin{frame}
  %% Moby Dick

  \includegraphics[width=\textwidth]{moby-dick-new-yorker.png}

  {\small
    The New Yorker,
    December 16, 2013, p.\ 56.}

\end{frame}

\begin{frame}

  \begin{block}{Ron Swanson:}
    \youtubevideo{afWLwPZZv2w}{}{}

    \begin{itemize}
    \item<+->
    ``I hate metaphors. 
    \visible<+->{
      That's why my favorite book is Moby Dick. 
    }
    \visible<+->{
    No frou-frou symbolism. 
    }
    \visible<+->{
    Just a good, simple tale about a man who hates an animal.''
    }
    \end{itemize}

  \end{block}
  
\end{frame}

\begin{frame}[plain]

  \includegraphics[width=1.20\textwidth]{swanson-pyramid-of-greatness-pc-2560x1600.png}

\end{frame}

\begin{frame}
  %% Moby Dick

  \begin{block}{The emotional shapes of stories---Moby Dick:}
    \begin{columns}
      \column{0.7\textwidth}
      \includegraphics[width=\textwidth]{figemotion_in_books_timeseries001_moby_dick_021_002_noname.pdf}\\
      \includegraphics[width=\textwidth]{figemotion_in_books_timeseries001_moby_dick_021_003_noname.pdf}\\
      \includegraphics[width=\textwidth]{figemotion_in_books_timeseries001_moby_dick_021_004_noname.pdf}
      \column{0.3\textwidth}
      \includegraphics[width=\textwidth]{figuniversal_wordshift_insert007_figtestuniversal_wordshift007_noname.pdf}
    \end{columns}
  \end{block}

  \small
  Partly inspired by 
  \wordwikilink{https://www.youtube.com/watch?v=oP3c1h8v2ZQ}{Vonnegut's Shapes of Stories}.
  
\end{frame}

%% \begin{frame}
%%   \centering
%%   \includegraphics[height=0.95\textheight]{figuniversal_wordshift007_figtestuniversal_wordshift008_noname.pdf}
%% \end{frame}

%% \begin{frame}
%%   
%%   \centering
%%   \includegraphics[height=0.95\textheight]{figmanyhapplang_instrument001_noname.pdf}
%%   
%% \end{frame}

\begin{frame}[plain]
  \wordwikilink{http://hedonometer.org/books.html}{Online, interactive Emotional Shapes of Stories}
  for 10,000+ books:
  
  \centering
  \includegraphics[height=0.9\textheight]{2014-09-15frankenstein.png}
  
\end{frame}

\begin{frame}[plain]
  \wordwikilink{http://hedonometer.org/books.html}{Online, interactive Emotional Shapes of Stories}
  for 10,000+ books:
  
  \centering
  \includegraphics[height=0.9\textheight]{2014-09-15harry-potter-all-books.png}
  
\end{frame}

\begin{frame}[plain]
  \wordwikilink{http://hedonometer.org/movies.html}{Online, interactive Emotional Shapes of Stories}
  for 1,000+ movie scripts:
  
  \includegraphics[height=0.7\textheight]{pulpfiction.png}
  
\end{frame}

\begin{frame}
  \href{http://hedonometer.org/books.html?book=Moby\%20Dick}{\includegraphics[width=\textwidth]{mobydick_timeseries.jpg}}

  \footnotesize
  \wordwikilink{http://whyfiles.org/2015/in-10-languages-happy-words-beat-sad-ones/}{http://whyfiles.org/2015/in-10-languages-happy-words-beat-sad-ones/}

\end{frame}

\begin{frame}[plain]
  \href{http://hedonometer.org/books.html?book=Harry\%20Potter\%20and\%20the\%20Deathly\%20Hallows}{\includegraphics[width=1.2\textwidth]{HP-005-01.png}}

  \begin{itemize}
  \item 
    \small
    Reagan et al. in preparation, 2016.
  \end{itemize}

\end{frame}

\begin{frame}

%%   \href{http://hedonometer.org/books.html?book=Harry\%20Potter\%20and\%20the\%20Sorcerer\%27s\%20Stone}{

  \begin{center}
    \includegraphics[height=0.95\textheight]{harry-potter-uvm-tweet.png}
  \end{center}

\end{frame}

\begin{frame}[plain]

  \includegraphics[width=1.2\textwidth]{harry-potter-voltage.jpg}

\end{frame}

\begin{frame}

  \begin{block}{Harry Potter and the Chamber of Plot Devices:}
    \localvideo{2016-06-22andy-reagan-hedonometer.mp4}
  \end{block}

\end{frame}


\begin{frame}
  \displaypaperfullauthor{kiley2016a}{1}

  \begin{center}
    \includegraphics[width=0.9\textwidth,height=0.55\textheight,keepaspectratio]{kiley2015a_fig13_piece_1200px-square.png}
  \end{center}
\end{frame}

\begin{frame}

  \begin{center}
    \includegraphics[height=0.95\textheight]{fig015_2016-01-30Biased_Motifs_71_Ratio_AFL_Ghosts_V2.pdf}
  \end{center}

\end{frame}

\section{Taxonomy}

\begin{frame}

  \begin{block}<+->{The ``I wonder who wrote this?'' Great Man Theory:}
    \displayamazonbook{campbell1991a}
    \bigskip
    \displayamazonbook{campbell2008a}
  \end{block}

  \begin{block}<+->{}
    Highly influential but     
    \wordwikilink{https://www.youtube.com/watch?v=4F4qzPbcFiA}{it's a trap!}
  \end{block}

\end{frame}

\begin{frame}

  \begin{block}{How to write a screenplay:}
    \displayamazonbook{snyder2005a}
    \begin{itemize}
    \item<+-> 
      9 acts.
    \item<+-> 
      Someone important to the main characters gets
      toasted in the second act, blah, blah.
    \item<+-> 
      Believes irony is key.
    \item<+-> 
      Logline = one or two sentence summary.
    \item<+-> 
      Logline fails to be a summary of logline.
    \end{itemize}
  \end{block}

\end{frame}


\begin{frame}

  \begin{block}<+->{Seven ``good'' stories?:}
    \displayamazonbook{booker2005a}
    \begin{itemize}
    \item<+->
      Seven Gateways to the Underworld (?)
    \item<+->
      Overcoming the Monster $\times$2 and the Thrilling escape from
      Death (plot).
    \item<+->
      Rags to Riches (plot).
    \item<+->
      The Quest (plot).
    \item<+->
      Voyage and Return (plot).
    \item<+->
      Comedy $\times$2 (plot but really structure).
    \item<+->
      Tragedy $\times$3 (plot).
    \item<+->
      Rebirth (plot).
    \item<+->
      The Dark Power: From Shadow into Light (master structure).
    \end{itemize}

  \end{block}

\end{frame}

\begin{frame}

  \frametitle{The taxonomy of stories:}

  \begin{columns}
    \column{0.02\textwidth}
    \column{0.45\textwidth}
    \includegraphics[width=\textwidth]{aarne-thompson-uther.png}
    \column{0.02\textwidth}    
    \column{0.51\textwidth}    
    \begin{block}<+->{\wordwikilink{https://en.wikipedia.org/wiki/Folkloristics}{Folkloristics:}}
      \begin{itemize}
      \item
        Academic area formally started around 1900.
      \item
        \wordwikilink{https://en.wikipedia.org/wiki/Aarne–Thompson\_classification\_systems}{Aarne–Thompson classification systems}
      \item
        Motif-based taxonomy.
      \item
        \wordwikilink{http://www.mftd.org/index.php?action=atu}{Online
          classification database}
      \end{itemize}
    \end{block}
  \end{columns}
  
\end{frame}

\begin{frame}
  \footnotesize

  \begin{block}{}
    \displaypaper{abello2012a}{1} 

    \begin{itemize}
    \item<+->
      Motivation: ``As a simple, historical example from the Danish
      materials, no one has yet classified (according to the ATU index) the
      several thousand fairy tales in the collections of the Danish Folklore
      Archive (\wordwikilink{http://www.dafos.dk}{http://www.dafos.dk}),
      nor does it seem anyone ever will.''
    \item<+->
      `Imagine a system in which the complexities of a folklore corpus
      can be explored at different levels of resolution, from the broad
      perspective of ``distant reading'' down to the narrow perspective of
      traditional \wordwikilink{https://en.wikipedia.org/wiki/Close\_reading}{``close reading.''}'
    \end{itemize}
  \end{block}

\end{frame}

%% red riding hood analysis
\framedisplaypaper{tehrani2013a}{1}{fig2}

\begin{frame}

  \begin{block}{Famous folklore scholar:}
    \begin{columns}
      \column{0.02\textwidth}
      \column{0.48\textwidth}
      \includegraphics[width=\textwidth]{The_Simpsons-Jeff_Albertson.png}
      \column{0.02\textwidth}
      \column{0.48\textwidth}
      \begin{itemize}
      \item<+->
        \wordwikilink{https://en.wikipedia.org/wiki/Comic\_Book\_Guy}{Comic
          Book Guy (CBG).}
      \item<+->
        Real name: Jeffrey ``Jeff'' Albertson.
      \item<+->
        Master's degree in Folklore and Mythology.
      \item<+->
        Thesis: translated Lord of the Rings into Klingon.
      \end{itemize}
    \end{columns}
  \end{block}
\end{frame}

\begin{frame}

  \frametitle{The taxonomy of stories:}

  \begin{block}{Fundamental arcs:}
    \begin{itemize}
    \item<+->
      Kill the Monster.
    \item<+->
      Rags to Riches (and Riches to Rags---\textit{Metamophosis}).
    \item<+->
      The Journey: a Search or a Quest.
    \item<+->
      Romance.
    \item<+->
      Narratives in Left Nullspace: All Stories of The Many.
    \end{itemize}
  \end{block}

  \begin{block}<+->{What about comedies?}
    \begin{itemize}
    \item<+-> 
      Comedies are not in themselves a story, but a way
      of telling stories.
    \end{itemize}
  \end{block}
  
\end{frame}

\section{Essence}

\begin{frame}
  \frametitle{Stories are algorithms for life:}

  \begin{block}<+->{Homo narrativus:}
    \begin{itemize}
    \item<+->
      Provide dynamic paths and trajectories.
    \item<+->
      If this, then that.
    \item<+->
      Convey and reinforce how to behave, how not to behave.
    \item<+->
      Full ecology of stories 
      = \\
      Competing, self-defending operating system
      for people's minds.
    \end{itemize}
  \end{block}

  \begin{block}<+->{Aphorisms as algorithms:}
    \begin{itemize}
    \item<+->
      Pride cometh before the fall.
    \item<+->
      A stitch in time saves nine.
    \item<+->
      Look before you leap.
    \item<+->
      Anti-aphorism: The one who hesitates is lost.
    \end{itemize}
  \end{block}
  
\end{frame}

\begin{frame}
  \small
  
  \frametitle{The unifying theme of existence is existence:}

  \begin{block}<+->{The three fundamental events of (non-clone) life:}
    \begin{columns}
      \column{0.02\textwidth}
      \column{0.31\textwidth}
      \begin{itemize}
      \item
        Hatchings.
      \end{itemize}
      \column{0.02\textwidth}
      \column{0.31\textwidth}
      \begin{itemize}
      \item
        Matchings.
      \end{itemize}
      \column{0.02\textwidth}
      \column{0.31\textwidth}
      \begin{itemize}
      \item
        Dispatchings.
      \end{itemize}
      \column{0.02\textwidth}
    \end{columns}
  \end{block}

\end{frame}


\begin{frame}

  \frametitle{The essence of all stories?}

  \begin{block}<+->{It's survival---Life and Death:}
    \begin{itemize}
    \item<+->
      Kill the Monster: Bare survival.
    \item<+->
      Rags to Riches: Flourishing.
    \item<+->
      Romance: Matchings and Hatchings.
    \item<+->
      Journey/Odyssey: Search for a salvation, a ``Holy Grail''.
    \end{itemize}
  \end{block}
  
\end{frame}
