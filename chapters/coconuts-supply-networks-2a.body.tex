\subsection{Damuth's Law}d

Damuth's law relates the population density, $\rho$, of mammalian primary consumers
to their body mass, $m$.  In support of the scaling relationship,
Damuth cites statistics on 307 different groups of mammals\cite{damuth81}.  A more extensive
set of data is used in a more recent paper\cite{damuth93}, where a least squared
regression test of $\log\rho$ vs. $m$ leads to an exponent of $-0.77 \pm 0.05$.
The spread of population density, $\delta\rho$, is typically two to three orders of magnitude
which is not surprising given the varied ecological conditions over which the data was collected.

Because of the large spread of the data, it is clear that many fits
may lead to similar values of least square error.  Given a possible allometric exponent,
$\alpha$, the power law of best fit may be calculated directly from the data,
$\rho(m)=C(\alpha)m^{\alpha}$.  The flucutations, $\rho_i'$ about this possible scaling law are
\begin{equation}
\rho_i'={\rho_i\over C(\alpha)m^{\alpha}}.
\end{equation}
Since evolution is a growth process, it is natural to study the correlation coefficient, $r$,
of $\log\rho_i'$ vs. $m_i$ as a function of the scaling exponent, $\alpha$.  We find $r=0$ 
when $\alpha=0.65$, much closer to 2/3 than to 3/4.  For $\alpha=0.66$, $r=0.009$, while
for $\alpha=0.75$, $r=0.078$, nearly an order of magnitude difference.  

Damuth's law was also intended to give ecologists
a sense of population densities in (relatively) isolated
ecological communities.  In support of this proposition, Damuth extracts the
109 species whose population density was sampled in ecological communities
of three or more species.  Damuth concludes that the mean allometric exponent
of these pooled communities is -0.74, e.g.\ ecological communities obey the 
same allometric scaling structure as the entire ecological community.
This result is tenuous at best.  A closer examination of Damuths constructed communities
reveals that seven of the fifteen examples have less than five individual species,
which leads Damuth to offer the warning that the "individual values for
these regressions are not very reliable".

If we only consider communities with seven or more species, e.g.\ all
of the communities whom Damuth considers to have reliable regressions,
we find something completely different.  The mean of the scaling exponents for the 
communities with reliable regressions is $-0.68\pm 0.10$.
Using 10000 bootstrap samples, we calculated a 95\% bootstrap interval on the 
mean of the exponent.  
The mean exponent from bootstrap analysis is $-0.6803\pm 0.03167$, and the 95\% bootstrap confidence 
interval stretches from -0.6200 to -0.7438.  Note that this interval does not include -3/4!

Any scaling law based on the behavior of a limited number of ecological
communities is clearly subject to criticism.  Our objective here is
to show that the results for population density in ecological communities
subject to similar ecological conditions is by no means inconsistent
with a 2/3 law for basal metabolic rate.  There is still a great need
for further data collection, studies on the role of sample size\cite{woman},
as well as further studies on equipartitioning of ecological resources.
