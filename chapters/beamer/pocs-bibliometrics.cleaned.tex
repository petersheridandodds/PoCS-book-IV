  \textbf{Bradford's law of scattering}

  Fix link...
%  \wikilink{http://en.wikipedia.org/wiki/Bradford%27s_law}

  

  from
http://www.gslis.utexas.edu/~palmquis/courses/biblio.html


  \textbf{Laws of Bibliometrics}

  One of the main areas in bibliometric research concerns the
  application of bibliometric laws. The three most commonly used laws in
  bibliometrics are: Lotka's law of scientific productivity, Bradford's
  law of scatter, and Zipf's law of word occurrence.  Lotka's Law


  \textbf{Lotka's Law}

  Lotka's Law describes the frequency of publication by authors in a
  given field. It states that " . . . the number (of authors) making n
  contributions is about 1/n? of those making one; and the proportion
  of all contributors, that make a single contribution, is about 60
  percent" (Lotka 1926, cited in Potter 1988). This means that out of
  all the authors in a given field, 60 percent will have just one
  publication, and 15 percent will have two publications (1/2? times
  .60). 7 percent of authors will have three publications (1/3? times
  .60), and so on. According to Lotka's Law of scientific
  productivity, only six percent of the authors in a field will
  produce more than 10 articles. Lotka's Law, when applied to large
  bodies of literature over a fairly long period of time, can be
  accurate in general, but not statistically exact. It is often used
  to estimate the frequency with which authors will appear in an
  online catalog (Potter 1988).


  \textbf{Bradford's Law}

  Bradford's Law serves as a general guideline to librarians in
  determining the number of core journals in any given field. It states
  that journals in a single field can be divided into three parts, each
  containing the same number of articles: 1) a core of journals on the
  subject, relatively few in number, that produces approximately
  one-third of all the articles, 2) a second zone, containing the same
  number of articles as the first, but a greater number of journals, and
  3) a third zone, containing the same number of articles as the second,
  but a still greater number of journals. The mathematical relationship
  of the number of journals in the core to the first zone is a constant
  n and to the second zone the relationship is n?. Bradford expressed
  this relationship as 1:n:n?. Bradford formulated his law after
  studying a bibliography of geophysics, covering 326 journals in the
  field. He discovered that 9 journals contained 429 articles, 59
  contained 499 articles, and 258 contained 404 articles. So it took 9
  journals to contribute one-third of the articles, 5 times 9, or 45, to
  produce the next third, and 5 times 5 times 9, or 225, to produce the
  last third. As may be seen, Bradford's Law is not statistically
  accurate, strictly speaking. But it is still commonly used as a
  general rule of thumb (Potter 1988).


  \textbf{Zipf's Law}

  Zipf's Law is often used to predict the frequency of words within a
  text. The Law states that in a relatively lengthy text, if you "list
  the words occurring within that text in order of decreasing
  frequency, the rank of a word on that list multiplied by its
  frequency will equal a constant. The equation for this relationship
  is: r x f = k where r is the rank of the word, f is the frequency,
  and k is the constant (Potter 1988). Zipf illustrated his law with
  an analysis of James Joyce's Ulysses. "He showed that the tenth most
  frequent word occurred 2,653 times, the hundredth most frequent word
  occurred 265 times, the two hundredth word occurred 133 times, and
  so on. Zipf found, then that the rank of the word multiplied by the
  frequency of the word equals a constant that is approximately
  26,500" (Potter 1988). Zipf's Law, again, is not statistically
  perfect, but it is very useful for indexers.


